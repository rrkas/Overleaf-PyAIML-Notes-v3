\usepackage{fontawesome5}


\RequirePackage[hyphens]{url} 
% This package allows proper handling of URLs in the document, ensuring they are properly hyphenated if they span multiple lines.

\usepackage{amssymb} % for symbols like checkmark

\usepackage[top=2cm,bottom=2cm,left=2.2cm,right=2.2cm,headsep=10pt,a4paper]{geometry} 
% Configures the page layout with specified margins and paper size (A4), including top, bottom, left, and right margins, and the header separation.

\usepackage[dvipsnames,table]{xcolor} 
% Provides color support in the document, allowing named colors and coloring of tables and other elements.

\usepackage{mathptmx} 
% Changes the default text font to Adobe Times Roman and the math font to the Symbol, Chancery, and Computer Modern fonts for mathematical symbols.

\usepackage{mathrsfs} 
% Loads the mathrsfs package to provide a script-like math font (commonly used for calligraphic letters in mathematics).

\DeclareMathAlphabet{\mathcal}{OMS}{cmsy}{m}{n} 
% Redefines the \mathcal font to make it less curvy and more suited for mathematical characters.

\usepackage[utf8]{inputenc} 
% Specifies the input encoding of the document as UTF-8, enabling support for special characters and accented letters.

\usepackage[T1]{fontenc} 
% Ensures the use of an 8-bit font encoding that provides access to 256 glyphs, improving the document's character handling, especially for European languages.

\usepackage{amsmath} 
% Provides advanced mathematical features, including various environments and tools for handling mathematical equations and symbols.

\everymath{\displaystyle}   
% Ensures that inline math expressions are displayed in "display style," which is usually used for equations to make them larger and more readable.

\everydisplay{\displaystyle} 
% Ensures that display math (e.g., equations shown on their own line) uses the "display style" for larger symbols and better readability.

\setcounter{MaxMatrixCols}{20} 
% Increases the maximum number of columns allowed in a matrix to 20 (the default is 10), useful for large matrices in math environments.

\usepackage{mathtools} 
% Extends the functionality of the amsmath package, adding additional tools and enhancements for working with mathematical expressions.

\usepackage{float} 
% Allows better control over the placement of figures and tables, particularly with the "H" option that forces figures to appear exactly where they are in the code.

\usepackage{longtable} 
% Provides support for tables that span multiple pages, useful for large tables that need to be split across different pages.

\usepackage{tabularx}


\usepackage{multicol,multirow} 
% The multicol package allows creating multi-column text layouts, and multirow enables the merging of rows in tables.

\usepackage{graphicx} 
% Required for including graphics (images) in your document, allowing for resizing, rotation, and positioning.

\usepackage{enumitem} 
% Enhances the customization of lists (enumerated and itemized), providing control over label format, spacing, and more.
\setlist{nolistsep} % Reduce spacing between bullet points and numbered lists

\usepackage[none]{hyphenat} 
% Disables hyphenation in the document, preventing words from being broken at the end of lines.

\usepackage{bbm} 
% Provides blackboard bold math symbols (e.g., \mathbb{N} for natural numbers) used in mathematics.

\usepackage{bm} 
% Provides support for bolding mathematical symbols and expressions (e.g., \bm{a} for bold vector a).

\usepackage{nameref} 
% Allows for referencing sections, equations, figures, and tables by their names instead of numbers.

\usepackage{datetime} 
% Provides tools for working with dates and times in LaTeX, such as displaying the current date in different formats.

\usepackage{cancel} 
% Adds the ability to "cancel" or strike out parts of mathematical expressions (e.g., crossing out terms).

\usepackage{scrextend} 
% Provides additional functionality for document layout and formatting, such as custom margins and page styles.

% Import the pifont package to use special symbols like checkmarks and crosses
\usepackage{pifont}% http://ctan.org/pkg/pifont

% Define a custom command \xmark to represent a cross mark using the 'ding' symbol (code 55)
\newcommand{\xmark}{\ding{55}}

\usepackage{lipsum} % Inserts dummy text


\usepackage{wrapfig}





\usepackage[style=numeric,sortcites=true,autopunct=true,autolang=hyphen,hyperref=true,abbreviate=false,backref=false,backend=biber]{biblatex}
% The `biblatex` package is used for creating and formatting bibliographies. The options here configure how the bibliography is displayed:
% - `style=numeric`: Uses a numeric citation style (e.g., [1], [2], etc.).
% - `sortcites=true`: Sorts the citations in the order they are referenced.
% - `autopunct=true`: Automatically adds punctuation marks after citations where necessary.
% - `autolang=hyphen`: Ensures correct hyphenation rules based on the language setting.
% - `hyperref=true`: Enables hyperlinks in the bibliography and citation links.
% - `abbreviate=false`: Avoids abbreviation of publisher names.
% - `backref=false`: Disables back references, where the bibliography shows which pages a citation appears on.
% - `backend=biber`: Specifies the backend for processing the bibliography, using `biber` as the citation manager.


\addbibresource{bibliography.bib} 
% Specifies the `.bib` file (bibliography file) containing the references to be used in the document. Here, it's named `bibliography.bib`.

\DeclareSortingTemplate{sort-tmpl}{
    \sort{
        \field{type}
    }
    \sort{
        \field{title}
    }
    \sort{
        \field{year}
    }
}

% This custom sorting scheme defines how references should be sorted in the bibliography. It sorts first by title, then by year, and finally by author.

\ExecuteBibliographyOptions{sorting=sort-tmpl} 
% Applies the sorting scheme defined earlier, sorting the bibliography entries based on the `title` field.

% remove full stop at end of bib entry
\renewcommand{\finentrypunct}{}


\DeclareBibliographyDriver{book}{% 
  \usebibmacro{bibindex}%
  \usebibmacro{begentry}%
  \iffieldundef{type}{}{%
    \textsc{\printfield{type}}:
  }
  \normalfont
  \thefield{title}%
  \iffieldundef{series}{}{
    \\\textsc{Series}: \printfield{series}
  }
  \\\textsc{Author(s)}: \printnames{author}%
  \\\textsc{Publisher \& Year}: \printlist{publisher},%
  \ \printfield{year}%
  \iffieldundef{url}{}{%
    \\ \textsc{url}: \url{\thefield{url}}
  }%
  \usebibmacro{finentry}%
} 


\DeclareBibliographyDriver{online}{% 
    \usebibmacro{bibindex}%
    \usebibmacro{begentry}%
    \iffieldundef{type}{}{%
        \textsc{\printfield{type}}:
      }
    \normalfont\thefield{title}%
    \iffieldundef{url}{}{%
        \\ \textsc{url}: \url{\thefield{url}}
    }%
    \usebibmacro{finentry}%
    } 







\usepackage{array} 
% Provides additional tools for defining and customizing tables. It allows for more control over column formatting, alignment, and other table features.

\usepackage{pdflscape} 
% This package allows you to rotate entire pages to landscape orientation. It is useful when you have wide tables or figures that don't fit in portrait mode.

\usepackage{ragged2e} 
% Provides more flexible options for text alignment, such as ragged-right (`\RaggedRight`) and ragged-left (`\RaggedLeft`), allowing you to control the alignment of text more precisely.

\usepackage{imakeidx} 
% Simplifies the creation of an index in your document. The `\makeindex` command is used to generate an index, with options for customization.

\makeindex 
% This command enables the creation of an index, which can then be populated with terms throughout the document using `\index{}`.

\usepackage{hhline}

\usepackage{adjustbox} 
% Provides a versatile way to adjust and position content (like images or tables), including scaling, rotating, or framing, without affecting the content's layout.

\usepackage{tikz} 
% A powerful package for creating high-quality graphics, including diagrams, plots, and illustrations. The `\usetikzlibrary{arrows.meta, positioning}` enables the use of specific TikZ libraries for arrows and positioning of nodes.

\usetikzlibrary{arrows.meta, positioning} 
% Loads specific TikZ libraries:
% - `arrows.meta`: Provides advanced arrow styles for creating flowcharts and diagrams.
% - `positioning`: Helps with positioning nodes relative to each other, allowing for more flexible layout in diagrams.

\usepackage{caption} 
% Customizes captions for figures, tables, and other floating objects. The `\captionsetup` command allows you to adjust settings like label font style, label name, and separator (e.g., colon after the "Fig." label).

\captionsetup[figure]{labelfont=bf, name=Fig., labelsep=colon} 
% Customizes the caption format for figures. It sets the label font to bold (`labelfont=bf`), changes the name to "Fig." instead of "Figure" (`name=Fig.`), and sets the separator between the label and caption to a colon (`labelsep=colon`).

\usepackage{changepage} 
% Allows for dynamic changes to page margins. For example, it can be used to change the width of the text block for specific parts of the document, useful for formatting long quotes or wide tables.

% \usepackage{lastpage} 
% Provides a reference to the last page in the document. It can be used in headers or footers, for example, to display "Page X of Y" by referencing the last page.





\usepackage[english]{babel} % Load the babel package with the English language option
% - `babel`: Handles language-specific conventions, hyphenation, and typography.
% - `[english]`: Specifies that the document is in English. This enables proper English hyphenation and text formatting.
% This will also ensure that any language-specific elements (like "Table of Contents", "Figure", etc.) are correctly set for English.

\usepackage{csquotes} % Load the csquotes package for correct quotation handling
% The `csquotes` package helps manage quotations according to language conventions, especially when using multiple languages.
% It is recommended when using `biblatex` with `babel` or `polyglossia` to ensure proper formatting of quotes.










% Import the fancyhdr package for customizing headers and footers
\usepackage{fancyhdr} % Required for header and footer configuration


% Customize the chapter and section headers with specific fonts and styles
\renewcommand{\chaptermark}[1]{\markboth{\sffamily\normalsize\bfseries\chaptername\ \thechapter.\ #1}{}} % Chapter text font settings
\renewcommand{\sectionmark}[1]{\markright{\sffamily\normalsize\thesection\hspace{5pt}#1}{}} % Section text font settings

% Set the page style to fancy for applying custom headers and footers
\fancypagestyle{mainmatter}{    
    % Clear all header and footer fields
    \fancyhf{}
    
    % Set the width of the header line to match the text width
    \setlength{\headwidth}{\textwidth} % Width of the header line
    
    % Configure left, center, and right header content
    \fancyhead[L]{{\bookname}} % Left side: Book title in blue
    \fancyhead[C]{\textsc{Chapter \thechapter}} % Center: Current chapter number
    \fancyhead[R]{By \textbf{\authorname}} % Right side: Author name
    
    % Configure center footer to show current page and total pages
    \fancyfoot[C]{Page \sffamily\normalsize\thepage\ of \protect\pageref*{MMLastPage}} % Footer with current and last page number
    
    % Customize the header and footer rule thickness
    \renewcommand{\headrulewidth}{0.5pt} % Width of the rule under the header
    \addtolength{\headheight}{2.5pt} % Increase the spacing around the header slightly
    \renewcommand{\footrulewidth}{0.5pt} % Removes the rule in the footer
}


\fancypagestyle{extra}{
    \fancyhf{} 
    \setlength{\headwidth}{\textwidth} % Width of the header line
    
    \fancyfoot[C]{\sffamily\normalsize\thepage}
    
    \renewcommand{\headrulewidth}{0pt} % Width of the rule under the header
    % \addtolength{\headheight}{2.5pt} % Increase the spacing around the header slightly
    \renewcommand{\footrulewidth}{0.5pt} % Removes the rule in the footer
}



\fancypagestyle{plain}{
    \fancyhead{}
    \renewcommand{\headrulewidth}{0pt}
} % Style for when a plain pagestyle is specified






% Ensure URLs in hyperlinks are broken correctly at hyphens if needed
\PassOptionsToPackage{hyphens}{url}\usepackage{hyperref} 
% Load the hyperref package with additional URL support


% Load the breakurl package for handling URL breaking in PDFs
\usepackage{breakurl} 
% Useful when compiling with DVI to PDF converters (e.g., dvipdfm)


% Configure hyperref settings using \hypersetup
\hypersetup{
    urlcolor=blue, % Set the color of URL links to blue
    linkcolor=blue, % Set the color of internal document links (e.g., sections) to blue
    citecolor=blue, % Set the color of citation links to blue
    % Backreferences to help navigate to citations and references
    backref=true, % Add back references from bibliography to citations
    pagebackref=true, % Include page numbers where citations appear
    % Additional options for improved indexing and appearance
    hyperindex=true, % Enable hyperlinking of index entries
    colorlinks=true, % Enable colored links instead of boxed links
    breaklinks=true, % Allow links to break across lines (useful for long URLs)
    bookmarks=true, % Enable PDF bookmarks for easy navigation
    bookmarksopen=true, % Automatically expand bookmarks in the PDF viewer    
    % Metadata for the PDF document
    pdftitle={\booknameSimple}, % Set the title of the PDF
    pdfauthor={\authornameSimple} % Set the author name in the PDF metadata
}


% Define a custom command to create a single clickable reference with both number and name
\newcommand*{\fullref}[1]{\hyperref[{#1}]{(\protect\ref*{#1}) \protect\nameref*{#1}}} 
% Example: \fullref{sec:example} → (1.1) Section Name
% \ref* provides the section number without a hyperlink
% \nameref* provides the section title without a hyperlink
% \hyperref combines them into one clickable link












\makeatletter % Allows the use of @ in command names, needed for low-level TeX commands

% Define how the chapter head is created
\def\@makechapterhead#1{%
  \vspace*{-20\p@}% Reduce space above chapter title (negative value moves it up)
  {\parindent \z@ \centering % No paragraph indent, center align the chapter title
    \fontsize{15}{17} % Use the normal font style
    \ifnum \c@secnumdepth >\m@ne % Check if section numbering depth is greater than -1
      \if@mainmatter % Ensure this is in the main part of the document (not frontmatter)
        %\huge\bfseries % (Commented out) Previously set large, bold text for chapter number
        \scshape % Use small caps for chapter number
        \@chapapp\space \thechapter % Display "Chapter X" where X is the chapter number
        \par\nobreak % Prevent a page break immediately after the chapter number
        \vskip 5\p@ % (Commented out) Was adding 20pt space after the chapter number
      \fi
    \fi
    % \interlinepenalty\@M % Prevent line breaks within the title (maximum penalty)
    \fontsize{30}{30} \bfseries #1\par\nobreak % Display the chapter title in huge, bold text
    \vskip 20\p@ % Add 40pt space below the chapter title
  }}

% Define how unnumbered chapter heads are created
\def\@schapter#1{\if@twocolumn % Check if the document is in two-column mode
       \@topnewpage[\@makeschapterhead{#1}]% Start a new page for the chapter title in two-column mode
     \else
       \@makeschapterhead{#1}% Otherwise, create the chapter title normally
       \@afterheading % Adjust spacing after the chapter heading
     \fi}
    
% Define how the chapter title is displayed for unnumbered chapters
\def\@makeschapterhead#1{%
  \vspace*{5\p@}% Add 20pt space above the chapter title
  {\parindent \z@ \centering % No indent, center align the title
    \fontsize{15}{17} % Use the normal font style
    \scshape % Use small caps for the chapter title
    % \interlinepenalty\@M % Prevent line breaks within the title
    \fontsize{30}{30} \bfseries #1\par\nobreak % Display the title in huge, bold text
    \vskip 20\p@ % Add 40pt space below the chapter title
  }}

%%%%%%%%%%%%%%%%%              section            %%%%%%%%%%%%%%%%%%%%%%%

% Define the \section command, checking for a star (*) to differentiate numbered and unnumbered sections
\def\section{\@ifstar\unnumberedsection\numberedsection}

% Handle numbered sections: Check for optional arguments using \@ifnextchar
\def\numberedsection{\@ifnextchar[%] 
  \numberedsectionwithtwoarguments\numberedsectionwithoneargument}

% Handle unnumbered sections: Similar to numbered sections but without numbering
\def\unnumberedsection{\@ifnextchar[%] 
  \unnumberedsectionwithtwoarguments\unnumberedsectionwithoneargument}

% Define one-argument version of numbered section by passing the same argument twice
\def\numberedsectionwithoneargument#1{\numberedsectionwithtwoarguments[#1]{#1}}

% Define one-argument version of unnumbered section by passing the same argument twice
\def\unnumberedsectionwithoneargument#1{\unnumberedsectionwithtwoarguments[#1]{#1}}

% Numbered section with two arguments: First argument is for TOC, second is the section title
\def\numberedsectionwithtwoarguments[#1]#2{%
  \ifhmode\par\fi % Ensure we are in vertical mode by breaking the current line if necessary
  \removelastskip % Remove any extra space above
  \vskip 5ex\goodbreak % Add 5ex vertical space and encourage a page break if needed
  \refstepcounter{section}% Increase the section counter and make it referencable
  \phantomsection % Create hyperlink target
  \edef\@currentlabelname{#2} % Set the section title for \nameref
  \hbox to \hsize{\hss\vbox{\advance\hsize by 1cm % Create a centered box for section heading
      \noindent % Prevent paragraph indent
      \leavevmode
      \fontsize{17pt}{17pt}\selectfont
      \bfseries\raggedright % Use large, bold, left-aligned text
      \thesection.\ \texorpdfstring{#2}{}\par % Display section number followed by title
      \vskip -1.5ex % Reduce space below the title
      \noindent\hrulefill % Draw a horizontal line across the page
      }}\nobreak
  \vskip 1ex\nobreak % Add 2ex space below the line and prevent a page break
  \addcontentsline{toc}{section}{%
    \protect\numberline{\thesection}% Add section number to the Table of Contents (TOC)
    #1}% Add section title to the TOC
}

% Unnumbered section with two arguments: Similar to numbered sections but without section number
\def\unnumberedsectionwithtwoarguments[#1]#2{%
  \ifhmode\par\fi % Ensure in vertical mode
  \removelastskip % Remove any extra space above
  \vskip 2ex\goodbreak % Add 5ex vertical space and allow good breakpoints
%  \refstepcounter{section}% (Commented out) No section number since it's unnumbered
  \phantomsection % Create hyperlink target
  \edef\@currentlabelname{#2} % Set the section title for \nameref
  \hbox to \hsize{\hss\vbox{\advance\hsize by 1cm % Center the title using a flexible box
      \noindent
      \leavevmode
      \fontsize{17pt}{17pt}\selectfont
      \bfseries\raggedright % Large, bold, left-aligned title
%      \thesection\ % (Commented out) Section number is not displayed
      \texorpdfstring{#2}{}\par % Display the section title
      \vskip -1.5ex % Reduce space below the title
      \noindent\hrulefill % Draw a horizontal line
      }}\nobreak
  \vskip 1ex\nobreak % Add 2ex space and prevent a page break
  % \addcontentsline{toc}{section}{%
%    \protect\numberline{\thesection}% (Commented out) No number in the TOC
    % #1}% Add section title to the TOC without a number
}


%%%%%%%%%%%%%%%%%              sub-section            %%%%%%%%%%%%%%%%%%%%%%%


% Define \subsection command to handle starred and unstarred subsections
\def\subsection{\@ifstar\unnumberedsubsection\numberedsubsection}
% If the \subsection command has a star (*) → Call \unnumberedsubsection
% If no star (*) → Call \numberedsubsection

% Handle numbered subsections, checking for optional arguments using \@ifnextchar
\def\numberedsubsection{\@ifnextchar[%] 
  \numberedsubsectionwithtwoarguments\numberedsubsectionwithoneargument}
% If optional argument ([...]) is present → Call \numberedsubsectionwithtwoarguments
% If no optional argument → Call \numberedsubsectionwithoneargument

% Handle unnumbered subsections similarly
\def\unnumberedsubsection{\@ifnextchar[%] 
  \unnumberedsubsectionwithtwoarguments\unnumberedsubsectionwithoneargument}

% Define single-argument version of numbered subsection
\def\numberedsubsectionwithoneargument#1{\numberedsubsectionwithtwoarguments[#1]{#1}}
% The same title is used for both display and Table of Contents (TOC)

% Define single-argument version of unnumbered subsection
\def\unnumberedsubsectionwithoneargument#1{\unnumberedsubsectionwithtwoarguments[#1]{#1}}

% Customize the appearance of numbered subsections
\def\numberedsubsectionwithtwoarguments[#1]#2{%
  \ifhmode\par\fi % Ensure the content starts in vertical mode (not in a paragraph)
  \removelastskip % Remove any unnecessary vertical space before the subsection
  \vskip 3ex\goodbreak % Add 3ex vertical space and allow a good break if necessary
  \refstepcounter{subsection}% Increment the subsection counter and make it referencable
  \phantomsection % Create hyperlink target
  \edef\@currentlabelname{#2} % Set the section title for \nameref
  \hbox to \hsize{\hss\vbox{\advance\hsize by 1cm % Create a box with extra width for aesthetics
      \noindent
      \leavevmode
      \fontsize{15pt}{15pt}\selectfont
      \bfseries\raggedright % Use large, bold, left-aligned font
      \thesubsection.\ \texorpdfstring{#2}{}\par % Display subsection number followed by the title
      \vskip -1ex % Reduce space below the title
      \noindent%\hrulefill % Draw a horizontal line below the subsection title
      }}\nobreak
  \vskip 1.5ex\nobreak % Add space below the title and prevent a page break
  \addcontentsline{toc}{subsection}{%
    \protect\numberline{\thesubsection}% Add the numbered subsection to the TOC
    #1}% The first argument (#1) is added to the TOC
}

% Customize the appearance of unnumbered subsections
\def\unnumberedsubsectionwithtwoarguments[#1]#2{%
  \ifhmode\par\fi
  \removelastskip
  \vskip 3ex\goodbreak % Add space before the subsection title
  \phantomsection % Create hyperlink target
  \edef\@currentlabelname{#2} % Set the section title for \nameref
  \hbox to \hsize{\hss\vbox{\advance\hsize by 1cm
      \noindent
      \leavevmode
      \fontsize{15pt}{15pt}\selectfont
      \bfseries\raggedright % Use large, bold, left-aligned font
      \texorpdfstring{#2}{}\par % Display only the subsection title (no number)
      \vskip -1ex
      \noindent%\hrulefill % Draw a horizontal line below the title
      }}\nobreak
  \vskip 1.5ex\nobreak
  % \addcontentsline{toc}{subsection}{%
  %   #1}% Add the subsection title to the TOC without numbering
}

%%%%%%%%%%%%%%%%%              sub-sub-section            %%%%%%%%%%%%%%%%%%%%%%%

% Define \subsubsection command, handling starred and unstarred types
\def\subsubsection{\@ifstar\unnumberedsubsubsection\numberedsubsubsection}

% Handle numbered subsubsections
\def\numberedsubsubsection{\@ifnextchar[%] 
  \numberedsubsubsectionwithtwoarguments\numberedsubsubsectionwithoneargument}

% Handle unnumbered subsubsections
\def\unnumberedsubsubsection{\@ifnextchar[%] 
  \unnumberedsubsubsectionwithtwoarguments\unnumberedsubsubsectionwithoneargument}

% Define single-argument version for numbered subsubsections
\def\numberedsubsubsectionwithoneargument#1{\numberedsubsubsectionwithtwoarguments[#1]{#1}}

% Define single-argument version for unnumbered subsubsections
\def\unnumberedsubsubsectionwithoneargument#1{\unnumberedsubsubsectionwithtwoarguments[#1]{#1}}

% Customize the appearance of numbered subsubsections
\def\numberedsubsubsectionwithtwoarguments[#1]#2{%
  \ifhmode\par\fi
  \removelastskip
  \vskip 2ex\goodbreak % Add smaller space for subsubsections
  \refstepcounter{subsubsection} % Increment the subsubsection counter
  \phantomsection % Create hyperlink target
  \edef\@currentlabelname{#2} % Set the section title for \nameref
  \hbox to \hsize{\hss\vbox{\advance\hsize by 1cm
      \noindent
      \leavevmode
      \fontsize{13pt}{15pt}\selectfont
      \bfseries\raggedright % Use normal size, bold, left-aligned text
      \thesubsubsection.\ #2\par
      \vskip -0.5ex
      \noindent%\hrulefill % Draw a thinner line below
      }}\nobreak
  \vskip 1ex\nobreak % Add small space below the title
  \addcontentsline{toc}{subsubsection}{%
    \protect\numberline{\thesubsubsection}%
    #1}% Add the subsubsection to the TOC
}

% Customize the appearance of unnumbered subsubsections
\def\unnumberedsubsubsectionwithtwoarguments[#1]#2{%
  \ifhmode\par\fi
  \removelastskip
  \vskip 2ex\goodbreak
  \phantomsection % Create hyperlink target
  \edef\@currentlabelname{#2} % Set the section title for \nameref
  \hbox to \hsize{\hss\vbox{\advance\hsize by 1cm
      \noindent
      \leavevmode
      \fontsize{13pt}{15pt}\selectfont
      \bfseries\raggedright
      #2\par
      \vskip -0.5ex
      \noindent%\hrulefill
      }}\nobreak
  \vskip 1ex\nobreak
  % \addcontentsline{toc}{subsubsection}{%
  %   #1}% Add to TOC without numbering
}

\makeatother % End of low-level commands, restoring normal @ behavior











\usepackage[ruled,vlined,linesnumbered,resetcount,algochapter]{algorithm2e} 
% This package provides a way to write algorithms in LaTeX. The options here configure the appearance of algorithms:
% - `ruled`: Adds horizontal lines at the top and bottom of the algorithm.
% - `vlined`: Adds vertical lines to separate the code blocks.
% - `linesnumbered`: Numbers each line in the algorithm.
% - `resetcount`: Resets the line numbering within each new algorithm.
% - `algochapter`: Allows for algorithm numbering to be tied to chapter numbers.

\SetKwComment{Comment}{/* }{ */} 
% This sets the style for comments within the algorithm. The comment text will be enclosed between `/*` and `*/`, and it will be formatted accordingly.




\usepackage{listings} 
% Provides tools for formatting and including source code in the document, with customizable options for syntax highlighting.


% Define custom colors for syntax highlighting
\definecolor{vscode-bg}{rgb}{0.97,0.97,0.97} % Background color (White-ish)
\definecolor{vscode-gray}{rgb}{0.5,0.5,0.5} % Line number color (Gray)
\definecolor{vscode-keyword}{rgb}{0.0,0.0,0.6} % Keywords (Blue)
\definecolor{vscode-string}{rgb}{0.627,0.126,0.941} % Strings (Purple)
\definecolor{vscode-comment}{rgb}{0.0,0.5,0.0} % Comments (Green)
\definecolor{vscode-number}{rgb}{0.098,0.098,0.439} % Numbers (Dark Blue)

% Define a custom style for Python-like code listings using the listings package
\lstdefinestyle{pystyle}{
    backgroundcolor=\color{vscode-bg},   % Set background color
    commentstyle=\color{vscode-comment}, % Style for comments
    keywordstyle=\color{vscode-keyword}, % Style for keywords
    numberstyle=\tiny\color{vscode-gray}, % Style for line numbers
    stringstyle=\color{vscode-string},   % Style for strings
    basicstyle=\ttfamily,                % Use a monospaced font
    breakatwhitespace=false,             % Allow breaking lines anywhere, not just at whitespace
    breaklines=true,                     % Enable automatic line breaking
    captionpos=b,                        % Position captions below listings
    keepspaces=true,                     % Preserve spaces in code
    numbers=left,                        % Display line numbers on the left side
    numbersep=5pt,                       % Distance between line numbers and code
    showspaces=false,                    % Don't show spaces as visible characters
    showstringspaces=false,              % Don't show spaces inside strings as visible characters
    showtabs=false,                      % Don't show tab characters
    tabsize=4,                           % Set tab size to 4 spaces
    postbreak=\mbox{\textcolor{red}{$\hookrightarrow$}\space}, % Add a red arrow at line breaks
    columns=fullflexible,                % Adjust column widths to fit the content
}

% Apply the custom style to all listings
\lstset{style=pystyle}

% Customize the name for code listings in captions and references
\renewcommand{\lstlistingname}{Python Snippet}
\renewcommand{\lstlistlistingname}{List of \lstlistingname s}




\setcounter{tocdepth}{5}
\setcounter{secnumdepth}{5}








\newcommand{\dsum}{\displaystyle\sum}
\newcommand{\tsum}{\textstyle\sum}

\newcommand{\dprod}{\displaystyle\prod}
\newcommand{\tprod}{\textstyle\prod}

\newcommand{\dint}{\displaystyle\int}
\newcommand{\tint}{\textstyle\int}

\newcommand{\dbigcup}{\displaystyle\bigcup}
\newcommand{\tbigcup}{\textstyle\bigcup}

\newcommand{\dParenBrac}[1]{\left(#1\right)}    % () 
\newcommand{\dSquareBrac}[1]{\left[#1\right]}   % []
\newcommand{\dCurlyBrac}[1]{\left\{#1\right\}}  % {}
\newcommand{\dAngleBrac}[1]{\left< #1\right >}  % <>

\newcommand{\dfloor}[1]{\left \lfloor #1 \right \rfloor}
\newcommand{\dceil}[1]{\left \lceil #1 \right \rceil}
\newcommand{\dabs}[1]{\left| #1 \right|}
\newcommand{\dnorm}[1]{\left\| #1 \right\|}

% ⫫ U+2AEB DOUBLE UP TACK
\newcommand{\doubleuptack}{\perp\!\!\!\!\perp}




\newenvironment{customArrayStretch}[1]{
    \begingroup
    \renewcommand{\arraystretch}{#1}
}{
    \renewcommand{\arraystretch}{1}
    \endgroup
}


\newcommand{\tableitemize}[1]{%
%\tightlist%
\vspace{-5pt}
\begin{itemize}[nosep,topsep=0pt,leftmargin=*]%
#1%
\end{itemize}%
\vspace{-\baselineskip}\mbox{}}


\newcommand{\tableenumerate}[1]{%
%\tightlist%
\vspace{-4pt}
\begin{enumerate}[nosep,topsep=0pt,leftmargin=*]%
#1%
\end{enumerate}%
\vspace{-\baselineskip}\mbox{}}






\newcommand{\textbfit}[1]{\textbf{\textit{#1}}}
\newcommand{\textbfsc}[1]{\textbf{\textsc{#1}}}















% Import the titletoc package for customizing the table of contents (TOC)
\usepackage{titletoc} % Required for manipulating the table of contents


% Define a counter for partition pages
\newcounter{partitioncounter} % Create a new counter called partitioncounter
\renewcommand{\thepartitioncounter}{\Roman{partitioncounter}} % Display the counter in Roman numerals (I, II, III...)


% Remove the default margin for the table of contents entries
\contentsmargin{0cm}


% --------------------
% partition text styling
% --------------------
\titlecontents{partition}[-0.25cm] % Indent chapter titles by 1.25 cm
{\addvspace{10pt}\fontsize{19pt}{19pt}\selectfont\bfseries} % Add vertical space and set chapter titles to large, bold font
{\contentslabel[\thecontentslabel]{1cm}\color{black}} % Display chapter numbers in large size with custom color
{} % No additional text before the chapter title
{} % Display page numbers with a dotted line separator


% --------------------
% Chapter text styling
% --------------------
\titlecontents{chapter}[0.75cm] % Indent chapter titles by 1.25 cm
{\addvspace{10pt}\fontsize{17pt}{17pt}\selectfont\bfseries} % Add vertical space and set chapter titles to large, bold font
{\contentslabel[\thecontentslabel]{1cm}\color{black}} % Display chapter numbers in large size with custom color
{} % No additional text before the chapter title
{\fontsize{11pt}{11pt}\selectfont\normalfont\sffamily\;\titlerule*[.5pc]{\_}\;\thecontentspage} % Display page numbers with a dotted line separator

% --------------------
% Section text styling
% --------------------
\titlecontents{section}[0.75cm] % Indent section titles by 1.5 cm
{\addvspace{3pt}\fontsize{14pt}{14pt}\selectfont} % Add vertical space before sections
{\contentslabel[\thecontentslabel]{1.25cm}\color{black}} % Display section numbers with appropriate alignment
{} % No additional text before the section title
{\fontsize{11pt}{11pt}\selectfont\normalfont\sffamily\;\titlerule*[.5pc]{- -}\;\thecontentspage} % Display page numbers with a dotted line separator
[] % No additional customization

% ------------------------
% Subsection text styling
% ------------------------
\titlecontents{subsection}[0.75cm] % Indent subsection titles by 1.75 cm
{\addvspace{3pt}\fontsize{12pt}{12pt}\selectfont} % Add minimal vertical space before subsections
{\contentslabel[\thecontentslabel]{1.5cm}\color{black}} % Display subsection numbers with alignment
{} % No additional text before the subsection title
{\fontsize{11pt}{11pt}\selectfont\normalfont\sffamily\;\titlerule*[.5pc]{.}\;\thecontentspage} % Display page numbers with a dotted line separator
[] % No additional customization

% ----------------------------
% Subsubsection text styling
% ----------------------------
\titlecontents{subsubsection}[0.75cm] % Indent subsubsection titles by 2 cm
{\addvspace{3pt}} % Add minimal vertical space
{\contentslabel[\thecontentslabel]{1.75cm}\color{black}} % Display subsubsection numbers with alignment
{} % No additional text before the subsubsection title
{\fontsize{11pt}{11pt}\selectfont\normalfont\sffamily\;\titlerule*[.5pc]{.}\;\thecontentspage} % Display page numbers with a dotted line separator
[] % No additional customization

% --------------------
% Paragraph text styling
% --------------------
\titlecontents{paragraph}[0.75cm] % Indent paragraph titles by 2.25 cm
{\addvspace{1pt}} % Add minimal vertical space
{\contentslabel[\thecontentslabel]{2cm}\color{black}} % Display paragraph numbers with alignment
{} % No additional text before the paragraph title
{\fontsize{11pt}{11pt}\selectfont\normalfont\sffamily\;\titlerule*[.5pc]{.}\;\thecontentspage} % Display page numbers with a dotted line separator
[] % No additional customization





% Define a reusable command to create partition pages
\newcommand{\partition}[1]{ % #1 is the argument for the partition name
    \cleardoublepage % Ensure the partition starts on an odd-numbered (right-hand) page in a book
    \stepcounter{partitioncounter} % Increment the partition counter
    \thispagestyle{empty} % Remove the page number from the partition page

    \null % Ensure proper alignment by adding an invisible object
    \vfill % Push the content to the center vertically

    \begin{center}
        {
            \fontsize{45}{50}
            \selectfont 
            \textsc{\thepartitioncounter}
        } \\[0.5cm] % Display "Part I, Part II..."
        {
            \fontsize{40}{45}
            \selectfont
            \textsc{#1}
            \par
        }
    \end{center}
    
    \vfill % Add vertical space to balance the design
    \cleardoublepage % Ensure the next chapter starts on an odd-numbered page

    % Add the partition to the table of contents without making it clickable
    \addtocontents{toc}{
        \protect\contentsline{partition}{
            \fontsize{19}{19}\selectfont 
            \textsc{
                \makebox[0.1cm][r]{\thepartitioncounter}
                \hspace{0.25cm}
                #1
            }
        }{}{}
    }
}




















