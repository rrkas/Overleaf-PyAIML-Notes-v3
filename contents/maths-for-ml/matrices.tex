\chapter{Matrices}

\begin{enumerate}
    \item With $m, n \in \mathbb{N}$ a real-valued $(m, n)$ matrix $\bm{A}$ is an $m\cdot n$-tuple of elements $a_{ij}$, $i = 1, \cdots , m$, $j = 1, \cdots , n$, which is ordered according to a rectangular scheme consisting of $m$ rows and $n$ columns:
    \\[0.2cm]
    $
        \bm{A}
        = 
        \begin{bmatrix}
            a_{11} & a_{12} & \cdots & a_{1n} \\
            a_{21} & a_{22} & \cdots & a_{2n} \\
            \vdots & \vdots & \ddots & \vdots \\
            a_{m1} & a_{m2} & \cdots & a_{mn} \\
        \end{bmatrix}
        \hfill
        (\ a_{ij} \in \mathbb{R} \ )
    $
    \hfill \cite{mfml/book/mml/Deisenroth-Faisal-Ong}

    \item $\mathbb{R}^{m\times n}$ is the set of all real-valued $(m, n)$-matrices. $\bm{A} \in \mathbb{R}^{m\times n}$ can be equivalently represented as $\bm{a} \in \mathbb{R}^{mn}$ by stacking all $n$ columns of the matrix into a long vector.
    \hfill \cite{mfml/book/mml/Deisenroth-Faisal-Ong}

    \vspace{0.5cm}

    \item They can be used to compactly represent systems of linear equations, but they also represent linear functions (linear mappings).
    \hfill \cite{mfml/book/mml/Deisenroth-Faisal-Ong}

    \item matrix represents a linear mapping or a collection of vectors.
    \hfill \cite{mfml/book/mml/Deisenroth-Faisal-Ong}
\end{enumerate}


\section{Matrix Operations}
\subsection{Matrix Addition ( $A+B$ ) \cite{mfml/book/mml/Deisenroth-Faisal-Ong}}

The sum of two matrices $A \in \mathbb{R}^{m\times n}$, $B \in \mathbb{R}^{m\times n}$ is defined as the element-wise sum:
\ \\

\hfill
\begin{customArrayStretch}{2}
$
    A + B
    := \begin{bmatrix}
        a_{11} + b_{11} &   \cdots  &  a_{1n} + b_{1n} \\
        \vdots          &   \ddots  &   \vdots  \\
        a_{m1} + b_{m1} &   \cdots  &  a_{mn} + b_{mn} \\
    \end{bmatrix}
    \in \mathbb{R}^{m\times n}
$
\end{customArrayStretch}
\hfill \cite{mfml/book/mml/Deisenroth-Faisal-Ong}

\begin{lstlisting}[
    language=Python,
    caption=Matrix Addition - numPy
]
import numpy as np
import random

m,n = 4,3

A = np.random.randint(-10, 10, size=(m,n))
B = np.random.randint(-10, 10, size=(m,n))

C = A + B

print("A:", A)
print("B:", B)
print("C:", C)
\end{lstlisting}




\subsection{Matrix Multiplication ( $AB$ OR $@$ OR $\cdot$ ) \cite{mfml/book/mml/Deisenroth-Faisal-Ong}}

For matrices $\bm{A} \in \mbbR^{m\times n}$, $\bm{B} \in \mbbR^{n\times k}$, the elements $c_{ij}$ of the product matrices. $\bm{C} = \bm{AB} \in \mbbR^{m\times k}$ are computed as:

\vspace{0.5cm}
\hfill
$
    c_{ij} = \dsum_{l=1}^n a_{il}\ b_{lj}
$
\hfill
$
    i = 1,\cdots,m
    \hspace{1cm}
    j = 1,\cdots,k
$
\hfill \cite{mfml/book/mml/Deisenroth-Faisal-Ong}





\begin{lstlisting}[
    language=Python,
    caption=Matrix Multiplication - numPy
]
import numpy as np

m,n,k = 4,3,5

A = np.random.randint(-10, 10, size=(m, n))
B = np.random.randint(-10, 10, size=(n, k))

C1 = A @ B
C2 = np.matmul(A, B)
C3 = np.dot(A, B)

print("A:\n", A)
print("B:\n", B)
print("C1:\n", C1)
print("C2:\n", C2)
print("C3:\n", C3)
print(np.all(C1 == C2), np.all(C1 == C3))
\end{lstlisting}




\vspace{0.5cm}

\begin{enumerate}
    \item To compute element $c_{ij}$ we multiply the elements of the $i$th row of $\bm{A}$ with the $j$th column of $\bm{B}$ and sum them up.
    \hfill \cite{mfml/book/mml/Deisenroth-Faisal-Ong}

    \item Matrices can only be multiplied if their “neighboring” dimensions match.
    \hfill \cite{mfml/book/mml/Deisenroth-Faisal-Ong}
    \\
    \hfill
    $
        \underset{n\times k}{\underbrace{\bm{A}}}\
        \underset{k\times m}{\underbrace{\bm{B}}}
        =
        \underset{n\times m}{\underbrace{\bm{C}}}
    $
    \hfill \cite{mfml/book/mml/Deisenroth-Faisal-Ong}
    \\
    The product $\bm{BA}$ is not defined if $m \neq n$ since the neighboring dimensions do not match.

    \item Matrix multiplication is \textbf{not} defined as an element-wise operation on matrix elements, i.e., 
    \\
    $c_{ij} \neq a_{ij}\ b_{ij}$ (even if the size of $\bm{A}$, $\bm{B}$ was chosen appropriately).
    \hfill \cite{mfml/book/mml/Deisenroth-Faisal-Ong}
\end{enumerate}


\subsubsection{Properties}

\begin{enumerate}
    \item matrix multiplication is \textbf{not} commutative: $\bm{AB} \neq \bm{BA}$
    \hfill \cite{mfml/book/mml/Deisenroth-Faisal-Ong}

    \item \textbf{Associativity}: 
    $
        \forall 
        \bm{A} \in \mbbR^{m\times n},\ 
        \bm{B} \in \mbbR^{n\times p},\ 
        \bm{C} \in \mbbR^{p\times q}
    $:
    
        \begin{enumerate}
            \item $\bm{ABC} = (\bm{AB})\ \bm{C} = \bm{A}\ (\bm{BC})$
            \hfill \cite{mfml/book/mml/Deisenroth-Faisal-Ong}
        \end{enumerate}

    \item \textbf{Distributivity}: 
    $
        \forall 
        \bm{A},\ \bm{B}\in \mbbR^{m\times n},\ 
        \bm{C},\ \bm{D}\in \mbbR^{n\times p}
    $:

        \begin{enumerate}
            \item $(\bm{A} + \bm{B})\ \bm{C} = \bm{AC} + \bm{BC}$
            \hfill \cite{mfml/book/mml/Deisenroth-Faisal-Ong}
            
            \item $\bm{A}\ (\bm{C} + \bm{D}) = \bm{AC} + \bm{AD}$
            \hfill \cite{mfml/book/mml/Deisenroth-Faisal-Ong}
        \end{enumerate}

    \item Multiplication with the identity matrix: 
    $
        \forall 
        \bm{A},\ \bm{B}\in \mbbR^{m\times n}
    $:
        \begin{enumerate}
            \item $\bm{I}_m\bm{A} = \bm{AI}_n = \bm{A}$
            \hfill $m\neq n \Rightarrow \bm{I}_m \neq \bm{I}_n$
            \hfill \cite{mfml/book/mml/Deisenroth-Faisal-Ong}
        \end{enumerate}
\end{enumerate}
















\subsection{Multiplication by a Scalar ( $\lambda A$ )}

Let $\bm{A} \in \mbbR^{m\times n}$ and $\lambda \in \mbbR$. 
Then $\lambda \bm{A} = \bm{K} \in \mbbR^{m\times n}$, $k_{ij} = \lambda a_{ij}$.
\hfill \cite{mfml/book/mml/Deisenroth-Faisal-Ong}


\begin{enumerate}
    \item $\lambda$ scales each element of $\bm{A}$
    \hfill \cite{mfml/book/mml/Deisenroth-Faisal-Ong}    
\end{enumerate}





\subsubsection{Properties}

$\forall\ \lambda,\ \psi \in \mbbR$:
\vspace{0.2cm}
\begin{enumerate}
    \item \textbf{Associativity}:
    $(\lambda \psi )\bm{C} = \lambda (\psi \bm{C})$ \hfill $\bm{C} \in  \mbbR^{m\times n}$
    \hfill \cite{mfml/book/mml/Deisenroth-Faisal-Ong}
    
    \item $
        \lambda (\bm{BC}) 
        = (\lambda \bm{B})\bm{C} 
        = \bm{B}(\lambda \bm{C}) 
        = (\bm{BC})\lambda 
        $ 
    \hfill $\bm{B} \in  \mbbR^{m\times n}, \bm{C} \in  \mbbR^{n\times k}$
    \hfill \cite{mfml/book/mml/Deisenroth-Faisal-Ong}
    \\
    Note that this allows us to move scalar values around.
    \hfill \cite{mfml/book/mml/Deisenroth-Faisal-Ong}

    \item $
        (\lambda \bm{C}) ^\top  
        = \bm{C}^\top \lambda ^\top  
        = \bm{C}^\top \lambda  
        = \lambda \bm{C}^\top 
    $
    \hfill $\lambda  = \lambda ^\top \  \forall \ \lambda  \in  \mbbR$
    \hfill \cite{mfml/book/mml/Deisenroth-Faisal-Ong}

    \item \textbf{Distributivity}:
    \begin{enumerate}
        \item $(\lambda  + \psi )\bm{C} = \lambda \bm{C} + \psi \bm{C}$
        \hfill $\bm{C} \in  \mbbR^{m\times n}$
        \hfill \cite{mfml/book/mml/Deisenroth-Faisal-Ong}

        \item $\lambda (\bm{B} + \bm{C}) = \lambda \bm{B} + \lambda \bm{C}$
        \hfill $\bm{B}, \bm{C} \in  \mbbR^{m\times n}$
        \hfill \cite{mfml/book/mml/Deisenroth-Faisal-Ong}        
    \end{enumerate}
\end{enumerate}


\begin{lstlisting}[
    language=Python,
    caption=Multiplication by a Scalar - numPy
]
import numpy as np

# Define a matrix or vector
A = np.array([
    [1, 2],
    [3, 4]
])

# Define a scalar
lambda = 5

# Multiply the matrix by the scalar
result = lambda * A

print("Original matrix A:\n", A)
print("Scalar lambda:", lambda)
print("Result of lambda * A:\n", result)
\end{lstlisting}





\subsection{Hadamard product ( $\circ$ OR $\ast$ ) \cite{mfml/book/mml/Deisenroth-Faisal-Ong}}

element-wise multiplication: For $\bm{A}, \bm{B} \in \mbbR^{m\times n}$
\\
\ 
\hfill
$
    \bm{C} = \bm{A} \circ \bm{B} \in \mbbR^{m\times n}
$
\hfill
$
    c_{ij} = a_{ij}\ b_{ij}
$
\hfill
\ 













\begin{lstlisting}[
    language=Python,
    caption=Hadamard product - numPy
]
import numpy as np

m,n = 4,3

A = np.random.randint(-10, 10, size=(m,n))
B = np.random.randint(-10, 10, size=(m,n))

C = A * B

print("A:\n", A)
print("B:\n", B)
print("C:\n", C)
\end{lstlisting}











\subsection{Matrix Transpose ( $A^{\top}$ ) \cite{mfml/book/mml/Deisenroth-Faisal-Ong}}

For $A \in \mathbb{R}^{m\times n}$ the matrix $B \in \mathbb{R}^{n\times m}$ with $b_{ij} = a_{ji}$ is called the transpose of $A$. We write $B = A^\top$.
\hfill \cite{mfml/book/mml/Deisenroth-Faisal-Ong}






\begin{lstlisting}[
    language=Python,
    caption=Matrix Inverse - numPy
]
import numpy as np

m,n = 4,3

A = np.random.randint(-10, 10, size=(m,n))

print("A:\n", A)
print("A transpose:\n", A.T)
\end{lstlisting}







\begin{enumerate}
    \item In general, $A^\top$ can be obtained by writing the columns of $A$ as the rows of $A^\top$
    \hfill \cite{mfml/book/mml/Deisenroth-Faisal-Ong}

\end{enumerate}



\subsubsection{Properties}

\begin{multicols}{2}
\begin{enumerate}[itemsep=0.2cm]
    \item $(A^\top)^\top = A$
    \hfill \cite{mfml/book/mml/Deisenroth-Faisal-Ong}

    \item $(AB)^\top = B^\top\ A^\top$
    \hfill \cite{mfml/book/mml/Deisenroth-Faisal-Ong}

    \item $(A+B)^\top = A^\top + B^\top$
    \hfill \cite{mfml/book/mml/Deisenroth-Faisal-Ong}

    \item $(A^{-1})^\top = (A^\top)^{-1} =: A^{-\top}$
    \hfill \cite{mfml/book/mml/Deisenroth-Faisal-Ong}

\end{enumerate}
\end{multicols}








\subsection{Matrix Inverse ( $A^{-1}$ ) \cite{mfml/book/mml/Deisenroth-Faisal-Ong}}

\begin{lstlisting}[
    language=Python,
    caption=Matrix Inverse - numPy
]
import numpy as np

n = 4

A = np.random.randint(-10, 10, size=(n,n)).astype(float)
A_inv = np.linalg.inv(A)

print("A:\n", A)
print("A_inv:\n", A_inv)
print(A @ A_inv)
print(np.allclose((A @ A_inv) , np.eye(n, n)))
\end{lstlisting}






\begin{enumerate}
    \item Consider a square matrix $\bm{A} \in \mbbR^{n\times n}$.
    Let matrix $\bm{B} \in \mbbR^{n\times n}$ have the property that $\bm{A}\bm{B} = \bm{I}_n = \bm{B}\bm{A}$.
    $\bm{B}$ is called the inverse of $\bm{A}$ and denoted by $\bm{A}^{-1}$.
    \hfill \cite{mfml/book/mml/Deisenroth-Faisal-Ong}

    \item
    \begin{theorem}[Square Matrix: Invertible]
        For any square matrix $\bm{A} \in \mbbR^{n\times n}$ it holds that $\bm{A}$ is invertible if and only if $\det(\bm{A})\neq 0$.
        \hfill \cite{mfml/book/mml/Deisenroth-Faisal-Ong}
    \end{theorem}

    \item \textbf{Not} every matrix $\bm{A}$ possesses an inverse $\bm{A}^{-1}$.
    \hfill \cite{mfml/book/mml/Deisenroth-Faisal-Ong}

    \item Only square matrices might have inverse. Non-square matrices \textbf{don't} have inverse.

    \item When the matrix inverse exists, it is \textbf{unique}.
    \hfill \cite{mfml/book/mml/Deisenroth-Faisal-Ong}

    \item $
        \bm{A} = \begin{bmatrix}
            a_{11} & a_{12} \\
            a_{21} & a_{22} \\
        \end{bmatrix}
        \in \mbbR^{2\times 2}
    $
    \hspace{1cm} and \hspace{1cm}
    $a_{11}\ a_{22} - a_{12}\ a_{21} \neq 0$ (determinant of $A$)\\[0.4cm]
    $\Rightarrow$
    $
        \bm{A}^{-1} =
        \dfrac{1}{a_{11}\ a_{22} - a_{12}\ a_{21}}
        \begin{bmatrix}
            a_{22} & -a_{12} \\
            -a_{21} & a_{11} \\
        \end{bmatrix}
    $
    \hfill \cite{mfml/book/mml/Deisenroth-Faisal-Ong}

\end{enumerate}



\subsubsection{Matrix Inverse using Gaussian Elimination}

\begin{enumerate}
    \item Given a matrix $\bm{A} \in \mbbR^{n\times n}$, we need to find $\bm{X} = \bm{A}^{-1}$ such that $\bm{A}\bm{X} = \bm{I}_n$
    \hfill \cite{mfml/book/mml/Deisenroth-Faisal-Ong}

    \item We can write this down as a set of simultaneous linear equations $\bm{A}\bm{X} = \bm{I}_n$, where we solve for $\bm{X} = [\bm{x}_1| \cdots |\bm{x}_n]$.
    \hfill \cite{mfml/book/mml/Deisenroth-Faisal-Ong}

    \item augmented matrix notation for a compact representation of this set of systems of linear equations:
    \hfill \cite{mfml/book/mml/Deisenroth-Faisal-Ong}
    \\
    .\hfill
    $
        [\bm{A}|\bm{I}_n]
        \curlyrightarrow \cdots \curlyrightarrow
        [\bm{I}_n|\bm{A}^{-1}]
    $
    \hfill \cite{mfml/book/mml/Deisenroth-Faisal-Ong}


\end{enumerate}




\subsubsection{Moore-Penrose pseudo-inverse ($({A}^\top {A})^{-1} {A}^\top$)}

$
    \begin{aligned}
                         & \bm{A}\bm{X} = \bm{I} \\
        \Leftrightarrow\ & \bm{A}^\top \bm{A}\bm{X} = \bm{A}^\top \\
        \Leftrightarrow\ & \bm{X} = (\bm{A}^\top \bm{A})^{-1} \bm{A}^\top \\
    \end{aligned}
$
\hfill \cite{mfml/book/mml/Deisenroth-Faisal-Ong}


\vspace{0.2cm}

\begin{enumerate}
    \item Disadvantages:
    \begin{enumerate}
        \item requires many computations for the matrix-matrix product and computing the inverse of $\bm{A}^\top \bm{A}$.
        \hfill \cite{mfml/book/mml/Deisenroth-Faisal-Ong}

        \item for reasons of numerical precision it is generally not recommended
        \hfill \cite{mfml/book/mml/Deisenroth-Faisal-Ong}
    \end{enumerate}
\end{enumerate}





\subsubsection{Properties}

\begin{multicols}{2}
\begin{enumerate}
    \item $\bm{A}\bm{A}^{-1} = \bm{I} = \bm{A}^{-1}\bm{A}$
    \hfill \cite{mfml/book/mml/Deisenroth-Faisal-Ong}

    \item $(\bm{A}\bm{B})^{-1} = \bm{B}^{-1}\ \bm{A}^{-1}$
    \hfill \cite{mfml/book/mml/Deisenroth-Faisal-Ong}

    \item $(\bm{A} + \bm{B})^{-1} \neq \bm{A}^{-1} + \bm{B}^{-1}$
    \hfill \cite{mfml/book/mml/Deisenroth-Faisal-Ong}

    \item $(\bm{A}^{-1})^\top = (\bm{A}^\top)^{-1} =: \bm{A}^{-\top}$
    \hfill \cite{mfml/book/mml/Deisenroth-Faisal-Ong}

    \item $(\bm{A}^{-1})^{-1} =  \bm{A}$

\end{enumerate}
\end{multicols}






\section{Rank of a matrix}

\begin{enumerate}
    \item \textbf{Definition}: The number of linearly independent columns of a matrix $\bm{A} \in \mathbb{R}^{m\times n}$ equals the number of linearly independent rows and is called the rank of $\bm{A}$ and is denoted by $\text{rk}(\bm{A})$.
    \hfill \cite{mfml/book/mml/Deisenroth-Faisal-Ong}

\end{enumerate}


\subsection{Properties of rank}

\begin{enumerate}
    \item $\text{rk}(\bm{A}) = \text{rk}(\bm{A}^\top)$, i.e., the column rank equals the row rank.
    \hfill \cite{mfml/book/mml/Deisenroth-Faisal-Ong}

    \item The columns of $\bm{A} \in \mathbb{R}^{m\times n}$ span a subspace $U \subseteq \mathbb{R}^m$ with $\dim(U) = \text{rk}(\bm{A})$. 
    We call this subspace the \textbf{image or range}. 
    A basis of $U$ can be found by applying Gaussian elimination to $\bm{A}$ to identify the \textbf{pivot columns}.
    \hfill \cite{mfml/book/mml/Deisenroth-Faisal-Ong}

    \item The rows of $\bm{A} \in  \mathbb{R}^{m\times n}$  span a subspace $W \subseteq \mathbb{R}^n$ with $\dim(W) = \text{rk}(\bm{A})$. 
    A basis of $W$ can be found by applying Gaussian elimination to $\bm{A}^\top$.
    \hfill \cite{mfml/book/mml/Deisenroth-Faisal-Ong}

    \item For all $\bm{A} \in  \mathbb{R}^{n\times n}$ it holds that $\bm{A}$ is regular (invertible) if and only if $\text{rk}(\bm{A}) = n$.
    \hfill \cite{mfml/book/mml/Deisenroth-Faisal-Ong}

    \item For all $\bm{A} \in  \mathbb{R}^{m\times n}$  and all $\bm{b} \in  \mathbb{R}^m$ it holds that the linear equation system $Ax = b$ can be solved if and only if $\text{rk}(\bm{A}) = \text{rk}(\bm{A}|\bm{b})$, where $\bm{A}|\bm{b}$ denotes the augmented system.
    \hfill \cite{mfml/book/mml/Deisenroth-Faisal-Ong}

    \item For $\bm{A} \in  \mathbb{R}^{m\times n}$  the subspace of solutions for $\bm{A}\bm{x} = \bm{0}$ possesses dimension $n - \text{rk}(\bm{A})$. 
    We call this subspace the \textbf{kernel} or the \textbf{null space}.
    \hfill \cite{mfml/book/mml/Deisenroth-Faisal-Ong}

    \item A matrix $\bm{A} \in  \mathbb{R}^{m\times n}$  has \textbf{full rank} if its rank equals the largest possible rank for a matrix of the same dimensions. 
    This means that the rank of a full-rank matrix is the lesser of the number of rows and columns, i.e., $\text{rk}(\bm{A}) = min(m, n)$. 
    A matrix is said to be \textbf{rank deficient} if it does not have full rank.
    \hfill \cite{mfml/book/mml/Deisenroth-Faisal-Ong}

    
\end{enumerate}



\begin{lstlisting}[
    language=Python,
    caption=Rank of a matrix - numPy
]
import numpy as np

# Define your matrix
A = np.array([
    [1, 2, 3],
    [2, 4, 6],
    [1, 0, 1]
])

# Compute the rank
rank = np.linalg.matrix_rank(A)

print("Rank of matrix A:", rank)
\end{lstlisting}









\section{Null Space and Column Space}

\begin{enumerate}
    \item Let us consider $\bm{A} \in \mathbb{R}^{m\times n}$ and a linear mapping $\Phi : \mathbb{R}^n \to \mathbb{R}^m$, $\bm{x} \mapsto \bm{Ax}$.
    \hfill \cite{mfml/book/mml/Deisenroth-Faisal-Ong}

    \item For $\bm{A} = [\bm{a}_1, \cdots , \bm{a}_n]$, where $\bm{a}_i$ are the columns of $\bm{A}$, we obtain
    \hfill \cite{mfml/book/mml/Deisenroth-Faisal-Ong}
    \\
    .\hfill
    $
        Im(\Phi) 
        = \dCurlyBrac{\bm{Ax} : \bm{x} \in \mathbb{R}^ n } 
        = \dCurlyBrac{\dsum ^n _{i=1} x_i \bm{a}_i : x_1, \cdots , x_n \in \mathbb{R}}
        = span[\bm{a}_1, \cdots , \bm{a}_n] \subseteq \mathbb{R}^m
    $
    \hfill \cite{mfml/book/mml/Deisenroth-Faisal-Ong}
    \\
    i.e., the image is the span of the columns of $\bm{A}$, also called the \textbf{column space}. 
    Therefore, the column space (image) is a subspace of $\mathbb{R}^m$, where $m$ is the “height” of the matrix.
    \hfill \cite{mfml/book/mml/Deisenroth-Faisal-Ong}

    \item $\text{rk}(\bm{A}) = \dim(\text{Im}(\Phi))$
    \hfill \cite{mfml/book/mml/Deisenroth-Faisal-Ong}

    \item The kernel/null space $\ker(\Phi)$ is the general solution to the homogeneous system of linear equations $\bm{Ax} = \bm{0}$ and captures all possible linear combinations of the elements in $\mathbb{R}^n$ that produce $\bm{0} \in \mathbb{R}^m$.
    \hfill \cite{mfml/book/mml/Deisenroth-Faisal-Ong}

    \item The kernel is a subspace of $\mathbb{R}^n$ , where $n$ is the “width” of the matrix.
    \hfill \cite{mfml/book/mml/Deisenroth-Faisal-Ong}

    \item The kernel focuses on the relationship among the columns, and we can use it to determine whether/how we can express a column as a linear combination of other columns.
    \hfill \cite{mfml/book/mml/Deisenroth-Faisal-Ong}
\end{enumerate}






\section{Inhomogeneous systems of linear equations and affine subspaces}

\begin{enumerate}
    \item For $\bm{A} \in \mathbb{R}^{m\times n}$ and $\bm{x} \in \mathbb{R}^m$, the solution of the system of linear equations $\bm{A} \bm{\lambda}  = \bm{x}$ is either the empty set or an affine subspace of $\mathbb{R}^n$ of dimension $n - \text{rk}(\bm{A})$. 
    In particular, the solution of the linear equation $\lambda _1 \bm{b}_1 + \cdots + \lambda _n \bm{b}_n = \bm{x}$, where $(\lambda _1, \cdots , \lambda _n) \neq (0, . . . , 0)$, is a hyperplane in $\mathbb{R}^n$ .
    \hfill \cite{mfml/book/mml/Deisenroth-Faisal-Ong}

    \item In $\mathbb{R}^n$ , every $k$-dimensional affine subspace is the solution of an inhomogeneous system of linear equations $\bm{Ax} = \bm{b}$, where $\bm{A} \in \mathbb{R}^{m\times n}$ , $\bm{b} \in \mathbb{R}^m$ and $\text{rk}(\bm{A}) = n - k$. 
    For homogeneous equation systems $\bm{Ax} = \bm{0}$ the solution was a vector subspace, which we can also think of as a special affine space with support point $\bm{x}_0 = \bm{0}$.
    \hfill \cite{mfml/book/mml/Deisenroth-Faisal-Ong}
\end{enumerate}

























\section{Types of matrices}
\subsection{Square matrix}

$A \in \mathbb{R}^{n\times n}$
we call $(n,\ n)$-matrices square matrices because they possess the same number of rows and columns.


\begin{lstlisting}[
    language=Python,
    caption=Square matrix - numPy
]
import numpy as np

n = 4

print(np.random.randint(-10, 10, size=(n, n)))
\end{lstlisting}



\subsection{Identity Matrix ( $I_n$ )}

In $\mathbb{R}^{n\times n}$, we define the identity matrix:\\
\vspace{0.5cm}
\hfill
$
    \matname{I}_n
    := \begin{bmatrix}
        1 & 0 & \cdots & 0 & \cdots & 0 \\
        0 & 1 & \cdots & 0 & \cdots & 0 \\
        \vdots & \vdots & \ddots & \vdots & \ddots & \vdots \\
        0 & 0 & \cdots & 1 & \cdots & 0 \\
        \vdots & \vdots & \ddots & \vdots & \ddots & \vdots \\
        0 & 0 & \cdots & 0 & \cdots & 1 \\
    \end{bmatrix}
    \in \mathbb{R}^{n\times n}
$
\hfill \cite{mfml/book/mml/Deisenroth-Faisal-Ong}
\\
as the $n \times n$-matrix containing $1$ on the diagonal and $0$ everywhere else.







\begin{lstlisting}[
    language=Python,
    caption=Identity Matrix - numPy
]
import numpy as np

n = 4

print(np.eye(n,n))
\end{lstlisting}









\subsection{Regular/ Invertible/ Non-singular matrix}

A matrix $\matname{A}$ is called regular/ invertible/ non-singular if $\matname{A}^{-1}$ exists.
\hfill \cite{mfml/book/mml/Deisenroth-Faisal-Ong}

\begin{enumerate}
    \item if $\matname{A}$ is invertible, then so is $\matname{A}^\top$
    \hfill \cite{mfml/book/mml/Deisenroth-Faisal-Ong}
\end{enumerate}



\subsection{Singular/ Non-invertible matrix}

A matrix $A$ is called singular/ non-invertible if $A^{-1}$ \textbf{doesn't} exists.
\hfill \cite{mfml/book/mml/Deisenroth-Faisal-Ong}





\subsection{Symmetric Matrix}

\begin{lstlisting}[
    language=Python,
    caption=Identity Matrix - numPy
]
import numpy as np

n = 4

A = np.random.randint(-10, 10, size=(n, n))

# converting A to a symmetric matrix
A = (A + A.T) / 2

print(A)
\end{lstlisting}

\begin{enumerate}
    \item A matrix $\bm{A} \in \mathbb{R}^{n\times n}$ is symmetric if $\bm{A} = \bm{A}^\top$
    \hfill \cite{mfml/book/mml/Deisenroth-Faisal-Ong}
    
    \item only $(n,\ n)$-matrices can be symmetric
    \hfill \cite{mfml/book/mml/Deisenroth-Faisal-Ong}

    \item The sum of symmetric matrices $\bm{A},\ \bm{B} \in \mathbb{R}^{n\times n}$ is \textbf{always symmetric}. 
    \hfill \cite{mfml/book/mml/Deisenroth-Faisal-Ong}

    \item although the product of 2 symmetric matrices is \textbf{always defined}, it is generally \textbf{not symmetric}
    \hfill \cite{mfml/book/mml/Deisenroth-Faisal-Ong}

    \item 
    \begin{theorem}[Spectral Theorem]
        If $\bm{A} \in \mathbb{R}^{n\times n}$ is symmetric, there exists an orthonormal basis of the corresponding vector space $V$ consisting of eigenvectors of $\bm{A}$, and each eigenvalue is real.
        \hfill \cite{mfml/book/mml/Deisenroth-Faisal-Ong}
    \end{theorem}
    \begin{enumerate}
        \item A direct implication of the spectral theorem is that the eigen-decomposition of a symmetric matrix $\bm{A}$ exists (with real eigenvalues), and that we can find an ONB of eigenvectors so that $\bm{A} = \bm{PDP}^\top$ , where $\bm{D}$ is diagonal and the columns of $\bm{P}$ contain the eigenvectors.
        \hfill \cite{mfml/book/mml/Deisenroth-Faisal-Ong}
    \end{enumerate}

    \item 
    \begin{theorem}[Symmetric Matrix: always diagonalizable]
        A symmetric matrix $\bm{S} \in \mathbb{R}^{n\times n}$ can always be diagonalized.
        \hfill \cite{mfml/book/mml/Deisenroth-Faisal-Ong}
    \end{theorem}
    \begin{enumerate}
        \item follows directly from the spectral theorem
        \hfill \cite{mfml/book/mml/Deisenroth-Faisal-Ong}

        \item $\bm{P}$ an orthogonal matrix
        \hfill \cite{mfml/book/mml/Deisenroth-Faisal-Ong}

        \item $\bm{A} = \bm{PDP}^{-1}$
        \hfill \cite{mfml/book/mml/Deisenroth-Faisal-Ong}
    \end{enumerate}
\end{enumerate}








\subsection{Symmetric, Positive Definite Matrix}

\begin{enumerate}
    \item \textbf{Definition}: A symmetric matrix $\bm{A} \in \mathbb{R}^{n\times n}$ that satisfies $\forall \bm{x} \in V \backslash \dCurlyBrac{\bm{0}} : \bm{x}^\top \bm{Ax} > 0$ is called \textbf{symmetric, positive definite}, or just \textbf{positive definite}.
    \hfill \cite{mfml/book/mml/Deisenroth-Faisal-Ong}

    \item Consider an $n$-dimensional vector space $V$ with an inner product $\dAngleBrac{\cdot, \cdot} : V \times V \to \mathbb{R}$ and an ordered basis $B = (\bm{b}_1, \cdots , \bm{b}_n)$ of $V$ . 
    Any vectors $\bm{x}, \bm{y} \in  V$ can be written as linear combinations of the basis vectors so that $\bm{x} = \dsum^n _{i=1} \psi _i \bm{b}_i \in  V$ and $\bm{y} = \dsum ^n _{j=1} \lambda _j \bm{b}_j \in  V$ for suitable $\psi _i , \lambda _j \in  \mathbb{R}$. 
    Due to the bilinearity of the inner product, it holds for all $\bm{x}, \bm{y} \in  V$ that
    \hfill \cite{mfml/book/mml/Deisenroth-Faisal-Ong}
    \\
    .\hfill
    $
        \dAngleBrac{\bm{x}, \bm{y}}
        = \dAngleBrac{
            \dsum^n _{i=1} \psi _i \bm{b}_i,
            \dsum ^n _{j=1} \lambda _j \bm{b}_j
        }
        = \dsum^n _{i=1} \dsum ^n _{j=1} \psi _i \dAngleBrac{\bm{b}_i, \bm{b}_j} \lambda _j
        = \hat{\bm{x}} \bm{A} \hat{\bm{y}}
    $
    \hfill \cite{mfml/book/mml/Deisenroth-Faisal-Ong}
    \\
    where $A_{ij} := \dAngleBrac{\bm{b}_i , \bm{b}_j}$ and $\hat{\bm{x}}, \hat{\bm{y}}$ are the coordinates of $\bm{x}$ and $\bm{y}$ with respect to the basis $B$.
    This implies that the inner product $\dAngleBrac{\cdot, \cdot}$ is uniquely determined through $\bm{A}$. 
    The symmetry of the inner product also means that $\bm{A}$ is symmetric. 
    Furthermore, the positive definiteness of the inner product implies that
    \\
    $\forall \bm{x} \in V \backslash \dCurlyBrac{\bm{0}} : \bm{x}^\top \bm{Ax} > 0$.
    \hfill \cite{mfml/book/mml/Deisenroth-Faisal-Ong}

    \item If only $\geq$ holds, then $\bm{A}$ is called \textbf{symmetric, positive semi-definite}.
    \hfill \cite{mfml/book/mml/Deisenroth-Faisal-Ong}

    \item If $\bm{A} \in \mathbb{R}^{n\times n}$ is symmetric, positive definite, then
    \hfill \cite{mfml/book/mml/Deisenroth-Faisal-Ong}
    \\
    .\hfill
    $⟨\bm{x}, \bm{y}⟩ = \hat{\bm{x}}^\top \bm{A} \hat{\bm{y}}$
    \hfill \cite{mfml/book/mml/Deisenroth-Faisal-Ong}
    \\
    defines an inner product with respect to an ordered basis $B$, where $\hat{\bm{x}}$ and $\hat{\bm{y}}$ are the coordinate representations of $\bm{x}, \bm{y} \in V$ with respect to $B$.
    \hfill \cite{mfml/book/mml/Deisenroth-Faisal-Ong}

    \item \textbf{Theorem}: For a real-valued, finite-dimensional vector space $V$ and an ordered basis $B$ of $V$ , it holds that $\dAngleBrac{\cdot, \cdot} : V \times V \to \mathbb{R}$ is an inner product if and only if there exists a symmetric, positive definite matrix $\bm{A} \in \mathbb{R}^{n\times n}$ with
    \hfill \cite{mfml/book/mml/Deisenroth-Faisal-Ong}
    \\
    .\hfill
    $\dAngleBrac{\bm{x}, \bm{y}} = \hat{\bm{x}} ^\top \bm{A} \hat{\bm{y}}$
    \hfill \cite{mfml/book/mml/Deisenroth-Faisal-Ong}
    \\
    The following properties hold if $\bm{A} \in \mathbb{R}^{n\times n}$ is symmetric and positive definite:
    \hfill \cite{mfml/book/mml/Deisenroth-Faisal-Ong}
    \begin{enumerate}
        \item The null space (kernel) of $\bm{A}$ consists only of $\bm{0}$ because $\bm{x} ^\top \bm{Ax} > 0$ for all $\bm{x} \neq \bm{0}$. 
        This implies that $\bm{Ax} \neq \bm{0}$ if $\bm{x} \neq \bm{0}$.
        \hfill \cite{mfml/book/mml/Deisenroth-Faisal-Ong}

        \item The diagonal elements $a_{ii}$ of $\bm{A}$ are positive because $a_{ii} = \bm{e}^\top _i \bm{A} \bm{e}_i > 0$, where $\bm{e}_i$ is the $i$-th vector of the standard basis in $\mathbb{R}^n$ .
        \hfill \cite{mfml/book/mml/Deisenroth-Faisal-Ong}
    \end{enumerate}
\end{enumerate}















\subsection{Equivalent Matrices}

\begin{enumerate}
    \item \textbf{Definition}: Two matrices $\bm{A}, \tilde{\bm{A}} \in \mathbb{R}^{m\times n}$ are equivalent if there exist regular matrices $S \in R^{n\times n}$ and $T \in R^{m\times m}$, such that $\tilde{\bm{A}} = \bm{T} ^{-1}\bm{AS}$.
    \hfill \cite{mfml/book/mml/Deisenroth-Faisal-Ong}

    \item equivalent matrices are not necessarily similar. 
    \hfill \cite{mfml/book/mml/Deisenroth-Faisal-Ong}
\end{enumerate}















\subsection{Similar Matrices}

\begin{enumerate}
    \item \textbf{Definition}: Two matrices $\bm{A}, \tilde{\bm{A}} \in \mathbb{R}^{n\times n}$ are similar if there exists a regular matrix $\bm{S} \in \mathbb{R}^{n\times n}$ with $\tilde{\bm{A}} = \bm{S} ^{-1}\bm{AS}$.
    \hfill \cite{mfml/book/mml/Deisenroth-Faisal-Ong}

    \item Similar matrices are always equivalent.
    \hfill \cite{mfml/book/mml/Deisenroth-Faisal-Ong}
\end{enumerate}








\subsection{Orthogonal Matrix}


\begin{enumerate}
    \item A square matrix $\bm{A} \in \mbbR^{n\times n}$ is an orthogonal matrix if and only if its columns are orthonormal so that
    \hfill \cite{mfml/book/mml/Deisenroth-Faisal-Ong}
    \\
    .\hfill
    $
        \bm{AA}^\top = \bm{I} = \bm{A}^\top \bm{A}
        \hspace{1cm}
        \Rightarrow
        \hspace{1cm}
        \bm{A}^{-1} = \bm{A}^\top
    $
    \hfill \cite{mfml/book/mml/Deisenroth-Faisal-Ong}
    \\
    i.e., the inverse is obtained by simply transposing the matrix.
    \hfill \cite{mfml/book/mml/Deisenroth-Faisal-Ong}
\end{enumerate}









\subsection{Triangular matrix ($T$)}

\begin{enumerate}
    \item We call a square matrix $\bm{T}$ an \textbf{upper-triangular matrix} if $\bm{T}_{ij} = 0$ for upper-triangular matrix $i > j$, i.e., the matrix is zero below its diagonal. 
    \hfill \cite{mfml/book/mml/Deisenroth-Faisal-Ong}

    \item Analogously, we define a \textbf{lower-triangular matrix} as a matrix with zeros above its diagonal. 
    \hfill \cite{mfml/book/mml/Deisenroth-Faisal-Ong}
\end{enumerate}









\section{Describing/summarizing Matrix}
\subsection{Determinant ($\det(A)$ or $\dabs{A}$)}


\begin{table}[H]
    \hfill
    \begin{minipage}{0.45\linewidth}
        \begin{figure}[H]
            \centering
            \includegraphics[
                width=\linewidth,
                height=3cm,
                keepaspectratio,
            ]{images/maths-for-ml/determinant-2d.png}
            \caption*{
                The area of the parallelogram (shaded region) spanned by the vectors $\bm{b}$ and $\bm{g}$ is $\dabs{\det([\bm{b}, \bm{g}])}$.
                \cite{mfml/book/mml/Deisenroth-Faisal-Ong}
            }
        \end{figure}
    \end{minipage}
    \hfill
    \begin{minipage}{0.45\linewidth}
        \begin{figure}[H]
            \centering
            \includegraphics[
                width=\linewidth,
                height=3cm,
                keepaspectratio,
            ]{images/maths-for-ml/determinant-3d.png}
            \caption*{
                The volume of the parallelepiped (shaded volume) spanned by vectors $\bm{r}$, $\bm{b}$, $\bm{g}$ is $\dabs{\det([\bm{r}, \bm{b}, \bm{g}])}$.
                \cite{mfml/book/mml/Deisenroth-Faisal-Ong}
            }
        \end{figure}
    \end{minipage}
    \hfill
\end{table}


.\hfill
$
    \det(\bm{A}) 
    = \dabs{\bm{A}}
    = \begin{vmatrix}
        a_{11} & a_{12} & \cdots & a_{1n} \\
        a_{21} & a_{22} & \cdots & a_{2n} \\
        \vdots & \vdots & \ddots & \vdots \\
        a_{n1} & a_{n2} & \cdots & a_{nn} \\
    \end{vmatrix}
$
\hfill \cite{mfml/book/mml/Deisenroth-Faisal-Ong}

\begin{enumerate}
    \item A determinant is a mathematical object in the analysis and solution of systems of linear equations.
    \hfill \cite{mfml/book/mml/Deisenroth-Faisal-Ong}

    \item Determinants are \textbf{only} defined for square matrices $\bm{A} \in \mathbb{R}^{n\times n}$ ,i.e., matrices with the same number of rows and columns.
    \hfill \cite{mfml/book/mml/Deisenroth-Faisal-Ong}

    \item The determinant of a square matrix $\bm{A} \in \mathbb{R}^{n\times n}$ is a function that maps $\bm{A}$ onto a real number.
    \hfill \cite{mfml/book/mml/Deisenroth-Faisal-Ong}

    \item For $n=1$, $\det(\bm{A}) = \det(a_{11}) = a_{11}$
    \hfill \cite{mfml/book/mml/Deisenroth-Faisal-Ong}

    \item For $n=2$, 
    $
        \det(\bm{A}) 
        = \begin{vmatrix}
            a_{11} & a_{12} \\
            a_{21} & a_{22}
        \end{vmatrix} 
        = a_{11} a_{22} - a_{12} a_{21}
    $
    \hfill \cite{mfml/book/mml/Deisenroth-Faisal-Ong}

    \item (\textbf{Sarrus’ rule}) For $n=3$, 
    \hfill \cite{mfml/book/mml/Deisenroth-Faisal-Ong}
    \\[0.3cm]
    $
        \det(\bm{A}) 
        = \begin{vmatrix}
            a_{11} & a_{12} & a_{13} \\
            a_{21} & a_{22} & a_{23} \\
            a_{31} & a_{32} & a_{33} \\
        \end{vmatrix} 
        = a_{11} a_{22} a_{33}  + a_{21} a_{32} a_{13}  + a_{31} a_{12} a_{23}  - a_{31} a_{22} a_{13}  - a_{11} a_{32} a_{23}  - a_{21} a_{12} a_{33} 
    $
    \hfill \cite{mfml/book/mml/Deisenroth-Faisal-Ong}

    \item For a triangular matrix $\bm{T} \in \mathbb{R}^{n\times n}$ , the determinant is the product of the diagonal elements, i.e., $\det(\bm{T}) = \dsum ^n _{i=1} \bm{T}_{ii}$
    \hfill \cite{mfml/book/mml/Deisenroth-Faisal-Ong}

    \item the determinant $\det(\bm{A})$ is the signed volume of an n-dimensional parallelepiped formed by columns of the matrix $\bm{A}$.
    \hfill \cite{mfml/book/mml/Deisenroth-Faisal-Ong}

    \item The sign of the determinant indicates the orientation of the spanning vectors.
    \hfill \cite{mfml/book/mml/Deisenroth-Faisal-Ong}

    \item \textbf{Theorem} (Laplace Expansion): Consider a matrix $\bm{A} \in \mathbb{R}^{n\times n}$. Then, for all $j = 1, \cdots , n$:
    \hfill \cite{mfml/book/mml/Deisenroth-Faisal-Ong}
    \begin{enumerate}
        \item Expansion along column $j$: 
        $
            \det(\bm{A}) = \dsum^n _{k=1} (-1)^{k+j}\  a_{kj}\ \det(\bm{A}_{k,j} )
        $
        \hfill \cite{mfml/book/mml/Deisenroth-Faisal-Ong}

        \item Expansion along row $j$:
        $
            \det(\bm{A}) = \dsum^n _{k=1} (-1)^{k+j}\  a_{jk}\ \det(\bm{A}_{j,k} )
        $
        \hfill \cite{mfml/book/mml/Deisenroth-Faisal-Ong}

        \item $\bm{A}_{k,j} \in \mathbb{R}^{(n-1)\times(n-1)}$ is the sub-matrix of $\bm{A}$ that we obtain when deleting row $k$ and column $j$.
        \hfill \cite{mfml/book/mml/Deisenroth-Faisal-Ong}

        \item $\det(\bm{A}_{k,j} )$ is called a \textbf{minor} and $(-1)^{k+j}\ \det(\bm{A}_{k,j} )$ a \textbf{cofactor}.
        \hfill \cite{mfml/book/mml/Deisenroth-Faisal-Ong}
    \end{enumerate}
\end{enumerate}











\subsection{Trace ($tr(A)$)}


\begin{enumerate}
    \item The trace of a square matrix $\bm{A} \in \mathbb{R}^{n\times n}$ is defined as $tr(\bm{A}) := \dsum^n _{i=1} a_{ii}$ i.e. , the trace is the \textbf{sum of the diagonal elements} of $\bm{A}$.
    \hfill \cite{mfml/book/mml/Deisenroth-Faisal-Ong}

    
\end{enumerate}


\subsubsection{Properties of Trace}

\begin{enumerate}
    \item $tr(\bm{A} + \bm{B}) = tr(\bm{A}) + tr(\bm{B})$ for $\bm{A}, \bm{B} \in \mathbb{R}^{n\times n}$
    \hfill \cite{mfml/book/mml/Deisenroth-Faisal-Ong}

    \item $tr(\alpha \bm{A}) = \alpha\ tr(\bm{A}), \alpha \in \mathbb{R}$ for $\bm{A} \in \mathbb{R}^{n\times n}$
    \hfill \cite{mfml/book/mml/Deisenroth-Faisal-Ong}
    
    \item $tr(\bm{I}_n) = n$
    \hfill \cite{mfml/book/mml/Deisenroth-Faisal-Ong}
    
    \item $tr(\bm{AB}) = tr(\bm{BA})$ for $\bm{A} \in \mathbb{R}^{n\times k}, \bm{B} \in \mathbb{R}^{k\times n}$
    \hfill \cite{mfml/book/mml/Deisenroth-Faisal-Ong}

    \item The trace is invariant under cyclic permutations, ie, $tr(\bm{AKL}) = tr(\bm{KLA})$ for matrices $\bm{A} \in  \mathbb{R}^{a \times k}, \bm{K} \in  \mathbb{R}^{k \times l}, \bm{L} \in  \mathbb{R}^{l \times a}$. 
    \hfill \cite{mfml/book/mml/Deisenroth-Faisal-Ong}

    \item $
        tr(\bm{xy}^\top) = tr(\bm{y} ^\top \bm{x}) = \bm{y} ^\top \bm{x} \in \mathbb{R}\ 
        \forall \bm{x}, \bm{y} \in \mathbb{R}^n
    $
    \hfill \cite{mfml/book/mml/Deisenroth-Faisal-Ong}

    \item Given $\Phi : V \to V$, then $tr(\Phi) = tr(\bm{A}_\Phi)$. while matrix representations of linear mappings are basis dependent the trace of a linear mapping $\Phi$ is independent of the basis.
    \hfill \cite{mfml/book/mml/Deisenroth-Faisal-Ong}
\end{enumerate}








\subsection{Characteristic Polynomial ($p_A(\lambda)$)}

\begin{enumerate}
    \item
    \begin{definition}[Characteristic Polynomial]
        For $\lambda \in \mbbR$ and a square matrix $\bm{A} \in \mbbR^{n\times n}$:
        \\
        $
            \begin{aligned}
                p_{\bm{A}}(\lambda ) &:= \det(\bm{A} - \lambda \bm{I}) \\
                &= c_0 + c_1\lambda  + c_2\lambda  ^2 + \cdots + c_{n-1}\lambda ^{( n-1)} + (-1)^n\lambda^  n
            \end{aligned}
        $
        \\
        $c_0, \cdots , c_{n-1} \in \mbbR$, is the characteristic polynomial of $\bm{A}$.
    \end{definition}

    \item $c_0 = \det(\bm{A}), c_{n-1} = (-1)^{n-1}\ \tr(\bm{A})$
\end{enumerate}













\section{Matrix Decomposition/Factorization}
















