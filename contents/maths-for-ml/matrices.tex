\chapter{Matrices}

\begin{enumerate}
    \item With $m, n \in \mathbb{N}$ a real-valued $(m, n)$ matrix $\bm{A}$ is an $m\cdot n$-tuple of elements $a_{ij}$, $i = 1, \cdots , m$, $j = 1, \cdots , n$, which is ordered according to a rectangular scheme consisting of $m$ rows and $n$ columns:
    \\[0.2cm]
    $
        \bm{A}
        =
        \begin{bmatrix}
            a_{11} & a_{12} & \cdots & a_{1n} \\
            a_{21} & a_{22} & \cdots & a_{2n} \\
            \vdots & \vdots & \ddots & \vdots \\
            a_{m1} & a_{m2} & \cdots & a_{mn} \\
        \end{bmatrix}
        \hfill
        (\ a_{ij} \in \mbbR \ )
    $
    \hfill \cite{mfml/book/mml/Deisenroth-Faisal-Ong}

    \item $\mbbR^{m\times n}$ is the set of all real-valued $(m, n)$-matrices. $\bm{A} \in \mbbR^{m\times n}$ can be equivalently represented as $\bm{a} \in \mbbR^{mn}$ by stacking all $n$ columns of the matrix into a long vector.
    \hfill \cite{mfml/book/mml/Deisenroth-Faisal-Ong}

    \vspace{0.5cm}

    \item They can be used to compactly represent systems of linear equations, but they also represent linear functions (linear mappings).
    \hfill \cite{mfml/book/mml/Deisenroth-Faisal-Ong}

    \item matrix represents a linear mapping or a collection of vectors.
    \hfill \cite{mfml/book/mml/Deisenroth-Faisal-Ong}
\end{enumerate}


\section{Matrix Operations}
\subsection{Matrix Addition ( $A+B$ ) \cite{mfml/book/mml/Deisenroth-Faisal-Ong}}

The sum of two matrices $A \in \mathbb{R}^{m\times n}$, $B \in \mathbb{R}^{m\times n}$ is defined as the element-wise sum:
\ \\

\hfill
\begin{customArrayStretch}{2}
$
    A + B
    := \begin{bmatrix}
        a_{11} + b_{11} &   \cdots  &  a_{1n} + b_{1n} \\
        \vdots          &   \ddots  &   \vdots  \\
        a_{m1} + b_{m1} &   \cdots  &  a_{mn} + b_{mn} \\
    \end{bmatrix}
    \in \mathbb{R}^{m\times n}
$
\end{customArrayStretch}
\hfill \cite{mfml/book/mml/Deisenroth-Faisal-Ong}

\begin{lstlisting}[
    language=Python,
    caption=Matrix Addition - numPy
]
import numpy as np
import random

m,n = 4,3

A = np.random.randint(-10, 10, size=(m,n))
B = np.random.randint(-10, 10, size=(m,n))

C = A + B

print("A:", A)
print("B:", B)
print("C:", C)
\end{lstlisting}




\subsection{Matrix Multiplication ( $AB$ OR $@$ OR $\cdot$ ) \cite{mfml/book/mml/Deisenroth-Faisal-Ong}}

For matrices $\bm{A} \in \mbbR^{m\times n}$, $\bm{B} \in \mbbR^{n\times k}$, the elements $c_{ij}$ of the product matrices. $\bm{C} = \bm{AB} \in \mbbR^{m\times k}$ are computed as:

\vspace{0.5cm}
\hfill
$
    c_{ij} = \dsum_{l=1}^n a_{il}\ b_{lj}
$
\hfill
$
    i = 1,\cdots,m
    \hspace{1cm}
    j = 1,\cdots,k
$
\hfill \cite{mfml/book/mml/Deisenroth-Faisal-Ong}





\begin{lstlisting}[
    language=Python,
    caption=Matrix Multiplication - numPy
]
import numpy as np

m,n,k = 4,3,5

A = np.random.randint(-10, 10, size=(m, n))
B = np.random.randint(-10, 10, size=(n, k))

C1 = A @ B
C2 = np.matmul(A, B)
C3 = np.dot(A, B)

print("A:\n", A)
print("B:\n", B)
print("C1:\n", C1)
print("C2:\n", C2)
print("C3:\n", C3)
print(np.all(C1 == C2), np.all(C1 == C3))
\end{lstlisting}




\vspace{0.5cm}

\begin{enumerate}
    \item To compute element $c_{ij}$ we multiply the elements of the $i$th row of $\bm{A}$ with the $j$th column of $\bm{B}$ and sum them up.
    \hfill \cite{mfml/book/mml/Deisenroth-Faisal-Ong}

    \item Matrices can only be multiplied if their “neighboring” dimensions match.
    \hfill \cite{mfml/book/mml/Deisenroth-Faisal-Ong}
    \\
    \hfill
    $
        \underset{n\times k}{\underbrace{\bm{A}}}\
        \underset{k\times m}{\underbrace{\bm{B}}}
        =
        \underset{n\times m}{\underbrace{\bm{C}}}
    $
    \hfill \cite{mfml/book/mml/Deisenroth-Faisal-Ong}
    \\
    The product $\bm{BA}$ is not defined if $m \neq n$ since the neighboring dimensions do not match.

    \item Matrix multiplication is \textbf{not} defined as an element-wise operation on matrix elements, i.e., 
    \\
    $c_{ij} \neq a_{ij}\ b_{ij}$ (even if the size of $\bm{A}$, $\bm{B}$ was chosen appropriately).
    \hfill \cite{mfml/book/mml/Deisenroth-Faisal-Ong}
\end{enumerate}


\subsubsection{Properties}

\begin{enumerate}
    \item matrix multiplication is \textbf{not} commutative: $\bm{AB} \neq \bm{BA}$
    \hfill \cite{mfml/book/mml/Deisenroth-Faisal-Ong}

    \item \textbf{Associativity}: 
    $
        \forall 
        \bm{A} \in \mbbR^{m\times n},\ 
        \bm{B} \in \mbbR^{n\times p},\ 
        \bm{C} \in \mbbR^{p\times q}
    $:
    
        \begin{enumerate}
            \item $\bm{ABC} = (\bm{AB})\ \bm{C} = \bm{A}\ (\bm{BC})$
            \hfill \cite{mfml/book/mml/Deisenroth-Faisal-Ong}
        \end{enumerate}

    \item \textbf{Distributivity}: 
    $
        \forall 
        \bm{A},\ \bm{B}\in \mbbR^{m\times n},\ 
        \bm{C},\ \bm{D}\in \mbbR^{n\times p}
    $:

        \begin{enumerate}
            \item $(\bm{A} + \bm{B})\ \bm{C} = \bm{AC} + \bm{BC}$
            \hfill \cite{mfml/book/mml/Deisenroth-Faisal-Ong}
            
            \item $\bm{A}\ (\bm{C} + \bm{D}) = \bm{AC} + \bm{AD}$
            \hfill \cite{mfml/book/mml/Deisenroth-Faisal-Ong}
        \end{enumerate}

    \item Multiplication with the identity matrix: 
    $
        \forall 
        \bm{A},\ \bm{B}\in \mbbR^{m\times n}
    $:
        \begin{enumerate}
            \item $\bm{I}_m\bm{A} = \bm{AI}_n = \bm{A}$
            \hfill $m\neq n \Rightarrow \bm{I}_m \neq \bm{I}_n$
            \hfill \cite{mfml/book/mml/Deisenroth-Faisal-Ong}
        \end{enumerate}
\end{enumerate}
















\subsection{Multiplication by a Scalar ( $\lambda A$ )}

Let $\bm{A} \in \mbbR^{m\times n}$ and $\lambda \in \mbbR$. 
Then $\lambda \bm{A} = \bm{K} \in \mbbR^{m\times n}$, $k_{ij} = \lambda a_{ij}$.
\hfill \cite{mfml/book/mml/Deisenroth-Faisal-Ong}


\begin{enumerate}
    \item $\lambda$ scales each element of $\bm{A}$
    \hfill \cite{mfml/book/mml/Deisenroth-Faisal-Ong}    
\end{enumerate}





\subsubsection{Properties}

$\forall\ \lambda,\ \psi \in \mbbR$:
\vspace{0.2cm}
\begin{enumerate}
    \item \textbf{Associativity}:
    $(\lambda \psi )\bm{C} = \lambda (\psi \bm{C})$ \hfill $\bm{C} \in  \mbbR^{m\times n}$
    \hfill \cite{mfml/book/mml/Deisenroth-Faisal-Ong}
    
    \item $
        \lambda (\bm{BC}) 
        = (\lambda \bm{B})\bm{C} 
        = \bm{B}(\lambda \bm{C}) 
        = (\bm{BC})\lambda 
        $ 
    \hfill $\bm{B} \in  \mbbR^{m\times n}, \bm{C} \in  \mbbR^{n\times k}$
    \hfill \cite{mfml/book/mml/Deisenroth-Faisal-Ong}
    \\
    Note that this allows us to move scalar values around.
    \hfill \cite{mfml/book/mml/Deisenroth-Faisal-Ong}

    \item $
        (\lambda \bm{C}) ^\top  
        = \bm{C}^\top \lambda ^\top  
        = \bm{C}^\top \lambda  
        = \lambda \bm{C}^\top 
    $
    \hfill $\lambda  = \lambda ^\top \  \forall \ \lambda  \in  \mbbR$
    \hfill \cite{mfml/book/mml/Deisenroth-Faisal-Ong}

    \item \textbf{Distributivity}:
    \begin{enumerate}
        \item $(\lambda  + \psi )\bm{C} = \lambda \bm{C} + \psi \bm{C}$
        \hfill $\bm{C} \in  \mbbR^{m\times n}$
        \hfill \cite{mfml/book/mml/Deisenroth-Faisal-Ong}

        \item $\lambda (\bm{B} + \bm{C}) = \lambda \bm{B} + \lambda \bm{C}$
        \hfill $\bm{B}, \bm{C} \in  \mbbR^{m\times n}$
        \hfill \cite{mfml/book/mml/Deisenroth-Faisal-Ong}        
    \end{enumerate}
\end{enumerate}


\begin{lstlisting}[
    language=Python,
    caption=Multiplication by a Scalar - numPy
]
import numpy as np

# Define a matrix or vector
A = np.array([
    [1, 2],
    [3, 4]
])

# Define a scalar
lambda = 5

# Multiply the matrix by the scalar
result = lambda * A

print("Original matrix A:\n", A)
print("Scalar lambda:", lambda)
print("Result of lambda * A:\n", result)
\end{lstlisting}





\subsection{Hadamard product ( $\circ$ OR $\ast$ ) \cite{mfml/book/mml/Deisenroth-Faisal-Ong}}

element-wise multiplication: For $\bm{A}, \bm{B} \in \mbbR^{m\times n}$
\\
\ 
\hfill
$
    \bm{C} = \bm{A} \circ \bm{B} \in \mbbR^{m\times n}
$
\hfill
$
    c_{ij} = a_{ij}\ b_{ij}
$
\hfill
\ 













\begin{lstlisting}[
    language=Python,
    caption=Hadamard product - numPy
]
import numpy as np

m,n = 4,3

A = np.random.randint(-10, 10, size=(m,n))
B = np.random.randint(-10, 10, size=(m,n))

C = A * B

print("A:\n", A)
print("B:\n", B)
print("C:\n", C)
\end{lstlisting}











\subsection{Matrix Transpose ( $A^{\top}$ ) \cite{mfml/book/mml/Deisenroth-Faisal-Ong}}

For $A \in \mathbb{R}^{m\times n}$ the matrix $B \in \mathbb{R}^{n\times m}$ with $b_{ij} = a_{ji}$ is called the transpose of $A$. We write $B = A^\top$.
\hfill \cite{mfml/book/mml/Deisenroth-Faisal-Ong}






\begin{lstlisting}[
    language=Python,
    caption=Matrix Inverse - numPy
]
import numpy as np

m,n = 4,3

A = np.random.randint(-10, 10, size=(m,n))

print("A:\n", A)
print("A transpose:\n", A.T)
\end{lstlisting}







\begin{enumerate}
    \item In general, $A^\top$ can be obtained by writing the columns of $A$ as the rows of $A^\top$
    \hfill \cite{mfml/book/mml/Deisenroth-Faisal-Ong}

\end{enumerate}



\subsubsection{Properties}

\begin{multicols}{2}
\begin{enumerate}[itemsep=0.2cm]
    \item $(A^\top)^\top = A$
    \hfill \cite{mfml/book/mml/Deisenroth-Faisal-Ong}

    \item $(AB)^\top = B^\top\ A^\top$
    \hfill \cite{mfml/book/mml/Deisenroth-Faisal-Ong}

    \item $(A+B)^\top = A^\top + B^\top$
    \hfill \cite{mfml/book/mml/Deisenroth-Faisal-Ong}

    \item $(A^{-1})^\top = (A^\top)^{-1} =: A^{-\top}$
    \hfill \cite{mfml/book/mml/Deisenroth-Faisal-Ong}

\end{enumerate}
\end{multicols}








\subsection{Matrix Inverse ( $A^{-1}$ ) \cite{mfml/book/mml/Deisenroth-Faisal-Ong}}

\begin{lstlisting}[
    language=Python,
    caption=Matrix Inverse - numPy
]
import numpy as np

n = 4

A = np.random.randint(-10, 10, size=(n,n)).astype(float)
A_inv = np.linalg.inv(A)

print("A:\n", A)
print("A_inv:\n", A_inv)
print(A @ A_inv)
print(np.allclose((A @ A_inv) , np.eye(n, n)))
\end{lstlisting}






\begin{enumerate}
    \item Consider a square matrix $\bm{A} \in \mbbR^{n\times n}$.
    Let matrix $\bm{B} \in \mbbR^{n\times n}$ have the property that $\bm{A}\bm{B} = \bm{I}_n = \bm{B}\bm{A}$.
    $\bm{B}$ is called the inverse of $\bm{A}$ and denoted by $\bm{A}^{-1}$.
    \hfill \cite{mfml/book/mml/Deisenroth-Faisal-Ong}

    \item
    \begin{theorem}[Square Matrix: Invertible]
        For any square matrix $\bm{A} \in \mbbR^{n\times n}$ it holds that $\bm{A}$ is invertible if and only if $\det(\bm{A})\neq 0$.
        \hfill \cite{mfml/book/mml/Deisenroth-Faisal-Ong}
    \end{theorem}

    \item \textbf{Not} every matrix $\bm{A}$ possesses an inverse $\bm{A}^{-1}$.
    \hfill \cite{mfml/book/mml/Deisenroth-Faisal-Ong}

    \item Only square matrices might have inverse. Non-square matrices \textbf{don't} have inverse.

    \item When the matrix inverse exists, it is \textbf{unique}.
    \hfill \cite{mfml/book/mml/Deisenroth-Faisal-Ong}

    \item $
        \bm{A} = \begin{bmatrix}
            a_{11} & a_{12} \\
            a_{21} & a_{22} \\
        \end{bmatrix}
        \in \mbbR^{2\times 2}
    $
    \hspace{1cm} and \hspace{1cm}
    $a_{11}\ a_{22} - a_{12}\ a_{21} \neq 0$ (determinant of $A$)\\[0.4cm]
    $\Rightarrow$
    $
        \bm{A}^{-1} =
        \dfrac{1}{a_{11}\ a_{22} - a_{12}\ a_{21}}
        \begin{bmatrix}
            a_{22} & -a_{12} \\
            -a_{21} & a_{11} \\
        \end{bmatrix}
    $
    \hfill \cite{mfml/book/mml/Deisenroth-Faisal-Ong}

\end{enumerate}



\subsubsection{Matrix Inverse using Gaussian Elimination}

\begin{enumerate}
    \item Given a matrix $\bm{A} \in \mbbR^{n\times n}$, we need to find $\bm{X} = \bm{A}^{-1}$ such that $\bm{A}\bm{X} = \bm{I}_n$
    \hfill \cite{mfml/book/mml/Deisenroth-Faisal-Ong}

    \item We can write this down as a set of simultaneous linear equations $\bm{A}\bm{X} = \bm{I}_n$, where we solve for $\bm{X} = [\bm{x}_1| \cdots |\bm{x}_n]$.
    \hfill \cite{mfml/book/mml/Deisenroth-Faisal-Ong}

    \item augmented matrix notation for a compact representation of this set of systems of linear equations:
    \hfill \cite{mfml/book/mml/Deisenroth-Faisal-Ong}
    \\
    .\hfill
    $
        [\bm{A}|\bm{I}_n]
        \curlyrightarrow \cdots \curlyrightarrow
        [\bm{I}_n|\bm{A}^{-1}]
    $
    \hfill \cite{mfml/book/mml/Deisenroth-Faisal-Ong}


\end{enumerate}




\subsubsection{Moore-Penrose pseudo-inverse ($({A}^\top {A})^{-1} {A}^\top$)}

$
    \begin{aligned}
                         & \bm{A}\bm{X} = \bm{I} \\
        \Leftrightarrow\ & \bm{A}^\top \bm{A}\bm{X} = \bm{A}^\top \\
        \Leftrightarrow\ & \bm{X} = (\bm{A}^\top \bm{A})^{-1} \bm{A}^\top \\
    \end{aligned}
$
\hfill \cite{mfml/book/mml/Deisenroth-Faisal-Ong}


\vspace{0.2cm}

\begin{enumerate}
    \item Disadvantages:
    \begin{enumerate}
        \item requires many computations for the matrix-matrix product and computing the inverse of $\bm{A}^\top \bm{A}$.
        \hfill \cite{mfml/book/mml/Deisenroth-Faisal-Ong}

        \item for reasons of numerical precision it is generally not recommended
        \hfill \cite{mfml/book/mml/Deisenroth-Faisal-Ong}
    \end{enumerate}
\end{enumerate}





\subsubsection{Properties}

\begin{multicols}{2}
\begin{enumerate}
    \item $\bm{A}\bm{A}^{-1} = \bm{I} = \bm{A}^{-1}\bm{A}$
    \hfill \cite{mfml/book/mml/Deisenroth-Faisal-Ong}

    \item $(\bm{A}\bm{B})^{-1} = \bm{B}^{-1}\ \bm{A}^{-1}$
    \hfill \cite{mfml/book/mml/Deisenroth-Faisal-Ong}

    \item $(\bm{A} + \bm{B})^{-1} \neq \bm{A}^{-1} + \bm{B}^{-1}$
    \hfill \cite{mfml/book/mml/Deisenroth-Faisal-Ong}

    \item $(\bm{A}^{-1})^\top = (\bm{A}^\top)^{-1} =: \bm{A}^{-\top}$
    \hfill \cite{mfml/book/mml/Deisenroth-Faisal-Ong}

    \item $(\bm{A}^{-1})^{-1} =  \bm{A}$

\end{enumerate}
\end{multicols}






\section{Rank of a matrix}

\begin{enumerate}
    \item
    \begin{definition}[Rank of a matrix]
        The number of linearly independent columns of a matrix $\bm{A} \in \mbbR^{m\times n}$ equals the number of linearly independent rows and is called the rank of $\bm{A}$ and is denoted by $\text{rk}(\bm{A})$.
        \hfill \cite{mfml/book/mml/Deisenroth-Faisal-Ong}
    \end{definition}

\end{enumerate}


\subsection{Properties of rank}

\begin{enumerate}
    \item $\text{rk}(\bm{A}) = \text{rk}(\bm{A}^\top)$, i.e., the column rank equals the row rank.
    \hfill \cite{mfml/book/mml/Deisenroth-Faisal-Ong}

    \item The columns of $\bm{A} \in \mbbR^{m\times n}$ span a subspace $U \subseteq \mbbR^m$ with $\dim(U) = \text{rk}(\bm{A})$.
    We call this subspace the \textbf{image or range}.
    A basis of $U$ can be found by applying Gaussian elimination to $\bm{A}$ to identify the \textbf{pivot columns}.
    \hfill \cite{mfml/book/mml/Deisenroth-Faisal-Ong}

    \item The rows of $\bm{A} \in  \mbbR^{m\times n}$  span a subspace $W \subseteq \mbbR^n$ with $\dim(W) = \text{rk}(\bm{A})$.
    A basis of $W$ can be found by applying Gaussian elimination to $\bm{A}^\top$.
    \hfill \cite{mfml/book/mml/Deisenroth-Faisal-Ong}

    \item For all $\bm{A} \in  \mbbR^{n\times n}$ it holds that $\bm{A}$ is regular (invertible) if and only if $\text{rk}(\bm{A}) = n$.
    \hfill \cite{mfml/book/mml/Deisenroth-Faisal-Ong}

    \item For all $\bm{A} \in  \mbbR^{m\times n}$  and all $\bm{b} \in  \mbbR^m$ it holds that the linear equation system $Ax = b$ can be solved if and only if $\text{rk}(\bm{A}) = \text{rk}(\bm{A}|\bm{b})$, where $\bm{A}|\bm{b}$ denotes the augmented system.
    \hfill \cite{mfml/book/mml/Deisenroth-Faisal-Ong}

    \item For $\bm{A} \in  \mbbR^{m\times n}$  the subspace of solutions for $\bm{A}\bm{x} = \bm{0}$ possesses dimension $n - \text{rk}(\bm{A})$.
    We call this subspace the \textbf{kernel} or the \textbf{null space}.
    \hfill \cite{mfml/book/mml/Deisenroth-Faisal-Ong}

    \item A matrix $\bm{A} \in  \mbbR^{m\times n}$  has \textbf{full rank} if its rank equals the largest possible rank for a matrix of the same dimensions.
    This means that the rank of a full-rank matrix is the lesser of the number of rows and columns, i.e., $\text{rk}(\bm{A}) = min(m, n)$.
    A matrix is said to be \textbf{rank deficient} if it does not have full rank.
    \hfill \cite{mfml/book/mml/Deisenroth-Faisal-Ong}


\end{enumerate}



\begin{lstlisting}[
    language=Python,
    caption=Rank of a matrix - numPy
]
import numpy as np

# Define your matrix
A = np.array([
    [1, 2, 3],
    [2, 4, 6],
    [1, 0, 1]
])

# Compute the rank
rank = np.linalg.matrix_rank(A)

print("Rank of matrix A:", rank)
\end{lstlisting}









\section{Null Space and Column Space}

\begin{enumerate}
    \item Let us consider $\bm{A} \in \mbbR^{m\times n}$ and a linear mapping $\Phi : \mbbR^n \to \mbbR^m$, $\bm{x} \mapsto \bm{Ax}$.
    \hfill \cite{mfml/book/mml/Deisenroth-Faisal-Ong}

    \item For $\bm{A} = [\bm{a}_1, \cdots , \bm{a}_n]$, where $\bm{a}_i$ are the columns of $\bm{A}$, we obtain
    \hfill \cite{mfml/book/mml/Deisenroth-Faisal-Ong}
    \\
    .\hfill
    $
        Im(\Phi)
        = \dCurlyBrac{\bm{Ax} : \bm{x} \in \mbbR^ n }
        = \dCurlyBrac{\dsum ^n _{i=1} x_i \bm{a}_i : x_1, \cdots , x_n \in \mbbR}
        = span[\bm{a}_1, \cdots , \bm{a}_n] \subseteq \mbbR^m
    $
    \hfill \cite{mfml/book/mml/Deisenroth-Faisal-Ong}
    \\
    i.e., the image is the span of the columns of $\bm{A}$, also called the \textbf{column space}.
    Therefore, the column space (image) is a subspace of $\mbbR^m$, where $m$ is the “height” of the matrix.
    \hfill \cite{mfml/book/mml/Deisenroth-Faisal-Ong}

    \item $\text{rk}(\bm{A}) = \dim(\text{Im}(\Phi))$
    \hfill \cite{mfml/book/mml/Deisenroth-Faisal-Ong}

    \item The kernel/null space $\ker(\Phi)$ is the general solution to the homogeneous system of linear equations $\bm{Ax} = \bm{0}$ and captures all possible linear combinations of the elements in $\mbbR^n$ that produce $\bm{0} \in \mbbR^m$.
    \hfill \cite{mfml/book/mml/Deisenroth-Faisal-Ong}

    \item The kernel is a subspace of $\mbbR^n$ , where $n$ is the “width” of the matrix.
    \hfill \cite{mfml/book/mml/Deisenroth-Faisal-Ong}

    \item The kernel focuses on the relationship among the columns, and we can use it to determine whether/how we can express a column as a linear combination of other columns.
    \hfill \cite{mfml/book/mml/Deisenroth-Faisal-Ong}
\end{enumerate}






\section{Inhomogeneous systems of linear equations and affine subspaces}

\begin{enumerate}
    \item For $\bm{A} \in \mbbR^{m\times n}$ and $\bm{x} \in \mbbR^m$, the solution of the system of linear equations $\bm{A} \bm{\lambda}  = \bm{x}$ is either the empty set or an affine subspace of $\mbbR^n$ of dimension $n - \text{rk}(\bm{A})$.
    In particular, the solution of the linear equation $\lambda _1 \bm{b}_1 + \cdots + \lambda _n \bm{b}_n = \bm{x}$, where $(\lambda _1, \cdots , \lambda _n) \neq (0, . . . , 0)$, is a hyperplane in $\mbbR^n$ .
    \hfill \cite{mfml/book/mml/Deisenroth-Faisal-Ong}

    \item In $\mbbR^n$ , every $k$-dimensional affine subspace is the solution of an inhomogeneous system of linear equations $\bm{Ax} = \bm{b}$, where $\bm{A} \in \mbbR^{m\times n}$ , $\bm{b} \in \mbbR^m$ and $\text{rk}(\bm{A}) = n - k$.
    For homogeneous equation systems $\bm{Ax} = \bm{0}$ the solution was a vector subspace, which we can also think of as a special affine space with support point $\bm{x}_0 = \bm{0}$.
    \hfill \cite{mfml/book/mml/Deisenroth-Faisal-Ong}
\end{enumerate}

























\section{Types of matrices}
\subsection{Square matrix}

$A \in \mathbb{R}^{n\times n}$
we call $(n,\ n)$-matrices square matrices because they possess the same number of rows and columns.


\begin{lstlisting}[
    language=Python,
    caption=Square matrix - numPy
]
import numpy as np

n = 4

print(np.random.randint(-10, 10, size=(n, n)))
\end{lstlisting}



\subsection{Identity Matrix ( $I_n$ )}

In $\mathbb{R}^{n\times n}$, we define the identity matrix:\\
\vspace{0.5cm}
\hfill
$
    \matname{I}_n
    := \begin{bmatrix}
        1 & 0 & \cdots & 0 & \cdots & 0 \\
        0 & 1 & \cdots & 0 & \cdots & 0 \\
        \vdots & \vdots & \ddots & \vdots & \ddots & \vdots \\
        0 & 0 & \cdots & 1 & \cdots & 0 \\
        \vdots & \vdots & \ddots & \vdots & \ddots & \vdots \\
        0 & 0 & \cdots & 0 & \cdots & 1 \\
    \end{bmatrix}
    \in \mathbb{R}^{n\times n}
$
\hfill \cite{mfml/book/mml/Deisenroth-Faisal-Ong}
\\
as the $n \times n$-matrix containing $1$ on the diagonal and $0$ everywhere else.







\begin{lstlisting}[
    language=Python,
    caption=Identity Matrix - numPy
]
import numpy as np

n = 4

print(np.eye(n,n))
\end{lstlisting}









\subsection{Regular/ Invertible/ Non-singular matrix}

A matrix $\matname{A}$ is called regular/ invertible/ non-singular if $\matname{A}^{-1}$ exists.
\hfill \cite{mfml/book/mml/Deisenroth-Faisal-Ong}

\begin{enumerate}
    \item if $\matname{A}$ is invertible, then so is $\matname{A}^\top$
    \hfill \cite{mfml/book/mml/Deisenroth-Faisal-Ong}
\end{enumerate}



\subsection{Singular/ Non-invertible matrix}

A matrix $A$ is called singular/ non-invertible if $A^{-1}$ \textbf{doesn't} exists.
\hfill \cite{mfml/book/mml/Deisenroth-Faisal-Ong}





\subsection{Symmetric Matrix}

\begin{lstlisting}[
    language=Python,
    caption=Identity Matrix - numPy
]
import numpy as np

n = 4

A = np.random.randint(-10, 10, size=(n, n))

# converting A to a symmetric matrix
A = (A + A.T) / 2

print(A)
\end{lstlisting}

\begin{enumerate}
    \item A matrix $\bm{A} \in \mathbb{R}^{n\times n}$ is symmetric if $\bm{A} = \bm{A}^\top$
    \hfill \cite{mfml/book/mml/Deisenroth-Faisal-Ong}
    
    \item only $(n,\ n)$-matrices can be symmetric
    \hfill \cite{mfml/book/mml/Deisenroth-Faisal-Ong}

    \item The sum of symmetric matrices $\bm{A},\ \bm{B} \in \mathbb{R}^{n\times n}$ is \textbf{always symmetric}. 
    \hfill \cite{mfml/book/mml/Deisenroth-Faisal-Ong}

    \item although the product of 2 symmetric matrices is \textbf{always defined}, it is generally \textbf{not symmetric}
    \hfill \cite{mfml/book/mml/Deisenroth-Faisal-Ong}

    \item 
    \begin{theorem}[Spectral Theorem]
        If $\bm{A} \in \mathbb{R}^{n\times n}$ is symmetric, there exists an orthonormal basis of the corresponding vector space $V$ consisting of eigenvectors of $\bm{A}$, and each eigenvalue is real.
        \hfill \cite{mfml/book/mml/Deisenroth-Faisal-Ong}
    \end{theorem}
    \begin{enumerate}
        \item A direct implication of the spectral theorem is that the eigen-decomposition of a symmetric matrix $\bm{A}$ exists (with real eigenvalues), and that we can find an ONB of eigenvectors so that $\bm{A} = \bm{PDP}^\top$ , where $\bm{D}$ is diagonal and the columns of $\bm{P}$ contain the eigenvectors.
        \hfill \cite{mfml/book/mml/Deisenroth-Faisal-Ong}
    \end{enumerate}

    \item 
    \begin{theorem}[Symmetric Matrix: always diagonalizable]
        A symmetric matrix $\bm{S} \in \mathbb{R}^{n\times n}$ can always be diagonalized.
        \hfill \cite{mfml/book/mml/Deisenroth-Faisal-Ong}
    \end{theorem}
    \begin{enumerate}
        \item follows directly from the spectral theorem
        \hfill \cite{mfml/book/mml/Deisenroth-Faisal-Ong}

        \item $\bm{P}$ an orthogonal matrix
        \hfill \cite{mfml/book/mml/Deisenroth-Faisal-Ong}

        \item $\bm{A} = \bm{PDP}^{-1}$
        \hfill \cite{mfml/book/mml/Deisenroth-Faisal-Ong}
    \end{enumerate}
\end{enumerate}








\subsection{Symmetric, Positive Definite Matrix}

\begin{enumerate}
    \item \textbf{Definition}: A symmetric matrix $\bm{A} \in \mathbb{R}^{n\times n}$ that satisfies $\forall \bm{x} \in V \backslash \dCurlyBrac{\bm{0}} : \bm{x}^\top \bm{Ax} > 0$ is called \textbf{symmetric, positive definite}, or just \textbf{positive definite}.
    \hfill \cite{mfml/book/mml/Deisenroth-Faisal-Ong}

    \item Consider an $n$-dimensional vector space $V$ with an inner product $\dAngleBrac{\cdot, \cdot} : V \times V \to \mathbb{R}$ and an ordered basis $B = (\bm{b}_1, \cdots , \bm{b}_n)$ of $V$ . 
    Any vectors $\bm{x}, \bm{y} \in  V$ can be written as linear combinations of the basis vectors so that $\bm{x} = \dsum^n _{i=1} \psi _i \bm{b}_i \in  V$ and $\bm{y} = \dsum ^n _{j=1} \lambda _j \bm{b}_j \in  V$ for suitable $\psi _i , \lambda _j \in  \mathbb{R}$. 
    Due to the bilinearity of the inner product, it holds for all $\bm{x}, \bm{y} \in  V$ that
    \hfill \cite{mfml/book/mml/Deisenroth-Faisal-Ong}
    \\
    .\hfill
    $
        \dAngleBrac{\bm{x}, \bm{y}}
        = \dAngleBrac{
            \dsum^n _{i=1} \psi _i \bm{b}_i,
            \dsum ^n _{j=1} \lambda _j \bm{b}_j
        }
        = \dsum^n _{i=1} \dsum ^n _{j=1} \psi _i \dAngleBrac{\bm{b}_i, \bm{b}_j} \lambda _j
        = \hat{\bm{x}} \bm{A} \hat{\bm{y}}
    $
    \hfill \cite{mfml/book/mml/Deisenroth-Faisal-Ong}
    \\
    where $A_{ij} := \dAngleBrac{\bm{b}_i , \bm{b}_j}$ and $\hat{\bm{x}}, \hat{\bm{y}}$ are the coordinates of $\bm{x}$ and $\bm{y}$ with respect to the basis $B$.
    This implies that the inner product $\dAngleBrac{\cdot, \cdot}$ is uniquely determined through $\bm{A}$. 
    The symmetry of the inner product also means that $\bm{A}$ is symmetric. 
    Furthermore, the positive definiteness of the inner product implies that
    \\
    $\forall \bm{x} \in V \backslash \dCurlyBrac{\bm{0}} : \bm{x}^\top \bm{Ax} > 0$.
    \hfill \cite{mfml/book/mml/Deisenroth-Faisal-Ong}

    \item If only $\geq$ holds, then $\bm{A}$ is called \textbf{symmetric, positive semi-definite}.
    \hfill \cite{mfml/book/mml/Deisenroth-Faisal-Ong}

    \item If $\bm{A} \in \mathbb{R}^{n\times n}$ is symmetric, positive definite, then
    \hfill \cite{mfml/book/mml/Deisenroth-Faisal-Ong}
    \\
    .\hfill
    $⟨\bm{x}, \bm{y}⟩ = \hat{\bm{x}}^\top \bm{A} \hat{\bm{y}}$
    \hfill \cite{mfml/book/mml/Deisenroth-Faisal-Ong}
    \\
    defines an inner product with respect to an ordered basis $B$, where $\hat{\bm{x}}$ and $\hat{\bm{y}}$ are the coordinate representations of $\bm{x}, \bm{y} \in V$ with respect to $B$.
    \hfill \cite{mfml/book/mml/Deisenroth-Faisal-Ong}

    \item \textbf{Theorem}: For a real-valued, finite-dimensional vector space $V$ and an ordered basis $B$ of $V$ , it holds that $\dAngleBrac{\cdot, \cdot} : V \times V \to \mathbb{R}$ is an inner product if and only if there exists a symmetric, positive definite matrix $\bm{A} \in \mathbb{R}^{n\times n}$ with
    \hfill \cite{mfml/book/mml/Deisenroth-Faisal-Ong}
    \\
    .\hfill
    $\dAngleBrac{\bm{x}, \bm{y}} = \hat{\bm{x}} ^\top \bm{A} \hat{\bm{y}}$
    \hfill \cite{mfml/book/mml/Deisenroth-Faisal-Ong}
    \\
    The following properties hold if $\bm{A} \in \mathbb{R}^{n\times n}$ is symmetric and positive definite:
    \hfill \cite{mfml/book/mml/Deisenroth-Faisal-Ong}
    \begin{enumerate}
        \item The null space (kernel) of $\bm{A}$ consists only of $\bm{0}$ because $\bm{x} ^\top \bm{Ax} > 0$ for all $\bm{x} \neq \bm{0}$. 
        This implies that $\bm{Ax} \neq \bm{0}$ if $\bm{x} \neq \bm{0}$.
        \hfill \cite{mfml/book/mml/Deisenroth-Faisal-Ong}

        \item The diagonal elements $a_{ii}$ of $\bm{A}$ are positive because $a_{ii} = \bm{e}^\top _i \bm{A} \bm{e}_i > 0$, where $\bm{e}_i$ is the $i$-th vector of the standard basis in $\mathbb{R}^n$ .
        \hfill \cite{mfml/book/mml/Deisenroth-Faisal-Ong}
    \end{enumerate}
\end{enumerate}















\subsection{Equivalent Matrices}

\begin{enumerate}
    \item \textbf{Definition}: Two matrices $\bm{A}, \tilde{\bm{A}} \in \mathbb{R}^{m\times n}$ are equivalent if there exist regular matrices $S \in R^{n\times n}$ and $T \in R^{m\times m}$, such that $\tilde{\bm{A}} = \bm{T} ^{-1}\bm{AS}$.
    \hfill \cite{mfml/book/mml/Deisenroth-Faisal-Ong}

    \item equivalent matrices are not necessarily similar. 
    \hfill \cite{mfml/book/mml/Deisenroth-Faisal-Ong}
\end{enumerate}















\subsection{Similar Matrices}

\begin{enumerate}
    \item \textbf{Definition}: Two matrices $\bm{A}, \tilde{\bm{A}} \in \mathbb{R}^{n\times n}$ are similar if there exists a regular matrix $\bm{S} \in \mathbb{R}^{n\times n}$ with $\tilde{\bm{A}} = \bm{S} ^{-1}\bm{AS}$.
    \hfill \cite{mfml/book/mml/Deisenroth-Faisal-Ong}

    \item Similar matrices are always equivalent.
    \hfill \cite{mfml/book/mml/Deisenroth-Faisal-Ong}
\end{enumerate}








\subsection{Orthogonal Matrix}


\begin{enumerate}
    \item A square matrix $\bm{A} \in \mbbR^{n\times n}$ is an orthogonal matrix if and only if its columns are orthonormal so that
    \hfill \cite{mfml/book/mml/Deisenroth-Faisal-Ong}
    \\
    .\hfill
    $
        \bm{AA}^\top = \bm{I} = \bm{A}^\top \bm{A}
        \hspace{1cm}
        \Rightarrow
        \hspace{1cm}
        \bm{A}^{-1} = \bm{A}^\top
    $
    \hfill \cite{mfml/book/mml/Deisenroth-Faisal-Ong}
    \\
    i.e., the inverse is obtained by simply transposing the matrix.
    \hfill \cite{mfml/book/mml/Deisenroth-Faisal-Ong}
\end{enumerate}









\subsection{Triangular matrix ($T$)}

\begin{enumerate}
    \item We call a square matrix $\bm{T}$ an \textbf{upper-triangular matrix} if $\bm{T}_{ij} = 0$ for upper-triangular matrix $i > j$, i.e., the matrix is zero below its diagonal. 
    \hfill \cite{mfml/book/mml/Deisenroth-Faisal-Ong}

    \item Analogously, we define a \textbf{lower-triangular matrix} as a matrix with zeros above its diagonal. 
    \hfill \cite{mfml/book/mml/Deisenroth-Faisal-Ong}
\end{enumerate}






\subsection{Diagonal Matrix ($D$)}


\begin{enumerate}
    \item A diagonal matrix is a matrix that has value zero on all off-diagonal elements, i.e., they are of the form:
    \hfill \cite{mfml/book/mml/Deisenroth-Faisal-Ong}
    \\
    .\hfill
    $
        \bm{D} = \begin{bmatrix}
        c_1 & \cdots & 0 \\
        \vdots & \ddots & \vdots \\
        0 & \cdots & c_n
        \end{bmatrix}
    $
    \hfill \cite{mfml/book/mml/Deisenroth-Faisal-Ong}

    \item They allow fast computation of determinants, powers, and inverses.
    \begin{enumerate}
        \item The determinant is the product of its diagonal entries
        \hfill \cite{mfml/book/mml/Deisenroth-Faisal-Ong}
        
        \item a matrix power $\bm{D}^k$ is given by each diagonal element raised to the power $k$
        \hfill \cite{mfml/book/mml/Deisenroth-Faisal-Ong}
        
        \item the inverse $\bm{D}^{-1}$ is the reciprocal of its diagonal elements if all of them are nonzero
        \hfill \cite{mfml/book/mml/Deisenroth-Faisal-Ong}
        
        \item[] 
        $
            \dabs{\bm{D}} 
            = \begin{vmatrix}
                c_1 & \cdots & 0 \\
                \vdots & \ddots & \vdots \\
                0 & \cdots & c_n
            \end{vmatrix}
            = \dprod_{i=1}^n c_i
        $
        \hfill
        $
            \bm{D}^k 
            = \begin{bmatrix}
                c_1^k & \cdots & 0 \\
                \vdots & \ddots & \vdots \\
                0 & \cdots & c_n^k
            \end{bmatrix}
        $
        \hfill
        $
            \bm{D}^{-1} 
            = \begin{bmatrix}
                \dfrac{1}{c_1} & \cdots & 0 \\
                \vdots & \ddots & \vdots \\
                0 & \cdots & \dfrac{1}{c_n}
            \end{bmatrix}
        $
    \end{enumerate}

    \item 
    \begin{definition}[Diagonalizable]
        A matrix $\bm{A} \in \mathbb{R}^{n\times n}$ is diagonalizable if it is similar to a diagonal matrix, i.e., if there exists an invertible matrix $\bm{P} \in \mathbb{R}^{n\times n}$ such that $\bm{D} = \bm{P}^{-1}\bm{AP}$.
        \hfill \cite{mfml/book/mml/Deisenroth-Faisal-Ong}
    \end{definition}

    \item 
    \begin{theorem}[Symmetric Matrix: always diagonalizable]
        A symmetric matrix $\bm{S} \in \mathbb{R}^{n\times n}$ can always be diagonalized.
        \hfill \cite{mfml/book/mml/Deisenroth-Faisal-Ong}
    \end{theorem}
    \begin{enumerate}
        \item follows directly from the spectral theorem
        \hfill \cite{mfml/book/mml/Deisenroth-Faisal-Ong}

        \item $\bm{P}$ an orthogonal matrix
        \hfill \cite{mfml/book/mml/Deisenroth-Faisal-Ong}

        \item $\bm{A} = \bm{PDP}^{-1}$
        \hfill \cite{mfml/book/mml/Deisenroth-Faisal-Ong}
    \end{enumerate}
\end{enumerate}




















\section{Norms of a Matrix}

\begin{enumerate}
    \item \begin{definition}[Spectral Norm of a Matrix]
        For $\bm{x} \in \mbbR^n\backslash \dCurlyBrac{0}$, the spectral norm of a matrix $\bm{A} \in \mbbR^{m\times n}$ is defined as
        $
            \dnorm{\bm{A}}_2
            := \displaystyle\max_{\bm{x}} \dfrac{\dnorm{\bm{Ax}}_2}{\dnorm{\bm{x}}_2}
        $
        \hfill \cite{mfml/book/mml/Deisenroth-Faisal-Ong}
    \end{definition}
\end{enumerate}





\section{Describing/summarizing Matrix}
\subsection{Determinant ($\det(A)$ or $\dabs{A}$)}


\begin{table}[H]
    \hfill
    \begin{minipage}{0.45\linewidth}
        \begin{figure}[H]
            \centering
            \includegraphics[
                width=\linewidth,
                height=3cm,
                keepaspectratio,
            ]{images/maths-for-ml/determinant-2d.png}
            \caption*{
                The area of the parallelogram (shaded region) spanned by the vectors $\bm{b}$ and $\bm{g}$ is $\dabs{\det([\bm{b}, \bm{g}])}$.
                \cite{mfml/book/mml/Deisenroth-Faisal-Ong}
            }
        \end{figure}
    \end{minipage}
    \hfill
    \begin{minipage}{0.45\linewidth}
        \begin{figure}[H]
            \centering
            \includegraphics[
                width=\linewidth,
                height=3cm,
                keepaspectratio,
            ]{images/maths-for-ml/determinant-3d.png}
            \caption*{
                The volume of the parallelepiped (shaded volume) spanned by vectors $\bm{r}$, $\bm{b}$, $\bm{g}$ is $\dabs{\det([\bm{r}, \bm{b}, \bm{g}])}$.
                \cite{mfml/book/mml/Deisenroth-Faisal-Ong}
            }
        \end{figure}
    \end{minipage}
    \hfill
\end{table}


.\hfill
$
    \det(\bm{A}) 
    = \dabs{\bm{A}}
    = \begin{vmatrix}
        a_{11} & a_{12} & \cdots & a_{1n} \\
        a_{21} & a_{22} & \cdots & a_{2n} \\
        \vdots & \vdots & \ddots & \vdots \\
        a_{n1} & a_{n2} & \cdots & a_{nn} \\
    \end{vmatrix}
$
\hfill \cite{mfml/book/mml/Deisenroth-Faisal-Ong}

\begin{enumerate}
    \item A determinant is a mathematical object in the analysis and solution of systems of linear equations.
    \hfill \cite{mfml/book/mml/Deisenroth-Faisal-Ong}

    \item Determinants are \textbf{only} defined for square matrices $\bm{A} \in \mathbb{R}^{n\times n}$ ,i.e., matrices with the same number of rows and columns.
    \hfill \cite{mfml/book/mml/Deisenroth-Faisal-Ong}

    \item The determinant of a square matrix $\bm{A} \in \mathbb{R}^{n\times n}$ is a function that maps $\bm{A}$ onto a real number.
    \hfill \cite{mfml/book/mml/Deisenroth-Faisal-Ong}

    \item For $n=1$, $\det(\bm{A}) = \det(a_{11}) = a_{11}$
    \hfill \cite{mfml/book/mml/Deisenroth-Faisal-Ong}

    \item For $n=2$, 
    $
        \det(\bm{A}) 
        = \begin{vmatrix}
            a_{11} & a_{12} \\
            a_{21} & a_{22}
        \end{vmatrix} 
        = a_{11} a_{22} - a_{12} a_{21}
    $
    \hfill \cite{mfml/book/mml/Deisenroth-Faisal-Ong}

    \item (\textbf{Sarrus’ rule}) For $n=3$, 
    \hfill \cite{mfml/book/mml/Deisenroth-Faisal-Ong}
    \\[0.3cm]
    $
        \det(\bm{A}) 
        = \begin{vmatrix}
            a_{11} & a_{12} & a_{13} \\
            a_{21} & a_{22} & a_{23} \\
            a_{31} & a_{32} & a_{33} \\
        \end{vmatrix} 
        = a_{11} a_{22} a_{33}  + a_{21} a_{32} a_{13}  + a_{31} a_{12} a_{23}  - a_{31} a_{22} a_{13}  - a_{11} a_{32} a_{23}  - a_{21} a_{12} a_{33} 
    $
    \hfill \cite{mfml/book/mml/Deisenroth-Faisal-Ong}

    \item For a triangular matrix $\bm{T} \in \mathbb{R}^{n\times n}$ , the determinant is the product of the diagonal elements, i.e., $\det(\bm{T}) = \dsum ^n _{i=1} \bm{T}_{ii}$
    \hfill \cite{mfml/book/mml/Deisenroth-Faisal-Ong}

    \item the determinant $\det(\bm{A})$ is the signed volume of an n-dimensional parallelepiped formed by columns of the matrix $\bm{A}$.
    \hfill \cite{mfml/book/mml/Deisenroth-Faisal-Ong}

    \item The sign of the determinant indicates the orientation of the spanning vectors.
    \hfill \cite{mfml/book/mml/Deisenroth-Faisal-Ong}

    \item \textbf{Theorem} (Laplace Expansion): Consider a matrix $\bm{A} \in \mathbb{R}^{n\times n}$. Then, for all $j = 1, \cdots , n$:
    \hfill \cite{mfml/book/mml/Deisenroth-Faisal-Ong}
    \begin{enumerate}
        \item Expansion along column $j$: 
        $
            \det(\bm{A}) = \dsum^n _{k=1} (-1)^{k+j}\  a_{kj}\ \det(\bm{A}_{k,j} )
        $
        \hfill \cite{mfml/book/mml/Deisenroth-Faisal-Ong}

        \item Expansion along row $j$:
        $
            \det(\bm{A}) = \dsum^n _{k=1} (-1)^{k+j}\  a_{jk}\ \det(\bm{A}_{j,k} )
        $
        \hfill \cite{mfml/book/mml/Deisenroth-Faisal-Ong}

        \item $\bm{A}_{k,j} \in \mathbb{R}^{(n-1)\times(n-1)}$ is the sub-matrix of $\bm{A}$ that we obtain when deleting row $k$ and column $j$.
        \hfill \cite{mfml/book/mml/Deisenroth-Faisal-Ong}

        \item $\det(\bm{A}_{k,j} )$ is called a \textbf{minor} and $(-1)^{k+j}\ \det(\bm{A}_{k,j} )$ a \textbf{cofactor}.
        \hfill \cite{mfml/book/mml/Deisenroth-Faisal-Ong}
    \end{enumerate}
\end{enumerate}











\subsection{Trace ($tr(A)$)}


\begin{enumerate}
    \item The trace of a square matrix $\bm{A} \in \mathbb{R}^{n\times n}$ is defined as $tr(\bm{A}) := \dsum^n _{i=1} a_{ii}$ i.e. , the trace is the \textbf{sum of the diagonal elements} of $\bm{A}$.
    \hfill \cite{mfml/book/mml/Deisenroth-Faisal-Ong}

    
\end{enumerate}


\subsubsection{Properties of Trace}

\begin{enumerate}
    \item $tr(\bm{A} + \bm{B}) = tr(\bm{A}) + tr(\bm{B})$ for $\bm{A}, \bm{B} \in \mathbb{R}^{n\times n}$
    \hfill \cite{mfml/book/mml/Deisenroth-Faisal-Ong}

    \item $tr(\alpha \bm{A}) = \alpha\ tr(\bm{A}), \alpha \in \mathbb{R}$ for $\bm{A} \in \mathbb{R}^{n\times n}$
    \hfill \cite{mfml/book/mml/Deisenroth-Faisal-Ong}
    
    \item $tr(\bm{I}_n) = n$
    \hfill \cite{mfml/book/mml/Deisenroth-Faisal-Ong}
    
    \item $tr(\bm{AB}) = tr(\bm{BA})$ for $\bm{A} \in \mathbb{R}^{n\times k}, \bm{B} \in \mathbb{R}^{k\times n}$
    \hfill \cite{mfml/book/mml/Deisenroth-Faisal-Ong}

    \item The trace is invariant under cyclic permutations, ie, $tr(\bm{AKL}) = tr(\bm{KLA})$ for matrices $\bm{A} \in  \mathbb{R}^{a \times k}, \bm{K} \in  \mathbb{R}^{k \times l}, \bm{L} \in  \mathbb{R}^{l \times a}$. 
    \hfill \cite{mfml/book/mml/Deisenroth-Faisal-Ong}

    \item $
        tr(\bm{xy}^\top) = tr(\bm{y} ^\top \bm{x}) = \bm{y} ^\top \bm{x} \in \mathbb{R}\ 
        \forall \bm{x}, \bm{y} \in \mathbb{R}^n
    $
    \hfill \cite{mfml/book/mml/Deisenroth-Faisal-Ong}

    \item Given $\Phi : V \to V$, then $tr(\Phi) = tr(\bm{A}_\Phi)$. while matrix representations of linear mappings are basis dependent the trace of a linear mapping $\Phi$ is independent of the basis.
    \hfill \cite{mfml/book/mml/Deisenroth-Faisal-Ong}
\end{enumerate}








\subsection{Characteristic Polynomial ($p_A(\lambda)$)}

\begin{enumerate}
    \item
    \begin{definition}[Characteristic Polynomial]
        For $\lambda \in \mbbR$ and a square matrix $\bm{A} \in \mbbR^{n\times n}$:
        \\
        $
            \begin{aligned}
                p_{\bm{A}}(\lambda ) &:= \det(\bm{A} - \lambda \bm{I}) \\
                &= c_0 + c_1\lambda  + c_2\lambda  ^2 + \cdots + c_{n-1}\lambda ^{( n-1)} + (-1)^n\lambda^  n
            \end{aligned}
        $
        \\
        $c_0, \cdots , c_{n-1} \in \mbbR$, is the characteristic polynomial of $\bm{A}$.
    \end{definition}

    \item $c_0 = \det(\bm{A}), c_{n-1} = (-1)^{n-1}\ \tr(\bm{A})$
\end{enumerate}










\subsection{Eigenvalues ($\lambda$) \& Eigenvectors}

\begin{enumerate}
    \item Eigen is a German word meaning “characteristic”, “self”, or “own”.
    \hfill \cite{mfml/book/mml/Deisenroth-Faisal-Ong}

    \item 
    \begin{definition}[Eigenvalues \& Eigenvectors]
        Let $\bm{A} \in \mbbR^{n\times n}$ be a square matrix. 
        Then $\lambda \in \mbbR$ is an eigenvalue of $\bm{A}$ and $\bm{x} \in \mbbR^n\backslash\dCurlyBrac{0}$ is the corresponding eigenvector of $\bm{A}$ if
        $\bm{Ax} = \lambda \bm{x}$.
        \hfill \cite{mfml/book/mml/Deisenroth-Faisal-Ong}
    \end{definition}

    \item $\bm{Ax} = \lambda \bm{x}$ is called eigenvalue equation.
    \hfill \cite{mfml/book/mml/Deisenroth-Faisal-Ong}

    \item it is often a convention that eigenvalues are sorted in descending order, so that the largest eigenvalue and associated eigenvector are called the \textbf{first eigenvalue} and its associated eigenvector, and the second largest called the \textbf{second eigenvalue} and its associated eigenvector, and so on.
    \hfill \cite{mfml/book/mml/Deisenroth-Faisal-Ong}

    \item The following statements are equivalent:
    \begin{enumerate}
        \item $\lambda$ is an eigenvalue of $\bm{A} \in \mbbR^{n\times n}$
        \hfill \cite{mfml/book/mml/Deisenroth-Faisal-Ong}

        \item There exists an $\bm{x} \in \mbbR^n\backslash\dCurlyBrac{0}$ with $\bm{Ax} = \lambda \bm{x}$, or equivalently, $(\bm{A} - \lambda \bm{I}_n)\bm{x} = 0$ can be solved non-trivially, i.e., $\bm{x} \neq 0$.
        \hfill \cite{mfml/book/mml/Deisenroth-Faisal-Ong}

        \item $rk(\bm{A} - \lambda \bm{I}_n) < n$
        \hfill \cite{mfml/book/mml/Deisenroth-Faisal-Ong}

        \item $\det(\bm{A} - \lambda \bm{I}_n) = 0$
        \hfill \cite{mfml/book/mml/Deisenroth-Faisal-Ong}
    \end{enumerate}

    \item \textbf{Non-uniqueness of eigenvectors}: If $\bm{x}$ is an eigenvector of $\bm{A}$ associated with eigenvalue $\lambda$, then for any $c \in \mbbR\backslash \dCurlyBrac{0}$ it holds that $c\bm{x}$ is an eigenvector of $\bm{A}$ with the same eigenvalue since $\bm{A}(c\bm{x}) = c\bm{Ax} = c\lambda\bm{x} = \lambda(c\bm{x})$. All vectors that are collinear to $\bm{x}$ are also eigenvectors of $\bm{A}$.
    \hfill \cite{mfml/book/mml/Deisenroth-Faisal-Ong}

    \item 
    \begin{theorem}[Eigenvalue: root of the characteristic polynomial]
        $\lambda  \in \mbbR$ is an eigenvalue of $\bm{A} \in \mbbR^{n\times n}$ if and only if $\lambda$  is a root of the characteristic polynomial $p_{\bm{A}}(\lambda )$ of $\bm{A}$.
        \hfill \cite{mfml/book/mml/Deisenroth-Faisal-Ong}
    \end{theorem}

    \item 
    \begin{definition}[Algebraic Multiplicity]
        Let a square matrix $\bm{A}$ have an eigenvalue $\lambda_i$ . 
        The \textbf{algebraic multiplicity} of $\lambda_i$ is the number of times the root appears in the characteristic polynomial.
        \hfill \cite{mfml/book/mml/Deisenroth-Faisal-Ong}
    \end{definition}

    \item
    \begin{definition}[Geometric Multiplicity]
        Let $\lambda_i$ be an eigenvalue of a square matrix $\bm{A}$. 
        Then the \textbf{geometric multiplicity} of $\lambda_i$ is the number of linearly independent eigenvectors associated with $\lambda_i$. 
        In other words, it is the dimensionality of the eigenspace spanned by the eigenvectors associated with $\lambda_i$.
        \hfill \cite{mfml/book/mml/Deisenroth-Faisal-Ong}
    \end{definition}

    \item A specific eigenvalue’s geometric multiplicity must be at least one because every eigenvalue has at least one associated eigenvector. 
    An eigenvalue’s geometric multiplicity cannot exceed its algebraic multiplicity, but it may be lower.

    \item Geometrically, the eigenvector corresponding to a nonzero eigenvalue points in a direction that is stretched by the linear mapping.
    The eigenvalue is the factor by which it is stretched. 
    If the eigenvalue is negative, the direction of the stretching is flipped.
    \hfill \cite{mfml/book/mml/Deisenroth-Faisal-Ong}

    \item \textbf{The Case of the Identity Matrix}: The identity matrix $\bm{I} \in \mbbR^{n\times n}$ has characteristic polynomial $p_{\bm{I}} (\lambda ) = \det(\bm{I} -\lambda \bm{I}) = (1-\bm{\lambda }) ^n = 0$, which has only one eigenvalue $\lambda  = 1$ that occurs $n$ times. 
    Moreover, $\bm{Ix} = \lambda \bm{x} = 1\bm{x}$ holds for all vectors $\bm{x} \in \mbbR^n\backslash \dCurlyBrac{0}$. 
    Because of this, the sole eigenspace $E_1$ of the identity matrix spans $n$ dimensions, and all $n$ standard basis vectors of $\mbbR^n$ are eigenvectors of $\bm{I}$.
    \hfill \cite{mfml/book/mml/Deisenroth-Faisal-Ong}

    \item 
    \begin{theorem}[eigenvectors: linearly independent]
        The eigenvectors $\bm{x}_1, \cdots , \bm{x}_n$ of a matrix $\bm{A} \in \mbbR^{n\times n}$ with $n$ distinct eigenvalues $\lambda _1, \cdots , \lambda _n$ are linearly independent.
        This theorem states that eigenvectors of a matrix with $n$ distinct eigenvalues form a basis of $\mbbR^n$.
        \hfill \cite{mfml/book/mml/Deisenroth-Faisal-Ong}
    \end{theorem}

    \item 
    \begin{theorem}[Spectral Theorem]
        If $\bm{A} \in \mbbR^{n\times n}$ is symmetric, there exists an orthonormal basis of the corresponding vector space $V$ consisting of eigenvectors of $\bm{A}$, and each eigenvalue is real.
        \hfill \cite{mfml/book/mml/Deisenroth-Faisal-Ong}
    \end{theorem}
    \begin{enumerate}
        \item A direct implication of the spectral theorem is that the eigen-decomposition of a symmetric matrix $\bm{A}$ exists (with real eigenvalues), and that we can find an ONB of eigenvectors so that $\bm{A} = \bm{PDP}^\top$ , where $\bm{D}$ is diagonal and the columns of $\bm{P}$ contain the eigenvectors.
        \hfill \cite{mfml/book/mml/Deisenroth-Faisal-Ong}
    \end{enumerate}
\end{enumerate}




\begin{lstlisting}[
    language=Python,
    caption=Eigenvalues \& Eigenvectors of a Matrix - numPy
]
import numpy as np

# Define a square matrix
A = np.random.randint(-10, 10, size=(3,3), dtype=int)
print(A)

# Compute eigenvalues and eigenvectors
# eigenvalues: 1D array of eigenvalues.
# eigenvectors: 2D array where each column is an eigenvector.
eigenvalues, eigenvectors = np.linalg.eig(A)
print(eigenvalues)
print(eigenvectors)
\end{lstlisting}



\subsubsection{Properties of eigenvalues \& eigenvectors}

\begin{enumerate}
    \item A matrix $\bm{A}$ and its transpose $\bm{A}^\top$ possess the same eigenvalues, but not necessarily the same eigenvectors.
    \hfill \cite{mfml/book/mml/Deisenroth-Faisal-Ong}

    \item Similar matrices possess the same eigenvalues.
    Therefore, a linear mapping $\Phi$ has eigenvalues that are independent of the choice of basis of its transformation matrix.
    \hfill \cite{mfml/book/mml/Deisenroth-Faisal-Ong}

    \item Symmetric, positive definite (SPD) matrices \textbf{always} have positive, real eigenvalues.
    \hfill \cite{mfml/book/mml/Deisenroth-Faisal-Ong}
\end{enumerate}




\subsection{Eigenspace ($E_\lambda $) \& Eigenspectrum}

\begin{enumerate}
    \item 
    \begin{definition}[Eigenspace]
        For $\bm{A} \in \mbbR^{n\times n}$ , the set of all eigenvectors of $\bm{A}$ associated with an eigenvalue $\lambda $ spans a subspace of $\mbbR^n$, which is called the \textbf{eigenspace} of $\bm{A}$ with respect to $\lambda $ and is denoted by $E_\lambda $.
        \hfill \cite{mfml/book/mml/Deisenroth-Faisal-Ong}
    \end{definition}

    \item 
    \begin{definition}[Eigenspectrum]
        The set of all eigenvalues of $\bm{A}$ is called the \textbf{eigenspectrum}, or just spectrum, of $\bm{A}$.
        \hfill \cite{mfml/book/mml/Deisenroth-Faisal-Ong}
    \end{definition}

    \item If $\lambda $ is an eigenvalue of $\bm{A} \in \mbbR^{n\times n}$ , then the corresponding eigenspace $E_\lambda $ is the solution space of the homogeneous system of linear equations $(\bm{A} - \lambda \bm{I})\bm{x} = 0$.
    \hfill \cite{mfml/book/mml/Deisenroth-Faisal-Ong}
\end{enumerate}


\subsubsection{Properties of Eigenspace \& Eigenspectrum}

\begin{enumerate}
    \item The eigenspace $E_\lambda$  is the null space of $\bm{A} - \lambda \bm{I}$ since:
    \hfill \cite{mfml/book/mml/Deisenroth-Faisal-Ong}
    \\
    .\hfill
    $
        \begin{aligned}
            \bm{Ax} = \lambda \bm{x}
            & \Longleftrightarrow \bm{Ax} - \lambda \bm{x} = 0 \\
            & \Longleftrightarrow (\bm{A} - \lambda \bm{I})\bm{x} = 0  
            & \Longleftrightarrow \bm{x} \in \ker(\bm{A} - \lambda \bm{I})
        \end{aligned}
    $
    \hfill \cite{mfml/book/mml/Deisenroth-Faisal-Ong}
\end{enumerate}










\section{Matrix Decomposition/Factorization}
\subsection{Cholesky Decomposition ($A=LL^\top$)}

\begin{enumerate}
    \item 
    \begin{theorem}[Cholesky Decomposition]
        A symmetric, positive definite matrix $\bm{A}$ can be factorized into a product $\bm{A} = \bm{LL}^\top$, where L is a lower-triangular matrix with positive diagonal elements:
        \hfill \cite{mfml/book/mml/Deisenroth-Faisal-Ong}
        \\
        .\hfill
        $
            \underset
            {
                \displaystyle
                \bm{A}
            } 
            {\underbrace{
                \displaystyle
                \begin{bmatrix}
                    a_{11} & \cdots & a_{1n} \\
                    \vdots & \ddots & \vdots \\
                    a_{n1} & \cdots & a_{nn}
                \end{bmatrix}
            }}
            =
            \underset
            {
                \displaystyle
                \bm{L}
            } 
            {\underbrace{
                \displaystyle
                \begin{bmatrix}
                    l_{11} & \cdots & 0 \\
                    \vdots & \ddots & \vdots \\
                    l_{n1} & \cdots & l_{nn}
                \end{bmatrix}
            }}
            \underset
            {
                \displaystyle
                \bm{L}^\top
            } 
            {\underbrace{
                \displaystyle
                \begin{bmatrix}
                    l_{11} & \cdots & l_{n1} \\
                    \vdots & \ddots & \vdots \\
                    0 & \cdots & l_{nn}
                \end{bmatrix}
            }}
        $
        \hfill \cite{mfml/book/mml/Deisenroth-Faisal-Ong}
        \\
        $\bm{L}$ is called the \textbf{Cholesky factor} of $\bm{A}$, and $\bm{L}$ is unique.
        \hfill \cite{mfml/book/mml/Deisenroth-Faisal-Ong}
    \end{theorem}

    \item In machine learning, symmetric positive definite matrices require frequent manipulation, e.g., the covariance matrix of a multivariate Gaussian variable is symmetric, positive definite. 
    The Cholesky factorization of this covariance matrix allows us to generate samples from a Gaussian distribution.
    \hfill \cite{mfml/book/mml/Deisenroth-Faisal-Ong}

    \item  It also allows us to perform a linear transformation of random variables, which is heavily exploited when computing gradients in deep stochastic models, such as the variational auto-encoder.
    \hfill \cite{mfml/book/mml/Deisenroth-Faisal-Ong}

    \item The Cholesky decomposition also allows us to compute determinants very efficiently: 
    $
        \det(\bm{A}) 
        = \det(\bm{L}) \det(\bm{L}^\top) 
        = \det(\bm{L})^2
        = \dprod_i l_{ii}^2
    $
    \hfill \cite{mfml/book/mml/Deisenroth-Faisal-Ong}
\end{enumerate}


\begin{lstlisting}[
    language=Python,
    caption=Cholesky Decomposition - numPy
]
import numpy as np

# generate a random matrix
A = np.random.randint(-10, 10, size=(3,3))
# Create a symmetric positive definite matrix
A = A.T @ A

# Perform Cholesky decomposition
L = np.linalg.cholesky(A)

print("Matrix A:")
print(A)
print("\nCholesky factor L:")
print(L)
\end{lstlisting}








\subsection{Eigendecomposition ($A = P DP ^{-1}$)}

\begin{figure}[H]
    \centering
    \includegraphics[
        width=\linewidth,
        height=5cm,
        keepaspectratio,
    ]{images/maths-for-ml/eigendecomposition.png}
    \caption*{
        Intuition behind the eigendecomposition as sequential transformations.
        \\
        \textbf{Top-left to bottom-left}: $\bm{P}^{-1}$ performs a basis change (here drawn in $\mbbR^2$ and depicted as a rotation-like operation) from the standard basis into the eigenbasis.
        \\
        \textbf{Bottom-left to bottom-right}: $\bm{D}$ performs a scaling along the remapped orthogonal eigenvectors, depicted here by a circle being stretched to an ellipse. 
        \\
        \textbf{Bottom-right to top-right}: $\bm{P}$ undoes the basis change (depicted as a reverse rotation) and restores the original coordinate frame.
    }
\end{figure}


\begin{enumerate}
    \item 
    \begin{theorem}[Eigendecomposition]
        A square matrix $\bm{A} \in \mbbR^{n\times n}$ can be factored into $\bm{A} = \bm{PDP}^{-1}$ where $\bm{P} \in \mbbR^{n\times n}$ and $\bm{D}$ is a diagonal matrix whose diagonal entries are the eigenvalues of $\bm{A}$, if and only if the eigenvectors of $\bm{A}$ form a basis of $\mbbR^n$.
        \hfill \cite{mfml/book/mml/Deisenroth-Faisal-Ong}
    \end{theorem}

    \item \textbf{Steps}:
    \begin{enumerate}
        \item Compute eigenvalues and eigenvectors

        \item Check for existence (eigenvectors span $\mbbR^n$ or not)

        \item Construct the matrix P to diagonalize A
        \begin{enumerate}
            \item $\bm{P} = [\bm{p}_1, \cdots, \bm{p}_n]$

            \item $\bm{D} = \bm{P}^{-1}\bm{AP} = \bm{P}^{\top}\bm{AP}$
            \hfill (if $\bm{A}$ is symmetric, $\bm{P}^{-1} = \bm{P}^{\top}$)
        \end{enumerate}
    \end{enumerate}

    \item we can find a matrix power for a matrix $\bm{A} \in \mbbR^{n\times n}$ via the eigenvalue decomposition (if it exists) so that:
    \hfill \cite{mfml/book/mml/Deisenroth-Faisal-Ong}
    \\
    .\hfill
    $
        \bm{A}^k 
        = (\bm{PDP}^{-1})^k 
        = \bm{PDP}^{-1}\bm{PDP}^{-1}\cdots \bm{PDP}^{-1} 
        = \bm{PD}^k \bm{P}^{-1}
    $
    \hfill \cite{mfml/book/mml/Deisenroth-Faisal-Ong}

    \item Assume that the eigendecomposition $\bm{A} = (\bm{PDP}^{-1})$ exists. Then: 
    \hfill \cite{mfml/book/mml/Deisenroth-Faisal-Ong}
    \\
    .\hfill
    $
        \det(\bm{A}) 
        = \det(\bm{PDP}^{ -1}) 
        = \det(\bm{P}) \det(\bm{D}) \det(\bm{P}^{-1})
        = \det(\bm{D})
        = \dprod_{i} d_{ii}
    $
    \hfill \cite{mfml/book/mml/Deisenroth-Faisal-Ong}

    \item eigendecomposition operates within the same vector space, where the same basis change is applied and then undone. 
    \hfill \cite{mfml/book/mml/Deisenroth-Faisal-Ong}
\end{enumerate}



\begin{lstlisting}[
    language=Python,
    caption=Eigendecomposition - numPy
]
import numpy as np

# Define a square matrix
A = np.array([[4, -2],
              [1,  1]])

# Perform eigendecomposition
eigenvalues, eigenvectors = np.linalg.eig(A)

# Construct D and P
D = np.diag(eigenvalues)
P = eigenvectors

# Inverse of P
P_inv = np.linalg.inv(P)

# Reconstruct A
A_reconstructed = P @ D @ P_inv

# Print results
print("Original matrix A:")
print(A)

print("\nEigenvalues (D):")
print(D)

print("\nEigenvectors (P):")
print(P)

print("\nReconstructed A (P D P_inv):")
print(A_reconstructed)

print("\nCheck reconstruction accuracy:", np.allclose(A, A_reconstructed))
\end{lstlisting}







\subsection{Singular Value Decomposition (SVD) ($A = U \Sigma V^\top$)}



\begin{enumerate}
    \item The singular value decomposition (SVD) of a matrix is a central matrix decomposition method in linear algebra.
    \hfill \cite{mfml/book/mml/Deisenroth-Faisal-Ong}

    \item It has been referred to as the “fundamental theorem of linear algebra” because it can be applied to \textbf{all matrices}, not only to square matrices, and it \textbf{always exists}.
    \hfill \cite{mfml/book/mml/Deisenroth-Faisal-Ong}

    \item the SVD of a matrix $\bm{A}$, which represents a linear mapping $\Phi : V \to W$, quantifies the change between the underlying geometry of these two vector spaces.
    \hfill \cite{mfml/book/mml/Deisenroth-Faisal-Ong}

    \item
    \begin{theorem}[SVD Theorem]
        Let $\bm{A} \in \mbbR^{m\times n}$ be a rectangular matrix of rank $r \in [0,\ \min(m, n)]$.
        The SVD of $\bm{A}$ is a decomposition of the form
        $
            \underset{\displaystyle m\times n}{\bm{A}} =
            \underset{\displaystyle m\times m}{\bm{U}}\
            \underset{\displaystyle m\times n}{\bm{\Sigma}}\
            \underset{\displaystyle n\times n}{\bm{V}^\top}
        $
        with an orthogonal matrix $\bm{U} \in \mbbR^{m\times m}$ with column vectors $\bm{u}_i$ , $i = 1, \cdots , m$, and an orthogonal matrix $\bm{V} \in \mbbR^{n\times n}$ with column vectors $\bm{v}_j$ , $j = 1, \cdots , n$.
        Moreover, $\bm{\Sigma}$ is an $m \times n$ matrix with $\Sigma_{ii} = \sigma_i \geq 0$ and $\Sigma_{ij} = 0, i \neq j$.
        \hfill \cite{mfml/book/mml/Deisenroth-Faisal-Ong}
    \end{theorem}

    \item $\bm{U}$
    \begin{enumerate}
        \item $\bm{u}_i$ are called the left-singular vectors
        \hfill \cite{mfml/book/mml/Deisenroth-Faisal-Ong}
    \end{enumerate}

    \item $\bm{\Sigma}$:
    \begin{enumerate}
        \item The diagonal entries $\sigma _i$ , $i = 1, \cdots , r$, of $\bm{\Sigma}$  are called the singular values
        \hfill \cite{mfml/book/mml/Deisenroth-Faisal-Ong}

        \item By convention, the singular values are ordered, i.e., $\sigma _1 \geq \sigma _2 \geq \sigma _r \geq 0$.
        \hfill \cite{mfml/book/mml/Deisenroth-Faisal-Ong}

        \item The singular value matrix $\bm{\Sigma}$ is unique
        \hfill \cite{mfml/book/mml/Deisenroth-Faisal-Ong}

        \item the $\bm{\Sigma}$ is rectangular, particularly, it is of the same size as $\bm{A}$
        \hfill \cite{mfml/book/mml/Deisenroth-Faisal-Ong}

        \item $\bm{\Sigma}$ has a diagonal submatrix that contains the singular values and needs additional zero padding
        \hfill \cite{mfml/book/mml/Deisenroth-Faisal-Ong}
        \\
        if $m > n$, then the matrix $\bm{\Sigma}$ has diagonal structure up to row $n$ and then consists of $\bm{0}^\top$ row vectors from $n + 1$ to $m$ below so that:
        \hfill \cite{mfml/book/mml/Deisenroth-Faisal-Ong}
        \\
        .\hfill
        $
            \begin{bmatrix}
                \sigma_1 & \cdots  & 0 \\
                \vdots & \ddots & \vdots \\
                0 & \cdots & \sigma_n \\
                0 & \cdots & 0\\
                0 & \ddots & 0 \\
                0 & \cdots & 0\\
            \end{bmatrix}
        $
        \hfill \cite{mfml/book/mml/Deisenroth-Faisal-Ong}
        \\
        If $m < n$, the matrix $\bm{\Sigma}$ has a diagonal structure up to column $m$ and columns that consist of $\bm{0}$ from $m + 1$ to $n$:
        \hfill \cite{mfml/book/mml/Deisenroth-Faisal-Ong}
        \\
        .\hfill
        $
            \begin{bmatrix}
                \sigma_1 & \cdots  & 0 & 0 & 0 & 0 \\
                \vdots & \ddots & \vdots & \vdots  & \ddots & \vdots \\
                0 & \cdots & \sigma_m & 0 & 0 & 0 \\
            \end{bmatrix}
        $
        \hfill \cite{mfml/book/mml/Deisenroth-Faisal-Ong}
    \end{enumerate}



    \item $\bm{V}$
    \begin{enumerate}
        \item $\bm{v}_j$ are called the right-singular vectors
        \hfill \cite{mfml/book/mml/Deisenroth-Faisal-Ong}

    \end{enumerate}


    \item \begin{definition}[Standard SVD/ Full SVD ($A = U\Sigma V^\top$)]
        SVD notation where the SVD is described as having two square left- and right-singular vector matrices, but a non-square singular value matrix.
        \hfill \cite{mfml/book/mml/Deisenroth-Faisal-Ong}
        \\
        $
            \underset{\displaystyle m\times n}{\bm{A}} =
            \underset{\displaystyle m\times m}{\bm{U}}\
            \underset{\displaystyle m\times n}{\bm{\Sigma}}\
            \underset{\displaystyle n\times n}{\bm{V}^\top}
        $
        \hfill \cite{mfml/book/mml/Deisenroth-Faisal-Ong}
    \end{definition}

    \item \begin{definition}[Reduced SVD ($A = U_r\Sigma_r V_r^\top$)]
         focus on square singular matrices
         \hfill \cite{mfml/book/mml/Deisenroth-Faisal-Ong}
         \\
         $
            \underset{\displaystyle m\times n}{\bm{A}} =
            \underset{\displaystyle m\times n}{\bm{U}_r}\
            \underset{\displaystyle n\times n}{\bm{\Sigma}_r}\
            \underset{\displaystyle n\times n}{\bm{V}_r^\top}
        $
        \hfill ($m \geq n$)
        \hfill \cite{mfml/book/mml/Deisenroth-Faisal-Ong}
        \\
        This alternative format changes merely how the matrices are constructed but leaves the mathematical structure of the SVD unchanged.
        The convenience of this alternative formulation is that $\bm{\Sigma}$ is diagonal, as in the eigenvalue decomposition.
        \hfill \cite{mfml/book/mml/Deisenroth-Faisal-Ong}
    \end{definition}

    \item
    \begin{definition}[Compact SVD/ Thin SVD/ Truncated SVD ($A = U\Sigma V$)]
        It is possible to define the SVD of a rank-$r$ matrix $\bm{A}$ so that $\bm{U}$ is an $m \times r$ matrix, $\bm{\Sigma}$ a diagonal matrix $r \times r$, and $\bm{V}$ an $r \times n$ matrix.
        This construction is very similar to our definition, and ensures that the diagonal matrix $\bm{\Sigma}$ has only nonzero entries along the diagonal.
        The main convenience of this alternative notation is that $\bm{\Sigma}$ is diagonal, as in the eigenvalue decomposition.
        \\
         $
            \underset{\displaystyle m\times n}{\bm{A}} =
            \underset{\displaystyle m\times r}{\bm{U}}\
            \underset{\displaystyle r\times r}{\bm{\Sigma}}\
            \underset{\displaystyle r\times n}{\bm{V}}
        $
        \hfill \cite{mfml/book/mml/Deisenroth-Faisal-Ong}
    \end{definition}

    \item A restriction that the SVD for $\bm{A}$ only applies to $m \times n$ matrices with $m > n$ is practically unnecessary.
    When $m < n$, the SVD decomposition will yield $\bm{\Sigma}$ with more zero columns than rows and, consequently, the singular values $\sigma _{m+1}, \cdots , \sigma _n$ are $0$.
    \hfill \cite{mfml/book/mml/Deisenroth-Faisal-Ong}
\end{enumerate}





\subsubsection{Geometric Intuitions for the SVD}

\begin{figure}[H]
    \centering
    \includegraphics[
        width=\linewidth,
        height=5cm,
        keepaspectratio,
    ]{images/maths-for-ml/singular-value-decomposition.png}
    \caption*{
        Intuition behind the SVD of a matrix $\bm{A} \in \mbbR^{3\times 2}$ as sequential transformations.
        \hfill \cite{mfml/book/mml/Deisenroth-Faisal-Ong}
        \\
        \textbf{Top-left to bottom-left}: $\bm{V}^\top$ performs a basis change in $\mbbR^2$.
        \hfill \cite{mfml/book/mml/Deisenroth-Faisal-Ong}
        \\
        \textbf{Bottom-left to bottom-right}: $\Sigma$ scales and maps from $\mbbR^2$ to $\mbbR^3$ . The ellipse in the bottom-right lives in $\mbbR^3$. The third dimension is orthogonal to the surface of the elliptical disk.
        \hfill \cite{mfml/book/mml/Deisenroth-Faisal-Ong}
        \\
        \textbf{Bottom-right to top-right}: $\bm{U}$ performs a basis change within $\mbbR^3$.
        \hfill \cite{mfml/book/mml/Deisenroth-Faisal-Ong}
    }
\end{figure}

SVD as sequential linear transformations performed on the bases.
The SVD of a matrix can be interpreted as a decomposition of a corresponding linear mapping $\Phi : \mbbR^n \to \mbbR^m$ into three operations.
Assume we are given a transformation matrix of a linear mapping $\Phi : \mbbR^n \to \mbbR^m$ with respect to the standard bases $B$ and $C$ of $\mbbR^n$ and $\mbbR^m$, respectively.
Moreover, assume a second basis $\tilde{B}$ of $\mbbR^n$ and $\tilde{C}$ of $\mbbR^m$. Then:
\hfill \cite{mfml/book/mml/Deisenroth-Faisal-Ong}

\begin{enumerate}
    \item The matrix $\bm{V}$ performs a basis change in the domain $\mbbR^n$ from $\tilde{B}$ (represented by the red and orange vectors $\bm{v}_1$ and $\bm{v}_2$ in the top-left) to the standard basis $B$.
    $\bm{V}^\top = \bm{V}^{-1}$ performs a basis change from $B$ to $\tilde{B}$.
    The red and orange vectors are now aligned with the canonical basis in the bottom-left.
    \hfill \cite{mfml/book/mml/Deisenroth-Faisal-Ong}

    \item Having changed the coordinate system to $\tilde{B}$, $\bm{\Sigma}$ scales the new coordinates by the singular values $\sigma _i$ (and adds or deletes dimensions), i.e., $\bm{\Sigma}$  is the transformation matrix of $\Phi$ with respect to $\tilde{B}$ and $\tilde{C}$, represented by the red and orange vectors being stretched and lying in the $\bm{e}_1-\bm{e}_2$ plane, which is now embedded in a third dimension in the bottom-right.
    \hfill \cite{mfml/book/mml/Deisenroth-Faisal-Ong}

    \item $\bm{U}$ performs a basis change in the co-domain $\mbbR^m$ from $\tilde{C}$ into the canonical basis of $\mbbR^m$, represented by a rotation of the red and orange vectors out of the $\bm{e}_1-\bm{e}_2$ plane.
    This is shown in the top-right.
    \hfill \cite{mfml/book/mml/Deisenroth-Faisal-Ong}

    \item The SVD expresses a change of basis in both the domain and codomain.
    What makes the SVD special is that these two different bases are simultaneously linked by the singular value matrix $\bm{\Sigma}$.
    \hfill \cite{mfml/book/mml/Deisenroth-Faisal-Ong}
\end{enumerate}



\subsubsection{Construction of the SVD}

\begin{enumerate}
    \item Finding $\bm{V}$:
    \begin{enumerate}
        \item $
            \bm{A}^\top \bm{A}
            = \bm{PDP} ^\top
            = \bm{P} \begin{bmatrix}
                \lambda_1 & \cdots & 0 \\
                \vdots & \ddots & \vdots \\
                0 & \cdots & \lambda_n
            \end{bmatrix} \bm{P} ^\top
            \in \mbbR^{n\times n}
        $
        \hfill \cite{mfml/book/mml/Deisenroth-Faisal-Ong}

        \item $\bm{P}$ is an orthogonal matrix, which is composed of the orthonormal eigenbasis.
        \hfill \cite{mfml/book/mml/Deisenroth-Faisal-Ong}

        \item The $\lambda_i \geq 0$ are the eigenvalues of $\bm{A}^\top \bm{A}$.
        \hfill \cite{mfml/book/mml/Deisenroth-Faisal-Ong}

        \item Let SVD exist. Then:
        \hfill \cite{mfml/book/mml/Deisenroth-Faisal-Ong}
        \\
        $
            \bm{A}^\top \bm{A}
            = (\bm{U\Sigma V}^\top)^\top (\bm{U\Sigma V}^\top)
            = \bm{V} \bm{\Sigma}^\top \bm{U}^\top \bm{U\Sigma V}^\top
            = \bm{V} \bm{\Sigma}^\top \bm{\Sigma V}^\top
            = \bm{V} \begin{bmatrix}
                \sigma_1^2 & \cdots & 0 \\
                \vdots & \ddots & \vdots \\
                0 & \cdots & \sigma_n^2
            \end{bmatrix} \bm{V} ^\top
            \in \mbbR^{n\times n}
        $
        \hfill \cite{mfml/book/mml/Deisenroth-Faisal-Ong}

        \item $\bm{V}^\top = \bm{P}^\top$ and $\sigma^2_i = \lambda_i$
        \hfill \cite{mfml/book/mml/Deisenroth-Faisal-Ong}

        \item Therefore, the eigenvectors of $\bm{A}^\top \bm{A}$ that compose $\bm{P}$ are the right-singular vectors $\bm{V}$ of $\bm{A}$
        \hfill \cite{mfml/book/mml/Deisenroth-Faisal-Ong}

        \item The eigenvalues of $\bm{A}^\top \bm{A}$ are the squared singular values of $\bm{\Sigma}$
        \hfill \cite{mfml/book/mml/Deisenroth-Faisal-Ong}
    \end{enumerate}

    \item Finding $\bm{U}$:
    \begin{enumerate}
        \item $
            \bm{A}^\top \bm{A}
            = \bm{SDS} ^\top
            = \bm{S} \begin{bmatrix}
                \lambda_1 & \cdots & 0 \\
                \vdots & \ddots & \vdots \\
                0 & \cdots & \lambda_m
            \end{bmatrix} \bm{S} ^\top
            \in \mbbR^{m\times m}
        $
        \hfill \cite{mfml/book/mml/Deisenroth-Faisal-Ong}

        \item $\bm{P}$ is an orthogonal matrix, which is composed of the orthonormal eigenbasis.
        \hfill \cite{mfml/book/mml/Deisenroth-Faisal-Ong}

        \item Let SVD exist. Then:
        \hfill \cite{mfml/book/mml/Deisenroth-Faisal-Ong}
        \\
        $
            \bm{A} \bm{A} ^\top
            = (\bm{U\Sigma V}^\top) (\bm{U\Sigma V}^\top) ^\top
            = \bm{U\Sigma V}^\top \bm{V} \bm{\Sigma}^\top \bm{U}^\top
            = \bm{U} \bm{\Sigma} \bm{\Sigma}^\top \bm{U}^\top
            = \bm{U} \begin{bmatrix}
                \sigma_1^2 & \cdots & 0 \\
                \vdots & \ddots & \vdots \\
                0 & \cdots & \sigma_m^2
            \end{bmatrix} \bm{U} ^\top
            \in \mbbR^{m\times m}
        $
        \hfill \cite{mfml/book/mml/Deisenroth-Faisal-Ong}

        \item $\bm{U}^\top = \bm{S}^\top$ and $\sigma^2_i = \lambda_i$
        \hfill \cite{mfml/book/mml/Deisenroth-Faisal-Ong}

        \item The orthonormal eigenvectors of $\bm{AA}^\top$ are the left-singular vectors $\bm{U}$ and form an orthonormal basis in the codomain of the SVD.
        \hfill \cite{mfml/book/mml/Deisenroth-Faisal-Ong}
    \end{enumerate}

    \item Finding $\bm{\Sigma}$:
    \begin{enumerate}
        \item Since $\bm{AA^\top}$ and $\bm{A ^\top A}$ have the same nonzero eigenvalues, the nonzero entries of the $\bm{\Sigma}$ matrices in the SVD for both cases have to be the same.
        \hfill \cite{mfml/book/mml/Deisenroth-Faisal-Ong}

        \item  the images of the $\bm{v}_i$ under $\bm{A}$ have to be orthogonal, too.
        \hfill \cite{mfml/book/mml/Deisenroth-Faisal-Ong}
        \\
        $
            (\bm{Av}_i)^\top (\bm{Av}_j ) =
            \bm{v}^\top _i(\bm{A}^\top \bm{A})\bm{v}_j =
            \bm{v}^\top _i(\lambda _j\bm{v}_j ) =
            \lambda _j\bm{v}^\top _i \bm{v}_j
            = 0
        $
        \hfill \cite{mfml/book/mml/Deisenroth-Faisal-Ong}

        \item For the case $m \geq r$ (where $r = rk(\bm{A})$), it holds that $\dCurlyBrac{\bm{Av}_1, \cdots , \bm{Av}_r}$ is a basis of an r-dimensional subspace of $\mbbR^m$.
        \hfill \cite{mfml/book/mml/Deisenroth-Faisal-Ong}

        \item To complete the SVD construction, we need left-singular vectors that are orthonormal:
        We normalize the images of the right-singular vectors $\bm{Av}_i$ and obtain:
        \hfill \cite{mfml/book/mml/Deisenroth-Faisal-Ong}
        \\
        $
            \bm{u}_i = \dfrac{\bm{Av}_i}{\dnorm{\bm{Av}_i}}
            = \dfrac{\bm{Av}_i}{\sqrt{\lambda_i}}
            = \dfrac{\bm{Av}_i}{\sigma_i}
            \hspace{0.3cm}
            \Rightarrow
            \bm{Av}_i = \sigma_i \bm{u}_i
        $
        \hfill
        $
            (i = 1, \cdots , r)
        $
        \hfill \cite{mfml/book/mml/Deisenroth-Faisal-Ong}

        \item singular value equation: $\bm{Av}_i = \sigma_i \bm{u}_i$
        \hfill \cite{mfml/book/mml/Deisenroth-Faisal-Ong}

        \item \textbf{Author's Note}: As $\bm{V}^T = \bm{V}^{-1}$,
        \\
        $
            \bm{A} = \bm{U\Sigma V}^\top = \bm{U\Sigma V}^{-1}
            \hspace{0.5cm}
            \Rightarrow
            \bm{AV} = \bm{U\Sigma}
            \hspace{0.5cm}
            \Rightarrow
            \bm{U} = \bm{AV\Sigma}^{-1}
            \hspace{0.5cm}
            \Rightarrow
            \bm{U} = \dfrac{\bm{AV}}{\bm{\Sigma}}
        $
        \\
        As $\bm{\Sigma}$ is a diagonal matrix (more or less), $\bm{\Sigma}^{-1} = \dfrac{1}{\bm{\Sigma}}$ (element-wise)

    \end{enumerate}

    \item the eigenvectors of $\bm{A} ^\top \bm{A}$, which we know are the right-singular vectors $\bm{v}_i$ , and their normalized images under $\bm{A}$, the left-singular vectors $\bm{u}_i$ , form two self-consistent ONBs that are connected through the singular value matrix $\bm{\Sigma}$.
    \hfill \cite{mfml/book/mml/Deisenroth-Faisal-Ong}

    \item Case 1: $n<m$:
    \begin{enumerate}
        \item $\bm{Av}_i = \sigma_i \bm{u}_i$ holds for $i\leq n$, i.e., only for the nonzero singular values.
        \hfill \cite{mfml/book/mml/Deisenroth-Faisal-Ong, common/online/chatgpt}

        \item For $i>n$, $\bm{u} _i$  are not defined via $\bm{Av}_ i$ .
        \hfill \cite{mfml/book/mml/Deisenroth-Faisal-Ong, common/online/chatgpt}

        \item But by construction of SVD, we choose $\bm{u}_ i$  to complete an orthonormal basis for $\mbbR^ m$ , so the extra $\bm{u}_ i$ 's (for $i>n$) are orthonormal anyway.
        \hfill \cite{mfml/book/mml/Deisenroth-Faisal-Ong, common/online/chatgpt}
    \end{enumerate}

    \item Case 2: $m<n$
    \begin{enumerate}
        \item $\bm{Av}_i = \sigma_i \bm{u}_i$ only holds for $i\leq m$.
        \hfill \cite{mfml/book/mml/Deisenroth-Faisal-Ong, common/online/chatgpt}

        \item For $i>m$, $\bm{Av}_ i =0$. So, these $\bm{v}_ i$  lie in the null space of $\bm{A}$.
        \hfill \cite{mfml/book/mml/Deisenroth-Faisal-Ong, common/online/chatgpt}

        \item The set of all $\bm{v}_ i$  (even for $i>m$) is still orthonormal.
        \hfill \cite{mfml/book/mml/Deisenroth-Faisal-Ong, common/online/chatgpt}

        \item So the columns $\bm{v}_ i$  with $i>m$ form an orthonormal basis for the null space of $\bm{A}$.
        \hfill \cite{mfml/book/mml/Deisenroth-Faisal-Ong, common/online/chatgpt}
    \end{enumerate}
\end{enumerate}








\subsection{Eigenvalue Decomposition ($A = P DP ^{-1}$) vs. Singular Value Decomposition ($A = U \Sigma V^\top$)}

\begin{enumerate}
    \item The SVD always exists for any matrix $\mbbR^{m\times n}$ .
    The eigendecomposition is only defined for square matrices $\mbbR^{n\times n}$ and only exists if we can find a basis of eigenvectors of $\mbbR^n$.
    \hfill \cite{mfml/book/mml/Deisenroth-Faisal-Ong}

    \item The vectors in the eigendecomposition matrix $\bm{P}$ are not necessarily orthogonal, i.e., the change of basis is not a simple rotation and scaling.
    On the other hand, the vectors in the matrices $\bm{U}$ and $\bm{V}$ in the SVD are orthonormal, so they do represent rotations.
    \hfill \cite{mfml/book/mml/Deisenroth-Faisal-Ong}

    \item Both the eigendecomposition and the SVD are compositions of three linear mappings:
    \hfill \cite{mfml/book/mml/Deisenroth-Faisal-Ong}
    \begin{enumerate}
        \item Change of basis in the domain
        \hfill \cite{mfml/book/mml/Deisenroth-Faisal-Ong}

        \item Independent scaling of each new basis vector and mapping from domain to codomain
        \hfill \cite{mfml/book/mml/Deisenroth-Faisal-Ong}

        \item Change of basis in the codomain
        \hfill \cite{mfml/book/mml/Deisenroth-Faisal-Ong}
    \end{enumerate}
    A key difference between the eigendecomposition and the SVD is that in the SVD, domain and codomain can be vector spaces of different dimensions.
    \hfill \cite{mfml/book/mml/Deisenroth-Faisal-Ong}

    \item In the SVD, the left- and right-singular vector matrices $\bm{U}$ and $\bm{V}$ are generally not inverse of each other (they perform basis changes in different vector spaces).
    In the eigendecomposition, the basis change matrices $\bm{P}$ and $\bm{P}^{ -1}$ are inverses of each other.
    \hfill \cite{mfml/book/mml/Deisenroth-Faisal-Ong}

    \item In the SVD, the entries in the diagonal matrix $\bm{\Sigma}$ are all real and nonnegative, which is not generally true for the diagonal matrix in the eigendecomposition.
    \hfill \cite{mfml/book/mml/Deisenroth-Faisal-Ong}

    \item The SVD and the eigendecomposition are closely related through their projections:
    \hfill \cite{mfml/book/mml/Deisenroth-Faisal-Ong}
    \begin{enumerate}
        \item The left-singular vectors of $\bm{A}$ are eigenvectors of $\bm{AA}^\top$
        \hfill \cite{mfml/book/mml/Deisenroth-Faisal-Ong}

        \item The right-singular vectors of $\bm{A}$ are eigenvectors of $\bm{A}^\top \bm{A}$.
        \hfill \cite{mfml/book/mml/Deisenroth-Faisal-Ong}

        \item The nonzero singular values of $\bm{A}$ are the square roots of the nonzero eigenvalues of both $\bm{AA}^\top$ and $\bm{A} ^\top \bm{A}$.
        \hfill \cite{mfml/book/mml/Deisenroth-Faisal-Ong}
    \end{enumerate}

    \item For symmetric matrices $\bm{A} \in \mbbR^{n\times n}$ , the eigenvalue decomposition and the SVD are one and the same, which follows from the spectral theorem.
    \hfill \cite{mfml/book/mml/Deisenroth-Faisal-Ong}
\end{enumerate}






\section{Matrix Approximation}

\begin{enumerate}
    \item Instead of doing the full SVD factorization, SVD allows us to represent a matrix $\bm{A}$ as a sum of simpler (low-rank) matrices $\bm{A}_i$, which lends itself to a matrix approximation scheme that is cheaper to compute than the full SVD.
    \hfill \cite{mfml/book/mml/Deisenroth-Faisal-Ong}

    \item We construct a rank-1 matrix
    $
        \bm{A}_i
        := \bm{u}_i \bm{v}_i^\top
        \in \mbbR^{m\times n}
    $
    which is formed by the outer product of the $i$-th orthogonal column vector of $\bm{U}$ and $\bm{V}$ .
    \hfill \cite{mfml/book/mml/Deisenroth-Faisal-Ong}

    \item A matrix $\bm{A} \in \mbbR^{m\times n}$ of rank $r$ can be written as a sum of rank-1 matrices $\bm{A}_i$ so that
    $
        \bm{A}
        = \dsum_{i=1}^r \sigma_i \bm{u}_i \bm{v}_i^\top
        = \dsum_{i=1}^r \sigma_i \bm{A}_i
    $
    where the outer-product matrices $\bm{A}_i$ are weighted by the $i$-th singular value $\sigma_i$.
    \hfill \cite{mfml/book/mml/Deisenroth-Faisal-Ong}

    \item If the sum does not run over all matrices $\bm{A}_i$, $i = 1, \cdots , r$, but only up to an intermediate value $k < r$, we obtain a rank-$k$ approximation
    $
        \hat{\bm{A}}(k)
        = \dsum_{i=1}^k \sigma_i \bm{u}_i \bm{v}_i^\top
        = \dsum_{i=1}^k \sigma_i \bm{A}_i
    $
    of $\bm{A}$ with $rk(\hat{\bm{A}}(k)) = k$.
    \hfill \cite{mfml/book/mml/Deisenroth-Faisal-Ong}

    \item \begin{theorem}[Eckart-Young Theorem]
        Consider a matrix $\bm{A} \in \mbbR^{m\times n}$ of rank $r$ and let $\bm{B} \in \mbbR^{m\times n}$ be a matrix of rank $k$.
        For any $k \leq r$ with $\hat{\bm{A}}(k) = \dsum^k_{i=1} \sigma_i \bm{u}_i \bm{v}_i^\top$ it holds that
        ${
            \displaystyle
            \hat{\bm{A}}(k)
            = {\arg\max}_{\text{rk}(\bm{B})=k} \dnorm{\bm{A} - \bm{B}}_2
        }$,
        \hspace{0.5cm}
        ${
            \displaystyle
            \dnorm{\bm{A} - \hat{\bm{A}}(k)}_2 = \sigma_{k+1}.
        }$
        \hfill \cite{mfml/book/mml/Deisenroth-Faisal-Ong}
    \end{theorem}
    \begin{enumerate}
        \item The Eckart-Young theorem states explicitly how much error we introduce by approximating $\bm{A}$ using a rank-$k$ approximation.
        \hfill \cite{mfml/book/mml/Deisenroth-Faisal-Ong}

        \item We can interpret the rank-$k$ approximation obtained with the SVD as a projection of the full-rank matrix $\bm{A}$ onto a lower-dimensional space of rank-at-most-$k$ matrices.
        \hfill \cite{mfml/book/mml/Deisenroth-Faisal-Ong}

        \item The SVD minimizes the error (with respect to the spectral norm) between $\bm{A}$ and any rank-$k$ approximation.
        \hfill \cite{mfml/book/mml/Deisenroth-Faisal-Ong}

        \item The Eckart-Young theorem implies that we can use SVD to reduce a rank-$r$ matrix $\bm{A}$ to a rank-$k$ matrix $\hat{\bm{A}}$ in a principled, optimal (in the spectral norm sense) manner.
        \hfill \cite{mfml/book/mml/Deisenroth-Faisal-Ong}
    \end{enumerate}

    \item the difference between $\bm{A} - \hat{\bm{A}}(k)$ is a matrix containing the sum of the remaining rank-$1$ matrices
    $
        \bm{A} - \hat{\bm{A}}(k)
        = \dsum_{i=k+1}^r \sigma_i \bm{u}_i \bm{v}_i^\top
    $
    \hfill \cite{mfml/book/mml/Deisenroth-Faisal-Ong}

    \item If we assume that there is another matrix $\bm{B}$ with $\text{rk}(\bm{B}) \leq k$, such that
    $
        \dnorm{\bm{A} - \bm{B}}_2 \leq \dnorm{\bm{A} - \hat{\bm{A}}(k)}_2
    $,
    then there exists an at least $(n - k)$-dimensional null space $Z \subseteq \mbbR^n$, such that $\bm{x} \in \mathbb{Z}$ implies that $\bm{Bx} = \bm{0}$.
    \hfill \cite{mfml/book/mml/Deisenroth-Faisal-Ong}
    \\
    Then it follows that
    $
        \dnorm{\bm{Ax}}_2 = \dnorm{(\bm{A} - \bm{B})\bm{x}}_2
    $
    and by using a version of the Cauchy-Schwartz inequality that encompasses norms of matrices, we obtain
    $
        \dnorm{\bm{Ax}}_2
        \leq \dnorm{\bm{A} - \bm{B}}_2 \dnorm{\bm{x}}_2
        < \sigma_{k+2} \dnorm{\bm{x}}_2
    $.
    \hfill \cite{mfml/book/mml/Deisenroth-Faisal-Ong}
    \\
    However, there exists a ($k + 1$)-dimensional subspace where $\dnorm{\bm{Ax}}_2 \geq \sigma_{k+1} \dnorm{\bm{x}}_2$, which is spanned by the right-singular vectors $\bm{v}_j$ , $j \leq k + 1$ of $\bm{A}$.
    \\
    Adding up dimensions of these two spaces yields a number greater than $n$, as there must be a nonzero vector in both spaces. This is a contradiction of the rank-nullity theorem
    \hfill \cite{mfml/book/mml/Deisenroth-Faisal-Ong}
    \vspace{0.5cm}
    \\
    \textbf{Explaining the contradiction part}:
    \hfill \cite{common/online/chatgpt}
    \begin{enumerate}
        \item $\text{rank}(\bm{B})+\text{nullity}(\bm{B})=n$

        \item So if $\text{rank}(\bm{B})< k$, then nullity $>n-k$.

        \item Consider:
        \begin{enumerate}
            \item The subspace where $\bm{Bx}=\bm{0}$: null space of dimension $>n-k$
            \item The subspace spanned by $\bm{v}_{k+1},\cdots,\bm{v}_r$: has dimension $\geq n-k$
        \end{enumerate}

        \item If both have more than $n-k$ dimensions, their intersection must be nonzero (they must share a direction). That means:
        \begin{enumerate}
            \item There exists a nonzero vector xx that’s in both: in the null space of BB and in the span of $\bm{v}_{k+1},\cdots,\bm{v}_r$

            \item For that $\bm{x}$, we must have: $\dnorm{\bm{Ax}}_2 \geq \sigma_{k+1}$.
            But this contradicts the earlier assumption that $\dnorm{\bm{Ax}}_2 < \sigma_{k+1}$.
        \end{enumerate}

        \item Thus, assuming that $\dnorm{\bm{A}-\bm{B}}_2<\sigma_{k+1}$ leads to a contradiction with rank-nullity.
        Therefore: $\dnorm{\bm{A}-\bm{B}}_2 \geq \sigma_{k+1}$ which confirms the theorem.

        \item \textbf{in Simple Terms}:
        \begin{enumerate}
            \item You’re trying to approximate a matrix using fewer components (rank $k$).

            \item The best you can do (smallest error) is using the top $k$ singular values from SVD.

            \item If you try to do better, you run into a contradiction with basic linear algebra — namely, the rank-nullity theorem.

            \item So, SVD gives the best possible low-rank approximation, and the error is exactly the $(k+1)$-th singular value.
        \end{enumerate}
    \end{enumerate}

    \item We can interpret the approximation of $\bm{A}$ by a rank-$k$ matrix as a form of \textbf{lossy compression}.
    \hfill \cite{mfml/book/mml/Deisenroth-Faisal-Ong}
\end{enumerate}














