\chapter{Matrices}

\begin{enumerate}
    \item With $m, n \in \mathbb{N}$ a real-valued $(m, n)$ matrix $\matname{A}$ is an $m\cdot n$-tuple of elements $a_{ij}$, $i = 1, \cdots , m$, $j = 1, \cdots , n$, which is ordered according to a rectangular scheme consisting of $m$ rows and $n$ columns:
    \\[0.2cm]
    $
        \matname{A}
        = 
        \begin{bmatrix}
            a_{11} & a_{12} & \cdots & a_{1n} \\
            a_{21} & a_{22} & \cdots & a_{2n} \\
            \vdots & \vdots & \ddots & \vdots \\
            a_{m1} & a_{m2} & \cdots & a_{mn} \\
        \end{bmatrix}
        \hfill
        (\ a_{ij} \in \mathbb{R} \ )
    $
    \hfill \cite{mfml/book/mml/Deisenroth-Faisal-Ong}

    \item $\mathbb{R}^{m\times n}$ is the set of all real-valued $(m, n)$-matrices. $\matname{A} \in \mathbb{R}^{m\times n}$ can be equivalently represented as $\matname{a} \in \mathbb{R}^{mn}$ by stacking all $n$ columns of the matrix into a long vector.
    \hfill \cite{mfml/book/mml/Deisenroth-Faisal-Ong}

    \vspace{0.5cm}

    \item They can be used to compactly represent systems of linear equations, but they also represent linear functions (linear mappings).
    \hfill \cite{mfml/book/mml/Deisenroth-Faisal-Ong}

    \item matrix represents a linear mapping or a collection of vectors.
    \hfill \cite{mfml/book/mml/Deisenroth-Faisal-Ong}
\end{enumerate}


\section{Matrix Operations}
\subsection{Matrix Addition ( $A+B$ ) \cite{mfml/book/mml/Deisenroth-Faisal-Ong}}

The sum of two matrices $A \in \mathbb{R}^{m\times n}$, $B \in \mathbb{R}^{m\times n}$ is defined as the element-wise sum:\\
. \hfill
\begin{customArrayStretch}{2}
$
    A + B
    := \begin{bmatrix}
        a_{11} + b_{11} &   \cdots  &  a_{1n} + b_{1n} \\
        \vdots          &   \ddots  &   \vdots  \\
        a_{m1} + b_{m1} &   \cdots  &  a_{mn} + b_{mn} \\
    \end{bmatrix}
    \in \mathbb{R}^{m\times n}
$
\end{customArrayStretch}
\hfill \cite{mfml/book/mml/Deisenroth-Faisal-Ong}







\begin{lstlisting}[
    language=Python,
    caption=Matrix Addition - numPy
]
import numpy as np

m,n = 4,3

A = np.random.randint(-10, 10, size=(m,n))
B = np.random.randint(-10, 10, size=(m,n))

C = A + B

print("A:\n", A)
print("B:\n", B)
print("C:\n", C)
\end{lstlisting}









\subsection{Matrix Multiplication ( $AB$ OR $@$ OR $\cdot$ ) \cite{mfml/book/mml/Deisenroth-Faisal-Ong}}

For matrices $\matname{A} \in \mathbb{R}^{m\times n}$, $\matname{B} \in \mathbb{R}^{n\times k}$, the elements $c_{ij}$ of the product matrices. $\matname{C} = \matname{AB} \in \mathbb{R}^{m\times k}$ are computed as:

\vspace{0.5cm}
\hfill
$
    c_{ij} = \dsum_{l=1}^n a_{il}\ b_{lj}
$
\hfill
$
    i = 1,\cdots,m
    \hspace{1cm}
    j = 1,\cdots,k
$
\hfill \cite{mfml/book/mml/Deisenroth-Faisal-Ong}





\begin{lstlisting}[
    language=Python,
    caption=Matrix Multiplication - numPy
]
import numpy as np

m,n,k = 4,3,5

A = np.random.randint(-10, 10, size=(m, n))
B = np.random.randint(-10, 10, size=(n, k))

C1 = A @ B
C2 = np.matmul(A, B)
C3 = np.dot(A, B)

print("A:\n", A)
print("B:\n", B)
print("C1:\n", C1)
print("C2:\n", C2)
print("C3:\n", C3)
print(np.all(C1 == C2), np.all(C1 == C3))
\end{lstlisting}




\vspace{0.5cm}

\begin{enumerate}
    \item To compute element $c_{ij}$ we multiply the elements of the $i$th row of $\matname{A}$ with the $j$th column of $\matname{B}$ and sum them up.
    \hfill \cite{mfml/book/mml/Deisenroth-Faisal-Ong}

    \item Matrices can only be multiplied if their “neighboring” dimensions match.
    \hfill \cite{mfml/book/mml/Deisenroth-Faisal-Ong}
    \\
    \hfill
    $
        \underset{n\times k}{\underbrace{\matname{A}}}\
        \underset{k\times m}{\underbrace{\matname{B}}}
        =
        \underset{n\times m}{\underbrace{\matname{C}}}
    $
    \hfill \cite{mfml/book/mml/Deisenroth-Faisal-Ong}
    \\
    The product $\matname{BA}$ is not defined if $m \neq n$ since the neighboring dimensions do not match.

    \item Matrix multiplication is \textbf{not} defined as an element-wise operation on matrix elements, i.e., 
    \\
    $c_{ij} \neq a_{ij}\ b_{ij}$ (even if the size of $\matname{A}$, $\matname{B}$ was chosen appropriately).
    \hfill \cite{mfml/book/mml/Deisenroth-Faisal-Ong}
\end{enumerate}


\subsubsection{Properties}

\begin{enumerate}
    \item matrix multiplication is \textbf{not} commutative: $\matname{AB} \neq \matname{BA}$
    \hfill \cite{mfml/book/mml/Deisenroth-Faisal-Ong}

    \item \textbf{Associativity}: 
    $
        \forall 
        \matname{A} \in \mathbb{R}^{m\times n},\ 
        \matname{B} \in \mathbb{R}^{n\times p},\ 
        \matname{C} \in \mathbb{R}^{p\times q}
    $:
    
        \begin{enumerate}
            \item $\matname{ABC} = (\matname{AB})\ \matname{C} = \matname{A}\ (\matname{BC})$
            \hfill \cite{mfml/book/mml/Deisenroth-Faisal-Ong}
        \end{enumerate}

    \item \textbf{Distributivity}: 
    $
        \forall 
        \matname{A},\ \matname{B}\in \mathbb{R}^{m\times n},\ 
        \matname{C},\ \matname{D}\in \mathbb{R}^{n\times p}
    $:

        \begin{enumerate}
            \item $(\matname{A} + \matname{B})\ \matname{C} = \matname{AC} + \matname{BC}$
            \hfill \cite{mfml/book/mml/Deisenroth-Faisal-Ong}
            
            \item $\matname{A}\ (\matname{C} + \matname{D}) = \matname{AC} + \matname{AD}$
            \hfill \cite{mfml/book/mml/Deisenroth-Faisal-Ong}
        \end{enumerate}

    \item Multiplication with the identity matrix: 
    $
        \forall 
        \matname{A},\ \matname{B}\in \mathbb{R}^{m\times n}
    $:
        \begin{enumerate}
            \item $\matname{I}_m\matname{A} = \matname{AI}_n = \matname{A}$
            \hfill $m\neq n \Rightarrow \matname{I}_m \neq \matname{I}_n$
            \hfill \cite{mfml/book/mml/Deisenroth-Faisal-Ong}
        \end{enumerate}
\end{enumerate}
















\subsection{Multiplication by a Scalar ( $\lambda A$ )}

Let $\matname{A} \in \mathbb{R}^{m\times n}$ and $\lambda \in \mathbb{R}$. 
Then $\lambda \matname{A} = \matname{K} \in \mathbb{R}^{m\times n}$, $k_{ij} = \lambda a_{ij}$.
\hfill \cite{mfml/book/mml/Deisenroth-Faisal-Ong}


\begin{enumerate}
    \item $\lambda$ scales each element of $\matname{A}$
    \hfill \cite{mfml/book/mml/Deisenroth-Faisal-Ong}    
\end{enumerate}





\subsubsection{Properties}

$\forall\ \lambda,\ \psi \in \mathbb{R}$:
\vspace{0.2cm}
\begin{enumerate}
    \item \textbf{Associativity}:
    $(\lambda \psi )\matname{C} = \lambda (\psi \matname{C})$ \hfill $\matname{C} \in  \mathbb{R}^{m\times n}$
    \hfill \cite{mfml/book/mml/Deisenroth-Faisal-Ong}
    
    \item $
        \lambda (\matname{BC}) 
        = (\lambda \matname{B})\matname{C} 
        = \matname{B}(\lambda \matname{C}) 
        = (\matname{BC})\lambda 
        $ 
    \hfill $\matname{B} \in  \mathbb{R}^{m\times n}, \matname{C} \in  \mathbb{R}^{n\times k}$
    \hfill \cite{mfml/book/mml/Deisenroth-Faisal-Ong}
    \\
    Note that this allows us to move scalar values around.
    \hfill \cite{mfml/book/mml/Deisenroth-Faisal-Ong}

    \item $
        (\lambda \matname{C}) ^\top  
        = \matname{C}^\top \lambda ^\top  
        = \matname{C}^\top \lambda  
        = \lambda \matname{C}^\top 
    $
    \hfill $\lambda  = \lambda ^\top \  \forall \ \lambda  \in  \mathbb{R}$
    \hfill \cite{mfml/book/mml/Deisenroth-Faisal-Ong}

    \item \textbf{Distributivity}:
    \begin{enumerate}
        \item $(\lambda  + \psi )\matname{C} = \lambda \matname{C} + \psi \matname{C}$
        \hfill $\matname{C} \in  \mathbb{R}^{m\times n}$
        \hfill \cite{mfml/book/mml/Deisenroth-Faisal-Ong}

        \item $\lambda (\matname{B} + \matname{C}) = \lambda \matname{B} + \lambda \matname{C}$
        \hfill $\matname{B}, \matname{C} \in  \mathbb{R}^{m\times n}$
        \hfill \cite{mfml/book/mml/Deisenroth-Faisal-Ong}        
    \end{enumerate}
\end{enumerate}






\subsection{Hadamard product ( $A \circ B$ OR $A \ast B$ ) \cite{mfml/book/mml/Deisenroth-Faisal-Ong}}

element-wise multiplication: For $A,B \in \mathbb{R}^{m\times n}$
\\
\ 
\hfill
$
    C = A \circ B \in \mathbb{R}^{m\times n}
$
\hfill
$
    c_{ij} = a_{ij}\ b_{ij}
$
\hfill
\ 













\begin{lstlisting}[
    language=Python,
    caption=Hadamard product - numPy
]
import numpy as np

m,n = 4,3

A = np.random.randint(-10, 10, size=(m,n))
B = np.random.randint(-10, 10, size=(m,n))

C = A * B

print("A:\n", A)
print("B:\n", B)
print("C:\n", C)
\end{lstlisting}











% \subsection{Matrix Inverse ( $A^{-1}$ ) \cite{mfml/book/mml/Deisenroth-Faisal-Ong}}
\subsection{Matrix Inverse \cite{mfml/book/mml/Deisenroth-Faisal-Ong}}

Consider a square matrix $\bm{A} \in \mathbb{R}^{n\times n}$. 
Let matrix $\bm{B} \in \mathbb{R}^{n\times n}$ have the property that $\bm{A}\bm{B} = \bm{I}_n = \bm{B}\bm{A}$. 
$\bm{B}$ is called the inverse of $\bm{A}$ and denoted by $\bm{A}^{-1}$.
\hfill \cite{mfml/book/mml/Deisenroth-Faisal-Ong}





\begin{lstlisting}[
    language=Python,
    caption=Matrix Inverse - numPy
]
import numpy as np

n = 4

A = np.random.randint(-10, 10, size=(n,n)).astype(float)
A_inv = np.linalg.inv(A)

print("A:\n", A)
print("A_inv:\n", A_inv)
print(A @ A_inv)
print(np.allclose((A @ A_inv) , np.eye(n, n)))
\end{lstlisting}






\begin{enumerate}
    \item \textbf{Not} every matrix $\bm{A}$ possesses an inverse $\bm{A}^{-1}$.
    \hfill \cite{mfml/book/mml/Deisenroth-Faisal-Ong}

    \item Only square matrices might have inverse. Non-square matrices \textbf{don't} have inverse.

    \item When the matrix inverse exists, it is \textbf{unique}.
    \hfill \cite{mfml/book/mml/Deisenroth-Faisal-Ong}

    \item $
        \bm{A} = \begin{bmatrix}
            a_{11} & a_{12} \\
            a_{21} & a_{22} \\
        \end{bmatrix} 
        \in \mathbb{R}^{2\times 2}
    $
    \hspace{1cm} and \hspace{1cm}
    $a_{11}\ a_{22} - a_{12}\ a_{21} \neq 0$ (determinant of $A$)\\[0.4cm] 
    $\Rightarrow$
    $
        \bm{A}^{-1} = 
        \dfrac{1}{a_{11}\ a_{22} - a_{12}\ a_{21}}
        \begin{bmatrix}
            a_{22} & -a_{12} \\
            -a_{21} & a_{11} \\
        \end{bmatrix}
    $
    \hfill \cite{mfml/book/mml/Deisenroth-Faisal-Ong}

\end{enumerate}



\subsubsection{Matrix Inverse using Gaussian Elimination}

\begin{enumerate}
    \item Given a matrix $\bm{A} \in \mathbb{R}^{n\times n}$, we need to find $\bm{X} = \bm{A}^{-1}$ such that $\bm{A}\bm{X} = \bm{I}_n$
    \hfill \cite{mfml/book/mml/Deisenroth-Faisal-Ong}
    
    \item We can write this down as a set of simultaneous linear equations $\bm{A}\bm{X} = \bm{I}_n$, where we solve for $\bm{X} = [\bm{x}_1| \cdots |\bm{x}_n]$. 
    \hfill \cite{mfml/book/mml/Deisenroth-Faisal-Ong}

    \item augmented matrix notation for a compact representation of this set of systems of linear equations:
    \hfill \cite{mfml/book/mml/Deisenroth-Faisal-Ong}
    \\
    .\hfill
    $
        [\bm{A}|\bm{I}_n] 
        \curlyrightarrow \cdots \curlyrightarrow 
        [\bm{I}_n|\bm{A}^{-1}]
    $
    \hfill \cite{mfml/book/mml/Deisenroth-Faisal-Ong}

    
\end{enumerate}




\subsubsection{Moore-Penrose pseudo-inverse}

$
    \begin{aligned}
                         & \bm{A}\bm{X} = \bm{I} \\
        \Leftrightarrow\ & \bm{A}^\top \bm{A}\bm{X} = \bm{A}^\top \\
        \Leftrightarrow\ & \bm{X} = (\bm{A}^\top \bm{A})^{-1} \bm{A}^\top \\
    \end{aligned}
$
\hfill \cite{mfml/book/mml/Deisenroth-Faisal-Ong}


\vspace{0.2cm}

\begin{enumerate}
    \item Disadvantages:
    \begin{enumerate}
        \item requires many computations for the matrix-matrix product and computing the inverse of $\bm{A}^\top \bm{A}$. 
        \hfill \cite{mfml/book/mml/Deisenroth-Faisal-Ong}

        \item for reasons of numerical precision it is generally not recommended
        \hfill \cite{mfml/book/mml/Deisenroth-Faisal-Ong}
    \end{enumerate}
\end{enumerate}





\subsubsection{Properties}

\begin{multicols}{2}
\begin{enumerate}
    \item $\bm{A}\bm{A}^{-1} = \bm{I} = \bm{A}^{-1}\bm{A}$
    \hfill \cite{mfml/book/mml/Deisenroth-Faisal-Ong}

    \item $(\bm{A}\bm{B})^{-1} = \bm{B}^{-1}\ \bm{A}^{-1}$
    \hfill \cite{mfml/book/mml/Deisenroth-Faisal-Ong}

    \item $(\bm{A} + \bm{B})^{-1} \neq \bm{A}^{-1} + \bm{B}^{-1}$
    \hfill \cite{mfml/book/mml/Deisenroth-Faisal-Ong}

    \item $(\bm{A}^{-1})^\top = (\bm{A}^\top)^{-1} =: \bm{A}^{-\top}$
    \hfill \cite{mfml/book/mml/Deisenroth-Faisal-Ong}

    \item $(\bm{A}^{-1})^{-1} =  \bm{A}$
    
\end{enumerate}
\end{multicols}


\subsection{Matrix Transpose ( $A^{\top}$ ) \cite{mfml/book/mml/Deisenroth-Faisal-Ong}}

For $\matname{A} \in \mathbb{R}^{m\times n}$ the matrix $\matname{B} \in \mathbb{R}^{n\times m}$ with $b_{ij} = a_{ji}$ is called the transpose of $\matname{A}$. 
We write $\matname{B} = \matname{A}^\top$.
\hfill \cite{mfml/book/mml/Deisenroth-Faisal-Ong}






\begin{lstlisting}[
    language=Python,
    caption=Matrix Inverse - numPy
]
import numpy as np

m,n = 4,3

A = np.random.randint(-10, 10, size=(m,n))

print("A:\n", A)
print("A transpose:\n", A.T)
\end{lstlisting}







\begin{enumerate}
    \item In general, $\matname{A}^\top$ can be obtained by writing the columns of $\matname{A}$ as the rows of $\matname{A}^\top$
    \hfill \cite{mfml/book/mml/Deisenroth-Faisal-Ong}
\end{enumerate}



\subsubsection{Properties}

\begin{multicols}{2}
\begin{enumerate}
    \item $(\matname{A}^\top)^\top = \matname{A}$
    \hfill \cite{mfml/book/mml/Deisenroth-Faisal-Ong}

    \item $(\matname{AB})^\top = \matname{B}^\top\ \matname{A}^\top$
    \hfill \cite{mfml/book/mml/Deisenroth-Faisal-Ong}

    \item $(\matname{A}+\matname{B})^\top = \matname{A}^\top + \matname{B}^\top$
    \hfill \cite{mfml/book/mml/Deisenroth-Faisal-Ong}

    \item $(\matname{A}^{-1})^\top = (\matname{A}^\top)^{-1} =: \matname{A}^{-\top}$
    \hfill \cite{mfml/book/mml/Deisenroth-Faisal-Ong}

\end{enumerate}
\end{multicols}













\section{Rank of a matrix}

\begin{enumerate}
    \item \textbf{Definition}: The number of linearly independent columns of a matrix $\matname{A} \in \mathbb{R}^{m\times n}$ equals the number of linearly independent rows and is called the rank of $\matname{A}$ and is denoted by $\text{rk}(\matname{A})$.
    \hfill \cite{mfml/book/mml/Deisenroth-Faisal-Ong}

\end{enumerate}


\subsection{Properties of rank}

\begin{enumerate}
    \item $\text{rk}(\matname{A}) = \text{rk}(\matname{A}^\top)$, i.e., the column rank equals the row rank.
    \hfill \cite{mfml/book/mml/Deisenroth-Faisal-Ong}

    \item The columns of $\matname{A} \in \mathbb{R}^{m\times n}$ span a subspace $U \subseteq \mathbb{R}^m$ with $\dim(U) = \text{rk}(\matname{A})$. 
    We call this subspace the \textbf{image or range}. 
    A basis of $U$ can be found by applying Gaussian elimination to $\matname{A}$ to identify the \textbf{pivot columns}.
    \hfill \cite{mfml/book/mml/Deisenroth-Faisal-Ong}

    \item The rows of $\matname{A} \in  \mathbb{R}^{m\times n}$  span a subspace $W \subseteq \mathbb{R}^n$ with $\dim(W) = \text{rk}(\matname{A})$. 
    A basis of $W$ can be found by applying Gaussian elimination to $\matname{A}^\top$.
    \hfill \cite{mfml/book/mml/Deisenroth-Faisal-Ong}

    \item For all $\matname{A} \in  \mathbb{R}^{n\times n}$ it holds that $\matname{A}$ is regular (invertible) if and only if $\text{rk}(\matname{A}) = n$.
    \hfill \cite{mfml/book/mml/Deisenroth-Faisal-Ong}

    \item For all $\matname{A} \in  \mathbb{R}^{m\times n}$  and all $\matname{b} \in  \mathbb{R}^m$ it holds that the linear equation system $Ax = b$ can be solved if and only if $\text{rk}(\matname{A}) = \text{rk}(\matname{A}|\matname{b})$, where $\matname{A}|\matname{b}$ denotes the augmented system.
    \hfill \cite{mfml/book/mml/Deisenroth-Faisal-Ong}

    \item For $\matname{A} \in  \mathbb{R}^{m\times n}$  the subspace of solutions for $\matname{A}\matname{x} = \matname{0}$ possesses dimension $n - \text{rk}(\matname{A})$. 
    We call this subspace the \textbf{kernel} or the \textbf{null space}.
    \hfill \cite{mfml/book/mml/Deisenroth-Faisal-Ong}

    \item A matrix $\matname{A} \in  \mathbb{R}^{m\times n}$  has \textbf{full rank} if its rank equals the largest possible rank for a matrix of the same dimensions. 
    This means that the rank of a full-rank matrix is the lesser of the number of rows and columns, i.e., $\text{rk}(\matname{A}) = min(m, n)$. 
    A matrix is said to be \textbf{rank deficient} if it does not have full rank.
    \hfill \cite{mfml/book/mml/Deisenroth-Faisal-Ong}

    
\end{enumerate}




\section{Types of matrices}
\subsection{Square matrix}

\begin{enumerate}
    \item $\bm{A} \in \mathbb{R}^{n\times n}$ we call $(n,\ n)$-matrices square matrices because they possess the same number of rows and columns.
    \hfill \cite{mfml/book/mml/Deisenroth-Faisal-Ong}

    \item
    \begin{definition}[Defective (square) matrix]
        A square matrix $\bm{A} \in \mathbb{R}^{n\times n}$ is \textbf{defective} if it possesses  fewer than $n$ linearly independent eigenvectors.
        \hfill \cite{mfml/book/mml/Deisenroth-Faisal-Ong}
    \end{definition}
    \begin{enumerate}
        \item A non-defective matrix $\bm{A} \in \mathbb{R}^{n\times n}$ does not necessarily require $n$ distinct eigenvalues, but it does require that the eigenvectors form a basis of $\mathbb{R}^n$.
        \hfill \cite{mfml/book/mml/Deisenroth-Faisal-Ong}

        \item sum of the dimensions of the eigenspaces is less than $n$
        \hfill \cite{mfml/book/mml/Deisenroth-Faisal-Ong}

        \item a defective matrix has at least one eigenvalue $\lambda_i$ with an algebraic multiplicity $m > 1$ and a geometric multiplicity of less than $m$.
        \hfill \cite{mfml/book/mml/Deisenroth-Faisal-Ong}

        \item A defective matrix cannot have $n$ distinct eigenvalues, as distinct eigenvalues have linearly independent eigenvectors
        \hfill \cite{mfml/book/mml/Deisenroth-Faisal-Ong}
    \end{enumerate}
\end{enumerate}


\begin{lstlisting}[
    language=Python,
    caption=Square matrix - numPy
]
import numpy as np

n = 4

print(np.random.randint(-10, 10, size=(n, n)))
\end{lstlisting}



\subsection{Identity Matrix ( $I_n$ )}

In $\mbbR^{n\times n}$, we define the identity matrix:\\
\vspace{0.5cm}
\hfill
$
    \bm{I}_n
    := \begin{bmatrix}
        1 & 0 & \cdots & 0 & \cdots & 0 \\
        0 & 1 & \cdots & 0 & \cdots & 0 \\
        \vdots & \vdots & \ddots & \vdots & \ddots & \vdots \\
        0 & 0 & \cdots & 1 & \cdots & 0 \\
        \vdots & \vdots & \ddots & \vdots & \ddots & \vdots \\
        0 & 0 & \cdots & 0 & \cdots & 1 \\
    \end{bmatrix}
    \in \mbbR^{n\times n}
$
\hfill \cite{mfml/book/mml/Deisenroth-Faisal-Ong}
\\
as the $n \times n$-matrix containing $1$ on the diagonal and $0$ everywhere else.







\begin{lstlisting}[
    language=Python,
    caption=Identity Matrix - numPy
]
import numpy as np

n = 4

print(np.eye(n,n))
\end{lstlisting}









\subsection{regular/invertible/non-singular matrix}

A matrix $A$ is called regular/invertible/non-singular if $A^{-1}$ exists.






\subsection{singular/non-invertible matrix}

A matrix $A$ is called singular/non-invertible if $A^{-1}$ \textbf{doesn't} exists.






\subsection{Symmetric Matrix}

A matrix $\matname{A} \in \mathbb{R}^{n\times n}$ is symmetric if $\matname{A} = \matname{A}^\top$
\hfill \cite{mfml/book/mml/Deisenroth-Faisal-Ong}


\begin{lstlisting}[
    language=Python,
    caption=Identity Matrix - numPy
]
import numpy as np

n = 4

A = np.random.randint(-10, 10, size=(n, n))

# converting A to a symmetric matrix
A = (A + A.T) / 2

print(A)
\end{lstlisting}

\begin{enumerate}
    \item only $(n,\ n)$-matrices can be symmetric
    \hfill \cite{mfml/book/mml/Deisenroth-Faisal-Ong}

    \item The sum of symmetric matrices $\matname{A},\ \matname{B} \in \mathbb{R}^{n\times n}$ is \textbf{always symmetric}. 
    \hfill \cite{mfml/book/mml/Deisenroth-Faisal-Ong}

    \item although the product of 2 symmetric matrices is \textbf{always defined}, it is generally \textbf{not symmetric}
    \hfill \cite{mfml/book/mml/Deisenroth-Faisal-Ong}
\end{enumerate}













