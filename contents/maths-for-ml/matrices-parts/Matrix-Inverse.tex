\subsection{Matrix Inverse ( $A^{-1}$ ) \cite{mfml/book/mml/Deisenroth-Faisal-Ong}}

Consider a square matrix $A \in \mathbb{R}^{n\times n}$. Let matrix $B \in \mathbb{R}^{n\times n}$ have the property that $AB = I_n = BA$. $B$ is called the inverse of $A$ and denoted by $A^{-1}$.
\hfill \cite{mfml/book/mml/Deisenroth-Faisal-Ong}





\begin{lstlisting}[
    language=Python,
    caption=Matrix Inverse - numPy
]
import numpy as np

n = 4

A = np.random.randint(-10, 10, size=(n,n)).astype(float)
A_inv = np.linalg.inv(A)

print("A:\n", A)
print("A_inv:\n", A_inv)
print(A @ A_inv)
print(np.allclose((A @ A_inv) , np.eye(n, n)))
\end{lstlisting}






\begin{enumerate}
    \item \textbf{Not} every matrix $A$ possesses an inverse $A^{-1}$.
    \hfill \cite{mfml/book/mml/Deisenroth-Faisal-Ong}

    \item Only square matrices might have inverse. Non-square matrices \textbf{don't} have inverse.

    \item When the matrix inverse exists, it is \textbf{unique}.
    \hfill \cite{mfml/book/mml/Deisenroth-Faisal-Ong}

    \item $
        A = \begin{bmatrix}
            a_{11} & a_{12} \\
            a_{21} & a_{22} \\
        \end{bmatrix} 
        \in \mathbb{R}^{2\times 2}
    $
    \hspace{1cm} and \hspace{1cm}
    $a_{11}\ a_{22} - a_{12}\ a_{21} \neq 0$ (determinant of $A$)\\[0.4cm] 
    $\Rightarrow$
    $
        A^{-1} = 
        \dfrac{1}{a_{11}\ a_{22} - a_{12}\ a_{21}}
        \begin{bmatrix}
            a_{22} & -a_{12} \\
            -a_{21} & a_{11} \\
        \end{bmatrix}
    $
    \hfill \cite{mfml/book/mml/Deisenroth-Faisal-Ong}

\end{enumerate}



\subsubsection{Matrix Inverse using Gaussian Elimination}

\begin{enumerate}
    \item Given a matrix $A \in \mathbb{R}^{n\times n}$, we need to find $X = A^{-1}$ such that $AX = I_n$
    \hfill \cite{mfml/book/mml/Deisenroth-Faisal-Ong}
    
    \item We can write this down as a set of simultaneous linear equations $AX = I_n$, where we solve for $X = [x_1| \cdots |x_n]$. 
    \hfill \cite{mfml/book/mml/Deisenroth-Faisal-Ong}

    \item augmented matrix notation for a compact representation of this set of systems of linear equations:
    \hfill \cite{mfml/book/mml/Deisenroth-Faisal-Ong}
    \\
    .\hfill
    $
        [A|I_n] \curlyrightarrow \cdots \curlyrightarrow [I_n|A^{-1}]
    $
    \hfill \cite{mfml/book/mml/Deisenroth-Faisal-Ong}

    
\end{enumerate}




\subsubsection{Moore-Penrose pseudo-inverse}

$
    \begin{aligned}
                         & AX = I \\
        \Leftrightarrow\ & A^\top AX = A^\top \\
        \Leftrightarrow\ & X = (A^\top A)^{-1} A^\top \\
    \end{aligned}
$
\hfill \cite{mfml/book/mml/Deisenroth-Faisal-Ong}


\vspace{0.2cm}

\begin{enumerate}
    \item Disadvantages:
    \begin{enumerate}
        \item requires many computations for the matrix-matrix product and computing the inverse of $A^\top A$. 
        \hfill \cite{mfml/book/mml/Deisenroth-Faisal-Ong}

        \item for reasons of numerical precision it is generally not recommended
        \hfill \cite{mfml/book/mml/Deisenroth-Faisal-Ong}
    \end{enumerate}
\end{enumerate}





\subsubsection{Properties}

\begin{multicols}{2}
\begin{enumerate}
    \item $AA^{-1} = I = A^{-1}A$
    \hfill \cite{mfml/book/mml/Deisenroth-Faisal-Ong}

    \item $(AB)^{-1} = B^{-1}\ A^{-1}$
    \hfill \cite{mfml/book/mml/Deisenroth-Faisal-Ong}

    \item $(A + B)^{-1} \neq A^{-1} + B^{-1}$
    \hfill \cite{mfml/book/mml/Deisenroth-Faisal-Ong}

    \item $(A^{-1})^\top = (A^\top)^{-1} =: A^{-\top}$
    \hfill \cite{mfml/book/mml/Deisenroth-Faisal-Ong}
    
\end{enumerate}
\end{multicols}

