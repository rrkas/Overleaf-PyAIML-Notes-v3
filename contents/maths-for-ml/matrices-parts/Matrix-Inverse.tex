% \subsection{Matrix Inverse ( $A^{-1}$ ) \cite{mfml/book/mml/Deisenroth-Faisal-Ong}}
\subsection{Matrix Inverse \cite{mfml/book/mml/Deisenroth-Faisal-Ong}}

Consider a square matrix $\matname{A} \in \mathbb{R}^{n\times n}$. 
Let matrix $\matname{B} \in \mathbb{R}^{n\times n}$ have the property that $\matname{A}\matname{B} = \matname{I}_n = \matname{B}\matname{A}$. 
$\matname{B}$ is called the inverse of $\matname{A}$ and denoted by $\matname{A}^{-1}$.
\hfill \cite{mfml/book/mml/Deisenroth-Faisal-Ong}





\begin{lstlisting}[
    language=Python,
    caption=Matrix Inverse - numPy
]
import numpy as np

n = 4

A = np.random.randint(-10, 10, size=(n,n)).astype(float)
A_inv = np.linalg.inv(A)

print("A:\n", A)
print("A_inv:\n", A_inv)
print(A @ A_inv)
print(np.allclose((A @ A_inv) , np.eye(n, n)))
\end{lstlisting}






\begin{enumerate}
    \item \textbf{Not} every matrix $\matname{A}$ possesses an inverse $\matname{A}^{-1}$.
    \hfill \cite{mfml/book/mml/Deisenroth-Faisal-Ong}

    \item Only square matrices might have inverse. Non-square matrices \textbf{don't} have inverse.

    \item When the matrix inverse exists, it is \textbf{unique}.
    \hfill \cite{mfml/book/mml/Deisenroth-Faisal-Ong}

    \item $
        \matname{A} = \begin{bmatrix}
            a_{11} & a_{12} \\
            a_{21} & a_{22} \\
        \end{bmatrix} 
        \in \mathbb{R}^{2\times 2}
    $
    \hspace{1cm} and \hspace{1cm}
    $a_{11}\ a_{22} - a_{12}\ a_{21} \neq 0$ (determinant of $A$)\\[0.4cm] 
    $\Rightarrow$
    $
        \matname{A}^{-1} = 
        \dfrac{1}{a_{11}\ a_{22} - a_{12}\ a_{21}}
        \begin{bmatrix}
            a_{22} & -a_{12} \\
            -a_{21} & a_{11} \\
        \end{bmatrix}
    $
    \hfill \cite{mfml/book/mml/Deisenroth-Faisal-Ong}

\end{enumerate}



\subsubsection{Matrix Inverse using Gaussian Elimination}

\begin{enumerate}
    \item Given a matrix $\matname{A} \in \mathbb{R}^{n\times n}$, we need to find $\matname{X} = \matname{A}^{-1}$ such that $\matname{A}\matname{X} = \matname{I}_n$
    \hfill \cite{mfml/book/mml/Deisenroth-Faisal-Ong}
    
    \item We can write this down as a set of simultaneous linear equations $\matname{A}\matname{X} = \matname{I}_n$, where we solve for $\matname{X} = [\matname{x}_1| \cdots |\matname{x}_n]$. 
    \hfill \cite{mfml/book/mml/Deisenroth-Faisal-Ong}

    \item augmented matrix notation for a compact representation of this set of systems of linear equations:
    \hfill \cite{mfml/book/mml/Deisenroth-Faisal-Ong}
    \\
    .\hfill
    $
        [\matname{A}|\matname{I}_n] 
        \curlyrightarrow \cdots \curlyrightarrow 
        [\matname{I}_n|\matname{A}^{-1}]
    $
    \hfill \cite{mfml/book/mml/Deisenroth-Faisal-Ong}

    
\end{enumerate}




\subsubsection{Moore-Penrose pseudo-inverse}

$
    \begin{aligned}
                         & \matname{A}\matname{X} = \matname{I} \\
        \Leftrightarrow\ & \matname{A}^\top \matname{A}\matname{X} = \matname{A}^\top \\
        \Leftrightarrow\ & \matname{X} = (\matname{A}^\top \matname{A})^{-1} \matname{A}^\top \\
    \end{aligned}
$
\hfill \cite{mfml/book/mml/Deisenroth-Faisal-Ong}


\vspace{0.2cm}

\begin{enumerate}
    \item Disadvantages:
    \begin{enumerate}
        \item requires many computations for the matrix-matrix product and computing the inverse of $\matname{A}^\top \matname{A}$. 
        \hfill \cite{mfml/book/mml/Deisenroth-Faisal-Ong}

        \item for reasons of numerical precision it is generally not recommended
        \hfill \cite{mfml/book/mml/Deisenroth-Faisal-Ong}
    \end{enumerate}
\end{enumerate}





\subsubsection{Properties}

\begin{multicols}{2}
\begin{enumerate}
    \item $\matname{A}\matname{A}^{-1} = \matname{I} = \matname{A}^{-1}\matname{A}$
    \hfill \cite{mfml/book/mml/Deisenroth-Faisal-Ong}

    \item $(\matname{A}\matname{B})^{-1} = \matname{B}^{-1}\ \matname{A}^{-1}$
    \hfill \cite{mfml/book/mml/Deisenroth-Faisal-Ong}

    \item $(\matname{A} + \matname{B})^{-1} \neq \matname{A}^{-1} + \matname{B}^{-1}$
    \hfill \cite{mfml/book/mml/Deisenroth-Faisal-Ong}

    \item $(\matname{A}^{-1})^\top = (\matname{A}^\top)^{-1} =: \matname{A}^{-\top}$
    \hfill \cite{mfml/book/mml/Deisenroth-Faisal-Ong}

    \item $(\matname{A}^{-1})^{-1} =  \matname{A}$
    
\end{enumerate}
\end{multicols}

