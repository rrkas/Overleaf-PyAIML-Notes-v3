\subsection{Matrix Transpose ( $A^{\top}$ ) \cite{mfml/book/mml/Deisenroth-Faisal-Ong}}

For $A \in \mathbb{R}^{m\times n}$ the matrix $B \in \mathbb{R}^{n\times m}$ with $b_{ij} = a_{ji}$ is called the transpose of $A$. We write $B = A^\top$.
\hfill \cite{mfml/book/mml/Deisenroth-Faisal-Ong}






\begin{lstlisting}[
    language=Python,
    caption=Matrix Inverse - numPy
]
import numpy as np

m,n = 4,3

A = np.random.randint(-10, 10, size=(m,n))

print("A:\n", A)
print("A transpose:\n", A.T)
\end{lstlisting}







\begin{enumerate}
    \item In general, $A^\top$ can be obtained by writing the columns of $A$ as the rows of $A^\top$
    \hfill \cite{mfml/book/mml/Deisenroth-Faisal-Ong}

\end{enumerate}



\subsubsection{Properties}

\begin{multicols}{2}
\begin{enumerate}[itemsep=0.2cm]
    \item $(A^\top)^\top = A$
    \hfill \cite{mfml/book/mml/Deisenroth-Faisal-Ong}

    \item $(AB)^\top = B^\top\ A^\top$
    \hfill \cite{mfml/book/mml/Deisenroth-Faisal-Ong}

    \item $(A+B)^\top = A^\top + B^\top$
    \hfill \cite{mfml/book/mml/Deisenroth-Faisal-Ong}

    \item $(A^{-1})^\top = (A^\top)^{-1} =: A^{-\top}$
    \hfill \cite{mfml/book/mml/Deisenroth-Faisal-Ong}

\end{enumerate}
\end{multicols}







