\chapter{Group Theory}




\section{Groups}

\begin{enumerate}
    \item \textbf{Definition}: Consider a set $\mathcal{G}$ and an (inner) operation $\otimes : \mathcal{G} \times \mathcal{G} \to \mathcal{G}$ group defined on $G$. Then $G := (\mathcal{G}, \otimes)$ is called a group if the following hold:
    \hfill \cite{mfml/book/mml/Deisenroth-Faisal-Ong}
    \begin{enumerate}
        \item Closure of $\mathcal{G}$ under $\otimes$: $\forall x, y \in \mathcal{G} : x \otimes y \in \mathcal{G}$
        \hfill \cite{mfml/book/mml/Deisenroth-Faisal-Ong}

        \item Associativity: $\forall x, y, z \in  \mathcal{G} : (x \otimes  y) \otimes  z = x \otimes  (y \otimes  z)$
        \hfill \cite{mfml/book/mml/Deisenroth-Faisal-Ong}

        \item Neutral element: $\exists e \in  \mathcal{G} \forall x \in  \mathcal{G} : x \otimes  e = x and e \otimes  x = x$
        \hfill \cite{mfml/book/mml/Deisenroth-Faisal-Ong}

        \item Inverse element: $\forall x \in  \mathcal{G} \exists y \in  \mathcal{G}$ : $x \otimes  y = e$ and $y \otimes  x = e$, where $e$ is the neutral element. We often write $x^{-1}$ to denote the inverse element of $x$.
        \hfill \cite{mfml/book/mml/Deisenroth-Faisal-Ong}
    \end{enumerate}

    \item The inverse element is defined with respect to the operation $\otimes$ and does not necessarily mean $\dfrac{1}{x}$.
    \hfill \cite{mfml/book/mml/Deisenroth-Faisal-Ong}

    \item Group allows only inner operation $\otimes$, means the operands \textbf{must be} elements from $\mathcal{G}$.

    \item Examples:
    \begin{enumerate}
        \item $(\mathbb{N}_0, +)$ is \textbf{not} a group: Although $(\mathbb{N}_0, +)$ possesses a neutral element ($0$), the inverse elements are missing.
        \hfill $\mathbb{N}_0 := \mathbb{N} \cup \dCurlyBrac{0}$
        \hfill \cite{mfml/book/mml/Deisenroth-Faisal-Ong}

        \item $(\mathbb{Z}, \cdot)$ is not a group: Although $(\mathbb{Z}, \cdot)$ contains a neutral element ($1$), the inverse elements for any $z \in \mathbb{Z}, z \neq \pm1$, are missing.
        \hfill \cite{mfml/book/mml/Deisenroth-Faisal-Ong}

        \item $(\mbbR, \cdot)$ is not a group since $0$ does not possess an inverse element.
        \hfill \cite{mfml/book/mml/Deisenroth-Faisal-Ong}

        
    \end{enumerate}
\end{enumerate}






\section{Abelian Group}

\begin{enumerate}
    \item \textbf{Definition}: A group $G = (\mathcal{G}, \otimes)$ is called Abelian group if $\forall x, y \in \mathcal{G} : x \otimes y = y \otimes x$ (commutative)
    \hfill \cite{mfml/book/mml/Deisenroth-Faisal-Ong}

    \item Examples:
    \begin{enumerate}
        \item $(\mathbb{Z}, +)$ is an Abelian group
        \hfill \cite{mfml/book/mml/Deisenroth-Faisal-Ong}

        \item $( \mbbR \backslash \dCurlyBrac{0}, \cdot)$ is Abelian
        \hfill \cite{mfml/book/mml/Deisenroth-Faisal-Ong}

        \item $(\mbbR^n, +),(\mathbb{Z}^n, +), n \in \mathbb{N}$ are Abelian if + is defined component-wise:
        \\
        $
            (x_1, \cdots , x_n) + (y_1, \cdots , y_n) = (x_1 + y_1, \cdots , x_n + y_n)
        $
        \\
        Then, $(x_1, \cdots , x_n)^{-1} := (-x_1, \cdots , -x_n)$ is the inverse element and $e = (0, \cdots , 0)$ is the neutral element.

        \item $(\mbbR^{m\times n} , +)$, the set of $m \times n$-matrices is Abelian  (with component-wise addition)

        
    \end{enumerate}
\end{enumerate}





\section{General Linear Group}

\begin{enumerate}
    \item \textbf{Definition}: The set of regular (invertible) matrices $A \in \mbbR^{n\times n}$ is a group with respect to matrix multiplication and is called general linear group $GL(n, \mbbR)$. 
    
    \item However, since matrix multiplication is not commutative, the group is not Abelian.

    
\end{enumerate}

