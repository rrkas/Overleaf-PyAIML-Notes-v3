\subsection{Trace ($\tr(A)$)}


\begin{lstlisting}[
    language=Python,
    caption=Trace of a matrix - numPy
]
import numpy as np

# Define a square matrix
A = np.array([
    [1, 2, 3],
    [4, 5, 6],
    [7, 8, 9]
])

# Compute the trace (sum of diagonal elements)
trace_A = np.trace(A)

print("Matrix A:\n", A)
print("Trace of A:", trace_A)
\end{lstlisting}

\begin{enumerate}
    \item 
    \begin{definition}[Trace]
        The trace of a square matrix $\bm{A} \in \mbbR^{n\times n}$ is defined as $\tr(\bm{A}) := \dsum^n _{i=1} a_{ii}$ i.e. , the trace is the \textbf{sum of the diagonal elements} of $\bm{A}$.
        \hfill \cite{mfml/book/mml/Deisenroth-Faisal-Ong}
    \end{definition}

    \item 
    \begin{theorem}[Trace: sum of eigenvalues]
        The trace of a matrix $\bm{A} \in \mbbR^{n\times n}$ is the sum of its eigenvalues, i.e.,
        \hfill \cite{mfml/book/mml/Deisenroth-Faisal-Ong}
        \\
        .\hfill
        $
            \det(\bm{A}) = \dsum_{i=1}^{n} \lambda_i
        $
        \hfill \cite{mfml/book/mml/Deisenroth-Faisal-Ong}
        \\
        where $\lambda_i \in \mathbb{C}$ are (possibly repeated) eigenvalues of $\bm{A}$.
        \hfill \cite{mfml/book/mml/Deisenroth-Faisal-Ong}
    \end{theorem}
\end{enumerate}


\subsubsection{Properties of Trace}

\begin{enumerate}
    \item $\tr(\bm{A} + \bm{B}) = \tr(\bm{A}) + \tr(\bm{B})$ for $\bm{A}, \bm{B} \in \mbbR^{n\times n}$
    \hfill \cite{mfml/book/mml/Deisenroth-Faisal-Ong}

    \item $\tr(\alpha \bm{A}) = \alpha\ \tr(\bm{A}), \alpha \in \mbbR$ for $\bm{A} \in \mbbR^{n\times n}$
    \hfill \cite{mfml/book/mml/Deisenroth-Faisal-Ong}
    
    \item $\tr(\bm{I}_n) = n$
    \hfill \cite{mfml/book/mml/Deisenroth-Faisal-Ong}
    
    \item $\tr(\bm{AB}) = \tr(\bm{BA})$ for $\bm{A} \in \mbbR^{n\times k}, \bm{B} \in \mbbR^{k\times n}$
    \hfill \cite{mfml/book/mml/Deisenroth-Faisal-Ong}

    \item The trace is invariant under cyclic permutations, ie, $\tr(\bm{AKL}) = \tr(\bm{KLA})$ for matrices $\bm{A} \in  \mbbR^{a \times k}, \bm{K} \in  \mbbR^{k \times l}, \bm{L} \in  \mbbR^{l \times a}$. 
    \hfill \cite{mfml/book/mml/Deisenroth-Faisal-Ong}

    \item $
        \tr(\bm{xy}^\top) = \tr(\bm{y} ^\top \bm{x}) = \bm{y} ^\top \bm{x} \in \mbbR\ 
        \forall \bm{x}, \bm{y} \in \mbbR^n
    $
    \hfill \cite{mfml/book/mml/Deisenroth-Faisal-Ong}

    \item Given $\Phi : V \to V$, then $\tr(\Phi) = \tr(\bm{A}_\Phi)$. while matrix representations of linear mappings are basis dependent the trace of a linear mapping $\Phi$ is independent of the basis.
    \hfill \cite{mfml/book/mml/Deisenroth-Faisal-Ong}
\end{enumerate}







