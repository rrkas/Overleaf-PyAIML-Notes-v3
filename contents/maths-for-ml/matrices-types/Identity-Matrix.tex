\subsection{Identity Matrix ( $I_n$ )}

In $\mathbb{R}^{n\times n}$, we define the identity matrix:\\
\vspace{0.5cm}
\hfill
$
    I_n
    := \begin{bmatrix}
        1 & 0 & \cdots & 0 & \cdots & 0 \\
        0 & 1 & \cdots & 0 & \cdots & 0 \\
        \vdots & \vdots & \ddots & \vdots & \ddots & \vdots \\
        0 & 0 & \cdots & 1 & \cdots & 0 \\
        \vdots & \vdots & \ddots & \vdots & \ddots & \vdots \\
        0 & 0 & \cdots & 0 & \cdots & 1 \\
    \end{bmatrix}
    \in \mathbb{R}^{n\times n}
$
\hfill \cite{mfml/book/mml/Deisenroth-Faisal-Ong}
\\
as the $n \times n$-matrix containing $1$ on the diagonal and $0$ everywhere else.







\begin{lstlisting}[
    language=Python,
    caption=Identity Matrix - numPy
]
import numpy as np

n = 4

print(np.eye(n,n))
\end{lstlisting}








\subsubsection{Properties}

\begin{enumerate}[itemsep=0.2cm]
    \item \textbf{Associativity}: 
    $
        \forall 
        A\in \mathbb{R}^{m\times n},\ 
        B\in \mathbb{R}^{n\times p},\ 
        C \in \mathbb{R}^{p\times q}
    $:
    
        \begin{enumerate}
            \item $ABC = (AB)\ C = A\ (BC)$
            \hfill \cite{mfml/book/mml/Deisenroth-Faisal-Ong}
        \end{enumerate}

    \item \textbf{Distributivity}: 
    $
        \forall 
        A,\ B\in \mathbb{R}^{m\times n},\ 
        C,\ D\in \mathbb{R}^{n\times p}
    $:

        \begin{enumerate}
            \item $(A + B)\ C = AC + BC$
            \hfill \cite{mfml/book/mml/Deisenroth-Faisal-Ong}
            
            \item $A\ (C + D) = AC + AD$
            \hfill \cite{mfml/book/mml/Deisenroth-Faisal-Ong}
        \end{enumerate}

    \item Multiplication with the identity matrix: 
    $
        \forall 
        A,\ B\in \mathbb{R}^{m\times n}
    $:
        \begin{enumerate}
            \item $I_m\ A = A\ I_n = A$
            \hfill $m\neq n \Rightarrow I_m \neq I_n$
            \hfill \cite{mfml/book/mml/Deisenroth-Faisal-Ong}

            
        \end{enumerate}
\end{enumerate}










