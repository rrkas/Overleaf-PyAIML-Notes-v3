\subsection{Symmetric Matrix}

\begin{lstlisting}[
    language=Python,
    caption=Identity Matrix - numPy
]
import numpy as np

n = 4

A = np.random.randint(-10, 10, size=(n, n))

# converting A to a symmetric matrix
A = (A + A.T) / 2

print(A)
\end{lstlisting}

\begin{enumerate}
    \item A matrix $\bm{A} \in \mathbb{R}^{n\times n}$ is symmetric if $\bm{A} = \bm{A}^\top$
    \hfill \cite{mfml/book/mml/Deisenroth-Faisal-Ong}
    
    \item only $(n,\ n)$-matrices can be symmetric
    \hfill \cite{mfml/book/mml/Deisenroth-Faisal-Ong}

    \item The sum of symmetric matrices $\bm{A},\ \bm{B} \in \mathbb{R}^{n\times n}$ is \textbf{always symmetric}. 
    \hfill \cite{mfml/book/mml/Deisenroth-Faisal-Ong}

    \item although the product of 2 symmetric matrices is \textbf{always defined}, it is generally \textbf{not symmetric}
    \hfill \cite{mfml/book/mml/Deisenroth-Faisal-Ong}

    \item 
    \begin{theorem}[Spectral Theorem]
        If $\bm{A} \in \mathbb{R}^{n\times n}$ is symmetric, there exists an orthonormal basis of the corresponding vector space $V$ consisting of eigenvectors of $\bm{A}$, and each eigenvalue is real.
        \hfill \cite{mfml/book/mml/Deisenroth-Faisal-Ong}
    \end{theorem}
    \begin{enumerate}
        \item A direct implication of the spectral theorem is that the eigen-decomposition of a symmetric matrix $\bm{A}$ exists (with real eigenvalues), and that we can find an ONB of eigenvectors so that $\bm{A} = \bm{PDP}^\top$ , where $\bm{D}$ is diagonal and the columns of $\bm{P}$ contain the eigenvectors.
        \hfill \cite{mfml/book/mml/Deisenroth-Faisal-Ong}
    \end{enumerate}

    \item 
    \begin{theorem}[Symmetric Matrix: always diagonalizable]
        A symmetric matrix $\bm{S} \in \mathbb{R}^{n\times n}$ can always be diagonalized.
        \hfill \cite{mfml/book/mml/Deisenroth-Faisal-Ong}
    \end{theorem}
    \begin{enumerate}
        \item follows directly from the spectral theorem
        \hfill \cite{mfml/book/mml/Deisenroth-Faisal-Ong}

        \item $\bm{P}$ an orthogonal matrix
        \hfill \cite{mfml/book/mml/Deisenroth-Faisal-Ong}

        \item $\bm{A} = \bm{PDP}^{-1}$
        \hfill \cite{mfml/book/mml/Deisenroth-Faisal-Ong}
    \end{enumerate}
\end{enumerate}







