\subsection{Diagonal Matrix ($D$)}


\begin{enumerate}
    \item A diagonal matrix is a matrix that has value zero on all off-diagonal elements, i.e., they are of the form:
    \hfill \cite{mfml/book/mml/Deisenroth-Faisal-Ong}
    \\
    .\hfill
    $
        \bm{D} = \begin{bmatrix}
        c_1 & \cdots & 0 \\
        \vdots & \ddots & \vdots \\
        0 & \cdots & c_n
        \end{bmatrix}
    $
    \hfill \cite{mfml/book/mml/Deisenroth-Faisal-Ong}

    \item They allow fast computation of determinants, powers, and inverses.
    \begin{enumerate}
        \item The determinant is the product of its diagonal entries
        \hfill \cite{mfml/book/mml/Deisenroth-Faisal-Ong}
        
        \item a matrix power $\bm{D}^k$ is given by each diagonal element raised to the power $k$
        \hfill \cite{mfml/book/mml/Deisenroth-Faisal-Ong}
        
        \item the inverse $\bm{D}^{-1}$ is the reciprocal of its diagonal elements if all of them are nonzero
        \hfill \cite{mfml/book/mml/Deisenroth-Faisal-Ong}
        
        \item[] 
        $
            \dabs{\bm{D}} 
            = \begin{vmatrix}
                c_1 & \cdots & 0 \\
                \vdots & \ddots & \vdots \\
                0 & \cdots & c_n
            \end{vmatrix}
            = \dprod_{i=1}^n c_i
        $
        \hfill
        $
            \bm{D}^k 
            = \begin{bmatrix}
                c_1^k & \cdots & 0 \\
                \vdots & \ddots & \vdots \\
                0 & \cdots & c_n^k
            \end{bmatrix}
        $
        \hfill
        $
            \bm{D}^{-1} 
            = \begin{bmatrix}
                \dfrac{1}{c_1} & \cdots & 0 \\
                \vdots & \ddots & \vdots \\
                0 & \cdots & \dfrac{1}{c_n}
            \end{bmatrix}
        $
    \end{enumerate}

    \item 
    \begin{definition}[Diagonalizable]
        A matrix $\bm{A} \in \mathbb{R}^{n\times n}$ is diagonalizable if it is similar to a diagonal matrix, i.e., if there exists an invertible matrix $\bm{P} \in \mathbb{R}^{n\times n}$ such that $\bm{D} = \bm{P}^{-1}\bm{AP}$.
        \hfill \cite{mfml/book/mml/Deisenroth-Faisal-Ong}
    \end{definition}
\end{enumerate}














