\subsection{Matrix Multiplication ( $AB$ OR $@$ OR $\cdot$ ) \cite{mfml/book/mml/Deisenroth-Faisal-Ong}}

For matrices $\bm{A} \in \mathbb{R}^{m\times n}$, $\bm{B} \in \mathbb{R}^{n\times k}$, the elements $c_{ij}$ of the product matrices. $\bm{C} = \bm{AB} \in \mathbb{R}^{m\times k}$ are computed as:

\vspace{0.5cm}
\hfill
$
    c_{ij} = \dsum_{l=1}^n a_{il}\ b_{lj}
$
\hfill
$
    i = 1,\cdots,m
    \hspace{1cm}
    j = 1,\cdots,k
$
\hfill \cite{mfml/book/mml/Deisenroth-Faisal-Ong}





\begin{lstlisting}[
    language=Python,
    caption=Matrix Multiplication - numPy
]
import numpy as np

m,n,k = 4,3,5

A = np.random.randint(-10, 10, size=(m, n))
B = np.random.randint(-10, 10, size=(n, k))

C1 = A @ B
C2 = np.matmul(A, B)
C3 = np.dot(A, B)

print("A:\n", A)
print("B:\n", B)
print("C1:\n", C1)
print("C2:\n", C2)
print("C3:\n", C3)
print(np.all(C1 == C2), np.all(C1 == C3))
\end{lstlisting}




\vspace{0.5cm}

\begin{enumerate}
    \item To compute element $c_{ij}$ we multiply the elements of the $i$th row of $\bm{A}$ with the $j$th column of $\bm{B}$ and sum them up.
    \hfill \cite{mfml/book/mml/Deisenroth-Faisal-Ong}

    \item Matrices can only be multiplied if their “neighboring” dimensions match.
    \hfill \cite{mfml/book/mml/Deisenroth-Faisal-Ong}
    \\
    \hfill
    $
        \underset{n\times k}{\underbrace{\bm{A}}}\
        \underset{k\times m}{\underbrace{\bm{B}}}
        =
        \underset{n\times m}{\underbrace{\bm{C}}}
    $
    \hfill \cite{mfml/book/mml/Deisenroth-Faisal-Ong}
    \\
    The product $\bm{BA}$ is not defined if $m \neq n$ since the neighboring dimensions do not match.

    \item Matrix multiplication is \textbf{not} defined as an element-wise operation on matrix elements, i.e., 
    \\
    $c_{ij} \neq a_{ij}\ b_{ij}$ (even if the size of $\bm{A}$, $\bm{B}$ was chosen appropriately).
    \hfill \cite{mfml/book/mml/Deisenroth-Faisal-Ong}
\end{enumerate}


\subsubsection{Properties}

\begin{enumerate}
    \item matrix multiplication is \textbf{not} commutative: $\bm{AB} \neq \bm{BA}$
    \hfill \cite{mfml/book/mml/Deisenroth-Faisal-Ong}

    \item \textbf{Associativity}: 
    $
        \forall 
        \bm{A} \in \mathbb{R}^{m\times n},\ 
        \bm{B} \in \mathbb{R}^{n\times p},\ 
        \bm{C} \in \mathbb{R}^{p\times q}
    $:
    
        \begin{enumerate}
            \item $\bm{ABC} = (\bm{AB})\ \bm{C} = \bm{A}\ (\bm{BC})$
            \hfill \cite{mfml/book/mml/Deisenroth-Faisal-Ong}
        \end{enumerate}

    \item \textbf{Distributivity}: 
    $
        \forall 
        \bm{A},\ \bm{B}\in \mathbb{R}^{m\times n},\ 
        \bm{C},\ \bm{D}\in \mathbb{R}^{n\times p}
    $:

        \begin{enumerate}
            \item $(\bm{A} + \bm{B})\ \bm{C} = \bm{AC} + \bm{BC}$
            \hfill \cite{mfml/book/mml/Deisenroth-Faisal-Ong}
            
            \item $\bm{A}\ (\bm{C} + \bm{D}) = \bm{AC} + \bm{AD}$
            \hfill \cite{mfml/book/mml/Deisenroth-Faisal-Ong}
        \end{enumerate}

    \item Multiplication with the identity matrix: 
    $
        \forall 
        \bm{A},\ \bm{B}\in \mathbb{R}^{m\times n}
    $:
        \begin{enumerate}
            \item $\bm{I}_m\bm{A} = \bm{AI}_n = \bm{A}$
            \hfill $m\neq n \Rightarrow \bm{I}_m \neq \bm{I}_n$
            \hfill \cite{mfml/book/mml/Deisenroth-Faisal-Ong}
        \end{enumerate}
\end{enumerate}















