\chapter{Linear Algebra}

\begin{enumerate}
    \item \textbf{Algebra} is constructing a set of objects (symbols) and a set of rules to manipulate these objects.
    \hfill \cite{mfml/book/mml/Deisenroth-Faisal-Ong}

    \item \textbf{Linear algebra} is the study of vectors and certain rules to manipulate vectors.
    \hfill \cite{mfml/book/mml/Deisenroth-Faisal-Ong}
\end{enumerate}


\section{Systems of Linear Equations}

\begin{customArrayStretch}{1.3}
\begin{table}[H]
    \centering
    \begin{tabular}{| l | l  l |}
        \hline
        
        coefficients & $a_{ij}$    & $\in \mathbb{R}$ \\ \hline
        
        constants & $b_{i}$     & $\in \mathbb{R}$ \\ \hline

        unknowns & $x_{i}$     & $\in \mathbb{R}$ \\ \hline

    \end{tabular}
    \caption*{Notations}
\end{table}
\end{customArrayStretch}


\vspace{0.5cm}


\begin{enumerate}[itemsep=0.3cm]
    \item \textbf{general form} of a system of linear equations:
    \\
    . \hfill
    $
        \begin{aligned}
            a_{11}x_1 & + & \cdots & + & a_{1n}x_n & = & b_1 \\
            & & & & \vdots \\
            a_{m1}x_1 & + & \cdots & + & a_{mn}x_n & = & b_m
        \end{aligned}
    $
    \hfill \cite{mfml/book/mml/Deisenroth-Faisal-Ong}
    \vspace{0.2cm}
    \begin{enumerate}
        \item $x_1, \cdots , x_n$ are the \textbf{unknowns} of this system.

        \item Every $n$-tuple $(x_1, \cdots , x_n) \in \mathbb{R}^n$ that satisfies this system is a \textbf{solution} of the linear equation system.
    \end{enumerate}


    \item \textbf{compact notation}:
    \\[0.2cm]
    $
        \begin{bmatrix}a_{11}\\ \vdots\\ a_{m1}\end{bmatrix} x_1 +
        \begin{bmatrix}a_{12}\\ \vdots\\ a_{m2}\end{bmatrix} x_2 +
        \cdots +
        \begin{bmatrix}a_{1n}\\ \vdots\\ a_{mn}\end{bmatrix} x_n =
        \begin{bmatrix}b_{1}\\ \vdots\\ b_{m}\end{bmatrix}
    \Longleftrightarrow
        \underset{A}{\underbrace{\begin{bmatrix}
            a_{11} & \cdots & a_{1n} \\
            \vdots & \ddots & \vdots \\
            a_{m1} & \cdots & a_{mn}
        \end{bmatrix}}} \
        \underset{x}{\underbrace{\begin{bmatrix} x_{1} \\ \vdots \\ x_{n} \end{bmatrix}}}
        =
        \underset{b}{\underbrace{\begin{bmatrix} b_{1} \\ \vdots \\ b_{m} \end{bmatrix}}}
    $
    \hfill \cite{mfml/book/mml/Deisenroth-Faisal-Ong}

    
    
\end{enumerate}


\vspace{0.5cm}
\textbf{Note}:
\begin{enumerate}[itemsep=0.2cm]
    \item In general, for a real-valued system of linear equations we obtain either no, exactly one, or infinitely many solutions. 
    \hfill \cite{mfml/book/mml/Deisenroth-Faisal-Ong}

    \item \textbf{Geometric Interpretation of Systems of Linear Equations}: 
    In a system of linear equations with two variables $x_1$, $x_2$, each linear equation defines a line on the $x_1x_2$-plane. Since a solution to a system of linear equations must satisfy all equations simultaneously, the solution set is the intersection of these lines. This intersection set can be a line (if the linear equations describe the same line), a point, or empty (when the lines are parallel).
    \hfill \cite{mfml/book/mml/Deisenroth-Faisal-Ong}
    \\
    Similarly, for three variables, each linear equation determines a plane in three-dimensional space. When we intersect these planes, i.e., satisfy all linear equations at the same time, we can obtain a solution set that is a plane, a line, a point or empty (when the planes have no common intersection).
    \hfill \cite{mfml/book/mml/Deisenroth-Faisal-Ong}

    \item the product $Ax$ is a (linear) combination of the columns of $A$
    \hfill \cite{mfml/book/mml/Deisenroth-Faisal-Ong}

    \item \textbf{Augmented Matrix} ( $\dSquareBrac{A\ |\ b}$ ): 
    \\[0.2cm]
    $
        \left[
        \begin{array}{ccc|c}
            a_{11} & \cdots & a_{1n} & b_{1}\\
            \vdots & \ddots & \vdots & \vdots \\
            a_{m1} & \cdots & a_{mn} & b_{m}
        \end{array}
        \right]
    $
\end{enumerate}








\subsection{Solutions}

\begin{enumerate}[itemsep=0.3cm]

\item \textbf{Unique Solution}:
\begin{enumerate}
    \item The system has exactly one solution.
    \hfill \cite{common/online/chatgpt}

    \item \textbf{Graphically}: The lines (or planes) intersect at a single point.
    \hfill \cite{common/online/chatgpt}

    \item \textbf{Algebraically}: The equations are independent and consistent.
    \hfill \cite{common/online/chatgpt}
\end{enumerate}

\item \textbf{Infinite Solutions}:
\begin{enumerate}
    \item The system has infinitely many solutions.
    \hfill \cite{common/online/chatgpt}

    \item \textbf{Graphically}: The lines (or planes) are exactly the same (i.e., they overlap).
    \hfill \cite{common/online/chatgpt}

    \item \textbf{Algebraically}: The equations are dependent and consistent.
    \hfill \cite{common/online/chatgpt}
\end{enumerate}

\item \textbf{No Solution}:
\begin{enumerate}
    \item The system has no common solution.
    \hfill \cite{common/online/chatgpt}

    \item \textbf{Graphically}: The lines are parallel (in 2D) and never meet.
    \hfill \cite{common/online/chatgpt}

    \item \textbf{Algebraically}: The system is inconsistent.
    \hfill \cite{common/online/chatgpt}
\end{enumerate}

\end{enumerate}












\subsubsection{Finding Solution: Elementary Transformations}

\begin{enumerate}[itemsep=0.2cm]
    \item keep the solution set the same, but that transform the equation system into a simpler form
    \hfill \cite{mfml/book/mml/Deisenroth-Faisal-Ong}

    \item Operations:
    \begin{enumerate}
        \item Exchange of two equations (rows in the matrix representing the system of equations)
        \hfill \cite{mfml/book/mml/Deisenroth-Faisal-Ong}

        \item Multiplication of an equation (row) with a constant $\lambda \in \mathbb{R} \backslash \dCurlyBrac{0}$
        \hfill \cite{mfml/book/mml/Deisenroth-Faisal-Ong}

        \item Addition of two equations (rows) 
        \hfill \cite{mfml/book/mml/Deisenroth-Faisal-Ong}
    \end{enumerate}
\end{enumerate}










\subsubsection{Finding Solution: Row-Echelon Form (REF)}

\begin{enumerate}[itemsep=0.2cm]
    \item \textbf{pivot}: The leading coefficient of a row (first nonzero number from the left) is called the pivot and is always strictly to the right of the pivot of the row above it.
    \hfill \cite{mfml/book/mml/Deisenroth-Faisal-Ong}

    \item any equation system in row-echelon form always has a “staircase” structure
    \hfill \cite{mfml/book/mml/Deisenroth-Faisal-Ong}

    \item A matrix is in row-echelon form if
    \begin{enumerate}
        \item All rows that contain only zeros are at the bottom of the matrix; correspondingly, all rows that contain at least one nonzero element are on top of rows that contain only zeros.
        \hfill \cite{mfml/book/mml/Deisenroth-Faisal-Ong}

        \item Looking at nonzero rows only, the first nonzero number from the left (also called the \textbf{pivot} or the \textbf{leading coefficient}) is always strictly to the right of the pivot of the row above it.
        \hfill \cite{mfml/book/mml/Deisenroth-Faisal-Ong}
    \end{enumerate}

    \item The variables corresponding to the pivots in the row-echelon form are called basic variables and the other variables are free variables.
    \hfill \cite{mfml/book/mml/Deisenroth-Faisal-Ong}

    \item we express the right-hand side of the equation system using the pivot columns, such that $b = \dsum^P_{i=1} \lambda_i\ p_i$, where $p_i$ , $i = 1, \cdots , P$, are the pivot columns. 
    \\
    The $\lambda_i$ are determined easiest if we start with the rightmost pivot column and work our way to the left.
    \hfill \cite{mfml/book/mml/Deisenroth-Faisal-Ong}

    
\end{enumerate}








\subsubsection{Finding Solution: Reduced Row-Echelon Form (RREF)/ row-reduced echelon form/ row canonical form \cite{mfml/book/mml/Deisenroth-Faisal-Ong}}

\begin{enumerate}[itemsep=0.2cm]
    \item An equation system is in reduced reduced row-echelon form if:
    \begin{enumerate}
        \item It is in row-echelon form.
        \hfill \cite{mfml/book/mml/Deisenroth-Faisal-Ong}
    
        \item Every pivot is $1$.
        \hfill \cite{mfml/book/mml/Deisenroth-Faisal-Ong}
    
        \item The pivot is the only non-zero entry in its column.
        \hfill \cite{mfml/book/mml/Deisenroth-Faisal-Ong}
    \end{enumerate}

    \item Gaussian elimination is an algorithm that performs elementary transformations to bring a system of linear equations into reduced row-echelon form.
    \hfill \cite{mfml/book/mml/Deisenroth-Faisal-Ong}
\end{enumerate}










\subsubsection{Particular Solution/ Special solution}

\begin{customArrayStretch}{1.3}
\begin{table}[H]
    \centering
    \begin{tabular}{|l|l|l|}
        \hline
        \textbf{Term} & 
            \textbf{Scope} & 
            \textbf{Context} \\ \hline \hline
        
        \textbf{Unique Solution} & 
            One and only one solution & 
            Systems of equations \\ \hline

        \textbf{Particular Solution} & 
            One of possibly many solutions & 
            Differential equations, infinite solution systems \\ \hline

    \end{tabular}
    \caption*{Unique Solution VS Particular Solution \cite{common/online/chatgpt}}
\end{table}
\end{customArrayStretch}



\begin{enumerate}
    \item this is not the only solution of this system of linear equations.
    \hfill \cite{mfml/book/mml/Deisenroth-Faisal-Ong}

    \item To capture all the other solutions, we need to be creative in generating $0$ in a non-trivial way using the columns of the matrix: Adding $0$ to our special solution \textbf{does not} change the special solution. 
    \hfill \cite{mfml/book/mml/Deisenroth-Faisal-Ong}

    
\end{enumerate}







\subsubsection{Finding Solutions to $Ax=0$: Minus-$1$ Trick \cite{mfml/book/mml/Deisenroth-Faisal-Ong}}

\begin{enumerate}
    \item 

\end{enumerate}










\subsubsection{General Solution (= Particular Solution + Solutions to $Ax=0$)}

\begin{customArrayStretch}{1.3}
\begin{table}[H]
    \centering
    \begin{tabular}{|l|l|l|}
        \hline
        \textbf{Term} & 
            \textbf{What it tells you} & 
            \textbf{Context} \\ \hline

        \textbf{Infinite Solutions} & 
            The quantity of solutions & 
            Systems of equations \\ \hline

        \textbf{General Solution} & 
            A formula for all solutions & 
            Differential equations, algebra \\ \hline

    \end{tabular}
    \caption*{Infinite Solutions VS General Solution \cite{common/online/chatgpt}}
\end{table}
\end{customArrayStretch}


\begin{enumerate}[itemsep=0.2cm]
    \item The general approach we followed consisted of the following three steps:
    \begin{enumerate}
        \item Find a particular solution to $Ax = b$
        \hfill \cite{mfml/book/mml/Deisenroth-Faisal-Ong}

        \item Find all solutions to $Ax = 0$
        \hfill \cite{mfml/book/mml/Deisenroth-Faisal-Ong}

        \item Combine the solutions from steps 1. and 2. to the general solution.
        \hfill \cite{mfml/book/mml/Deisenroth-Faisal-Ong}
    \end{enumerate}

    
\end{enumerate}



















