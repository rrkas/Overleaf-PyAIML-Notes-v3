\chapter{Linear Algebra}

\begin{enumerate}
    \item \textbf{Algebra} is constructing a set of objects (symbols) and a set of rules to manipulate these objects.
    \hfill \cite{mfml/book/mml/Deisenroth-Faisal-Ong}

    \item \textbf{Linear algebra} is the study of vectors and certain rules to manipulate vectors.
    \hfill \cite{mfml/book/mml/Deisenroth-Faisal-Ong}
\end{enumerate}


\section{Systems of Linear Equations}

\begin{customArrayStretch}{1.3}
\begin{table}[H]
    \centering
    \begin{tabular}{| l | l  l |}
        \hline
        
        coefficients & $a_{ij}$    & $\in \mathbb{R}$ \\ \hline
        
         & $b_{i}$     & $\in \mathbb{R}$ \\ \hline
    \end{tabular}
    \caption*{Notations}
\end{table}
\end{customArrayStretch}


\vspace{0.5cm}


\begin{enumerate}[itemsep=0.3cm]
    \item \textbf{general form} of a system of linear equations:
    \\
    $
        \begin{aligned}
            a_{11}x_1 & + & \cdots & + & a_{1n}x_n & = & b_1 \\
            & & & & \vdots \\
            a_{m1}x_1 & + & \cdots & + & a_{mn}x_n & = & b_m
        \end{aligned}
    $
    \hfill \cite{mfml/book/mml/Deisenroth-Faisal-Ong}
    \begin{enumerate}
        \item $x_1, \cdots , x_n$ are the \textbf{unknowns} of this system.

        \item Every $n$-tuple $(x_1, \cdots , x_n) \in \mathbb{R}^n$ that satisfies this system is a \textbf{solution} of the linear equation system.
    \end{enumerate}


    \item \textbf{compact notation}:
    \\[0.2cm]
    $
        \begin{bmatrix}a_{11}\\ \vdots\\ a_{m1}\end{bmatrix} x_1 +
        \begin{bmatrix}a_{12}\\ \vdots\\ a_{m2}\end{bmatrix} x_2 +
        \cdots +
        \begin{bmatrix}a_{1n}\\ \vdots\\ a_{mn}\end{bmatrix} x_n =
        \begin{bmatrix}b_{1}\\ \vdots\\ b_{m}\end{bmatrix}
    \Longleftrightarrow
        \underset{A}{\underbrace{\begin{bmatrix}
            a_{11} & \cdots & a_{1n} \\
            \vdots & \ddots & \vdots \\
            a_{m1} & \cdots & a_{mn}
        \end{bmatrix}}} \
        \underset{x}{\underbrace{\begin{bmatrix} x_{1} \\ \vdots \\ x_{n} \end{bmatrix}}}
        =
        \underset{b}{\underbrace{\begin{bmatrix} b_{1} \\ \vdots \\ b_{m} \end{bmatrix}}}
    $
    \hfill \cite{mfml/book/mml/Deisenroth-Faisal-Ong}

    
    
\end{enumerate}


\vspace{0.5cm}
\textbf{Note}:
\begin{enumerate}[itemsep=0.2cm]
    \item In general, for a real-valued system of linear equations we obtain either no, exactly one, or infinitely many solutions. 
    \hfill \cite{mfml/book/mml/Deisenroth-Faisal-Ong}

    \item \textbf{Geometric Interpretation of Systems of Linear Equations}: 
    In a system of linear equations with two variables $x_1$, $x_2$, each linear equation defines a line on the $x_1x_2$-plane. Since a solution to a system of linear equations must satisfy all equations simultaneously, the solution set is the intersection of these lines. This intersection set can be a line (if the linear equations describe the same line), a point, or empty (when the lines are parallel).
    \hfill \cite{mfml/book/mml/Deisenroth-Faisal-Ong}
    \\
    Similarly, for three variables, each linear equation determines a plane in three-dimensional space. When we intersect these planes, i.e., satisfy all linear equations at the same time, we can obtain a solution set that is a plane, a line, a point or empty (when the planes have no common intersection).
    \hfill \cite{mfml/book/mml/Deisenroth-Faisal-Ong}

    \item the product $Ax$ is a (linear) combination of the columns of $A$
    \hfill \cite{mfml/book/mml/Deisenroth-Faisal-Ong}

    
\end{enumerate}










