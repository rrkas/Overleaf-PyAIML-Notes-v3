\section{Standard Normal Distribution (${N}(0, 1)$)}

\begin{enumerate}
    \item When we choose $\mu = 0$ and $\sigma = 1$ in a Normal Distribution, its called Standard Normal Distribution.
    \hfill \cite{statistics/book/Statistics-for-Data-Scientists/Maurits-Kaptein}


\end{enumerate}


\subsection{PDF ($f(x)$ or $f(x)$ or $\phi(x)$)}

\begin{enumerate}
    \item[] $\phi(x) = \dfrac{\exp(-x^2/2)}{\sqrt{2\pi}}$
    \hfill \text{\cite{statistics/book/Statistics-for-Data-Scientists/Maurits-Kaptein}}


\end{enumerate}



\subsection{Bivariate Standard Normal Distribution}

\begin{enumerate}
    \item Consider a bivariate standard Gaussian random variable $X$ and performed a linear transformation $\bm{Ax}$ on it. 
    \hfill \cite{mfml/book/mml/Deisenroth-Faisal-Ong}

    \item The outcome is a Gaussian random variable with mean zero and covariance $\bm{AA}^\top$.
    \hfill \cite{mfml/book/mml/Deisenroth-Faisal-Ong}

    \item Adding a constant vector will change the mean of the distribution, without affecting its variance, that is, the random variable $\bm{x} + \bm{\mu}$ is Gaussian with mean $\bm{\mu}$ and identity covariance.
    Hence, any linear/affine transformation of a Gaussian random variable is Gaussian distributed.
    \hfill \cite{mfml/book/mml/Deisenroth-Faisal-Ong}

\end{enumerate}













