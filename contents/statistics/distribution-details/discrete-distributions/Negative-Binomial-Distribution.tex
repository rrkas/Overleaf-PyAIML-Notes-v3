\section{Negative binomial distribution ($ {\displaystyle  {NB} (r,\,p)}$)}


\subsection{PMF}

\begin{enumerate}
    \item The negative binomial PMF is often considered a Poisson PMF with an extra amount of variation
    \hfill \cite{statistics/book/Statistics-for-Data-Scientists/Maurits-Kaptein}

    \item \textbf{Parameters}:
    \begin{enumerate}
        \item $\lambda$: mean, the same as for the Poisson random variable
        \hfill \cite{statistics/book/Statistics-for-Data-Scientists/Maurits-Kaptein}

        \item $\delta$: overdispersion parameter, indicating the extra amount of variation on top of the Poisson variation.
        \hfill \cite{statistics/book/Statistics-for-Data-Scientists/Maurits-Kaptein}
    \end{enumerate}

    \item A negative binomial random variable $X$ has its outcomes in the set $\dCurlyBrac{0, 1, 2, 3, ....}$, like the Poisson random variable. 
    The PMF is defined by:
    \hfill \cite{statistics/book/Statistics-for-Data-Scientists/Maurits-Kaptein}
    \\
    $
        P(X=x)
        = f_{\lambda,\ \delta}(k)
        = \dfrac{\Gamma(k + \delta^{-1})}{\Gamma(k + 1)\ \Gamma(\delta^{-1})}
        \dfrac{(\delta\lambda)^k}{(1+\delta\lambda)^{k+\delta^{-1}}}
    $
    \hfill \cite{statistics/book/Statistics-for-Data-Scientists/Maurits-Kaptein}
    \\
    $
        \Gamma(z) = \begin{cases}
            \dint_{0}^{\infty} x^{z-1} \exp\dCurlyBrac{x} dx & \text{ in general} \\
            (k-1)! & \text{ when $k$ is integer}
        \end{cases}
    $
    \hfill \cite{statistics/book/Statistics-for-Data-Scientists/Maurits-Kaptein}

    \item $\mu = \mathbb{E}(X) = \lambda$
    \hfill \cite{statistics/book/Statistics-for-Data-Scientists/Maurits-Kaptein}

    \item $\sigma^2 = \mathbb{E}(X - \lambda)^2 = \lambda + \delta\lambda^2$
    \hfill \cite{statistics/book/Statistics-for-Data-Scientists/Maurits-Kaptein}

    \item In case the parameter $\delta$ converges to zero, the variance converges to the variance of a Poisson random variable. 
    This is the reason that the parameter $\delta$ is called the overdispersion.
\end{enumerate}




\subsection{Summary}

\begin{enumerate}
    \item \textbf{Notation}: 
    $
         {\displaystyle \mathrm {NB} (r,\,p)}
    $
    \hfill \cite{wiki/Negative_binomial_distribution}

    \item \textbf{Parameters}:
    $r > 0$ — number of successes until the experiment is stopped (integer, but the definition can also be extended to reals) $p \in [0, 1]$ — success probability in each experiment (real)
    \hfill \cite{wiki/Negative_binomial_distribution}

    \item \textbf{Support}: 
     $k \in \dCurlyBrac{0, 1, 2, 3, \cdots}$ — number of failures
    \hfill \cite{wiki/Negative_binomial_distribution}

    \item \textbf{PMF}:
    ${\displaystyle k\mapsto {k+r-1 \choose k}\cdot (1-p)^{k}p^{r}}$ 
    involving a binomial coefficient
    \hfill \cite{wiki/Negative_binomial_distribution}

    \item \textbf{CDF}:
    ${\displaystyle k\mapsto I_{p}(r,\,k+1)}$ 
    the regularized incomplete beta function
    \hfill \cite{wiki/Negative_binomial_distribution}

    \item \textbf{Mean}: 
    $
         {\displaystyle {\frac {r(1-p)}{p}}}
    $
    \hfill \cite{wiki/Negative_binomial_distribution}

    % \item \textbf{Median}: 
    % $
    % $
    % \hfill \cite{wiki/Negative_binomial_distribution}

    \item \textbf{Mode}: 
    $
         {\displaystyle {\begin{cases}\left\lfloor {\frac {(r-1)(1-p)}{p}}\right\rfloor &{\text{if }}r>1\\0&{\text{if }}r\leq 1\end{cases}}}
    $
    \hfill \cite{wiki/Negative_binomial_distribution}

    \item \textbf{Variance}: 
    $ 
         {\displaystyle {\frac {r(1-p)}{p^{2}}}}
    $
    \hfill \cite{wiki/Negative_binomial_distribution}

    % \item \textbf{Median absolute deviation (MAD)}: 
    % $    
    % $
    % \hfill \cite{wiki/Negative_binomial_distribution}

    \item \textbf{Skewness}:
    $
         {\displaystyle {\frac {2-p}{\sqrt {(1-p)r}}}}
    $
    \hfill \cite{wiki/Negative_binomial_distribution}

    \item \textbf{Excess kurtosis}: 
    $
         {\displaystyle {\frac {6}{r}}+{\frac {p^{2}}{(1-p)r}}}
    $
    \hfill \cite{wiki/Negative_binomial_distribution}

    % \item \textbf{Entropy}: 
    % \hfill \cite{wiki/Negative_binomial_distribution}

    \item \textbf{Moment-generating function (MGF)}: 
    $
         {\displaystyle {\biggl (}{\frac {p}{1-(1-p)e^{t}}}{\biggr )}^{\!r}{\text{ for }}t<-\log(1-p)}
    $
    \hfill \cite{wiki/Negative_binomial_distribution}
    
    \item \textbf{Characteristic function (CF)}:
    $   
         {\displaystyle {\biggl (}{\frac {p}{1-(1-p)e^{i\,t}}}{\biggr )}^{\!r}{\text{ with }}t\in \mathbb {R} }
    $
    \hfill \cite{wiki/Negative_binomial_distribution}

    \item \textbf{Probability-generating function (PGF)}:
    $
         {\displaystyle {\biggl (}{\frac {p}{1-(1-p)z}}{\biggr )}^{\!r}{\text{ for }}|z|<{\frac {1}{p}}}
    $
    \hfill \cite{wiki/Negative_binomial_distribution}

    \item \textbf{Fisher information}:
    $
         {\displaystyle {\frac {r}{p^{2}(1-p)}}}
    $
    \hfill \cite{wiki/Negative_binomial_distribution}

    \item \textbf{Method of moments}:
    $ {\displaystyle r={\frac {E[X]^{2}}{V[X]-E[X]}}}$
    \hspace{1cm}
    $ {\displaystyle p={\frac {E[X]}{V[X]}}}$
    \hfill \cite{wiki/Negative_binomial_distribution}
\end{enumerate}







