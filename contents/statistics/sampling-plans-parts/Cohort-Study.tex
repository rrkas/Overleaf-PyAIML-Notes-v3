\section{Cohort Study (exposure-based)}

\begin{enumerate}
    \item a simple random sample is taken from the population of units who are exposed and another simple random sample is taken from the population of units who are unexposed.
    Thus this way of sampling relates directly to \textbf{stratified sampling} with the strata being the group of exposed ($E$) and the group of unexposed ($E^c$).
    \hfill \cite{statistics/book/Statistics-for-Data-Scientists/Maurits-Kaptein}

    \item In each sample or stratum the outcome D is noted and the contingency table in Table \fullref{statistics/probability-theory/Conditional Probability/Conditional-probabilities-contingency-table} is filled.
    \hfill \cite{statistics/book/Statistics-for-Data-Scientists/Maurits-Kaptein}

    \item In this setting, the probabilities $P (E)$ and $P (E^c)$ are preselected before sampling and are fixed in the sample, whatever they are in the population.
    Thus the sample and the population may have very different probabilities.
    \hfill \cite{statistics/book/Statistics-for-Data-Scientists/Maurits-Kaptein}

    \item the probabilities in the cells of the contingency tables are no longer appropriate estimates for the population probabilities, since we have destroyed the ratio in probabilities for $E$ and $E^c$.
    \hfill \cite{statistics/book/Statistics-for-Data-Scientists/Maurits-Kaptein}

    \item Despite the fact that we cannot use the joint probabilities in the contingency table as estimates for the population probabilities, the risk difference, the relative risk, and the odds ratio in the sample are all appropriate estimates for the population when a cohort study is used.
    The reason is that these measures use the conditional probabilities only, where conditioning is done on the exposure. The $P (D|E)$ and $P (D|E^c)$ in the sample do represent the conditional population probabilities.
    \hfill \cite{statistics/book/Statistics-for-Data-Scientists/Maurits-Kaptein}
\end{enumerate}








