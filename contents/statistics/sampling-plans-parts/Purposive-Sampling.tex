\section{Purposive Sampling/ Judgmental Sampling \cite{statistics/book/Statistics-for-Data-Scientists/Maurits-Kaptein}}\label{Sampling Plans/Non-representative Sampling/Purposive Sampling or Judgmental Sampling}

\begin{enumerate}
    \item Purposive sampling or judgmental sampling tries to sample units for a specific purpose.
    \hfill \cite{statistics/book/Statistics-for-Data-Scientists/Maurits-Kaptein}

    \item This means that the collection of units is focused on one or more particular characteristics and hence it implies that only units that are more alike are sampled.
    \hfill \cite{statistics/book/Statistics-for-Data-Scientists/Maurits-Kaptein}

    \item This way of sampling is strongly related to the definition of the population, since deliberately excluding units from the sample is analogous to limiting the population of interest.
    \hfill \cite{statistics/book/Statistics-for-Data-Scientists/Maurits-Kaptein}

    \item Purposive sampling may be useful, but it is limited since it does not allow us in general to make statements about the whole population, and at best only about a limited part of the population (although we may not be sure either).
    \hfill \cite{statistics/book/Statistics-for-Data-Scientists/Maurits-Kaptein}

    \item It does most likely produce a biased sample with respect to the complete population.
    \hfill \cite{statistics/book/Statistics-for-Data-Scientists/Maurits-Kaptein}
\end{enumerate}
