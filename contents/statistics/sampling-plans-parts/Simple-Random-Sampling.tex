\section{Simple Random Sampling \cite{statistics/book/Statistics-for-Data-Scientists/Maurits-Kaptein}}\label{Sampling Plans/Representative Sampling/Simple Random Sampling}

\begin{table}[H]
    \centering
    \begin{tabular}{l l}
        $N$ & population size \\
        $n$ & sample size \\
        $K$ & number of possible samples \\
        $k$ & sample index ($ k \in \dParenBrac{1,\cdots,K}$) \\
        $S_k$ & sample\\
    \end{tabular}
\end{table}

\begin{enumerate}[itemsep=0.2cm]
    \item Implicitly assume that there is no particular group structure present in the population.
    \hfill \cite{statistics/book/Statistics-for-Data-Scientists/Maurits-Kaptein}
    
    \item Simple random sampling is a way of collecting samples such that each unit from the population has the exact same probability of becoming part of the sample. 
    \hfill \cite{statistics/book/Statistics-for-Data-Scientists/Maurits-Kaptein}

    
    \item Simple random sampling is a conceptually easy method of forming random samples but it can prove hard in practice.
    \hfill \cite{statistics/book/Statistics-for-Data-Scientists/Maurits-Kaptein}

    \item Simple random sampling is frequently combined with other choices or settings (see stratified and cluster sampling).
    \hfill \cite{statistics/book/Statistics-for-Data-Scientists/Maurits-Kaptein}

    \item Number of unique samples $
        = K 
        = \dfrac{N!}{n!(N-n)!}
    $
    \hfill \cite{statistics/book/Statistics-for-Data-Scientists/Maurits-Kaptein}

    \item the probability of collecting sample $S_k$, using sequential sampling $
        = \dfrac{1}{K} 
        = \dfrac{n!(N-n)!}{N!}
    $
    \hfill \cite{statistics/book/Statistics-for-Data-Scientists/Maurits-Kaptein}

    \item The probability that a specific unit is part of the sample $
        = \dfrac{n}{N}
    $
    \hfill \cite{statistics/book/Statistics-for-Data-Scientists/Maurits-Kaptein}

    \item The probability that a specific unit is \textbf{not} contained in the sample $
        = 1 - \dfrac{n}{N}
        = \dfrac{N-n}{N}
    $
    \hfill \cite{statistics/book/Statistics-for-Data-Scientists/Maurits-Kaptein}

    \item The number of samples that does \textbf{not} contain a specific unit $
        = \dfrac{(N - 1)!}{n!(N - 1 - n)!}
    $
    \hfill \cite{statistics/book/Statistics-for-Data-Scientists/Maurits-Kaptein}

    \item Number of samples that contain certain unit $i$ = $\dfrac{nK}{N}$ 
    \hfill \cite{statistics/book/Statistics-for-Data-Scientists/Maurits-Kaptein}
    \\
    $
        \Rightarrow
        \dsum_{k=1}^{K}\ \dsum_{i \in S_k} x_i
        = \dfrac{nK}{N}\dsum_{i=1}^{N} x_i
    $
    \hfill \cite{statistics/book/Statistics-for-Data-Scientists/Maurits-Kaptein}

    \item Population Variance: $
        \sigma^2
        = \dfrac{1}{N} \dsum_{i=1}^N (x_i-\mu)^2
    $
    \hfill \cite{statistics/book/Statistics-for-Data-Scientists/Maurits-Kaptein}

    \item \textbf{Disadvantage}: When the numbers of units across these subpopulations are (substantially) different, simple random may not collect units from each subgroup.
    \hfill \cite{statistics/book/Statistics-for-Data-Scientists/Maurits-Kaptein}
    
\end{enumerate}

\vspace{0.5cm}
\textbf{Example}:
\begin{enumerate}[itemsep=0.3cm]
    \item[] $N = 20,\ n = 5$

    \item[] $K = \dfrac{20!}{5!\cdot 15!}=15504$

    
\end{enumerate}


\subsection{Estimation for population mean}
\begin{enumerate}[itemsep=0.2cm]
    \item Estimator: $
        T 
        = \bar{x}
        = \dfrac{1}{n} \dsum_{i=1}^n x_i
    $
    \hfill \cite{statistics/book/Statistics-for-Data-Scientists/Maurits-Kaptein}
    
    \item Bias: $0$
    \hfill \cite{statistics/book/Statistics-for-Data-Scientists/Maurits-Kaptein}

    \item $
        MSE(\bar{x}_k)
        = \dfrac{\sigma^2 }{n}
        \dParenBrac{\dfrac{N}{N-1}}
        \dParenBrac{1-\dfrac{n}{N}}
        = \dfrac{\sigma^2\ (N-n)}{n\ (N-1)}
    $
    \hfill \cite{statistics/book/Statistics-for-Data-Scientists/Maurits-Kaptein}
    \\
    MSE shows that it becomes equal to zero when the sample size $n$ becomes equal to the population size $N$.
    \hfill \cite{statistics/book/Statistics-for-Data-Scientists/Maurits-Kaptein}
    \\
    The MSE will not become zero when the estimator is biased, even if the sample size is equal to the population size.
    \hfill \cite{statistics/book/Statistics-for-Data-Scientists/Maurits-Kaptein}
    

    \item $
        \mathbb{E}(T) 
        = \dfrac{1}{K}\dsum_{k=1}^{K} \bar{x}_k
        = \dfrac{1}{nK}\dsum_{k=1}^{K}\ \dsum_{i \in S_k} x_i
        = \dfrac{1}{N}\dsum_{i=1}^{N} x_i
        = \mu
    $
    \hfill \cite{statistics/book/Statistics-for-Data-Scientists/Maurits-Kaptein}
\end{enumerate}


\subsection{Estimation of the sample MSE}

\begin{enumerate}[itemsep=0.2cm]
    \item unbiased estimator for $\sigma^2$: \\
    $
       \hat{\sigma}^2 = \dfrac{N - 1}{N}\ s^2_k
    $
    \hfill 
    $
        \dParenBrac{\ 
            s^2_k = \dfrac{1}{n - 1} \dsum_{i\ \in\ S_k} (x_i - \bar{x}_k )^2
        \ }
    $
    \hfill \cite{statistics/book/Statistics-for-Data-Scientists/Maurits-Kaptein}

    \item $
        \hat{MSE}(\bar{x}_k) 
        = \dfrac{N - n}{N n}\ s^2_k
    $
    \hfill \cite{statistics/book/Statistics-for-Data-Scientists/Maurits-Kaptein}

    \item This is an unbiased estimator of the MSE of the arithmetic average in $MSE(\bar{x}_k)$ and it may also be used when the observations are binary.
    \hfill \cite{statistics/book/Statistics-for-Data-Scientists/Maurits-Kaptein}

    
\end{enumerate}





