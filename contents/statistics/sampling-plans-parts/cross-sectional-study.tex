\section{Cross-Sectional Study (population-based)}

SEE: \fullref{statistics/probability-theory/Conditional Probability}

\begin{enumerate}
    \item a \textbf{simple random sample} of size $n$ is taken from the population
    \hfill \cite{statistics/book/Statistics-for-Data-Scientists/Maurits-Kaptein}

    \item For each unit in the sample both the exposure and outcome are being observed and the units are then summarized into the four cells $(E, D)$, $(E, D^c)$, $(E^c, D)$, and $(E^c, D^c)$.
    The $2 \times 2$ contingency table would then contain the number of units in each cell.
    \hfill \cite{statistics/book/Statistics-for-Data-Scientists/Maurits-Kaptein}

    \item This way of sampling implies that the proportions in the last row ($P(D)$ and $P(Dc)$) and the proportions in the last column ($P(E)$ and $P(Ec)$) of contingency table would be unknown before sampling and they are being determined by the probability of outcome and exposure in the population.
    \hfill \cite{statistics/book/Statistics-for-Data-Scientists/Maurits-Kaptein}

    \item the observed probabilities in Table \fullref{statistics/probability-theory/Conditional Probability/Conditional-probabilities-contingency-table} obtained from the sample represent unbiased estimates of the population probabilities.
    \hfill \cite{statistics/book/Statistics-for-Data-Scientists/Maurits-Kaptein}

    \item we apply the theory of simple random sampling for estimation of a population proportion.
    \hfill \cite{statistics/book/Statistics-for-Data-Scientists/Maurits-Kaptein}
    \begin{enumerate}
        \item if we define the binary variable $x_i$ by $1$ if unit $i$ has both events $E$ and $D$ (thus $E \cap D$) and it is zero otherwise, the estimate of the population proportion $P(E \cap D)$ would be the sample average of this binary variable.

        \item This sample average is equal to the number of units in cell $(E, D)$ divided by the total sample size $n$
    \end{enumerate}
    \hfill \cite{statistics/book/Statistics-for-Data-Scientists/Maurits-Kaptein}

    \item calculation of the risk difference, the relative risk, and the odds ratio are all appropriate for cross-sectional studies.
    \hfill \cite{statistics/book/Statistics-for-Data-Scientists/Maurits-Kaptein}
\end{enumerate}















