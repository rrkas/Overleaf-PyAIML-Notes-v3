\section{Convenience Sampling \cite{statistics/book/Statistics-for-Data-Scientists/Maurits-Kaptein}}\label{Sampling Plans/Non-representative Sampling/Convenience Sampling}

\begin{enumerate}
    \item Convenience sampling collects only units from the population that can be easily obtained.
    \hfill \cite{statistics/book/Statistics-for-Data-Scientists/Maurits-Kaptein}

    \item This may provide a biased sample, as it represents only one small part or time window of the whole processing window for a batch of products. The term \textbf{bias}\label{Sampling Plans/Non-representative Sampling/Convenience Sampling/bias} indicates that we obtain the value of interest with a systematic mistake.
    \hfill \cite{statistics/book/Statistics-for-Data-Scientists/Maurits-Kaptein}

    \item  Convenience sampling is often justified by using the argument of population homogeneity. This insinuates that either the population units are not truly different or the process produces the population of units in random order. 
    \hfill \cite{statistics/book/Statistics-for-Data-Scientists/Maurits-Kaptein}
\end{enumerate}

