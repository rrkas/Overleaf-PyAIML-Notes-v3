\chapter{Sampling Plans}\label{Sampling Plans}


\begin{figure}[H]
    \centering
    \includegraphics[
        width=0.5\linewidth, 
        height=5cm, 
        keepaspectratio
    ]{images/statistics/sampling-plan.png}

    \caption{Sampling Plan: Relation between Population and Sample}
\end{figure}


\begin{enumerate}
    \item \textbf{Statistical Inference}\label{Sampling Plans/Statistical Inference}: To extend your conclusions beyond the observed data. 
    \hfill \cite{statistics/book/Statistics-for-Data-Scientists/Maurits-Kaptein}

    \item Sampling procedures are formal \textit{probabilistic approaches} to help collect units from the population for the sample.
    \hfill \cite{statistics/book/Statistics-for-Data-Scientists/Maurits-Kaptein}

    \item \textbf{Population}\label{Sampling Plans/Population}: The complete set of units that we would like to say something about is called the (target) population.
    \hfill \cite{statistics/book/Statistics-for-Data-Scientists/Maurits-Kaptein}
    \\
    In principle we would expect that a population is \textbf{always finite}, since an infinite number of units does not exist in real life. However, populations are often treated as infinite. One reason is that populations can be really really large.
    \hfill \cite{statistics/book/Statistics-for-Data-Scientists/Maurits-Kaptein}
    \\
    It is mathematically often more convenient (as we will see later) to assume that such a population is infinite.
    \hfill \cite{statistics/book/Statistics-for-Data-Scientists/Maurits-Kaptein}
    \\
    Properly defining or describing a population can be difficult. Furthermore, even if the population is established, measuring all units is often impossible or too elaborate. This means that information about the population can only be obtained by considering a subset of the population.
    \hfill \cite{statistics/book/Statistics-for-Data-Scientists/Maurits-Kaptein}

    \item \textbf{Sample}\label{Sampling Plans/Sample}: The set of units for which we have obtained data is referred to as the sample. 
    \hfill \cite{statistics/book/Statistics-for-Data-Scientists/Maurits-Kaptein}
    \\
    The sample is typically a subset of the population, although in theory the sample can form the whole population or the sample can contain units that are not from the target population. 
    \hfill \cite{statistics/book/Statistics-for-Data-Scientists/Maurits-Kaptein}

    \item \textbf{Representative Sample}\label{Sampling Plans/Representative Sample}:  A representative sample can be intuitively defined as a sample of units that has approximately the same distribution of characteristics as the population from which it was drawn.
    \hfill \cite{statistics/book/Statistics-for-Data-Scientists/Maurits-Kaptein}
    \\
    Representative sampling is also referred to as random or probability sampling.
    \hfill \cite{statistics/book/Statistics-for-Data-Scientists/Maurits-Kaptein}

    \item \textbf{Unit}\label{Sampling Plans/Unit}: A unit is usually a concrete or physical thing for which we would like to measure its characteristics.
    \hfill \cite{statistics/book/Statistics-for-Data-Scientists/Maurits-Kaptein}

    \item \textbf{Estimates}\label{Sampling Plans/Estimates}: In terms of statistical inference, the calculations on the sample data are referred to as estimates for the theoretical value in the whole population.
    \hfill \cite{statistics/book/Statistics-for-Data-Scientists/Maurits-Kaptein}


    %%%%%%%%%%%%%%%%%%%%%%%%%%%%%%%%%%%%%%%%%%%%%%%%%%%%%%%%%%%%%%%%%%%%%%%%%%%%%%
    \vspace{0.5cm}

    \item Reasons for sample instead of population:
    \begin{enumerate}
        \item In many applications we really can’t measure the complete population. For instance, one of the tests applied to aircraft engines is the “frozen bird test”. 
        \hfill \cite{statistics/book/Statistics-for-Data-Scientists/Maurits-Kaptein}

        \item Time, space, or budget restrictions often do not allow us to measure all units from a population.
        \hfill \cite{statistics/book/Statistics-for-Data-Scientists/Maurits-Kaptein}

        \item  Big data itself may be an argument for sampling. If we have a very large sample or we have been able to measure all units from the population, the resulting dataset can be so large that it becomes impossible to analyze the full data at one computer.
        \hfill \cite{statistics/book/Statistics-for-Data-Scientists/Maurits-Kaptein}
    \end{enumerate}

    \item A non-representative sample implies that we do not know the exact process by which units in the population became part of the sample.
    \hfill \cite{statistics/book/Statistics-for-Data-Scientists/Maurits-Kaptein}

    \item If we know which sampling procedure was applied to collect the units for the sample, we would also know how close the calculations or statistics would be to the theoretical value in the whole population.
    \hfill \cite{statistics/book/Statistics-for-Data-Scientists/Maurits-Kaptein}
    \\
    Thus the sampling procedure and the choice of calculation on the sample data (\\
    \nameref{Data/Describing Data/Central Tendency/(Arithmetic) mean or average}, \\
    \nameref{Data/Describing Data/Central Tendency/Median},\\
    \nameref{Data/Describing Data/Central Tendency/Quartiles/first quartile}, \\
    \nameref{Data/Describing Data/Central Tendency/Standard Deviation}\\
    etc.) would make statistical inference mathematically precise and it would therefore help us when making statements beyond the sample data.
    \hfill \cite{statistics/book/Statistics-for-Data-Scientists/Maurits-Kaptein}



    %%%%%%%%%%%%%%%%%%%%%%%%%%%%%%%%%%%%%%%%%%%%%%%%%%%%%%%%%%%%%%%%%%%%%%%%%%%%%%
    \vspace{0.5cm}

    \item Let \textit{Population}: $\dCurlyBrac{x_1, \cdots, x_N}$ ($N$ units) 
    \hfill \cite{statistics/book/Statistics-for-Data-Scientists/Maurits-Kaptein}

    \item Let \textit{Sample}: $\dCurlyBrac{x_{i_1}, \cdots, x_{i_n}}$ 
    ($n$ units, $n<N$) 
    ($i_h \in \dCurlyBrac{1, \cdots, N}$)
    ($h \neq l \Rightarrow i_h \neq i_l$)
    \hfill \cite{statistics/book/Statistics-for-Data-Scientists/Maurits-Kaptein}

    \item The values in the sample are referred to as a \textbf{realization}\label{Sampling Plans/realization} from the population.
    \hfill \cite{statistics/book/Statistics-for-Data-Scientists/Maurits-Kaptein}

    
\end{enumerate}





\section{Non-representative Sampling \cite{statistics/book/Statistics-for-Data-Scientists/Maurits-Kaptein}}\label{Sampling Plans/Non-representative Sampling}

\begin{enumerate}
    \item Although these sampling methods are frequently in use, it is strongly recommended not to apply these methods, unless knowledge is available on how to adjust or correct the sample for inferential purposes.
    \hfill \cite{statistics/book/Statistics-for-Data-Scientists/Maurits-Kaptein}
\end{enumerate}

\subsection{Convenience Sampling \cite{statistics/book/Statistics-for-Data-Scientists/Maurits-Kaptein}}\label{Sampling Plans/Non-representative Sampling/Convenience Sampling}

\begin{enumerate}
    \item Convenience sampling collects only units from the population that can be easily obtained.
    \hfill \cite{statistics/book/Statistics-for-Data-Scientists/Maurits-Kaptein}

    \item This may provide a biased sample, as it represents only one small part or time window of the whole processing window for a batch of products. The term \textbf{bias}\label{Sampling Plans/Non-representative Sampling/Convenience Sampling/bias} indicates that we obtain the value of interest with a systematic mistake.
    \hfill \cite{statistics/book/Statistics-for-Data-Scientists/Maurits-Kaptein}

    \item  Convenience sampling is often justified by using the argument of population homogeneity. This insinuates that either the population units are not truly different or the process produces the population of units in random order. 
    \hfill \cite{statistics/book/Statistics-for-Data-Scientists/Maurits-Kaptein}

    
\end{enumerate}


















