\chapter{Random Variables}

\section{Intro}


\begin{enumerate}
    \item A random variable is a numerical outcome of a random process.
    A random variable assigns a number to each possible outcome of a random experiment.
    \hfill \cite{common/online/chatgpt}

    \item An intuitive definition of a random variable or random quantity is a variable for which the value or outcome is unknown and for which the outcome is influenced by some form of random phenomenon.
    \hfill \cite{statistics/book/Statistics-for-Data-Scientists/Maurits-Kaptein}

    \item Why is it called "random"?
    Because the value it takes depends on chance.
    You don’t know the outcome in advance, but you know the possible values and can assign probabilities to them.
    \hfill \cite{common/online/chatgpt}

    \item This allows us to extend our theory on probability to other types of data without restricting it to specific events (i.e., binary data).
    \hfill \cite{statistics/book/Statistics-for-Data-Scientists/Maurits-Kaptein}

    \item The probability sampling approach makes the variable of interest a random variable, as the sampling approach is here the random phenomenon.
    \hfill \cite{statistics/book/Statistics-for-Data-Scientists/Maurits-Kaptein}

    \item A realization or an outcome of the random variable is then indicated by the same, but lower case, letter.
    \hfill \cite{statistics/book/Statistics-for-Data-Scientists/Maurits-Kaptein}

    \item a random variable may be seen as a variable that can in principle be equal to any of the values in the population, and after probability sampling the outcome(s) will become known.
    \hfill \cite{statistics/book/Statistics-for-Data-Scientists/Maurits-Kaptein}

    \item Types of Random Variables:
    \hfill \cite{common/online/chatgpt}
    \begin{enumerate}
        \item Discrete Random Variable:
        \hfill \cite{common/online/chatgpt}
        \begin{enumerate}
            \item Takes countable values (e.g., 0, 1, 2, 3…)
            \hfill \cite{common/online/chatgpt}

            \item Examples: number of heads in 10 coin tosses, number of emails received
            \hfill \cite{common/online/chatgpt}
        \end{enumerate}

        \item Continuous Random Variable
        \hfill \cite{common/online/chatgpt}
        \begin{enumerate}
            \item Takes uncountably many values in an interval (e.g., any real number)
            \hfill \cite{common/online/chatgpt}

            \item Examples: height of a person, temperature, time taken to run a race
            \hfill \cite{common/online/chatgpt}
        \end{enumerate}
    \end{enumerate}

    \item Notation:
    \begin{enumerate}
        \item Random variables are often written as capital letters like $X, Y, Z$
        \hfill \cite{common/online/chatgpt}

        \item Their values are written in lowercase, e.g., $X=x$
        \hfill \cite{common/online/chatgpt}
    \end{enumerate}

    \item The \textbf{probability function} $P(X \leq x)$ is a general concept and can be used for any random variable.
    \hfill \cite{statistics/book/Statistics-for-Data-Scientists/Maurits-Kaptein}

    \item The PMF for a discrete random variable is the equivalent of the PDF for a continuous random variable.
    \hfill \cite{statistics/book/Statistics-for-Data-Scientists/Maurits-Kaptein}

    \item Data points that are drawn independently from the same distribution ($D$ OR $D(\cdots)$) are said to be \textbf{independent and identically distributed}, which is often abbreviated to i.i.d. (denoted by $x \sim D$ OR $x \sim D(\cdots)$)
    \hfill \cite{ml/book/Pattern-Recognition-And-Machine-Learning/Christopher-M-Bishop}
\end{enumerate}





\section{Probability Density Functions (PDF) ($f(x) \neq P(X = x)$)}

\textbf{For Continuous Random Variable}

\begin{enumerate}
    \item The density function may be viewed as a smooth version of the histogram if we standardize the frequencies on the vertical axis to proportions.
    \hfill \cite{statistics/book/Statistics-for-Data-Scientists/Maurits-Kaptein}

    \item The density function characterizes the occurrence of values for a specific variable (as depicted on the x-axis) on all units from the population.
    \hfill \cite{statistics/book/Statistics-for-Data-Scientists/Maurits-Kaptein}

    \item Since in practice all populations are finite, the density function is an abstract formulation of, or a “model” for, the “frequencies” of all population values.
    \hfill \cite{statistics/book/Statistics-for-Data-Scientists/Maurits-Kaptein}

    \item PDF must satisfy two important conditions or properties:
    \hfill \cite{statistics/book/Statistics-for-Data-Scientists/Maurits-Kaptein}
    \begin{enumerate}
        \item  it cannot be negative, i.e., $f (x) \geq 0$ for every value $x$ that is present in the population
        For values of $x$ outside this domain, the PDF can then be defined equal to zero: $f (x) = 0$.
        \hfill \cite{statistics/book/Statistics-for-Data-Scientists/Maurits-Kaptein}

        \item  the “area under the curve” (AUC) is equal to one, i.e., $\dint_{\mbbR} f(x) dx = 1$
        \hfill \cite{statistics/book/Statistics-for-Data-Scientists/Maurits-Kaptein}
    \end{enumerate}

    \item Many different PDFs exist and they have been proposed over a period of more than two centuries to be able to describe populations and data in practical settings.
    \hfill \cite{statistics/book/Statistics-for-Data-Scientists/Maurits-Kaptein}
    \begin{enumerate}
        \item These functions are often parametric functions, i.e., the PDF is known up to a set of parameters.
        \hfill \cite{statistics/book/Statistics-for-Data-Scientists/Maurits-Kaptein}

        \item The PDF is then often denoted by $f_{\bm{\theta}}$ , where $\bm{\theta}$ represents the set or \textbf{vector} of $m$ density parameters:
        $\bm{\theta} =
        \begin{bmatrix}
            \theta_1 & \theta_2 & \cdots &\theta_m
        \end{bmatrix}
        ^\top
        $.
        \hfill \cite{statistics/book/Statistics-for-Data-Scientists/Maurits-Kaptein}
    \end{enumerate}

    \item  $f(x)$ is \textbf{not} be equal to $P(X = x)$
    \hfill \cite{statistics/book/Statistics-for-Data-Scientists/Maurits-Kaptein}
    \begin{enumerate}
        \item The probability $P(X = x)$ is equal to zero for continuous random variables, since there is no surface area under $f (x)$.
        \hfill \cite{statistics/book/Statistics-for-Data-Scientists/Maurits-Kaptein}
    \end{enumerate}
\end{enumerate}





\section{Probability Mass Function (PMF) ($f(x_k) = p_k = P(X = x_k)$)}

\textbf{For Discrete Random Variable}

\begin{enumerate}
    \item Discrete does not always mean that we observe values in $\mathbb{N}$.
    For instance, grades on a data science test may take values in $\dCurlyBrac{1, 1.5, 2.0, 2.5, \cdots , 9.0, 9.5, 10}$.
    Thus, it would be more rigorous to say that a discrete random variable $X$ takes its values in the set $\dCurlyBrac{x_0, x_1, x_2, \cdots , x _k , \cdots}$, with $x _k$ an element of the real line $(x _k \in \mbbR)$ and with an ordering of the values $x_0 < x_1 < x_2 < \cdots$ .
    However, in many practical settings we can map this set to a subset of $\mathbb{N}$ or to the whole set $\mathbb{N}$.
    \hfill \cite{statistics/book/Statistics-for-Data-Scientists/Maurits-Kaptein}

    \item For discrete random variables we can define $p _k = P(X = k)$ as the probability of observing the outcome $k$.
    \hfill \cite{statistics/book/Statistics-for-Data-Scientists/Maurits-Kaptein}

    \item This is referred to as the probability mass function (PMF) if the probabilities $p _k$ satisfy two conditions.
    \hfill \cite{statistics/book/Statistics-for-Data-Scientists/Maurits-Kaptein}
    \begin{enumerate}
        \item all probabilities $p _k$ should be nonnegative ( $p _k \geq 0, \forall\ k$)
        \hfill \cite{statistics/book/Statistics-for-Data-Scientists/Maurits-Kaptein}

        \item probabilities need to add up to one, i.e., $\dsum^{\infty}_{k=0} p _k = 1$
        \hfill \cite{statistics/book/Statistics-for-Data-Scientists/Maurits-Kaptein}
    \end{enumerate}
\end{enumerate}







\section{Cumulative Density Functions (CDF) ($F(x) = P(X \leq x)$)}

\subsection{Intro}

\begin{enumerate}
    \item The probability $P(X \leq x)$ is also referred to as the \textbf{distribution function} obtained in $x$ and it is denoted by $F(x) = P(X \leq x)$.
    Thus every random variable $X$ has a distribution function $F$ through $F(x) = P(X \leq x)$, but also every distribution function $F$ has a random variable $X$, namely the random variable $X$ that makes $P(X \leq x) = F(x)$.
    Thus the two concepts are directly related to each other and we then typically say that $X$ is distributed according to $F$, i.e., \colorbox{yellow}{$X \sim F$}.
    \hfill \cite{statistics/book/Statistics-for-Data-Scientists/Maurits-Kaptein}

    \item Each distribution function $F$ typically satisfies three conditions:
    \begin{enumerate}
        \item When the value $x$ increases to infinity, the distribution function becomes equal to one, i.e., $\lim _{x\to \infty} F(x) = 1$.
        \hfill \cite{statistics/book/Statistics-for-Data-Scientists/Maurits-Kaptein}

        \item When the value $x$ decreases to minus infinity the distribution function becomes equal to zero, i.e., $\lim _{x\to -\infty} F(x) = 0$.
        \hfill \cite{statistics/book/Statistics-for-Data-Scientists/Maurits-Kaptein}

        \item The distribution function is a non-decreasing function, i.e., $F(x_1) \leq F(x_2)$ when $x_1 \leq x_2$.
        \hfill \cite{statistics/book/Statistics-for-Data-Scientists/Maurits-Kaptein}
    \end{enumerate}


\end{enumerate}


\subsection{CDF of PDF (Continuous)}


\begin{enumerate}
    \item There is a direct relation between distribution functions and densities.
    If we start with a PDF, we can define a distribution function in the following way:
    \colorbox{yellow}{$
        F(x)
        = P(X \leq x)
        = \dint_{-\infty}^{x} f(z) dz
    $}
    \hfill \cite{statistics/book/Statistics-for-Data-Scientists/Maurits-Kaptein}

    \item The full distribution function of $\psi(X)$ can always be established, but it does not always have a simple workable form.
    \hfill \cite{statistics/book/Statistics-for-Data-Scientists/Maurits-Kaptein}
    \\[0.2cm]
    $
        P(\psi^{-1}(X) \leq x)
        = P(X \leq \psi(x))
        = F(\psi(x))
        =\dint_{-\infty}^{\psi(x)} f(z) dz
        =\dint_{-\infty}^{x} \psi^{\prime}(z)\ f(\psi(z))\ dz
    $
    \hfill \cite{statistics/book/Statistics-for-Data-Scientists/Maurits-Kaptein}
    \\[0.2cm]
    The calculations now show that the distribution function of random variable $\psi^{-1}(X)$ is equal to $F(\psi(x))$ and the PDF is $\psi^\prime(x)\ f (\psi(x))$.
    \hfill \cite{statistics/book/Statistics-for-Data-Scientists/Maurits-Kaptein}
    \\[0.5cm]
    \textbf{Example}: Let $X$ being normally distributed with parameters $\mu$ and $\sigma$, and consider the function $\psi(x) = \exp\dCurlyBrac{x}$
    \hfill \cite{statistics/book/Statistics-for-Data-Scientists/Maurits-Kaptein}
    \\[0.2cm]
    $
        P(X \leq x)
        = \dint_{-\infty}^{x} \dfrac{1}{\sigma} \phi\dParenBrac{\dfrac{z - \mu}{\sigma}}
        = \dint_{-\infty}^{(x - \mu)/\sigma} \phi(z)\ dz
        = \Phi\dParenBrac{\dfrac{x - \mu}{\sigma}}
    $
    \hfill \cite{statistics/book/Statistics-for-Data-Scientists/Maurits-Kaptein}
    \\[0.2cm]
    $
        \begin{aligned}
            P(\exp\dCurlyBrac{X} \leq x)
                &= P(X \leq \log(x))
                = \Phi\dParenBrac{\dfrac{\log(x) - \mu}{\sigma}} \\
                &= \dint_{-\infty}^{\log(x)} \dfrac{1}{\sigma} \phi\dParenBrac{\dfrac{z - \mu}{\sigma}} dz
                = \dint_{0}^{x} \dfrac{1}{z\sigma} \phi\dParenBrac{\dfrac{\log(z) - \mu}{\sigma}} dz
        \end{aligned}
    $
    \hfill \cite{statistics/book/Statistics-for-Data-Scientists/Maurits-Kaptein}






    \item $P(X < x) = P(X \leq x)$ for continuous random variables

    \item $P(X > x) = 1 - P(X \leq x)$
    \hfill \cite{statistics/book/Statistics-for-Data-Scientists/Maurits-Kaptein}

    \item $P( x_1 < X \leq x_2) = P(X \leq x_2) - P(X \leq x_1)$
    \hfill \cite{statistics/book/Statistics-for-Data-Scientists/Maurits-Kaptein}
\end{enumerate}



\subsection{CDF of PMF (Discrete)}

\begin{enumerate}
    \item The distribution function or cumulative density function (CDF) for a discrete random variable $X$ is now given by \colorbox{yellow}{$F(x) = P(X \leq x) = \dsum^{x}_ {k=0} f (k)$}.
    \hfill \cite{statistics/book/Statistics-for-Data-Scientists/Maurits-Kaptein}

    \item If the set is $\dCurlyBrac{x_0, x_1, x_2, \cdots , x _k , \cdots}$, with $x_0 < x_1 < x_2 < \cdots$ , then the CDF is defined as \colorbox{yellow}{$F(x) =\dsum^{m_x} _{k=0} f (x _k )$}, with $m _x$ the largest value for $k$ that satisfies $x_ k \leq x$.
    \hfill \cite{statistics/book/Statistics-for-Data-Scientists/Maurits-Kaptein}
\end{enumerate}




\section{Expected Values ($E(X)$)}

\subsection{For PDF (Continuous)}

\begin{enumerate}
    \item If we consider a (not necessarily continuous or differentiable) function $\psi : \mbbR \to \mbbR$, then the expected value of the random variable $\psi (X)$ is defined by
    \colorbox{yellow}{$
        \mathbb{E}\psi(X)
        = \mathbb{E}(\psi(X))
        = \dint_{\mbbR} \psi(x)\ f(x)\ dx
    $}
    \hfill \cite{statistics/book/Statistics-for-Data-Scientists/Maurits-Kaptein}
\end{enumerate}

\subsection{For PMF (Discrete)}

\begin{enumerate}
    \item The expectation of a discrete random variable $\psi(X)$ is given by
    \hfill \cite{statistics/book/Statistics-for-Data-Scientists/Maurits-Kaptein}
    \\[0.2cm]
    \colorbox{yellow}{$
        \mathbb{E} (\psi(X)) \
        = \dsum^{\infty}_{k=0}\ \psi(k)\ p _k \
        = \dsum^{\infty}_{k=0}\ \psi(k)\ P(X = k)
    $}
    \hfill \cite{statistics/book/Statistics-for-Data-Scientists/Maurits-Kaptein}

    \item If we would collect many realizations of the discrete random variable $X$, say $N$ realizations, we expect to see value $k$ with frequency $N \cdot p _k$ .
    \hfill \cite{statistics/book/Statistics-for-Data-Scientists/Maurits-Kaptein}
\end{enumerate}

\subsection{Common for PDF \& PMF}

\begin{enumerate}
    \item \textbf{First moment}: mean $\mu$ is obtained by \colorbox{yellow}{$\psi(x)=x$}
    \hfill \cite{statistics/book/Statistics-for-Data-Scientists/Maurits-Kaptein}

    \item \textbf{Second central moment}: The population variance $\sigma^ 2$ is obtained by \colorbox{yellow}{$\psi(x)=(x -\mu)^2$}
    \hfill \cite{statistics/book/Statistics-for-Data-Scientists/Maurits-Kaptein}

    \item \textbf{pth moment} of random variable $X$ is obtained by \colorbox{yellow}{$\psi(x) = x^p$}
    \hfill \cite{statistics/book/Statistics-for-Data-Scientists/Maurits-Kaptein}

    \item \textbf{pth central moment} of random variable $X$ is obtained by choosing \colorbox{yellow}{$\psi(x) = (x - \mu)^p$}
    \hfill \cite{statistics/book/Statistics-for-Data-Scientists/Maurits-Kaptein}

    \item \textbf{skewness}: \colorbox{yellow}{$\gamma_1 = \dfrac{\mu_3}{\sigma^3}$}, where $\mu_3 = \mathbb{E}[(x-\mu)^3]$
    \hfill \cite{statistics/book/Statistics-for-Data-Scientists/Maurits-Kaptein}

    \item \textbf{kurtosis}: \colorbox{yellow}{$\gamma_2 = \dfrac{\mu_4}{\sigma^4} - 3$}, where $\mu_4 = \mathbb{E}[(x-\mu)^4]$
    \hfill \cite{statistics/book/Statistics-for-Data-Scientists/Maurits-Kaptein}
\end{enumerate}

\vspace{0.5cm}
\textbf{Note}:
\begin{enumerate}
    \item the moments of a discrete random variable $X$ may \textbf{not} always exist: this depends on the choice of probabilities $p _k $.
    \hfill \cite{statistics/book/Statistics-for-Data-Scientists/Maurits-Kaptein}
\end{enumerate}


\section{Calculation Rules for Random Variables}

Let we have 2 random variables $X, \ Y$, either discrete or continuous, and a constant $c$.
\\
$F _X$ and $F_Y$ the CDFs for $X$ and $Y$ , respectively. $P(X \leq x, Y \leq y)$ is the joint CDF
of $X$ and $Y$ , denoted by $F _{X Y} (x, y)$
\\
Let events $A = \dCurlyBrac{X \leq x}$ and $B = \dCurlyBrac{Y \leq y}$

\begin{multicols}{2}
\begin{enumerate}[series=calcrulesrv]
    \item $\mathbb{E} (c) = c$
    \hfill \cite{statistics/book/Statistics-for-Data-Scientists/Maurits-Kaptein}

    \item $\mathbb{E} (cX) = c \mathbb{E} (X)$
    \hfill \cite{statistics/book/Statistics-for-Data-Scientists/Maurits-Kaptein}

    \item $\text{Var}(X) \geq 0$
    \hfill \cite{statistics/book/Statistics-for-Data-Scientists/Maurits-Kaptein}

    \item $\text{Var}(X + c) = \text{Var}(X)$
    \hfill \cite{statistics/book/Statistics-for-Data-Scientists/Maurits-Kaptein}

    \item $\text{Var}(cX) = c^2\ \text{Var}(X)$
    \hfill \cite{statistics/book/Statistics-for-Data-Scientists/Maurits-Kaptein}

    \item $\text{Var}(X) = \mathbb{E}(X^2) - [\mathbb{E}(X)]^2$
    \hfill \cite{statistics/book/Statistics-for-Data-Scientists/Maurits-Kaptein}

    \item $\mathbb{E} (X) = \mathbb{E} (Y )$ , when $X = Y$
    \hfill \cite{statistics/book/Statistics-for-Data-Scientists/Maurits-Kaptein}

    \item $\mathbb{E} (X + Y ) = \mathbb{E} (X) + \mathbb{E} (Y )$
    \hfill \cite{statistics/book/Statistics-for-Data-Scientists/Maurits-Kaptein}
\end{enumerate}
\end{multicols}

\textbf{if $X$ and $Y$ are independent:}

\begin{multicols}{2}
\begin{enumerate}[resume*=calcrulesrv]
    \item $\phi(X)$ and $\psi(Y)$ are also independent
    \hfill \cite{statistics/book/Statistics-for-Data-Scientists/Maurits-Kaptein}

    \item $\mathbb{E}(X Y ) = \mathbb{E}(X)\ \mathbb{E}(Y )$
    \hfill \cite{statistics/book/Statistics-for-Data-Scientists/Maurits-Kaptein}

    \item $\text{Var}(X \pm Y ) = \text{Var}(X) + \text{Var}(Y )$
    \hfill \cite{statistics/book/Statistics-for-Data-Scientists/Maurits-Kaptein}

\end{enumerate}
\end{multicols}


\begin{enumerate}[resume*=calcrulesrv]
    \item
    $
        P(X \leq x, Y \leq y)
        = P(A \cap B)
        = P(A) P(B)
        = P(X \leq x) P(Y \leq y)
        = F _X (x)\ F_Y (y)
    $
    \hfill \cite{statistics/book/Statistics-for-Data-Scientists/Maurits-Kaptein}

    \item
    $
        \text{Var}(X Y )
        = \text{Var}(X) \text{Var}(Y)
            + \text{Var}(X)(\mathbb{E}(Y))^2
            + \text{Var}(Y)(\mathbb{E}(X))^2
    $
    \hfill \cite{statistics/book/Statistics-for-Data-Scientists/Maurits-Kaptein}
\end{enumerate}
























