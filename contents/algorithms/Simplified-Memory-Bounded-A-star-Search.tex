\section{Simplified Memory-Bounded A$^\ast$ (SMA$^\ast$) Search \cite{ai/book/Artificial-Intelligence-A-Modern-Approach/Russell-Norvig}}
\label{AI: Algorithms/Simplified Memory-Bounded A* (SMA*) Search}


\begin{enumerate}
    \item SMA$^\ast$ proceeds just like A$^\ast$, expanding the best leaf until memory is full.
    At this point, it cannot add a new node to the search tree without dropping an old one. 
    SMA$^\ast$ always drops the \textbf{worst leaf node}—the one with the highest $f$-value.
    \hfill \cite{ai/book/Artificial-Intelligence-A-Modern-Approach/Russell-Norvig}

    \item  Like RBFS, SMA$^\ast$ then backs up the value of the forgotten node to its parent. 
    In this way, the ancestor of a forgotten sub-tree knows the quality of the best path in that sub-tree. 
    SMA$^\ast$ regenerates the sub-tree only when all other paths have been shown to look worse than the path it has forgotten.
    \hfill \cite{ai/book/Artificial-Intelligence-A-Modern-Approach/Russell-Norvig}

    \item  if all the descendants of a node $n$ are forgotten, then we will not know which way to go from $n$, but we will still have an idea of how worthwhile it is to go anywhere from $n$.
    \hfill \cite{ai/book/Artificial-Intelligence-A-Modern-Approach/Russell-Norvig}

    \item if all the leaf nodes have the same f-value, SMA$^\ast$ expands the \textbf{newest best} leaf and deletes the \textbf{oldest worst} leaf to avoid selecting the same node for deletion and expansion
    \hfill \cite{ai/book/Artificial-Intelligence-A-Modern-Approach/Russell-Norvig}

    \item If the leaf is not a goal node, then even if it is on an optimal solution path, that solution is not reachable with the available memory. 
    Therefore, the node can be discarded exactly as if it had no successors.
    \hfill \cite{ai/book/Artificial-Intelligence-A-Modern-Approach/Russell-Norvig}

    \item In practical terms, SMA$^\ast$ is a fairly robust choice for finding optimal solutions, particularly when the state space is a graph, step costs are not uniform, and node generation is expensive compared to the overhead of maintaining the frontier and the explored set.
    \hfill \cite{ai/book/Artificial-Intelligence-A-Modern-Approach/Russell-Norvig}

    \item \textbf{Disadvantages}:
    \begin{enumerate}
        \item On very hard problems, it will often be the case that SMA$^\ast$ is forced to switch back and forth continually among many candidate solution paths, only a small subset of which can fit in memory.
        \hfill \cite{ai/book/Artificial-Intelligence-A-Modern-Approach/Russell-Norvig}

        \item the extra time required for repeated regeneration of the same nodes means that problems that would be practically solvable by A$^\ast$, given unlimited memory, become intractable for SMA$^\ast$.
        \hfill \cite{ai/book/Artificial-Intelligence-A-Modern-Approach/Russell-Norvig}
    \end{enumerate}
\end{enumerate}







