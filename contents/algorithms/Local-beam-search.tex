\section{Local beam search \cite{ai/book/Artificial-Intelligence-A-Modern-Approach/Russell-Norvig}}
\label{AI: Algorithms/Local beam search}


\begin{enumerate}
    \item The local beam search algorithm keeps track of $k$ states rather than just one.
    \hfill \cite{ai/book/Artificial-Intelligence-A-Modern-Approach/Russell-Norvig}

    \item It begins with $k$ randomly generated states.
    At each step, all the successors of all $k$ states are generated.
    If any one is a goal, the algorithm halts.
    Otherwise, it selects the $k$ best successors from the complete list and repeats.
    The algorithm quickly abandons unfruitful searches and moves its resources to where the most progress is being made.
    \hfill \cite{ai/book/Artificial-Intelligence-A-Modern-Approach/Russell-Norvig}

    \item In a local beam search, useful information is passed among the parallel search threads.
    \hfill \cite{ai/book/Artificial-Intelligence-A-Modern-Approach/Russell-Norvig}

    \item \textbf{Disadvantages}:
    \begin{enumerate}
        \item local beam search can suffer from a lack of diversity among the $k$ states—they can quickly become concentrated in a small region of the state space, making the search little more than an expensive version of hill climbing.
        \hfill \cite{ai/book/Artificial-Intelligence-A-Modern-Approach/Russell-Norvig}
    \end{enumerate}
\end{enumerate}




















