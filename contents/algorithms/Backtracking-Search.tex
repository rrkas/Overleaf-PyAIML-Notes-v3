\section{Backtracking Search \cite{ai/book/Artificial-Intelligence-A-Modern-Approach/Russell-Norvig}}
\label{AI: Algorithms/Backtracking Search}


\begin{enumerate}
    \item A variant of depth-first search that uses still less memory.
    \hfill \cite{ai/book/Artificial-Intelligence-A-Modern-Approach/Russell-Norvig}

    \item \textbf{only one successor} is generated at a time rather than all successors; each partially expanded node remembers which successor to generate next. 
    \hfill \cite{ai/book/Artificial-Intelligence-A-Modern-Approach/Russell-Norvig}

    \item \textbf{Performance}:
    \begin{enumerate}
        \item \textbf{Space Complexity}: $\mathcal{O}(m)$
        \hfill \cite{ai/book/Artificial-Intelligence-A-Modern-Approach/Russell-Norvig}
    \end{enumerate}

    \item \textbf{Advantages}:
    \begin{enumerate}
        \item  Backtracking search facilitates memory-saving (and time-saving) trick: the idea of generating a successor by \textbf{modifying} the current state description directly rather than copying it first.
        For this to work, we must be able to \textbf{undo} each modification when we go back to generate the next successor.
        \hfill \cite{ai/book/Artificial-Intelligence-A-Modern-Approach/Russell-Norvig}

        \item For problems with large state descriptions, such as robotic assembly, these techniques are critical to success.
        \hfill \cite{ai/book/Artificial-Intelligence-A-Modern-Approach/Russell-Norvig}
    \end{enumerate}
\end{enumerate}



\begin{lstlisting}[
    language=Python,
    caption=Problem Solving Agent - Backtracking using recursion \cite{common/online/chatgpt}
]
def backtrack(node: Node, explored: set):
    if problem.goal_test(node.state):
        return solution(node)

    explored.add(node)

    for action in problem.actions(node.state):
        child = child_node(problem, node, action)
        if not any([child.state == n.state for n in explored]):
            result = backtrack(child, explored)
            if result is not None:
                return result
    
    # Optional: allow revisiting for other paths (depends on problem)
    explored.remove(node)  
    return None

def backtracking_search(problem: Problem):
    root = Node(problem.initial_state, None, None, 0)
    return backtrack(root, set())
\end{lstlisting}














