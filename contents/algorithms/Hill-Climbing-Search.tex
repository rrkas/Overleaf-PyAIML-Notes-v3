\section{Hill Climbing Search/ greedy local search}


\begin{enumerate}[itemsep=0.2cm]
    \item It is simply a loop that continually moves in the direction of increasing value—that is, uphill. 
    It terminates when it reaches a “peak” where no neighbor has a higher value.
    \hfill \cite{ai/book/Artificial-Intelligence-A-Modern-Approach/Russell-Norvig}

    \item The algorithm does not maintain a search tree, so the data structure for the current node need only record the state and the value of the objective function. 
    \hfill \cite{ai/book/Artificial-Intelligence-A-Modern-Approach/Russell-Norvig}

    \item Hill climbing does not look ahead beyond the immediate neighbors of the current state. This resembles trying to find the top of Mount Everest in a thick fog while suffering from amnesia.
    \hfill \cite{ai/book/Artificial-Intelligence-A-Modern-Approach/Russell-Norvig}

    \item 
    \hfill \cite{ai/book/Artificial-Intelligence-A-Modern-Approach/Russell-Norvig}
\end{enumerate}

\vspace{0.5cm}

\begin{algorithm}[H]
    \caption{
        The hill-climbing search algorithm (\textbf{steepest-ascent} version), which is the most basic local search technique. 
        At each step the current node is replaced by the best neighbor; in this version, that means the neighbor with the highest \textsc{Value}, but if a heuristic cost estimate $h$ is used, we would find the neighbor with the lowest $h$. 
        \cite{ai/book/Artificial-Intelligence-A-Modern-Approach/Russell-Norvig}}

    \SetKwFunction{FUNCTION}{\textsc{Hill-Climbing-Search}}
    \SetKwProg{Fn}{function}{ returns \normalfont{a state that is a local maximum}}{end}
    \Fn{\FUNCTION{problem}}{
        $curent \gets$ \textsc{Make-Node}($problem$.\textsc{Initial-State}) \\
        \ \\
        \While{}{
            $neighbor \gets$ a highest-valued successor of current \\
            \lIf{$neighbor$.\textsc{Value} $\leq$ $current$.\textsc{Value}}{
                \Return $current$.\textsc{State}
            }
            $current \gets neighbor$
        }
    }
\end{algorithm}















