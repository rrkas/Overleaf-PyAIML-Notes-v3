\chapter{Machine Learning: Introduction}

\begin{customArrayStretch}{1.5}
\begin{longtable}{l p{7cm}}

\hline\endfirsthead
\hline\endhead
\hline\endfoot
\hline
\caption*{Notations} \\
\endlastfoot

$N$ & number of training records\\ \hline

$\dCurlyBrac{\bm{x}_1, \cdots , \bm{x}_N }$ & input records \\ \hline

$\dCurlyBrac{\bm{y}_1, \cdots , \bm{y}_N }$ & (ground truth/ reference) outputs \\ \hline

$\bm{y}(\cdot)$ & model/ function \\ \hline

$\hat{\bm{y}}_i = \bm{y}(\bm{x_i})$ & (hypothesis) output of $i$-th input \\



\end{longtable}
\end{customArrayStretch}





\section{Training/ Learning Phase}

\begin{enumerate}
    \item \textbf{Training set}: large set of $N$ records $\dCurlyBrac{\bm{x}_1, \cdots , \bm{x}_N }$ that can be used to tune the parameters of an adaptive model. 
    \hfill \cite{ml/book/Pattern-Recognition-And-Machine-Learning/Christopher-M-Bishop}

    \item The result of running the machine learning algorithm can be expressed as a function $\bm{y}(\bm{x}_i)$ which takes a new input record $\bm{x}$ as input and that generates an output vector $\bm{y}_i$, encoded in the same way as the target vectors.
    The precise form of the function $\bm{y}(\cdot)$ is determined during the training phase, also known as the learning phase, on the basis of the training data.
    \hfill \cite{ml/book/Pattern-Recognition-And-Machine-Learning/Christopher-M-Bishop}
\end{enumerate}



\section{Testing/ Evaluation Phase}

\begin{enumerate}
    \item 
\end{enumerate}










