\chapter{Machine Learning: Introduction}

\begin{customArrayStretch}{1.5}
\begin{longtable}{l p{7cm}}

\hline\endfirsthead
\hline\endhead
\hline\endfoot
\hline
\caption*{Notations} \\
\endlastfoot

$N$ & number of training records\\ \hline

$\dCurlyBrac{\bm{x}_1, \cdots , \bm{x}_N }$ & input records \\ \hline

$\dCurlyBrac{\bm{y}_1, \cdots , \bm{y}_N }$ & (ground truth/ reference) outputs \\ \hline

$\bm{y}(\cdot)$ & model/ function \\ \hline

$\hat{\bm{y}}_i = \bm{y}(\bm{x_i})$ & (hypothesis) output of $i$-th input \\



\end{longtable}
\end{customArrayStretch}


\section{Intro}

\begin{enumerate}
    \item In practical applications, the variability of the input vectors will be such that the training data can comprise only a tiny fraction of all possible input vectors, and so generalization is a central goal in pattern recognition.
    \hfill \cite{ml/book/Pattern-Recognition-And-Machine-Learning/Christopher-M-Bishop}
\end{enumerate}





\section{Pre-processing: Feature Extraction}

\begin{enumerate}
    \item For most practical applications, the original input variables are typically preprocessed to transform them into some new space of variables where, it is hoped, the pattern recognition problem will be easier to solve.
    \hfill \cite{ml/book/Pattern-Recognition-And-Machine-Learning/Christopher-M-Bishop}

    \item Pre-processing might also be performed in order to speed up computation.
    \hfill \cite{ml/book/Pattern-Recognition-And-Machine-Learning/Christopher-M-Bishop}

    \item Care must be taken during pre-processing because often information is discarded, and if this information is important to the solution of the problem then the overall accuracy of the system can suffer.
    \hfill \cite{ml/book/Pattern-Recognition-And-Machine-Learning/Christopher-M-Bishop}
\end{enumerate}




\section{Training/ Learning Phase}

\begin{enumerate}
    \item \textbf{Training set}: large set of $N$ records $\dCurlyBrac{\bm{x}_1, \cdots , \bm{x}_N }$ that can be used to tune the parameters of an adaptive model. 
    \hfill \cite{ml/book/Pattern-Recognition-And-Machine-Learning/Christopher-M-Bishop}

    \item The result of running the machine learning algorithm can be expressed as a function $\bm{y}(\bm{x}_i)$ which takes a new input record $\bm{x}$ as input and that generates an output vector $\bm{y}_i$, encoded in the same way as the target vectors.
    The precise form of the function $\bm{y}(\cdot)$ is determined during the training phase, also known as the learning phase, on the basis of the \textbf{training data}.
    \hfill \cite{ml/book/Pattern-Recognition-And-Machine-Learning/Christopher-M-Bishop}
\end{enumerate}



\section{Testing/ Evaluation Phase}

\begin{enumerate}
    \item Once the model is trained it can then determine the identity of new digit images, which are said to comprise a \textbf{test set}. 
    \hfill \cite{ml/book/Pattern-Recognition-And-Machine-Learning/Christopher-M-Bishop}

    \item The ability to categorize correctly new examples that differ from those used for training is known as \textbf{generalization}. 
    \hfill \cite{ml/book/Pattern-Recognition-And-Machine-Learning/Christopher-M-Bishop}
\end{enumerate}





\section{Supervised learning}

\begin{enumerate}
    \item Applications in which the training data comprises examples of the input vectors along with their corresponding target vectors are known as \textbf{supervised learning} problems. 
    \hfill \cite{ml/book/Pattern-Recognition-And-Machine-Learning/Christopher-M-Bishop}
\end{enumerate}



\subsection{Regression}

\begin{enumerate}
    \item Cases in which the desired output consists of one or more continuous variables, then the task is called \textbf{regression}. 
    \hfill \cite{ml/book/Pattern-Recognition-And-Machine-Learning/Christopher-M-Bishop}
\end{enumerate}


\subsection{Classification}

\begin{enumerate}
    \item Cases in which the aim is to assign each input vector to one of a finite number of discrete categories, are called \textbf{classification} problems.
    \hfill \cite{ml/book/Pattern-Recognition-And-Machine-Learning/Christopher-M-Bishop}
\end{enumerate}





\section{Unsupervised learning}

\begin{enumerate}
    \item the training data consists of a set of input vectors $\bm{x}$ without any corresponding target values
    \hfill \cite{ml/book/Pattern-Recognition-And-Machine-Learning/Christopher-M-Bishop}
\end{enumerate}

\subsection{Clustering}

\begin{enumerate}
    \item Cases to discover groups of similar examples within the data
    \hfill \cite{ml/book/Pattern-Recognition-And-Machine-Learning/Christopher-M-Bishop}
\end{enumerate}


\subsection{Density Estimation}

\begin{enumerate}
    \item Cases to determine the distribution of data within the input space
    \hfill \cite{ml/book/Pattern-Recognition-And-Machine-Learning/Christopher-M-Bishop}
\end{enumerate}


\subsection{Visualization}

\begin{enumerate}
    \item Cases to project the data from a high-dimensional space down to two or three dimensions 
    \hfill \cite{ml/book/Pattern-Recognition-And-Machine-Learning/Christopher-M-Bishop}
\end{enumerate}


\section{Reinforcement Learning}

\begin{enumerate}
    \item It is concerned with the problem of finding suitable actions to take in a given situation in order to maximize a reward. 
    \hfill \cite{ml/book/Pattern-Recognition-And-Machine-Learning/Christopher-M-Bishop}
    
    \item Here the learning algorithm is not given examples of optimal outputs, in contrast to supervised learning, but must instead discover them by a process of trial and error. 
    \hfill \cite{ml/book/Pattern-Recognition-And-Machine-Learning/Christopher-M-Bishop}
    
    \item Typically there is a sequence of states and actions in which the learning algorithm is interacting with its environment. 
    \hfill \cite{ml/book/Pattern-Recognition-And-Machine-Learning/Christopher-M-Bishop}
    
    \item In many cases, the current action not only affects the immediate reward but also has an impact on the reward at all subsequent time steps. 
    \hfill \cite{ml/book/Pattern-Recognition-And-Machine-Learning/Christopher-M-Bishop}

    \item \textbf{Credit assignment problem}: 
    \hfill \cite{ml/book/Pattern-Recognition-And-Machine-Learning/Christopher-M-Bishop}
    \begin{enumerate}
        \item A major challenge is that a case can involve dozens of moves, and yet it is only at the end of the case that the reward, in the form of victory, is achieved. 
        \hfill \cite{ml/book/Pattern-Recognition-And-Machine-Learning/Christopher-M-Bishop}
        
        \item The reward must then be attributed appropriately to all of the moves that led to it, even though some moves will have been good ones and others less so. 
        \hfill \cite{ml/book/Pattern-Recognition-And-Machine-Learning/Christopher-M-Bishop}
    \end{enumerate}

    \item A general feature of reinforcement learning is the trade-off between \\
    \textbf{exploration}, in which the system tries out new kinds of actions to see how effective they are, and \\
    \textbf{exploitation}, in which the system makes use of actions that are known to yield a high reward. 
    \hfill \cite{ml/book/Pattern-Recognition-And-Machine-Learning/Christopher-M-Bishop}
    
    \item Too strong a focus on either exploration or exploitation will yield poor results.
    \hfill \cite{ml/book/Pattern-Recognition-And-Machine-Learning/Christopher-M-Bishop}
\end{enumerate}


















































