\chapter{Artificial Intelligence: Introduction}\label{Artificial Intelligence: Introduction}

\begin{enumerate}
    \item The field of artificial intelligence, or AI, attempts not just to \textbf{understand} but also to \textbf{build} intelligent entities.
    \hfill \cite{ai/book/Artificial-Intelligence-A-Modern-Approach/Russell-Norvig}

    \item AI currently encompasses a huge variety of subfields, ranging from the general (learning and perception) to the specific, such as playing chess, proving mathematical theorems, writing poetry, driving a car on a crowded street, and diagnosing diseases.    
    \hfill \cite{ai/book/Artificial-Intelligence-A-Modern-Approach/Russell-Norvig}

    \item AI is relevant to any intellectual task; it is truly a universal field.
    \hfill \cite{ai/book/Artificial-Intelligence-A-Modern-Approach/Russell-Norvig}

    \item A system is \textbf{rational} if it does the “right thing,” given what it knows.
    \hfill \cite{ai/book/Artificial-Intelligence-A-Modern-Approach/Russell-Norvig}
\end{enumerate}






\section{Approaches to AI}\label{Artificial Intelligence: Introduction/Approaches to AI}

\subsection{Acting humanly: The Turing Test approach}\label{Artificial Intelligence: Introduction/Approaches to AI/Acting humanly: The Turing Test approach}


\begin{enumerate}[itemsep=0.2cm]
    \item The \textbf{Turing Test}\label{Artificial Intelligence: Introduction/Approaches to AI/Acting humanly: The Turing Test approach/Turing Test}, proposed by \textbf{Alan Turing} (1950), was designed to provide a satisfactory operational definition of intelligence.
    \hfill \cite{ai/book/Artificial-Intelligence-A-Modern-Approach/Russell-Norvig}

    \item A computer passes the test if a human interrogator, after posing some written questions, cannot tell whether the written responses come from a person or from a computer. 
    \hfill \cite{ai/book/Artificial-Intelligence-A-Modern-Approach/Russell-Norvig}

    \item The computer would need to possess the following capabilities:
    \hfill \cite{ai/book/Artificial-Intelligence-A-Modern-Approach/Russell-Norvig}
    \begin{enumerate}
        \item \textbf{Natural Language Processing} to enable it to communicate successfully in English

        \item \textbf{Knowledge Representation} to store what it knows or hears

        \item \textbf{Automated Reasoning} to use the stored information to answer questions and to draw new conclusions

        \item \textbf{Machine Learning} to adapt to new circumstances and to detect and extrapolate patterns

    \end{enumerate}

    \item Turing’s test deliberately \textit{avoided direct physical} interaction between the interrogator and the computer, because physical simulation of a person is unnecessary for intelligence.
    \hfill \cite{ai/book/Artificial-Intelligence-A-Modern-Approach/Russell-Norvig}

    \item \textbf{Total Turing Test}\label{Artificial Intelligence: Introduction/Approaches to AI/Acting humanly: The Turing Test approach/Total Turing Test} includes a video signal so that the interrogator can test the subject’s perceptual abilities, as well as the opportunity for the interrogator to pass physical objects “through the hatch”. To pass the total Turing Test, the computer will additionally need:
    \hfill \cite{ai/book/Artificial-Intelligence-A-Modern-Approach/Russell-Norvig}
    \begin{enumerate}
        \item \textbf{computer vision} to perceive objects

        \item \textbf{robotics} to manipulate objects and move about
    \end{enumerate}
\end{enumerate}









\subsection{Thinking humanly: The cognitive modeling approach}\label{Artificial Intelligence: Introduction/Approaches to AI/Thinking humanly: The cognitive modeling approach}

\begin{enumerate}[itemsep=0.2cm]
    \item Knowing the actual workings of human minds:
    \begin{enumerate}
        \item \textbf{through introspection}: trying to catch our own thoughts as they go by
        \hfill \cite{ai/book/Artificial-Intelligence-A-Modern-Approach/Russell-Norvig}

        \item \textbf{through psychological experiments}: observing a person in action
        \hfill \cite{ai/book/Artificial-Intelligence-A-Modern-Approach/Russell-Norvig}

        \item \textbf{through brain imaging}: observing the brain in action
        \hfill \cite{ai/book/Artificial-Intelligence-A-Modern-Approach/Russell-Norvig}
    \end{enumerate}

    \item  If the program’s input–output behavior matches corresponding human behavior, that is evidence that some of the program’s mechanisms could also be operating in humans.
    \hfill \cite{ai/book/Artificial-Intelligence-A-Modern-Approach/Russell-Norvig}

    \item The interdisciplinary field of \textbf{cognitive science}\label{Artificial Intelligence: Introduction/Approaches to AI/Thinking humanly: The cognitive modeling approach/cognitive science} brings together computer models from AI and experimental techniques from psychology to construct precise and testable theories of the human mind.
    \hfill \cite{ai/book/Artificial-Intelligence-A-Modern-Approach/Russell-Norvig}
\end{enumerate}







\subsection{Thinking rationally: The “laws of thought” approach}\label{Artificial Intelligence Introduction/Approaches to AI/Thinking rationally The laws of thought approach}

\begin{enumerate}[itemsep=0.2cm]
    \item \textbf{Syllogisms}\label{Artificial Intelligence Introduction/Approaches to AI/Thinking rationally The laws of thought approach/Syllogisms}: It is a kind of logical argument that applies deductive reasoning to arrive at a conclusion based on two propositions that are asserted or assumed to be true.
    \hfill \cite{ai/online/wiki/Syllogism}
    \\
    \textbf{Example}: All men are mortal; Socrates is a man; Therefore, Socrates is mortal.
    \hfill \cite{ai/online/wiki/Syllogism}

    \item \textbf{Logic}\label{Artificial Intelligence: Introduction/Approaches to AI/Thinking rationally: The laws of thought approach/Logic}: Logic is the study of correct reasoning.
    \hfill \cite{ai/online/wiki/Logic}

    \item Emphasis is on correct inferences.
    \hfill \cite{ai/book/Artificial-Intelligence-A-Modern-Approach/Russell-Norvig}

    \item \textbf{Challenges} with logicist approach:
    \begin{enumerate}
        \item it is not easy to take informal knowledge and state it in the formal terms required by logical notation, particularly when the knowledge is less than $100\%$ certain. 
        \hfill\cite{ai/book/Artificial-Intelligence-A-Modern-Approach/Russell-Norvig}

        \item there is a big difference between solving a problem “in principle” and solving it in practice. Even problems with just a few hundred facts can exhaust the computational resources of any computer unless it has some guidance as to which reasoning steps to try first. 
        \hfill\cite{ai/book/Artificial-Intelligence-A-Modern-Approach/Russell-Norvig}
    \end{enumerate}
\end{enumerate}







\subsection{Acting rationally: The rational agent approach}\label{Artificial Intelligence Introduction/Approaches to AI/Acting rationally: The rational agent approach}

\begin{enumerate}[itemsep=0.2cm]
    \item \textbf{Agent}\label{Artificial Intelligence Introduction/Approaches to AI/Acting rationally: The rational agent approach/Agent}: An agent is just something that acts: operate autonomously, perceive their environment, persist over a prolonged time period, adapt to change, and create and pursue goals. 
    \hfill \cite{ai/book/Artificial-Intelligence-A-Modern-Approach/Russell-Norvig}

    \item \textbf{Rational Agent}\label{Artificial Intelligence Introduction/Approaches to AI/Acting rationally: The rational agent approach/Rational Agent}: A rational agent is one that acts so as to achieve the best outcome or, when there is uncertainty, the best expected outcome.

    \item \textbf{Limited Rationality}\label{Artificial Intelligence Introduction/Approaches to AI/Acting rationally: The rational agent approach/Limited Rationality}: acting appropriately when there is not enough time to do all the computations one might like.
    \hfill \cite{ai/book/Artificial-Intelligence-A-Modern-Approach/Russell-Norvig}

    \item Making correct inferences is sometimes part of being a rational agent, because one way to act rationally is to reason logically to the conclusion that a given action will achieve one’s goals and then to act on that conclusion.
    \hfill \cite{ai/book/Artificial-Intelligence-A-Modern-Approach/Russell-Norvig}

    \item Correct inference is \textbf{not all} of rationality; in some situations, there is no provably correct thing to do, but something must still be done.
    \hfill \cite{ai/book/Artificial-Intelligence-A-Modern-Approach/Russell-Norvig}

    \item There are also ways of acting rationally that cannot be said to involve inference.
    \hfill \cite{ai/book/Artificial-Intelligence-A-Modern-Approach/Russell-Norvig}
    \\
    \textbf{Example}: recoiling from a hot stove is a \textbf{reflex action} that is usually more successful than a slower action taken after careful deliberation.
    \hfill \cite{ai/book/Artificial-Intelligence-A-Modern-Approach/Russell-Norvig}

    \item All the skills needed for the Turing Test also allow an agent to act rationally.
    \hfill \cite{ai/book/Artificial-Intelligence-A-Modern-Approach/Russell-Norvig}

    \item \textbf{Knowledge representation} and \textbf{reasoning} enable agents to reach good decisions. 
    \hfill \cite{ai/book/Artificial-Intelligence-A-Modern-Approach/Russell-Norvig}

    \item We need learning not only for erudition, but also because it improves our ability to generate effective behavior.
    \hfill \cite{ai/book/Artificial-Intelligence-A-Modern-Approach/Russell-Norvig}

    \item The standard of rationality is mathematically well defined and completely general, and can be “unpacked” to generate agent designs that provably achieve it. 
    \hfill \cite{ai/book/Artificial-Intelligence-A-Modern-Approach/Russell-Norvig}
    \\
    Human behavior, on the other hand, is well adapted for one specific environment and is defined by, well, the sum total of all the things that humans do.
    \hfill \cite{ai/book/Artificial-Intelligence-A-Modern-Approach/Russell-Norvig}

    \item The rational-agent approach has two \textbf{advantages} over the other approaches:
    \begin{enumerate}
        \item it is more general than the “laws of thought” approach because correct inference is just one of several possible mechanisms for achieving rationality
        \hfill \cite{ai/book/Artificial-Intelligence-A-Modern-Approach/Russell-Norvig}

        \item it is more amenable to scientific development than are approaches based on human behavior or human thought.
        \hfill \cite{ai/book/Artificial-Intelligence-A-Modern-Approach/Russell-Norvig}
    \end{enumerate}

    \item Achieving \textit{perfect rationality} - always doing the right thing - is \textbf{not feasible} in complicated environments.
    \hfill \cite{ai/book/Artificial-Intelligence-A-Modern-Approach/Russell-Norvig}
\end{enumerate}













\section{AI: Disciplines}\label{Artificial Intelligence: Introduction/AI: Disciplines}

\subsection{Philosophy}

\textbf{Questions}
\begin{enumerate}[itemsep=0.1cm]
    \item Can formal rules be used to draw valid conclusions?
    \hfill \cite{ai/book/Artificial-Intelligence-A-Modern-Approach/Russell-Norvig}
    
    \item How does the mind arise from a physical brain?
    \hfill \cite{ai/book/Artificial-Intelligence-A-Modern-Approach/Russell-Norvig}
    
    \item Where does knowledge come from?
    \hfill \cite{ai/book/Artificial-Intelligence-A-Modern-Approach/Russell-Norvig}
    
    \item How does knowledge lead to action?
    \hfill \cite{ai/book/Artificial-Intelligence-A-Modern-Approach/Russell-Norvig}

\end{enumerate}

\vspace{1cm}

\textbf{Notes}
\begin{enumerate}[itemsep=0.2cm]

    \item It’s one thing to say that the mind operates, at least in part, according to logical rules, and to build physical systems that emulate some of those rules; it’s another to say that the mind itself is such a physical system.
    \hfill \cite{ai/book/Artificial-Intelligence-A-Modern-Approach/Russell-Norvig}

    \item One problem with a purely physical conception of the mind is that it seems to leave little room for free will: if the mind is governed entirely by physical laws, then it has no more free will than a rock “deciding” to fall toward the center of the earth.
    \hfill \cite{ai/book/Artificial-Intelligence-A-Modern-Approach/Russell-Norvig}

    \item \textbf{Rationalism}\label{Artificial Intelligence: Introduction/AI: Disciplines/Rationalism}:
    \textit{Descartes} was a strong advocate of the power of reasoning in understanding the world
    \hfill \cite{ai/book/Artificial-Intelligence-A-Modern-Approach/Russell-Norvig}

    \item \textbf{Dualism}\label{Artificial Intelligence: Introduction/AI: Disciplines/Dualism}:
    There is a part of the human mind (or soul or spirit) that is outside of nature, exempt from physical laws. Animals, on the other hand, did not possess this dual quality; they could be treated as machines.
    \hfill \cite{ai/book/Artificial-Intelligence-A-Modern-Approach/Russell-Norvig}
    

    \item \textbf{Materialism}\label{Artificial Intelligence: Introduction/AI: Disciplines/materialism}:
    It holds that the brain’s operation according to the laws of physics constitutes the mind. Free will is simply the way that the perception of available choices appears to the choosing entity.
    \hfill \cite{ai/book/Artificial-Intelligence-A-Modern-Approach/Russell-Norvig}

    \item \textbf{Empiricism Movement}: The empiricism movement, starting with \textbf{Francis Bacon}’s (1561–1626) \textit{Novum Organum}, is characterized by a dictum of \textbf{John Locke} (1632–1704): “\textit{Nothing is in the understanding, which was not first in the senses.}”
    \hfill \cite{ai/book/Artificial-Intelligence-A-Modern-Approach/Russell-Norvig}

    \item  \textbf{Principle of Induction}: general rules are acquired by exposure to repeated associations between their elements.
    \hfill \cite{ai/book/Artificial-Intelligence-A-Modern-Approach/Russell-Norvig}

    \item \textbf{Logical Positivism}: This doctrine holds that all knowledge can be characterized by logical theories connected, ultimately, to observation sentences that correspond to sensory inputs; thus logical positivism combines rationalism and empiricism.
    \hfill \cite{ai/book/Artificial-Intelligence-A-Modern-Approach/Russell-Norvig}

    
    \item \textbf{Confirmation Theory}: The confirmation theory of Carnap and Carl Hempel (1905–1997) attempted to analyze the acquisition of knowledge from experience. 
    \hfill \cite{ai/book/Artificial-Intelligence-A-Modern-Approach/Russell-Norvig}
\end{enumerate}


\subsection{Mathematics}

\textbf{Questions}
\begin{enumerate}[itemsep=0.1cm]
    \item What are the formal rules to draw valid conclusions?
    \hfill \cite{ai/book/Artificial-Intelligence-A-Modern-Approach/Russell-Norvig}
    
    \item What can be computed?
    \hfill \cite{ai/book/Artificial-Intelligence-A-Modern-Approach/Russell-Norvig}
    
    \item How do we reason with uncertain information?
    \hfill \cite{ai/book/Artificial-Intelligence-A-Modern-Approach/Russell-Norvig}
    
\end{enumerate}

\vspace{0.5cm}

\textbf{Parts}: Logic, Computation, Probability
\hfill \cite{ai/book/Artificial-Intelligence-A-Modern-Approach/Russell-Norvig}

\vspace{0.5cm}

\textbf{Notes}
\begin{enumerate}[itemsep=0.2cm]
    \item The idea of formal logic can be traced back to the philosophers of ancient Greece, but its mathematical development really began with the work of \textbf{George Boole} (1815–1864), who worked out the details of propositional, or Boolean, logic (Boole, 1847).
    \hfill \cite{ai/book/Artificial-Intelligence-A-Modern-Approach/Russell-Norvig}

    \item In 1879, \textbf{Gottlob Frege} (1848–1925) extended Boole’s logic to include objects and relations, creating the first-order logic that is used today.
    \hfill \cite{ai/book/Artificial-Intelligence-A-Modern-Approach/Russell-Norvig}
    
    \item \textbf{Alfred Tarski} (1902–1983) introduced a theory of reference that shows how to relate the objects in a logic to objects in the real world.
    \hfill \cite{ai/book/Artificial-Intelligence-A-Modern-Approach/Russell-Norvig}

    
\end{enumerate}




\subsection{Economics}

\textbf{Questions}
\begin{enumerate}[itemsep=0.1cm]
    \item How should we make decisions so as to maximize payoff?
    \hfill \cite{ai/book/Artificial-Intelligence-A-Modern-Approach/Russell-Norvig}

    \item How should we do this when others may not go along?
    \hfill \cite{ai/book/Artificial-Intelligence-A-Modern-Approach/Russell-Norvig}

    \item How should we do this when the payoff may be far in the future?
    \hfill \cite{ai/book/Artificial-Intelligence-A-Modern-Approach/Russell-Norvig}

\end{enumerate}




\subsection{Neuroscience}

\textbf{Questions}
\begin{enumerate}[itemsep=0.1cm]
    \item How do brains process information?
    \hfill \cite{ai/book/Artificial-Intelligence-A-Modern-Approach/Russell-Norvig}

\end{enumerate}



\subsection{Psychology}

\textbf{Questions}
\begin{enumerate}[itemsep=0.1cm]
    \item How do humans and animals think and act?
    \hfill \cite{ai/book/Artificial-Intelligence-A-Modern-Approach/Russell-Norvig}

\end{enumerate}




\subsection{Computer engineering}

\textbf{Questions}
\begin{enumerate}[itemsep=0.1cm]
    \item How can we build an efficient computer?
    \hfill \cite{ai/book/Artificial-Intelligence-A-Modern-Approach/Russell-Norvig}

\end{enumerate}




\subsection{Control theory and cybernetics}

\textbf{Questions}
\begin{enumerate}[itemsep=0.1cm]
    \item How can artifacts operate under their own control?
    \hfill \cite{ai/book/Artificial-Intelligence-A-Modern-Approach/Russell-Norvig}

\end{enumerate}




\subsection{Linguistics}

\textbf{Questions}
\begin{enumerate}[itemsep=0.1cm]
    \item How does language relate to thought?
    \hfill \cite{ai/book/Artificial-Intelligence-A-Modern-Approach/Russell-Norvig}

\end{enumerate}




















\clearpage
\section{AI: History}\label{Artificial Intelligence: Introduction/AI: History}

\begin{enumerate}
    \item Minsky supervised a series of students who chose limited problems that appeared to require intelligence to solve. These limited domains became known as \textbf{microworlds}. 
\end{enumerate}


\newcommand{\customTimeline}[1]{
    {
        \fontsize{10}{10}\selectfont
        \bfseries
        \textsc{#1}
    }
}

\begin{customArrayStretch}{1.3}
\begin{longtable}{ 
    p{2.5cm} 
    p{11.5cm} 
    >{\RaggedLeft\arraybackslash}p{1.3cm} 
}

\hhline{=:=:=}
\textbf{Date/ Time} $\uparrow$ & \textbf{Events} & \textbf{Ref(s)} \\ \hhline{=:=:=}
\endfirsthead

\hhline{=:=:=}
\textbf{Date/ Time} $\uparrow$ & \textbf{Events} & \textbf{Ref(s)} \\ \hhline{=:=:=}
\endhead

\hhline{=:=:=} \endfoot
\hhline{=:=:=} \endlastfoot

%%%%%%%%%%%%%%%%%%%%%%%%%%%%%%%%%%%%%%%%%%%%%%%%%%%%%%%%%%%%%%%%%%%%%%%%%%%%%%%%%%%%%%%%%%%



%%%%%%%%%%%%%%%%%%%%%%%%%%%%%%%%%%%%%%%%%%%%%%%%%%%%%%%%%%%%%%%%%%%%%%%%%%%%%%%%%%%%%%%%%%%
%                                  400-300 BC
%%%%%%%%%%%%%%%%%%%%%%%%%%%%%%%%%%%%%%%%%%%%%%%%%%%%%%%%%%%%%%%%%%%%%%%%%%%%%%%%%%%%%%%%%%%


\customTimeline{384–322 B.C.} & 
    \textbf{Aristotle} formulate a precise set of laws governing the rational part of the mind. He developed an informal system of syllogisms for proper reasoning, which in principle allowed one to generate conclusions mechanically, given initial premises.  &
    \cite{ai/book/Artificial-Intelligence-A-Modern-Approach/Russell-Norvig} \\ \hline


\customTimeline{350 BC} &
    The Greek philosopher \textbf{Aristotle} was one of the first to attempt to codify “right thinking,” that is, irrefutable reasoning processes. &
    \cite{ai/book/Artificial-Intelligence-A-Modern-Approach/Russell-Norvig} \\ \hline


%%%%%%%%%%%%%%%%%%%%%%%%%%%%%%%%%%%%%%%%%%%%%%%%%%%%%%%%%%%%%%%%%%%%%%%%%%%%%%%%%%%%%%%%%%%
%                                  1300-1900
%%%%%%%%%%%%%%%%%%%%%%%%%%%%%%%%%%%%%%%%%%%%%%%%%%%%%%%%%%%%%%%%%%%%%%%%%%%%%%%%%%%%%%%%%%%


\customTimeline{1315} & 
    \textbf{Ramon Lull} (d. 1315) had the idea that useful reasoning could actually be carried out by a mechanical artifact. &
    \cite{ai/book/Artificial-Intelligence-A-Modern-Approach/Russell-Norvig} \\ \hline

\customTimeline{1452-1519} &
    \textbf{Leonardo da Vinci} (1452–1519) designed but did not build a mechanical calculator; recent reconstructions have shown the design to be functional. &
    \cite{ai/book/Artificial-Intelligence-A-Modern-Approach/Russell-Norvig} \\ \hline


\customTimeline{1588-1679} & 
    \textbf{Thomas Hobbes} (1588–1679) proposed that reasoning was like numerical computation, that “we add and subtract in our silent thoughts.” &
    \cite{ai/book/Artificial-Intelligence-A-Modern-Approach/Russell-Norvig} \\ \hline


\customTimeline{1596–1650} &
    \textbf{Rene Descartes} (1596–1650) gave the first clear discussion of the distinction between mind and matter and of the problems that arise. &
    \cite{ai/book/Artificial-Intelligence-A-Modern-Approach/Russell-Norvig} \\ \hline


\customTimeline{1623} & 
    The first known calculating machine was constructed around 1623 by the German scientist \textbf{Wilhelm Schickard} (1592–1635) &
    \cite{ai/book/Artificial-Intelligence-A-Modern-Approach/Russell-Norvig} \\ \hline


\customTimeline{1642} & 
    \textit{Pascaline}, built in 1642 by \textbf{Blaise Pascal} (1623–1662), is more famous. Pascal wrote that “the arithmetical machine produces effects which appear nearer to thought than all the actions of animals.” Pascaline could only add and subtract.  &
    \cite{ai/book/Artificial-Intelligence-A-Modern-Approach/Russell-Norvig} \\ \hline


\customTimeline{1646–1716} &
    \textbf{Gottfried Wilhelm Leibniz} (1646–1716) built a mechanical device intended to carry out operations on concepts rather than numbers, but its scope was rather limited. Leibniz did surpass Pascal by building a calculator that could add, subtract, multiply, and take roots. &
    \cite{ai/book/Artificial-Intelligence-A-Modern-Approach/Russell-Norvig} \\ \hline


%%%%%%%%%%%%%%%%%%%%%%%%%%%%%%%%%%%%%%%%%%%%%%%%%%%%%%%%%%%%%%%%%%%%%%%%%%%%%%%%%%%%%%%%%%%
%                                  1900-2000
%%%%%%%%%%%%%%%%%%%%%%%%%%%%%%%%%%%%%%%%%%%%%%%%%%%%%%%%%%%%%%%%%%%%%%%%%%%%%%%%%%%%%%%%%%%

\customTimeline{1939-1945} & 
    Work in AI started after World War II &
    \cite{ai/book/Artificial-Intelligence-A-Modern-Approach/Russell-Norvig} \\ \hline

\customTimeline{1943} &
    The first work that is now generally recognized as AI was done by \textbf{Warren McCulloch} and \textbf{Walter Pitts} (1943). &
    \cite{ai/book/Artificial-Intelligence-A-Modern-Approach/Russell-Norvig} \\ \hline

\customTimeline{1950} &
    Two undergraduate students at Harvard, \textbf{Marvin Minsky} and \textbf{Dean Edmonds}, built the first neural network computer in 1950. The \textit{SNARC}, as it was called, used 3000 vacuum tubes and a surplus automatic pilot mechanism from a B-24 bomber to simulate a network of 40 neurons. &
    \cite{ai/book/Artificial-Intelligence-A-Modern-Approach/Russell-Norvig} \\ \hline

\customTimeline{1952} &
    \textbf{Arthur Samuel} wrote a series of programs for checkers (draughts) that eventually learned to play at a strong amateur level. He disproved the idea that computers can do only what they are told to: his program quickly learned to play a better game than its creator.  &
    \cite{ai/book/Artificial-Intelligence-A-Modern-Approach/Russell-Norvig} \\ \hline

\customTimeline{1956} &
    AI name was coined &
    \cite{ai/book/Artificial-Intelligence-A-Modern-Approach/Russell-Norvig} \\ \hline

\customTimeline{1958} & 
    In MIT AI Lab Memo No. 1, \textbf{McCarthy} defined the high-level language \textbf{Lisp}, which was to become the dominant AI programming language for the next 30 years. &
    \cite{ai/book/Artificial-Intelligence-A-Modern-Approach/Russell-Norvig} \\ \hline

\customTimeline{1959} &
    \textbf{Herbert Gelernter} (1959) constructed the \textbf{Geometry Theorem Prover}, which was able to prove theorems that many students of mathematics would find quite tricky. &
    \cite{ai/book/Artificial-Intelligence-A-Modern-Approach/Russell-Norvig} \\ \hline

\customTimeline{1961} &
    \textbf{General Problem Solver (GPS)}: \textit{Allen Newell and Herbert Simon}: 
    \label{Artificial Intelligence: Introduction/AI: History/1961 - General Problem Solver (GPS): Allen Newell and Herbert Simon}
    They were not content merely to have their program solve problems correctly. They were more concerned with comparing the trace of its reasoning steps to traces of human subjects solving the same problems. &
    \cite{ai/book/Artificial-Intelligence-A-Modern-Approach/Russell-Norvig} \\ \hline

\customTimeline{1963} &
    \begin{minipage}{11.5cm}
        \vspace{0.15cm}
        \begin{enumerate}
            \item \textbf{McCarthy} started the AI lab at Stanford. 
            \item \textbf{James Slagle}’s \textit{Saint program} (1963) was able to solve closed-form calculus integration problems typical of first-year college courses.
        \end{enumerate}
        \vspace{0.15cm}
    \end{minipage}
    &
    \cite{ai/book/Artificial-Intelligence-A-Modern-Approach/Russell-Norvig} \\ \hline



\customTimeline{1965} &
    McCarthy's plan to use logic to build the ultimate Advice Taker was advanced by \textbf{J. A. Robinson}’s discovery in 1965 of the resolution method (a complete theorem-proving algorithm for first-order logic). &
    \cite{ai/book/Artificial-Intelligence-A-Modern-Approach/Russell-Norvig} \\ \hline


\customTimeline{1967} &
    \textbf{Daniel Bobrow}’s \textit{Student program} (1967) solved algebra story problems. &
    \cite{ai/book/Artificial-Intelligence-A-Modern-Approach/Russell-Norvig} \\ \hline


\customTimeline{1968} &
    \textbf{Tom Evans}’s \textit{Analogy program} (1968) solved geometric analogy problems that appear in IQ tests. &
    \cite{ai/book/Artificial-Intelligence-A-Modern-Approach/Russell-Norvig} \\ \hline



























%%%%%%%%%%%%%%%%%%%%%%%%%%%%%%%%%%%%%%%%%%%%%%%%%%%%%%%%%%%%%%%%%%%%%%%%%%%%%%%%%%%%%%%%%%%

\end{longtable}
\end{customArrayStretch}



