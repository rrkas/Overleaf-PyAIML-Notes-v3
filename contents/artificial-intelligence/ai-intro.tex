\chapter{Artificial Intelligence: Introduction}\label{Artificial Intelligence: Introduction}

\begin{enumerate}
    \item The field of artificial intelligence, or AI, attempts not just to \textbf{understand} but also to \textbf{build} intelligent entities.
    \hfill \cite{ai/book/Artificial-Intelligence-A-Modern-Approach/Russell-Norvig}

    \item AI currently encompasses a huge variety of subfields, ranging from the general (learning and perception) to the specific, such as playing chess, proving mathematical theorems, writing poetry, driving a car on a crowded street, and diagnosing diseases.    
    \hfill \cite{ai/book/Artificial-Intelligence-A-Modern-Approach/Russell-Norvig}

    \item AI is relevant to any intellectual task; it is truly a universal field.
    \hfill \cite{ai/book/Artificial-Intelligence-A-Modern-Approach/Russell-Norvig}

    \item A system is \textbf{rational} if it does the “right thing,” given what it knows.
    \hfill \cite{ai/book/Artificial-Intelligence-A-Modern-Approach/Russell-Norvig}
\end{enumerate}






\section{Approaches to AI}\label{Artificial Intelligence: Introduction/Approaches to AI}

\subsection{Acting humanly: The Turing Test approach}\label{Artificial Intelligence: Introduction/Approaches to AI/Acting humanly: The Turing Test approach}


\begin{enumerate}[itemsep=0.2cm]
    \item The \textbf{Turing Test}\label{Artificial Intelligence: Introduction/Approaches to AI/Acting humanly: The Turing Test approach/Turing Test}, proposed by \textbf{Alan Turing} (1950), was designed to provide a satisfactory operational definition of intelligence.
    \hfill \cite{ai/book/Artificial-Intelligence-A-Modern-Approach/Russell-Norvig}

    \item A computer passes the test if a human interrogator, after posing some written questions, cannot tell whether the written responses come from a person or from a computer. 
    \hfill \cite{ai/book/Artificial-Intelligence-A-Modern-Approach/Russell-Norvig}

    \item The computer would need to possess the following capabilities:
    \hfill \cite{ai/book/Artificial-Intelligence-A-Modern-Approach/Russell-Norvig}
    \begin{enumerate}
        \item \textbf{Natural Language Processing} to enable it to communicate successfully in English

        \item \textbf{Knowledge Representation} to store what it knows or hears

        \item \textbf{Automated Reasoning} to use the stored information to answer questions and to draw new conclusions

        \item \textbf{Machine Learning} to adapt to new circumstances and to detect and extrapolate patterns

    \end{enumerate}

    \item Turing’s test deliberately \textit{avoided direct physical} interaction between the interrogator and the computer, because physical simulation of a person is unnecessary for intelligence.
    \hfill \cite{ai/book/Artificial-Intelligence-A-Modern-Approach/Russell-Norvig}

    \item \textbf{Total Turing Test}\label{Artificial Intelligence: Introduction/Approaches to AI/Acting humanly: The Turing Test approach/Total Turing Test} includes a video signal so that the interrogator can test the subject’s perceptual abilities, as well as the opportunity for the interrogator to pass physical objects “through the hatch”. To pass the total Turing Test, the computer will additionally need:
    \hfill \cite{ai/book/Artificial-Intelligence-A-Modern-Approach/Russell-Norvig}
    \begin{enumerate}
        \item \textbf{computer vision} to perceive objects

        \item \textbf{robotics} to manipulate objects and move about
    \end{enumerate}
\end{enumerate}









\subsection{Thinking humanly: The cognitive modeling approach}\label{Artificial Intelligence: Introduction/Approaches to AI/Thinking humanly: The cognitive modeling approach}

\begin{enumerate}
    \item 
\end{enumerate}






















\section{AI: History}\label{Artificial Intelligence: Introduction/AI: History}

\begin{enumerate}
    \item Work in AI started after World War II. AI name was coined in 1956. 
    \hfill \cite{ai/book/Artificial-Intelligence-A-Modern-Approach/Russell-Norvig}

    
\end{enumerate}




