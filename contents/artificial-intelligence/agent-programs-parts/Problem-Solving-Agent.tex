\section{Problem-Solving Agent}\label{AI: Agent Programs/Problem-Solving Agent}



\begin{enumerate}
    \item one kind of \textit{goal-based agent} that use \textbf{atomic} representations
    \hfill \cite{ai/book/Artificial-Intelligence-A-Modern-Approach/Russell-Norvig}

    \item \textbf{uninformed search algorithms}: algorithms that are given \textbf{no information} about the problem other than its definition
    \hfill \cite{ai/book/Artificial-Intelligence-A-Modern-Approach/Russell-Norvig}
    \\
    some of these algorithms can solve any solvable problem, none of them can do so efficiently
    \hfill \cite{ai/book/Artificial-Intelligence-A-Modern-Approach/Russell-Norvig}

    \item \textbf{Informed search algorithms}: can do quite well given some guidance on where to look for solutions.
    \hfill \cite{ai/book/Artificial-Intelligence-A-Modern-Approach/Russell-Norvig}

    \item \textbf{Goal formulation}, based on the current situation and the agent’s performance measure, is the first step in problem solving.
    \hfill \cite{ai/book/Artificial-Intelligence-A-Modern-Approach/Russell-Norvig}

    \item We consider a goal to be a set of world states—exactly those states in which the goal is satisfied.
    The agent’s task is to find out how to act, now and in the future, so that it reaches a goal state.
    \hfill \cite{ai/book/Artificial-Intelligence-A-Modern-Approach/Russell-Norvig}

\subsection*{Simple “formulate, search, execute” design}

    \item \textbf{Problem formulation} is the process of deciding what actions and states to consider, given a goal.
    \hfill \cite{ai/book/Artificial-Intelligence-A-Modern-Approach/Russell-Norvig}

    \item An agent with several immediate options of unknown value can decide what to do by first examining future actions that eventually lead to states of known value.
    \hfill \cite{ai/book/Artificial-Intelligence-A-Modern-Approach/Russell-Norvig}

    \item The process of looking for a sequence of actions that reaches the goal is called \textbf{search}.
    \hfill \cite{ai/book/Artificial-Intelligence-A-Modern-Approach/Russell-Norvig}

    \item A \textit{search algorithm} takes a problem as input and returns a \textbf{solution} in the form of an action sequence.
    \hfill \cite{ai/book/Artificial-Intelligence-A-Modern-Approach/Russell-Norvig}

    \item Once a solution is found, the actions it recommends can be carried out. This is called the \textbf{execution} phase.
    While the agent is executing the solution sequence it \textit{ignores} its percepts when choosing an action because it knows in advance what they will be.
    \hfill \cite{ai/book/Artificial-Intelligence-A-Modern-Approach/Russell-Norvig}

    \item Once the solution has been executed, the agent will formulate a new goal.
    \hfill \cite{ai/book/Artificial-Intelligence-A-Modern-Approach/Russell-Norvig}

    \item An agent that carries out its plans with its eyes closed, so to speak, must be quite certain of what is going on. Control theorists call this an \textbf{open-loop system}, because ignoring the percepts breaks the loop between agent and environment.
    \hfill \cite{ai/book/Artificial-Intelligence-A-Modern-Approach/Russell-Norvig}
\end{enumerate}








\vspace{0.5cm}


\begin{algorithm}[H]
    \caption{A simple problem-solving agent. It first formulates a goal and a problem, searches for a sequence of actions that would solve the problem, and then executes the actions one at a time. When this is complete, it formulates another goal and starts over \cite{ai/book/Artificial-Intelligence-A-Modern-Approach/Russell-Norvig}}

    \SetKwFunction{FUNCTION}{\textsc{Simple-Problem-Solving-Agent}}
    \SetKwProg{Fn}{function}{ returns \normalfont an action}{end}
    \Fn{\FUNCTION{ percept }}{
        \textbf{persistent}:\\
            \hspace{1cm} $seq$, an action sequence, initially empty \\
            \hspace{1cm} $state$, some description of the current world state \\
            \hspace{1cm} $goal$, a goal, initially null \\
            \hspace{1cm} $problem$, a problem formulation \\
        \ \\
        $state$ $\gets$ \textsc{Update-State}( $state,\ percept$ )

        \If{$seq$ \normalfont is empty}{
            $goal$ $\gets$ \textsc{Formulate-Goal}( $state$ ) \\
            $problem$ $\gets$ \textsc{Formulate-Problem}( $state,\ goal$ )\\
            $seq$ $\gets$ \textsc{Search}( $prolem$ ) \\

            \If{$seq\ = \ failure$ }{
                \Return a null action
            }
        }

        $action$ $\gets$ \textsc{First}( $seq$ )\\
        $seq$ $\gets$ \textsc{Rest}( $seq$ )

        \Return $action$
    }
\end{algorithm}



\subsection{Defining \& Formulating Problem}

\begin{enumerate}
    \item The \textbf{initial state} that the agent starts in.
    \hfill \cite{ai/book/Artificial-Intelligence-A-Modern-Approach/Russell-Norvig}

    \item A description of the possible \textbf{actions} available to the agent. Given a particular state $s$, \textsc{Actions}($s$) returns the set of actions that can be executed in $s$. We say that each of these actions is \textit{applicable} in $s$.
    \hfill \cite{ai/book/Artificial-Intelligence-A-Modern-Approach/Russell-Norvig}

    \item A description of what each action does; the formal name for this is the \textbf{transition model}, specified by a function \textsc{Result}($s,\ a$) that returns the state that results from  doing action $a$ in state $s$.
    We also use the term \textbf{successor} to refer to any state reachable from a given state by a single action.
    \hfill \cite{ai/book/Artificial-Intelligence-A-Modern-Approach/Russell-Norvig}

    \item The \textbf{goal test}, which determines whether a given state is a goal state.
    Sometimes there is an \textit{explicit} set of possible goal states, and the test simply checks whether the given state is one of them.
    Sometimes the goal is specified by an \textit{abstract property} rather than an explicitly enumerated set of states.
    \hfill \cite{ai/book/Artificial-Intelligence-A-Modern-Approach/Russell-Norvig}

    \item A \textbf{path cost} function that assigns a numeric cost to each path. The problem-solving agent chooses a cost function that reflects its own performance measure.
    We assume that the cost of a path can be described as the \textit{sum} of the costs of the individual actions along the path.
    \hfill \cite{ai/book/Artificial-Intelligence-A-Modern-Approach/Russell-Norvig}
\end{enumerate}

\vspace{0.5cm}

\textbf{Note}:

\begin{enumerate}
    \item Together, the initial state, actions, and transition model implicitly define the \textbf{state space} of the problem—the set of all states reachable from the initial state by any sequence of actions. The state space forms a directed network or \textbf{graph} in which the nodes are states and the links between nodes are actions.
    \hfill \cite{ai/book/Artificial-Intelligence-A-Modern-Approach/Russell-Norvig}

    \item \textbf{diameter of the state space}: The maximum number of steps (or actions) required to go from any one state to any other reachable state in the state space.
    \hfill \cite{ai/book/Artificial-Intelligence-A-Modern-Approach/Russell-Norvig}

    \item A \textbf{path} in the state space is a sequence of states connected by a sequence of actions.
    \hfill \cite{ai/book/Artificial-Intelligence-A-Modern-Approach/Russell-Norvig}

    \item The \textbf{step cost} of taking action $a$ in state $s$ to reach state $s^\prime$ is denoted by $c(s,\ a,\ s^\prime)$.
    \hfill \cite{ai/book/Artificial-Intelligence-A-Modern-Approach/Russell-Norvig}

    \item A \textbf{solution} to a problem is an \textit{action sequence} that leads from the initial state to a goal state.
    \hfill \cite{ai/book/Artificial-Intelligence-A-Modern-Approach/Russell-Norvig}

    \item Solution quality is measured by the path cost function, and an \textbf{optimal solution} has the lowest path cost among all solutions.
    \hfill \cite{ai/book/Artificial-Intelligence-A-Modern-Approach/Russell-Norvig}

    \item In addition to abstracting the state description, we must abstract the actions themselves.
    \hfill \cite{ai/book/Artificial-Intelligence-A-Modern-Approach/Russell-Norvig}

    \item The abstraction is \textbf{valid} if we can expand any abstract solution into a solution in the more detailed world
    \hfill \cite{ai/book/Artificial-Intelligence-A-Modern-Approach/Russell-Norvig}

    \item The abstraction is \textbf{useful} if carrying out each of the actions in the solution is easier than the original problem
    \hfill \cite{ai/book/Artificial-Intelligence-A-Modern-Approach/Russell-Norvig}

    \item \textbf{meta-level state space}: Each \textbfit{state} in a meta-level state space captures the internal (computational) state of a program that is searching in an \textbf{object-level state space}.
    Each \textbfit{action} in the meta-level state space is a computation step that alters the internal state.
    A sequence of larger and larger search trees, can be seen as depicting a path in the meta-level state space where each state on the path is an object-level search tree
    \hfill \cite{ai/book/Artificial-Intelligence-A-Modern-Approach/Russell-Norvig}
    \\
    \textbf{Example}: the internal state of the A$^\ast$ algorithm consists of the current search tree.
    each computation step in A$^\ast$ expands a leaf node and adds its successors to the tree
    \hfill \cite{ai/book/Artificial-Intelligence-A-Modern-Approach/Russell-Norvig}

    \item \textbf{meta-level learning algorithm} can learn from experiences (missteps in harder problems) to avoid exploring unpromising sub-trees.
    The goal of learning is to \textit{minimize} the \textbf{total cost} of problem solving, trading off computational expense and path cost.
    \hfill \cite{ai/book/Artificial-Intelligence-A-Modern-Approach/Russell-Norvig}

    \item \textbf{relaxed problem}: A problem with \textit{fewer restrictions} on the actions
    \hfill \cite{ai/book/Artificial-Intelligence-A-Modern-Approach/Russell-Norvig}
    \\
    The state-space graph of the relaxed problem is a \textbf{super-graph} of the original state space because the removal of restrictions creates added edges in the graph.
    \hfill \cite{ai/book/Artificial-Intelligence-A-Modern-Approach/Russell-Norvig}
    \\
    Because the relaxed problem adds edges to the state space, any optimal solution in the original problem is, by definition, also a solution in the relaxed problem; but the relaxed problem may have \textbf{better solutions} if the added edges provide \textbf{short cuts}. Hence, \textbfit{the cost of an optimal solution to a relaxed problem is an admissible heuristic for the original problem}.
    \hfill \cite{ai/book/Artificial-Intelligence-A-Modern-Approach/Russell-Norvig}
    \\
    because the derived heuristic is an exact cost for the relaxed problem, it must obey the triangle inequality and is therefore \textbf{consistent}
    \hfill \cite{ai/book/Artificial-Intelligence-A-Modern-Approach/Russell-Norvig}

    \item  it is crucial that the relaxed problems generated by this technique can be solved essentially \textbf{without search}, because the relaxed rules allow the problem to be decomposed into some independent subproblems.
    If the relaxed problem is hard to solve, then the values of the corresponding heuristic will be \textbf{expensive} to obtain.
    \hfill \cite{ai/book/Artificial-Intelligence-A-Modern-Approach/Russell-Norvig}
\end{enumerate}

\vspace{0.2cm}

\begin{lstlisting}[
    language=Python,
    caption=Problem Solving Agent - Problem Skeleton
]
class Problem:
    def __init__(self, initial_state: State):
        self.initial_state = initial_state
        # 1. initial_state: The initial state that the agent starts in.

    def actions(self, state: State):
        """
            2. A description of the possible actions available to the agent.
            Given a particular state s, ACTIONS(s) returns the set of actions
            that can be executed in s.
        """
        raise NotImplementedError()

    def result(self, state: State, action):
        """
            3. A description of what each action does; the formal name for this
            is the transition model, specified by a function RESULT(s, a) that
            returns the state that results from doing action a in state s.
        """
        raise NotImplementedError()

    def goal_test(self, state: State):
        # 4. The goal test, which determines whether a given state is a goal state.
        raise NotImplementedError()

    def step_cost(self, state: State, action, new_state: State):
        """
            5. The step cost of taking action
            a in state s to reach state s' is denoted by c(s, a, s').
        """
        raise NotImplementedError()

    def heuristic(self, state: State):
        """
            Returns estimated cost from state to goal.
        """
        raise NotImplementedError()
\end{lstlisting}



\subsection{Measuring problem-solving performance}

\begin{enumerate}
    \item \textbf{Completeness}: Is the algorithm guaranteed to find a solution when there is one?
    \hfill \cite{ai/book/Artificial-Intelligence-A-Modern-Approach/Russell-Norvig}

    \item \textbf{Optimality}: Does the strategy find the optimal solution?
    \hfill \cite{ai/book/Artificial-Intelligence-A-Modern-Approach/Russell-Norvig}

    \item \textbf{Time complexity}: How long does it take to find a solution?
    \hfill \cite{ai/book/Artificial-Intelligence-A-Modern-Approach/Russell-Norvig}

    \item \textbf{Space complexity}: How much memory is needed to perform the search?
    \hfill \cite{ai/book/Artificial-Intelligence-A-Modern-Approach/Russell-Norvig}
\end{enumerate}


\vspace{0.5cm}

\begin{customArrayStretch}{1.3}
\begin{table}[H]
\centering
\begin{tabular}{l l p{12cm}}

$V$ & set & set of vertices (nodes) of the graph \\

$E$ & set & set of edges (links) of the graph \\

$b$ & $\in \mbbR$ & \textbf{branching factor} or maximum number of successors of any node \\

$d$ & $\in \mbbR$ & \textbf{depth} of the \textbf{shallowest goal} node (i.e., the number of steps along the path from the root) \\

$m$ & $\in \mbbR$ & maximum length of any path in the state space \\

\end{tabular}
\caption*{Notations}
\end{table}
\end{customArrayStretch}


\textbf{Note}:
\begin{enumerate}
    \item the typical measure is the \textbf{size} of the \textbf{state space graph}: $|V| + |E|$
    \hfill \cite{ai/book/Artificial-Intelligence-A-Modern-Approach/Russell-Norvig}

    \item Time is often measured in terms of the number of nodes generated during the search, and space in terms of the maximum number of nodes stored in memory.
    \hfill \cite{ai/book/Artificial-Intelligence-A-Modern-Approach/Russell-Norvig}

    \item \textbf{effectiveness of a search algorithm}:
    \begin{enumerate}
        \item \textbf{search cost}: depends on the time complexity but can also include a term for memory usage
        \hfill \cite{ai/book/Artificial-Intelligence-A-Modern-Approach/Russell-Norvig}

        \item \textbf{total cost}: combines the search cost and the path cost of the solution found
        \hfill \cite{ai/book/Artificial-Intelligence-A-Modern-Approach/Russell-Norvig}
    \end{enumerate}

    \item In general, exponential-complexity search problems \textbf{cannot be solved} by uninformed methods for any but the smallest instances.
\end{enumerate}






