\chapter{Recurrent Neural Networks (RNN) (1982)}

\begin{enumerate}
    \item Recurrent neural networks (RNNs) are a promising architecture for general-purpose sequence transduction.
    \hfill \cite{arxiv/1211.3711/Sequence-Transduction-RNN}

    \item The combination of a high-dimensional multivariate internal state and nonlinear state-to-state dynamics offers more expressive power than conventional sequential algorithms such as hidden Markov models.
    \hfill \cite{arxiv/1211.3711/Sequence-Transduction-RNN}

    \item In particular, RNNs are better at storing and accessing information over long periods of time.
    \hfill \cite{arxiv/1211.3711/Sequence-Transduction-RNN}

    \item RNNs are usually restricted to problems where the alignment between the input and output sequence is known in advance.
    For example, RNNs may be used to classify every frame in a speech signal, or every amino acid in a protein chain. 
    If the network outputs are probabilistic this leads to a distribution over output sequences of the same length as the input sequence. 
    \hfill \cite{arxiv/1211.3711/Sequence-Transduction-RNN}
\end{enumerate}

\section{Disadvantages}

\begin{enumerate}
    \item Recurrent models typically factor computation along the symbol positions of the input and output sequences. 
    Aligning the positions to steps in computation time, they generate a sequence of hidden states $h_t$, as a function of the previous hidden state $h_{t-1}$ and the input for position $t$. 
    This inherently sequential nature precludes parallelization within training examples, which becomes critical at longer sequence lengths, as memory constraints limit batching across examples.
    \hfill \cite{arxiv/1706.03762/Attention-Is-All-You-Need}

    \item  RNNs traditionally require a pre-defined alignment between the input and output sequences to perform transduction. 
    This is a severe limitation since finding the alignment is the most difficult aspect of many sequence transduction problems. 
    \hfill \cite{arxiv/1211.3711/Sequence-Transduction-RNN}
\end{enumerate}
















