\section{Density Plot \cite{data/online/seaborn.displot}} \label{Visualizing Data/Density Plot}

\begin{table}[H]
\begin{minipage}[t]{0.35\linewidth}
\begin{figure}[H]
    \centering
    \includegraphics[width=0.9\linewidth, height=10cm, keepaspectratio]{images/data/__visualizations__/sns-density-kde-rating-face-data.png}
    \caption{Density plot (py-sns) output (face\_data.csv)}
\end{figure}
\end{minipage}
\hspace{0.2cm}
\vrule width 1pt
\hspace{0.5cm}
\begin{minipage}[t]{0.57\linewidth}
\begin{lstlisting}[
    language=Python,
    caption=Density Plot: py-sns: face\_data.csv
]
sns.displot(df["rating"], kind="kde")

plt.show()
\end{lstlisting}

\vspace{0.2cm}

\begin{enumerate}
    \item A density plot - at least in this setting - can be considered a “continuous approximation” of a histogram. \hfill \cite{statistics/book/Statistics-for-Data-Scientists/Maurits-Kaptein} \\
    SEE: \fullref{Visualizing Data/Histogram}
    
    \item It gives per range of values of the continuous variable the probability of observing a value within that range. \hfill \cite{statistics/book/Statistics-for-Data-Scientists/Maurits-Kaptein}

    
\end{enumerate}

\end{minipage}
\end{table}





