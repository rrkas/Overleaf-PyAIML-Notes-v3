\chapter{Data}\label{Data}

\section{Measurement Levels \cite{statistics/book/Statistics-for-Data-Scientists/Maurits-Kaptein}}\label{Data/Measurement-Levels}

\subsection{Nominal, Ordinal, Interval and Ratio \cite{statistics/book/Statistics-for-Data-Scientists/Maurits-Kaptein}}\label{Data/Measurement-Levels/Nominal, Ordinal, Interval and Ratio}

\label{Data/Measurement-Levels/Nominal, Ordinal, Interval and Ratio/Categorical Data}
\label{Data/Measurement-Levels/Nominal, Ordinal, Interval and Ratio/Numerical Data}
\label{Data/Measurement-Levels/Nominal, Ordinal, Interval and Ratio/Nominal}
\label{Data/Measurement-Levels/Nominal, Ordinal, Interval and Ratio/Ordinal}
\label{Data/Measurement-Levels/Nominal, Ordinal, Interval and Ratio/Interval}
\label{Data/Measurement-Levels/Nominal, Ordinal, Interval and Ratio/Ratio}

\begin{table}[H]
    \hfill
    \begin{minipage}[H]{0.20\linewidth}
        \textbf{Levels}: \hfill \cite{statistics/book/Statistics-for-Data-Scientists/Maurits-Kaptein}
        \begin{enumerate}
            \item Nominal
            \item Ordinal
            \item Interval
            \item Ratio
        \end{enumerate}
    \end{minipage}
    \begin{minipage}[H]{0.75\linewidth}
        \begin{table}[H]
            \centering
            \begin{tabular}{|p{5cm}|c|c|c|c|}
                \hline
                & \multicolumn{2}{c|}{\textbf{Categorical Data}} & \multicolumn{2}{c|}{\textbf{Numerical Data}} \\

                \hline
                & \textbf{Nominal} & \textbf{Ordinal} & \textbf{Interval} & \textbf{Ratio} \\ \hline

                Distinction between groups / individuals & \checkmark & \checkmark & \checkmark & \checkmark \\ \hline

                Imposes logical Order & \xmark & \checkmark & \checkmark & \checkmark \\ \hline

                Provides a magnitude of the differences in some unit & \xmark & \xmark & \checkmark & \checkmark \\ \hline

                A clear reference point or "0" & \xmark & \xmark & \xmark & \checkmark \\ \hline
            \end{tabular}
            \caption{Data: Measurement Levels: Nominal, Ordinal, Interval and Ratio \cite{statistics/book/Statistics-for-Data-Scientists/Maurits-Kaptein}}
        \end{table}
    \end{minipage}
    \hfill
\end{table}

\vspace{0.3cm}

\textbf{Note}:
\begin{enumerate}
    \item Each consecutive measurement level contains as much "information" - in a fairly loose sense of the word - as the previous one and more. \hfill \cite{statistics/book/Statistics-for-Data-Scientists/Maurits-Kaptein}
\end{enumerate}


\subsection{Continuous vs Discrete numerical data \cite{statistics/book/Statistics-for-Data-Scientists/Maurits-Kaptein}}\label{Data/Measurement-Levels/Continuous vs Discrete numerical data}

\label{Data/Measurement-Levels/Continuous vs Discrete numerical data/Continuous numerical data}
\label{Data/Measurement-Levels/Continuous vs Discrete numerical data/Discrete numerical data}

\begin{enumerate}
    \item Continuous variables can assume any value. \hfill \cite{statistics/book/Statistics-for-Data-Scientists/Maurits-Kaptein} \\
    This means that the continuous variable can attain any value between two different values, no matter how close the two values are. \hfill \cite{statistics/book/Statistics-for-Data-Scientists/Maurits-Kaptein}\\
    \textbf{Example}: temperature, weight, and age \hfill \cite{statistics/book/Statistics-for-Data-Scientists/Maurits-Kaptein}

    \item Discrete variables cannot assume any value between 2 values. \hfill \cite{statistics/book/Statistics-for-Data-Scientists/Maurits-Kaptein}\\
    \textbf{Example}: number of text messages, accidents, microorganisms, students, etc. \hfill \cite{statistics/book/Statistics-for-Data-Scientists/Maurits-Kaptein}
\end{enumerate}


\subsection{Outliers \cite{statistics/book/Statistics-for-Data-Scientists/Maurits-Kaptein}}\label{Data/Measurement-Levels/Outliers}

\begin{enumerate}
    \item An outlier is a data point that significantly deviates from other observations in a dataset. \hfill \cite{common/online/chatgpt}

    \item Caused by Natural variability in the data or measurement errors. \hfill \cite{common/online/chatgpt}

    \item Typically identified using statistical methods like the IQR (Interquartile Range), Z-score, or visualization techniques (e.g., box plots). \hfill \cite{common/online/chatgpt}

    \item Outliers are not necessarily incorrect; they may represent rare but valid observations. \hfill \cite{common/online/chatgpt}

\end{enumerate}


\vspace{0.3cm}

\textbf{Examples}:
\begin{enumerate}
    \item In a dataset of human heights, a person measuring 250 cm might be an outlier but not necessarily unrealistic if it’s a rare case of gigantism. \hfill \cite{common/online/chatgpt}
\end{enumerate}


\vspace{0.3cm}
\textbf{Handling/ Dealing with Outliers}:
\begin{enumerate}
    \item Ignore these abnormalities and go ahead with the data. \hfill \cite{statistics/book/Statistics-for-Data-Scientists/Maurits-Kaptein}

    \item Delete/ remove the suspected records/ entries. \hfill \cite{statistics/book/Statistics-for-Data-Scientists/Maurits-Kaptein}

    \item Substitute them, using statistical methods, with a more plausible alternative. (aka \textbf{imputation}) \hfill \cite{statistics/book/Statistics-for-Data-Scientists/Maurits-Kaptein}\label{Data/Outliers/imputation}
\end{enumerate}




\subsection{Unrealistic Values \cite{statistics/book/Statistics-for-Data-Scientists/Maurits-Kaptein}}\label{Data/Measurement-Levels/Unrealistic Values}

\begin{enumerate}
    \item An unrealistic value is a data point that is not plausible within the context of the dataset, often due to data entry errors or faulty sensors. \hfill \cite{common/online/chatgpt}

    \item Caused by Human error, sensor malfunction, or corruption during data transmission. \hfill \cite{common/online/chatgpt}

    \item Typically identified using domain knowledge or logical constraints. \hfill \cite{common/online/chatgpt}

    \item Unlike outliers, unrealistic values are generally not useful and need correction or removal. \hfill \cite{common/online/chatgpt}

\end{enumerate}

\vspace{0.3cm}

\textbf{Examples}:
\begin{enumerate}
    \item A recorded body temperature of 200°C for a human is unrealistic, as it’s physically impossible for a person to survive at that temperature. \hfill \cite{common/online/chatgpt}

    \item Negative age of a person \hfill \cite{common/online/chatgpt}

    \item Missing values \hfill \cite{statistics/book/Statistics-for-Data-Scientists/Maurits-Kaptein}

    \item Incorrect datatype of value \hfill \cite{statistics/book/Statistics-for-Data-Scientists/Maurits-Kaptein}
\end{enumerate}

\vspace{0.3cm}
\textbf{Handling/ Dealing with Unrealistic values}:
\begin{enumerate}
    \item Delete/ remove the suspected records/ entries. \hfill \cite{statistics/book/Statistics-for-Data-Scientists/Maurits-Kaptein}

\end{enumerate}





\section{Describing Data \cite{statistics/book/Statistics-for-Data-Scientists/Maurits-Kaptein}} \label{Data/Describing Data}

Some \textbf{descriptive statistics}\label{Data/Describing Data/descriptive statistics} (or just \textbf{descriptives}\label{Data/Describing Data/descriptives}) that we introduce are often used for data of a certain measurement level. \hfill \cite{statistics/book/Statistics-for-Data-Scientists/Maurits-Kaptein}



\subsection{Frequency/ Frequency table \cite{statistics/book/Statistics-for-Data-Scientists/Maurits-Kaptein}}\label{Data/Describing Data/Frequency or Frequency table}

\textbf{Measurement levels}: Nominal and ordinal data \hfill \cite{statistics/book/Statistics-for-Data-Scientists/Maurits-Kaptein}

\vspace{0.3cm}

\begin{enumerate}
    \item Frequencies are often uninformative for interval or ratio variables. \hfill \cite{statistics/book/Statistics-for-Data-Scientists/Maurits-Kaptein}\\
    if there are lots and lots of different possible values, all of them will have a count of just one. \hfill \cite{statistics/book/Statistics-for-Data-Scientists/Maurits-Kaptein}\\
    This is often tackled by discretizing (or "\textbf{binning}”\label{Data/Describing Data/Frequency or Frequency table/binning}) the variable (which, note, effectively "throws away” some of the information in the data). \hfill \cite{statistics/book/Statistics-for-Data-Scientists/Maurits-Kaptein}

    
\end{enumerate}


\begin{lstlisting}[
    language=Python
]
import random
import numpy as np
import pandas as pd
from collections import Counter

# random seeding
random.seed(0)
np.random.seed(0)

# using int instead of float to have 10 unique values only
data = np.random.randint(0, 10, size=100) # generate sample data
\end{lstlisting}


\subsubsection{(Absolute) Frequency/ (Absolute) Frequency table \cite{statistics/book/Statistics-for-Data-Scientists/Maurits-Kaptein}}\label{Data/Describing Data/Frequency or Frequency table/Absolute}

\begin{enumerate}
    \item It refers to the count of occurrences of a particular value or category in a dataset. \hfill \cite{common/online/chatgpt}

    \item Simple count, no further processing. \hfill \cite{common/online/chatgpt}

    \item \textbf{Use Case}: Helpful in creating bar charts or histograms. \hfill \cite{common/online/chatgpt}
\end{enumerate}


\begin{lstlisting}[
    language=Python, 
    caption=Absolute Frequency - from Scratch
]
def get_abs_freq_scratch(data):
    counter = Counter()
    for e in data:
        counter[e] += 1
    
    return counter
\end{lstlisting}

\begin{lstlisting}[
    language=Python, 
    caption=Absolute Frequency - using numPy
]
def get_abs_freq_np(data):
    data = np.array(data) # convert data to numpy array
    counter = np.unique_counts(data)
    return dict(zip(counter.values, counter.counts))
\end{lstlisting}

\subsubsection{Cumulative Frequency/ Cumulative Frequency table \cite{statistics/book/Statistics-for-Data-Scientists/Maurits-Kaptein}}\label{Data/Describing Data/Frequency or Frequency table/Cumulative}

\begin{enumerate}
    \item It is the running total of frequencies up to a certain value or class. \hfill \cite{common/online/chatgpt}

    \item Each cumulative frequency includes its own frequency plus all previous frequencies. \hfill \cite{common/online/chatgpt}

    \item \textbf{Use Case}: Useful in percentile calculations and ogive graphs. \hfill \cite{common/online/chatgpt}

    \item The cumulative frequency makes more sense for ordinal data than for nominal data, since ordinal data can be ordered in size, which is not possible for nominal data. \hfill \cite{statistics/book/Statistics-for-Data-Scientists/Maurits-Kaptein}
\end{enumerate}

\begin{table}[H]
    \begin{minipage}[H]{0.3\linewidth}
    $
        \begin{aligned}
            CF_i 
                &= CF_{i-1} + F_{i} \\
                &= \sum_{k=1}^{i} F_{k}
        \end{aligned}
    $
    \end{minipage}
    \begin{minipage}[H]{0.65\linewidth}
        \begin{table}[H]
            \begin{tabular}{l l}
                $CF_i$ & Cumulative Frequency at the current value \\ 
                $CF_{i-1}$ & Cumulative Frequency at the previous value \\ 
                $F_i$ & Frequency at the current value \\ 
            \end{tabular}
        \end{table}
    \end{minipage}
\end{table}

\begin{lstlisting}[
    language=Python, 
    caption=Cumulative Frequency - using numPy
]
def get_cum_freq(data):
    counts = get_abs_freq_np(data)
    cum_counts = {}
    _tot = 0
    for k, v in counts.items():
        _tot += v
        cum_counts[k] = _tot
    
    return cum_counts
\end{lstlisting}


\subsubsection{Relative Frequency/ Relative Frequency table \cite{statistics/book/Statistics-for-Data-Scientists/Maurits-Kaptein}}\label{Data/Describing Data/Frequency or Frequency table/Relative}

\begin{enumerate}
    \item It shows the proportion of each category relative to the total number of observations. \hfill \cite{common/online/chatgpt}

    \item Expressed as a fraction, decimal, or percentage. \hfill \cite{common/online/chatgpt}

    \item \textbf{Use Case}: Ideal for creating pie charts and understanding distribution proportions. \hfill \cite{common/online/chatgpt}
\end{enumerate}


\begin{table}[H]
    \begin{minipage}{0.3\linewidth}
        \[
            \begin{aligned}
                RF_i 
                    &= \dfrac{F_{i}}{\dsum_{k=1}^{N} F_{k}}
            \end{aligned}
        \]
    \end{minipage}
    \begin{minipage}{0.65\linewidth}
        \begin{table}[H]
            \begin{tabular}{l l}
                $RF_i$ & Relative Frequency \\
                $F_i$ & Frequency of the value \\ 
                $N$ & Total number of observations \\ 
            \end{tabular}
        \end{table}
    \end{minipage}
\end{table}


\begin{lstlisting}[
    language=Python, 
    caption=Relative Frequency - using numPy
]
def get_rel_freq(data):
    abs_freq = get_abs_freq_np(data)
    rel_freq = {}
    for k, v in abs_freq.items():
        rel_freq[k] = v / len(data)
    
    return rel_freq
\end{lstlisting}


\subsubsection{Cumulative Relative Frequency/ Cumulative Relative Frequency table \cite{statistics/book/Statistics-for-Data-Scientists/Maurits-Kaptein}}\label{Data/Describing Data/Frequency or Frequency table/Cumulative Relative}

\begin{enumerate}
    \item Cumulative relative frequency is the accumulation of the relative frequencies of data points up to a certain value. \hfill \cite{common/online/chatgpt}

    \item It indicates the proportion of data points that are less than or equal to a particular value. \hfill \cite{common/online/chatgpt}

    \item \textbf{Use Cases}: \hfill \cite{statistics/book/Statistics-for-Data-Scientists/Maurits-Kaptein}
    \begin{enumerate}
        \item Identifying percentiles and median. \hfill \cite{statistics/book/Statistics-for-Data-Scientists/Maurits-Kaptein}

        \item Visualizing with a cumulative relative frequency graph (Ogive). \hfill \cite{statistics/book/Statistics-for-Data-Scientists/Maurits-Kaptein}

        \item Understanding data distribution by determining the proportion of values below a specific threshold. \hfill \cite{statistics/book/Statistics-for-Data-Scientists/Maurits-Kaptein}
    \end{enumerate}
\end{enumerate}



\begin{table}[H]
    \begin{minipage}{0.3\linewidth}
        \[
            \begin{aligned}
                CRF_i 
                    &= \dfrac{\dsum_{k=1}^{i} F_{k}}{\dsum_{k=1}^{N} F_{k}}
            \end{aligned}
        \]
    \end{minipage}
    \begin{minipage}{0.65\linewidth}
        \begin{table}[H]
            \begin{tabular}{l l}
                $CRF_i$ & Cumulative Relative Frequency \\
                $F_i$ & Frequency of the value \\ 
                $N$ & Total number of observations \\ 
            \end{tabular}
        \end{table}
    \end{minipage}
\end{table}

\begin{lstlisting}[
    language=Python, 
    caption=Cumulative Relative Frequency - using numPy
]
def get_cum_rel_freq(data):
    cum_freq = get_cum_freq(data)
    cum_rel_freq = {}
    for k, v in cum_freq.items():
        cum_rel_freq[k] = v / len(data)
    
    return cum_rel_freq
\end{lstlisting}


\subsection{Central Tendency \cite{statistics/book/Statistics-for-Data-Scientists/Maurits-Kaptein}}\label{Data/Describing Data/Central Tendency}

\begin{enumerate}
     \item When we work with numerical data, we often want to know something about the "central value" or "middle value" of the variable, also referred to as the \textbf{location}\label{Data/Describing Data/Central Tendency/location} of the data. \hfill \cite{statistics/book/Statistics-for-Data-Scientists/Maurits-Kaptein}
\end{enumerate}

\begin{lstlisting}[
    language=Python
]
import random
import numpy as np
import pandas as pd

# random seeding
random.seed(0)
np.random.seed(0)

# generate sample data
# `data` is transformed to only 1 decimal number to avoid all unique values
low = -100
high = 100
data = low + (high - low) * np.random.random(size=1000)
data = np.round(data, 1)
\end{lstlisting}

\subsubsection{(Arithmetic) mean/ average (TODO / $\mu$ ) \cite{statistics/book/Statistics-for-Data-Scientists/Maurits-Kaptein}} \label{Data/Describing Data/Central Tendency/(Arithmetic) mean or average}

\begin{table}[H]
\begin{minipage}[t]{0.25\linewidth}

\textbf{Sample} \cite{statistics/book/Statistics-for-Data-Scientists/Maurits-Kaptein} \label{Data/Describing Data/Central Tendency/(Arithmetic) mean or average/Sample}

\vspace{0.3cm}

$
    \bar{x} = \dfrac{1}{n} \dsum_{i=1}^{n} x_i
$

\end{minipage}
\hspace{0.3cm}
\vrule width 1pt
\hspace{0.3cm}
\begin{minipage}[t]{0.25\linewidth}

\textbf{Population} \cite{statistics/book/Statistics-for-Data-Scientists/Maurits-Kaptein} \label{Data/Describing Data/Central Tendency/(Arithmetic) mean or average/Population}

\vspace{0.3cm}

$
    \mu = \dfrac{1}{N} \dsum_{i=1}^{n} x_i
$

\end{minipage}
\hspace{0.3cm}
\vrule width 1pt
\hspace{0.3cm}
\begin{minipage}[t]{0.25\linewidth}

{\hfill\textbf{Notations}\hfill}

\begin{table}[H]
    \begin{tabular}{l l}
        $\mu$ & Population mean \\
        $\bar{x}$ & Sample mean \\
        $x_i$ & population item \\
        $x_i$ & sample item \\
        $N$ & population size \\
        $n$ & sample size \\
    \end{tabular}
\end{table}


\end{minipage}
\end{table}






\begin{lstlisting}[
    language=Python,
    caption=Arithmetic mean - using numPy
]
print(data.sum() / len(data))
# OR, shortcut
print(data.mean())
\end{lstlisting}



\subsubsection{Mode \cite{statistics/book/Statistics-for-Data-Scientists/Maurits-Kaptein}} \label{Data/Describing Data/Central Tendency/Mode}

\begin{enumerate}
    \item The mode is merely the most frequently occurring value. \hfill \cite{statistics/book/Statistics-for-Data-Scientists/Maurits-Kaptein}

    \item There might be multiple modes. \hfill \cite{statistics/book/Statistics-for-Data-Scientists/Maurits-Kaptein}

\end{enumerate}

\begin{lstlisting}[
    language=Python,
    caption=Mode - using numPy
]
values, counts = np.unique(data, return_counts=True)
mode_idx = np.argmax(counts)
print(values[mode_idx], counts[mode_idx])
\end{lstlisting}



\subsubsection{Median \cite{statistics/book/Statistics-for-Data-Scientists/Maurits-Kaptein}} \label{Data/Describing Data/Central Tendency/Median}

\begin{enumerate}
    \item The median is a value that divides the ordered data from small to large (or large to small) into two equal parts: 50\% of the data is below the median and 50\% is above.  \hfill \cite{statistics/book/Statistics-for-Data-Scientists/Maurits-Kaptein}

    \item The median is not necessarily a value that is present in the data. \hfill \cite{statistics/book/Statistics-for-Data-Scientists/Maurits-Kaptein}
\end{enumerate}


\vspace{0.3cm}
\textbf{Steps}: \hfill \cite{statistics/book/Statistics-for-Data-Scientists/Maurits-Kaptein}
\begin{enumerate}
    \item sort the data

    \item choose the middle-most value when $n$ is \textbf{odd}\\
        average of the two middle values when $n$ is \textbf{even}
\end{enumerate}


\begin{lstlisting}[
    language=Python,
    caption=Median - using numPy
]
print(np.median(data))
\end{lstlisting}



\subsubsection{Quantiles ( $q_i$ ) \cite{statistics/book/Statistics-for-Data-Scientists/Maurits-Kaptein}} \label{Data/Describing Data/Central Tendency/Quantiles}

\begin{enumerate}
    \item A quantile $x_q$ is a value that splits the ordered data of a variable $x$ into two parts: \hfill \cite{statistics/book/Statistics-for-Data-Scientists/Maurits-Kaptein}
    \begin{enumerate}
        \item $q \cdot 100\%$ of the data is below the value $x_q$

        \item $(1 - q) \cdot 100\%$ of the data is above the value $x_q$
    \end{enumerate}

    \item The parameter $q$ can take any value in the interval $[0, 1]$. \hfill \cite{statistics/book/Statistics-for-Data-Scientists/Maurits-Kaptein}

    \item Quantiles can be calculated in different ways, depending on the way we "interpolate" between two values. \hfill \cite{statistics/book/Statistics-for-Data-Scientists/Maurits-Kaptein} \\
    We could map the ordered values \textit{equally spaced} on the interval $(0, 1)$, where the \textit{i}th ordered value of the data is positioned at the level $q_i = {i}/{(n + 1)}$ in the interval $(0, 1)$, with $n$ being the number of data points. \hfill \cite{statistics/book/Statistics-for-Data-Scientists/Maurits-Kaptein} \\
    R uses $q_i = (i - 1)/(n - 1)$ for quantiles. \hfill \cite{statistics/book/Statistics-for-Data-Scientists/Maurits-Kaptein} \\
    \textbf{Example}: \hfill \cite{statistics/book/Statistics-for-Data-Scientists/Maurits-Kaptein}
    \begin{enumerate}
        \item Data points: $\dCurlyBrac{2, 5, 6, 4}$ ( $n=4$ )
        \item Sorted Data points: $\dCurlyBrac{2, 4, 5, 6}$ ( $n=4$ )
        \item Quantiles:\\[0.2cm]
        \begin{tabular}{|l|c|c|c|}
            \hline
            $i$ & $x_i$ & $q_i = i/(n+1) = i/5$ & $q_i = (i-1)/(n-1) = (i-1)/3$ \\ [0.1cm]
            \hline
            $1$ & $2$ & $1/5 = 0.2$ & $0$ \\
            $2$ & $4$ & $2/5 = 0.4$ & $1/3$ \\
            $3$ & $5$ & $3/5 = 0.6$ & $2/3$ \\
            $4$ & $6$ & $4/5 = 0.8$ & $1$ \\
            \hline
        \end{tabular}\\

        \item If $x_i = 3 \Rightarrow q_i = 0.3$
    \end{enumerate}
\end{enumerate}

\begin{lstlisting}[
    language=Python,
    caption=Quantile - using numPy
]
q = np.round(random.random(), 2)
print(q, np.quantile(data, q))
\end{lstlisting}



\subsubsection{Quartiles ( $q_i$ ) \cite{statistics/book/Statistics-for-Data-Scientists/Maurits-Kaptein}} \label{Data/Describing Data/Central Tendency/Quartiles}

\begin{enumerate}
    \item When $q = 0.25$, $q = 0.50$, and $q = 0.75$ the quantiles are referred to as the first, second, and third quartiles, respectively. \hfill \cite{statistics/book/Statistics-for-Data-Scientists/Maurits-Kaptein}
    \label{Data/Describing Data/Central Tendency/Quartiles/first quartile}
    \label{Data/Describing Data/Central Tendency/Quartiles/second quartile}
    \label{Data/Describing Data/Central Tendency/Quartiles/third quartile}

    \item Splits: \\
    \begin{tabular}{r l l l} % Right-align first column, left-align second column
        1. & $q = 0$ & to & $q = 0.25$ \\
        2. & $q = 0.25$ & to & $q = 0.5$ \\
        3. & $q = 0.5$ & to & $q = 0.75$ \\
        4. & $q = 0.75$ & to & $q = 1$ \\
    \end{tabular}
\end{enumerate}

\begin{lstlisting}[
    language=Python,
    caption=Quartile - using numPy
]
_splits = 5
np.percentile(data, np.linspace(0, 1, _splits))
\end{lstlisting}


\subsubsection{Deciles ( $q_i$ ) \cite{statistics/book/Statistics-for-Data-Scientists/Maurits-Kaptein}} \label{Data/Describing Data/Central Tendency/Deciles}

\begin{enumerate}
    \item We call quantiles deciles when $q$ is restricted to the set $\dCurlyBrac{0.1, 0.2,\cdots, 0.9}$ \hfill \cite{statistics/book/Statistics-for-Data-Scientists/Maurits-Kaptein}

    \item Splits: \\
    \begin{tabular}{r l l l} % Right-align first column, left-align second column
        1. & $q = 0$ & to & $q = 0.1$ \\
        2. & $q = 0.1$ & to & $q = 0.2$ \\
        && \vdots & \\
        9. & $q = 0.8$ & to & $q = 0.9$ \\
        10. & $q = 0.9$ & to & $q = 1$ \\
    \end{tabular}
\end{enumerate}


\begin{lstlisting}[
    language=Python,
    caption=Decile - using numPy
]
_splits = 11
print(np.percentile(data, np.linspace(0, 1, _splits)))
\end{lstlisting}


\subsubsection{Percentiles ( $q_i$ ) \cite{statistics/book/Statistics-for-Data-Scientists/Maurits-Kaptein}} \label{Data/Describing Data/Central Tendency/Percentiles}

\begin{enumerate}
    \item We call quantiles percentiles when $q$ is restricted to the set $\dCurlyBrac{0.01, 0.02,\cdots, 0.99}$ \hfill \cite{statistics/book/Statistics-for-Data-Scientists/Maurits-Kaptein}

    \item Splits: \\
    \begin{tabular}{r l l l} % Right-align first column, left-align second column
        1. & $q = 0$ & to & $q = 0.01$ \\
        2. & $q = 0.01$ & to & $q = 0.02$ \\
        & & \vdots & \\
        99. & $q = 0.98$ & to & $q = 0.99$ \\
        100. & $q = 0.99$ & to & $q = 1$ \\
    \end{tabular}
\end{enumerate}


\begin{lstlisting}[
    language=Python,
    caption=Percentile - using numPy
]
_splits = 101
print(np.percentile(data, np.linspace(0, 1, _splits)))
\end{lstlisting}


\subsubsection{Range \cite{statistics/book/Statistics-for-Data-Scientists/Maurits-Kaptein}} \label{Data/Describing Data/Central Tendency/Range}

\begin{enumerate}
    \item Range is the difference between the maximum and minimum. \hfill \cite{statistics/book/Statistics-for-Data-Scientists/Maurits-Kaptein}

    \item It quantifies the maximum distance between any two data points. \hfill \cite{statistics/book/Statistics-for-Data-Scientists/Maurits-Kaptein}

    \item The range is sensitive to outliers. \hfill \cite{statistics/book/Statistics-for-Data-Scientists/Maurits-Kaptein}
\end{enumerate}


\begin{lstlisting}[
    language=Python,
    caption=Range - using numPy
]
print(np.max(data) - np.min(data))
\end{lstlisting}


\subsubsection{Interquartile Range (IQR) \cite{statistics/book/Statistics-for-Data-Scientists/Maurits-Kaptein}} \label{Data/Describing Data/Central Tendency/Interquartile Range (IQR)}

\begin{enumerate}
    \item The interquartile range (IQR) calculates the difference between the third quartile and the first quartile. \hfill \cite{statistics/book/Statistics-for-Data-Scientists/Maurits-Kaptein}

    \item It quantifies a range for which 50\% of the data falls within. \hfill \cite{statistics/book/Statistics-for-Data-Scientists/Maurits-Kaptein}

    \item The interquartile range is visualized in the boxplot. \hfill \cite{statistics/book/Statistics-for-Data-Scientists/Maurits-Kaptein}
\end{enumerate}

\begin{lstlisting}[
    language=Python,
    caption=IQR - using numPy
]
print(np.percentile(data, 0.75), np.percentile(data, 0.25))
print(np.percentile(data, 0.75) - np.percentile(data, 0.25))
\end{lstlisting}


\subsubsection{Mean Absolute Deviation (MAD) \cite{statistics/book/Statistics-for-Data-Scientists/Maurits-Kaptein}} \label{Data/Describing Data/Central Tendency/Mean Absolute Deviation (MAD)}

$
    MAD
    = \dfrac{1}{n} \dsum_{i=1}^{n} \dabs{x_i - \bar{x}}
$ \hfill \cite{statistics/book/Statistics-for-Data-Scientists/Maurits-Kaptein}

\vspace{0.2cm}

SEE: \fullref{Basic Functions/Absolute Value function or Modulus Function}

\vspace{0.2cm}

\begin{enumerate}
    \item average distance that data values are away from the mean. \hfill \cite{statistics/book/Statistics-for-Data-Scientists/Maurits-Kaptein}
\end{enumerate}

\begin{lstlisting}[
    language=Python,
    caption=Mean Absolute Deviation (MAD) - using numPy
]
x_bar = data.mean()
print(np.abs(data - x_bar).mean())
\end{lstlisting}



\subsubsection{Mean Squared Deviation (MSD) \cite{statistics/book/Statistics-for-Data-Scientists/Maurits-Kaptein}} \label{Data/Describing Data/Central Tendency/Mean Squared Deviation (MSD)}

$
    MSD
    = \dfrac{1}{n} \dsum_{i=1}^{n} (x_i - \bar{x})^2
$ \hfill \cite{statistics/book/Statistics-for-Data-Scientists/Maurits-Kaptein}

\begin{enumerate}
    \item It does the same as MAD, but now it uses squared distances with respect to the mean.  \hfill \cite{statistics/book/Statistics-for-Data-Scientists/Maurits-Kaptein}\\
    SEE: \fullref{Data/Describing Data/Central Tendency/Mean Absolute Deviation (MAD)}

\end{enumerate}


\begin{lstlisting}[
    language=Python,
    caption=Mean Squared Deviation (MSD)- using numPy
]
x_bar = data.mean()
print(np.pow(data - x_bar, 2).mean())
\end{lstlisting}


\subsubsection{Variance ( $s^2$ / $\sigma^2$ ) \cite{statistics/book/Statistics-for-Data-Scientists/Maurits-Kaptein}} \label{Data/Describing Data/Central Tendency/Variance}

\begin{table}[H]
\begin{minipage}[t]{0.25\linewidth}

\textbf{Sample}\label{Data/Describing Data/Central Tendency/Variance/sample} \cite{statistics/book/Statistics-for-Data-Scientists/Maurits-Kaptein}

\vspace{0.3cm}

$
    \begin{aligned}
        s^2
        &= \dfrac{1}{n-1} \dsum_{i=1}^{n} (x_i - \bar{x})^2 \\
        &= MSD \cdot \dfrac{n}{n-1}
    \end{aligned}
$

\end{minipage}
\hspace{0.3cm}
\vrule width 1pt
\hspace{0.3cm}
\begin{minipage}[t]{0.25\linewidth}

\textbf{Population}\label{Data/Describing Data/Central Tendency/Variance/population} \cite{statistics/book/Statistics-for-Data-Scientists/Maurits-Kaptein}

\vspace{0.3cm}

$
    \sigma^2
    = \dfrac{1}{N} \dsum_{i=1}^{n} (x_i - \mu)^2
$

\end{minipage}
\hspace{0.3cm}
\vrule width 1pt
\hspace{0.3cm}
\begin{minipage}[t]{0.25\linewidth}

{\hfill\textbf{Notations}\hfill}

\begin{table}[H]
    \begin{tabular}{l l}
        $\mu$ & Population mean \\
        $\bar{x}$ & Sample mean \\
        $x_i$ & population item \\
        $x_i$ & sample item \\
        $N$ & population size \\
        $n$ & sample size \\
    \end{tabular}
\end{table}

\end{minipage}
\end{table}


\vspace{0.2cm}
\textbf{Note}:
\begin{enumerate}
    \item The variance is almost identical to the mean squared deviation. \hfill \cite{statistics/book/Statistics-for-Data-Scientists/Maurits-Kaptein}

    \item For small sample sizes the MSD and variance are not the same, but for large sample sizes they are obviously very similar. \hfill \cite{statistics/book/Statistics-for-Data-Scientists/Maurits-Kaptein}

    \item The variance is often preferred over the MSD. \hfill \cite{statistics/book/Statistics-for-Data-Scientists/Maurits-Kaptein}
\end{enumerate}


\begin{lstlisting}[
    language=Python,
    caption=Variance - using numPy
]
x_bar = data.mean()

# sample
print(np.pow(data - x_bar, 2).sum() / (len(data) - 1))

# population
print(np.pow(data - x_bar, 2).mean())
\end{lstlisting}


\subsubsection{Standard Deviation ( $s$ / $\sigma$ ) \cite{statistics/book/Statistics-for-Data-Scientists/Maurits-Kaptein}} \label{Data/Describing Data/Central Tendency/Standard Deviation}

\textbf{Formulas}:
\begin{enumerate}
    \item Sample:
    \label{Data/Describing Data/Central Tendency/Standard Deviation/Sample} \\
        $
            \begin{aligned}
                s
                &=\sqrt{s^2}
                &= \sqrt{\dfrac{1}{n-1} \dsum_{i=1}^{n} (x_i - \bar{x})^2}
                &= \sqrt{MSD \cdot \dfrac{n}{n-1}}
            \end{aligned}
        $

    \item Population:
\end{enumerate}


\vspace{0.2cm}
\textbf{Note}:
\begin{enumerate}
    \item The standard deviation is on the same scale as the original variable, instead of a squared scale for the variance. \hfill \cite{statistics/book/Statistics-for-Data-Scientists/Maurits-Kaptein}


\end{enumerate}


\begin{lstlisting}[
    language=Python,
    caption=Standard Deviation - using numPy
]
x_bar = data.mean()

# sample
s2 = np.pow(data - x_bar, 2).sum() / (len(data) - 1)
print(np.sqrt(s2))

# population
sig2 = np.pow(data - x_bar, 2).mean()
print(np.sqrt(sig2))
\end{lstlisting}


\subsubsection{Skewness ( $g_1$ ) \cite{statistics/book/Statistics-for-Data-Scientists/Maurits-Kaptein}} \label{Data/Describing Data/Central Tendency/Skewness}

\begin{enumerate}
    \item Computed using so-called standardized values $z_i$. \hfill \cite{statistics/book/Statistics-for-Data-Scientists/Maurits-Kaptein} \\
    Standardized values have no unit and the mean and variance of the standardized values are equal to 0 and 1, respectively. \hfill \cite{statistics/book/Statistics-for-Data-Scientists/Maurits-Kaptein}

    \item Skewness is used to measure the asymmetry in data. \hfill \cite{statistics/book/Statistics-for-Data-Scientists/Maurits-Kaptein}

    \item Data is considered skewed or asymmetric when the variation on one side of the middle of the data is larger than the variation on the other side. \hfill \cite{statistics/book/Statistics-for-Data-Scientists/Maurits-Kaptein}

    \item In practice, researchers sometimes compare the mean with the median to get an impression of the skewness, since the median and mean are identical under symmetric data, but this measure is more difficult to interpret than $g_1$. \hfill \cite{statistics/book/Statistics-for-Data-Scientists/Maurits-Kaptein}

    \item Data with skewness values of $\dabs{g_1} \leq 0.3$ are considered close to symmetry, \hfill \cite{statistics/book/Statistics-for-Data-Scientists/Maurits-Kaptein}

    \item $g_1$ remains \textbf{unchanged} when all values $\dCurlyBrac{x_1, x_2, \cdots, x_n}$ are shifted by a fixed number or when they are multiplied with a fixed number. \hfill \cite{statistics/book/Statistics-for-Data-Scientists/Maurits-Kaptein} \\
    This means that shifting the data and/or multiplying the data with a fixed number does not change the “shape” of the data. \hfill \cite{statistics/book/Statistics-for-Data-Scientists/Maurits-Kaptein}
\end{enumerate}


\vspace{0.3cm}
\textbf{Formulas}:
\begin{enumerate}
    \item Sample:
    \label{Data/Describing Data/Central Tendency/Skewness/Sample}
    \label{Data/Describing Data/Central Tendency/Skewness/z-value}
    \hspace{1cm}
    $z_i = \dfrac{x_i - \bar{x}}{s}$
    \hspace{1cm}
    $g_1 = \dfrac{1}{n} \dsum_{i=1}^{n} z_i^3$

    \item Population:
\end{enumerate}


\begin{longtable}{|l|p{14cm}|}
    \hline
    \textbf{Condition} & \textbf{Conclusions}\\ \hline
    \endfirsthead

    \hline
    \textbf{Condition} & \textbf{Conclusions}\\ \hline
    \endhead

    \hline
    $g_1 > 0$ & \tableenumerate{
        \item data is called skewed to the right
        \item The values on the right side of the mean are further away from each other than the values on the left side of the mean.
        \item  In other words, the “tail” on the right is longer than the “tail” on the left.
    } \\[0.2cm] \hline

    $g_1 < 0$ & data is called skewed to the left and the tail on the left is longer than the tail on the right.\\ \hline

    $g_1 = 0$ & the data is considered symmetric around its mean.  \\ \hline
\end{longtable}


\begin{lstlisting}[
    language=Python,
    caption=Skewness - using numPy
]
x_bar = data.mean()

# sample
s2 = np.pow(data - x_bar, 2).sum() / (len(data) - 1)
s = np.sqrt(s2)
z = (data - data.mean()) / s
g1 = np.pow(z, 3).mean()
print(g1)

# population
sig2 = np.pow(data - x_bar, 2).mean()
sig = np.sqrt(sig2)
z = (data - data.mean()) / sig
g1 = np.pow(z, 3).mean()
print(g1)
\end{lstlisting}


\subsubsection{Kurtosis ( $g_2$ ) \cite{statistics/book/Statistics-for-Data-Scientists/Maurits-Kaptein}} \label{Data/Describing Data/Central Tendency/Kurtosis}

\begin{enumerate}
    \item Computed using so-called standardized values $z_i$. \hfill \cite{statistics/book/Statistics-for-Data-Scientists/Maurits-Kaptein}\\
    Standardized values have no unit and the mean and variance of the standardized values are equal to $0$ and $1$, respectively. \hfill \cite{statistics/book/Statistics-for-Data-Scientists/Maurits-Kaptein}

    \item Kurtosis is used to measure the “peakedness” of data. \hfill \cite{statistics/book/Statistics-for-Data-Scientists/Maurits-Kaptein}

    \item It is difficult to demonstrate that data is different from mesokurtic data when $g_2$ is close to zero, since it requires large sample sizes. \hfill \cite{statistics/book/Statistics-for-Data-Scientists/Maurits-Kaptein}

    \item Values of $g_2$ in the asymmetric interval of $[-0.5, 1.5]$ indicate near-mesokurtic data. \hfill \cite{statistics/book/Statistics-for-Data-Scientists/Maurits-Kaptein}

    \item $g_2$ remains \textbf{unchanged} when all values $\dCurlyBrac{x_1, x_2, \cdots, x_n}$ are shifted by a fixed number or when they are multiplied with a fixed number. \hfill \cite{statistics/book/Statistics-for-Data-Scientists/Maurits-Kaptein} \\
    This means that shifting the data and/or multiplying the data with a fixed number does not change the “shape” of the data. \hfill \cite{statistics/book/Statistics-for-Data-Scientists/Maurits-Kaptein}
\end{enumerate}


\vspace{0.3cm}
\textbf{Formulas}:
\begin{enumerate}
    \item Sample:
    \label{Data/Describing Data/Central Tendency/Kurtosis/Sample}
    \label{Data/Describing Data/Central Tendency/Kurtosis/z-value}
    \hspace{1cm}
    $z_i = \dfrac{x_i - \bar{x}}{s}$
    \hspace{1cm}
    $g_2 = \dfrac{1}{n} \dsum_{i=1}^{n} z_i^4 - 3$


    \item Population:
\end{enumerate}


\begin{longtable}{|l|l|p{10cm}|}
    \hline
    \textbf{Condition} & \textbf{Condition Name} & \textbf{Conclusions}\\ \hline
    \endfirsthead

    \hline
    \textbf{Condition} & \textbf{Condition Name} & \textbf{Conclusions}\\ \hline
    \endhead

    \hline

    $g_2 > 0$ & leptokurtic \label{Data/Describing Data/Central Tendency/Kurtosis/leptokurtic} & data has long heavy tails and is severely peaked in the middle \\ \hline

    $g_2 < 0$ & platykurtic \label{Data/Describing Data/Central Tendency/Kurtosis/platykurtic} & tails of the data are shorter with a flat peak in the middle \\ \hline

    $g_2 = 0$ & mesokurtic \label{Data/Describing Data/Central Tendency/Kurtosis/mesokurtic} &  \\ \hline
\end{longtable}

\begin{lstlisting}[
    language=Python,
    caption=Kurtosis - using numPy
]
x_bar = data.mean()

# sample
s2 = np.pow(data - x_bar, 2).sum() / (len(data) - 1)
s = np.sqrt(s2)
z = (data - data.mean()) / s
g2 = np.pow(z, 4).mean() - 3
print(g2)

# population
sig2 = np.pow(data - x_bar, 2).mean()
sig = np.sqrt(sig2)
z = (data - data.mean()) / sig
g2 = np.pow(z, 4).mean() - 3
print(g2)
\end{lstlisting}















\subsection{Aggregated Data \cite{statistics/book/Statistics-for-Data-Scientists/Maurits-Kaptein}}\label{Data/Describing Data/Aggregated Data}

\begin{enumerate}
    \item Data is often collected in intervals or groups with a frequency $f_j$ for each group $j$. ( $1 \leq j \leq m$ ) \hfill \cite{statistics/book/Statistics-for-Data-Scientists/Maurits-Kaptein}

    \item $m$: number of groups \hfill \cite{statistics/book/Statistics-for-Data-Scientists/Maurits-Kaptein}

    \item Measures of central tendency and spread can then still be computed (approximately) based on such grouped data. \hfill \cite{statistics/book/Statistics-for-Data-Scientists/Maurits-Kaptein}

    \item For each group $j$ we need to determine or set the value $x_j$ as a value that belongs to the group, before we can compute these measures. \hfill \cite{statistics/book/Statistics-for-Data-Scientists/Maurits-Kaptein}


\end{enumerate}


\begin{lstlisting}[
    language=Python
]
import random
import numpy as np

random.seed(0)
np.random.seed(0)

# generate aggregated data
low = -100
high = 100
bin_size = 5
count = 1000

data = low + (high - low) * np.random.random(size=count)
bin_edges = np.arange(low, high + bin_size, bin_size)
hist, edges = np.histogram(data, bins=bin_edges)

data_aggr = {(edges[i], edges[i+1]):hist[i] for i in range(len(hist))}
data_x = {k: np.mean(k) for k in data_aggr}
\end{lstlisting}

\subsubsection{Mean \cite{statistics/book/Statistics-for-Data-Scientists/Maurits-Kaptein}}\label{Data/Describing Data/Aggregated Data/Mean}

$
    \bar{x}
    = \dfrac{
        \dsum_{k=1}^{m} x_kf_k
    }{
        \dsum_{k=1}^{m} f_k
    }
$

\begin{lstlisting}[
    language=Python,
    caption=Aggregated Data - Mean - using numPy
]
xs = np.array(list(data_x.values()))
fs = np.array(list(data_aggr.values()))
print((xs * fs).sum()/ fs.sum())
\end{lstlisting}



\subsubsection{Variance \cite{statistics/book/Statistics-for-Data-Scientists/Maurits-Kaptein}}\label{Data/Describing Data/Aggregated Data/Variance}

$
    s^2
    = \dfrac{
        \dsum_{k=1}^{m} (x_k - \bar{x})^2 f_k
    }{
        -1 + \dsum_{k=1}^{m} f_k
    }
$

\begin{lstlisting}[
    language=Python,
    caption=Aggregated Data - Variance - using numPy
]
xs = np.array(list(data_x.values()))
fs = np.array(list(data_aggr.values()))

x_bar = xs.mean()

# sample
print((np.pow(xs - x_bar, 2) * fs).sum() / (fs.sum() - 1))

# population
print((np.pow(xs - x_bar, 2) * fs).sum() / fs.sum())
\end{lstlisting}














