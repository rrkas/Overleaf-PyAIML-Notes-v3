\chapter{Data}

\section{Measurement Levels \cite{statistics/book/Statistics-for-Data-Scientists/Maurits-Kaptein}}\label{Data/Measurement-Levels}

\subsection{Nominal, Ordinal, Interval and Ratio \cite{statistics/book/Statistics-for-Data-Scientists/Maurits-Kaptein}}\label{Data/Measurement-Levels/Nominal, Ordinal, Interval and Ratio}

\label{Data/Measurement-Levels/Nominal, Ordinal, Interval and Ratio/Categorical Data}
\label{Data/Measurement-Levels/Nominal, Ordinal, Interval and Ratio/Numerical Data}
\label{Data/Measurement-Levels/Nominal, Ordinal, Interval and Ratio/Nominal}
\label{Data/Measurement-Levels/Nominal, Ordinal, Interval and Ratio/Ordinal}
\label{Data/Measurement-Levels/Nominal, Ordinal, Interval and Ratio/Interval}
\label{Data/Measurement-Levels/Nominal, Ordinal, Interval and Ratio/Ratio}

\begin{table}[H]
    \hfill
    \begin{minipage}[H]{0.25\linewidth}
        \textbf{Levels}: \cite{statistics/book/Statistics-for-Data-Scientists/Maurits-Kaptein}
        \begin{enumerate}
            \item Nominal
            \item Ordinal
            \item Interval
            \item Ratio
        \end{enumerate}
    \end{minipage}
    \hfill
    \begin{minipage}[H]{0.65\linewidth}
        \begin{table}[H]
            \centering
            \begin{tabular}{|p{5cm}|c|c|c|c|}
                \hline
                & \multicolumn{2}{c|}{\textbf{Categorical Data}} & \multicolumn{2}{c|}{\textbf{Numerical Data}} \\ 
                
                \hline
                & \textbf{Nominal} & \textbf{Ordinal} & \textbf{Interval} & \textbf{Ratio} \\ \hline
                
                Distinction between groups / individuals & \checkmark & \checkmark & \checkmark & \checkmark \\ \hline
                
                Imposes logical Order & \xmark & \checkmark & \checkmark & \checkmark \\ \hline
                
                Provides a magnitude of the differences in some unit & \xmark & \xmark & \checkmark & \checkmark \\ \hline
                
                A clear reference point or "0" & \xmark & \xmark & \xmark & \checkmark \\ \hline
            \end{tabular}
            \caption{Data: Measurement Levels: Nominal, Ordinal, Interval and Ratio \cite{statistics/book/Statistics-for-Data-Scientists/Maurits-Kaptein}}
        \end{table}
    \end{minipage}
    \hfill
\end{table}

\vspace{0.3cm}

\textbf{Note}:
\begin{enumerate}
    \item Each consecutive measurement level contains as much "information" - in a fairly loose sense of the word - as the previous one and more. \cite{statistics/book/Statistics-for-Data-Scientists/Maurits-Kaptein}
\end{enumerate}


\subsection{Continuous vs Discrete numerical data \cite{statistics/book/Statistics-for-Data-Scientists/Maurits-Kaptein}}\label{Data/Measurement-Levels/Continuous vs Discrete numerical data}

\label{Data/Measurement-Levels/Continuous vs Discrete numerical data/Continuous numerical data}
\label{Data/Measurement-Levels/Continuous vs Discrete numerical data/Discrete numerical data}

\begin{enumerate}
    \item Continuous variables can assume any value.\\
    This means that the continuous variable can attain any value between two different values, no matter how close the two values are.\\
    \textbf{Example}: temperature, weight, and age

    \item Discrete variables cannot assume any value between 2 values\\
    \textbf{Example}: number of text messages, accidents, microorganisms, students, etc.
\end{enumerate}


\subsection{Outliers \cite{statistics/book/Statistics-for-Data-Scientists/Maurits-Kaptein}}\label{Data/Measurement-Levels/Outliers}

\begin{enumerate}
    \item An outlier is a data point that significantly deviates from other observations in a dataset. \cite{common/online/chatgpt}

    \item Caused by Natural variability in the data or measurement errors. \cite{common/online/chatgpt}

    \item Typically identified using statistical methods like the IQR (Interquartile Range), Z-score, or visualization techniques (e.g., box plots). \cite{common/online/chatgpt}

    \item Outliers are not necessarily incorrect; they may represent rare but valid observations. \cite{common/online/chatgpt}
    
\end{enumerate}


\vspace{0.3cm}

\textbf{Examples}:
\begin{enumerate}
    \item In a dataset of human heights, a person measuring 250 cm might be an outlier but not necessarily unrealistic if it’s a rare case of gigantism. \cite{common/online/chatgpt}
\end{enumerate}


\vspace{0.3cm}
\textbf{Handling/ Dealing with Outliers}:
\begin{enumerate}
    \item Ignore these abnormalities and go ahead with the data. \cite{statistics/book/Statistics-for-Data-Scientists/Maurits-Kaptein}

    \item Delete/ remove the suspected records/ entries. \cite{statistics/book/Statistics-for-Data-Scientists/Maurits-Kaptein}

    \item Substitute them, using statistical methods, with a more plausible alternative. (aka \textbf{imputation}) \cite{statistics/book/Statistics-for-Data-Scientists/Maurits-Kaptein}\label{Data/Outliers/imputation}
\end{enumerate}




\subsection{Unrealistic Values \cite{statistics/book/Statistics-for-Data-Scientists/Maurits-Kaptein}}\label{Data/Measurement-Levels/Unrealistic Values}

\begin{enumerate}
    \item An unrealistic value is a data point that is not plausible within the context of the dataset, often due to data entry errors or faulty sensors. \cite{common/online/chatgpt}

    \item Caused by Human error, sensor malfunction, or corruption during data transmission. \cite{common/online/chatgpt}

    \item Typically identified using domain knowledge or logical constraints. \cite{common/online/chatgpt}

    \item Unlike outliers, unrealistic values are generally not useful and need correction or removal. \cite{common/online/chatgpt}

\end{enumerate}

\vspace{0.3cm}

\textbf{Examples}:
\begin{enumerate}
    \item A recorded body temperature of 200°C for a human is unrealistic, as it’s physically impossible for a person to survive at that temperature. \cite{common/online/chatgpt}

    \item Negative age of a person \cite{common/online/chatgpt}

    \item Missing values \cite{statistics/book/Statistics-for-Data-Scientists/Maurits-Kaptein}

    \item Incorrect datatype of value \cite{statistics/book/Statistics-for-Data-Scientists/Maurits-Kaptein}
\end{enumerate}

\vspace{0.3cm}
\textbf{Handling/ Dealing with Unrealistic values}:
\begin{enumerate}
    \item Delete/ remove the suspected records/ entries. \cite{statistics/book/Statistics-for-Data-Scientists/Maurits-Kaptein}
    
\end{enumerate}





\section{Describing Data \cite{statistics/book/Statistics-for-Data-Scientists/Maurits-Kaptein}} \label{Data/Describing Data}

Some \textbf{descriptive statistics}\label{Data/Describing Data/descriptive statistics} (or just \textbf{descriptives}\label{Data/Describing Data/descriptives}) that we introduce are often used for data of a certain measurement level. \cite{statistics/book/Statistics-for-Data-Scientists/Maurits-Kaptein}

\subsection{Frequency/ Frequency table \cite{statistics/book/Statistics-for-Data-Scientists/Maurits-Kaptein}}\label{Data/Describing Data/Frequency or Frequency table}

\textbf{Measurement levels}: Nominal and ordinal data

\vspace{0.3cm}

\begin{enumerate}
    \item Frequencies are often uninformative for interval or ratio variables. \cite{statistics/book/Statistics-for-Data-Scientists/Maurits-Kaptein}\\
        if there are lots and lots of different possible values, all of them will have a count of just one. \cite{statistics/book/Statistics-for-Data-Scientists/Maurits-Kaptein}\\
        This is often tackled by discretizing (or "\textbf{binning}”\label{Data/Describing Data/Frequency or Frequency table/binning}) the variable (which, note, effectively "throws away” some of the information in the data). \cite{statistics/book/Statistics-for-Data-Scientists/Maurits-Kaptein}

    
\end{enumerate}


\subsubsection{(Absolute) Frequency/ (Absolute) Frequency table \cite{statistics/book/Statistics-for-Data-Scientists/Maurits-Kaptein}}\label{Data/Describing Data/Frequency or Frequency table/Absolute}

\begin{enumerate}
    \item It refers to the count of occurrences of a particular value or category in a dataset. \cite{common/online/chatgpt}

    \item Simple count, no further processing. \cite{common/online/chatgpt}

    \item \textbf{Use Case}: Helpful in creating bar charts or histograms. \cite{common/online/chatgpt}
\end{enumerate}



\subsubsection{Cumulative Frequency/ Cumulative Frequency table \cite{statistics/book/Statistics-for-Data-Scientists/Maurits-Kaptein}}\label{Data/Describing Data/Frequency or Frequency table/Cumulative}

\begin{enumerate}
    \item It is the running total of frequencies up to a certain value or class. \cite{common/online/chatgpt}

    \item Each cumulative frequency includes its own frequency plus all previous frequencies. \cite{common/online/chatgpt}

    \item \textbf{Use Case}: Useful in percentile calculations and ogive graphs. \cite{common/online/chatgpt}

    \item The cumulative frequency makes more sense for ordinal data than for nominal data, since ordinal data can be ordered in size, which is not possible for nominal data. \cite{statistics/book/Statistics-for-Data-Scientists/Maurits-Kaptein}
\end{enumerate}

\begin{table}[H]
    \begin{minipage}[H]{0.3\linewidth}
    $
        \begin{aligned}
            CF_i 
                &= CF_{i-1} + F_{i} \\
                &= \sum_{k=1}^{i} F_{k}
        \end{aligned}
    $
    \end{minipage}
    \begin{minipage}[H]{0.65\linewidth}
        \begin{table}[H]
            \begin{tabular}{l l}
                $CF_i$ & Cumulative Frequency at the current value \\ 
                $CF_{i-1}$ & Cumulative Frequency at the previous value \\ 
                $F_i$ & Frequency at the current value \\ 
            \end{tabular}
            \caption*{Notations}
        \end{table}
    \end{minipage}
\end{table}


\subsubsection{Relative Frequency/ Relative Frequency table \cite{statistics/book/Statistics-for-Data-Scientists/Maurits-Kaptein}}\label{Data/Describing Data/Frequency or Frequency table/Relative}

\begin{enumerate}
    \item It shows the proportion of each category relative to the total number of observations. \cite{common/online/chatgpt}

    \item Expressed as a fraction, decimal, or percentage. \cite{common/online/chatgpt}

    \item \textbf{Use Case}: Ideal for creating pie charts and understanding distribution proportions. \cite{common/online/chatgpt}
\end{enumerate}


\begin{table}[H]
    \begin{minipage}{0.3\linewidth}
        \[
            \begin{aligned}
                RF_i 
                    &= \dfrac{F_{i}}{\dsum_{k=1}^{N} F_{k}}
            \end{aligned}
        \]
    \end{minipage}
    \begin{minipage}{0.65\linewidth}
        \begin{table}[H]
            \begin{tabular}{l l}
                $RF_i$ & Relative Frequency \\
                $F_i$ & Frequency of the value \\ 
                $N$ & Total number of observations \\ 
            \end{tabular}
            \caption*{Notations}
        \end{table}
    \end{minipage}
\end{table}




\subsubsection{Cumulative Relative Frequency/ Cumulative Relative Frequency table \cite{statistics/book/Statistics-for-Data-Scientists/Maurits-Kaptein}}\label{Data/Describing Data/Frequency or Frequency table/Cumulative Relative}

\begin{enumerate}
    \item Cumulative relative frequency is the accumulation of the relative frequencies of data points up to a certain value. \cite{common/online/chatgpt}

    \item It indicates the proportion of data points that are less than or equal to a particular value. \cite{common/online/chatgpt}

    \item \textbf{Use Cases}:
    \begin{enumerate}
        \item Identifying percentiles and median.

        \item Visualizing with a cumulative relative frequency graph (Ogive).

        \item Understanding data distribution by determining the proportion of values below a specific threshold.
    \end{enumerate}
\end{enumerate}



\begin{table}[H]
    \begin{minipage}{0.3\linewidth}
        \[
            \begin{aligned}
                CRF_i 
                    &= \dfrac{\dsum_{k=1}^{i} F_{k}}{\dsum_{k=1}^{N} F_{k}}
            \end{aligned}
        \]
    \end{minipage}
    \begin{minipage}{0.65\linewidth}
        \begin{table}[H]
            \begin{tabular}{l l}
                $CRF_i$ & Cumulative Relative Frequency \\
                $F_i$ & Frequency of the value \\ 
                $N$ & Total number of observations \\ 
            \end{tabular}
            \caption*{Notations}
        \end{table}
    \end{minipage}
\end{table}





\subsection{Central Tendency \cite{statistics/book/Statistics-for-Data-Scientists/Maurits-Kaptein}}\label{Data/Describing Data/Central Tendency}

\begin{enumerate}
     \item When we work with numerical data, we often want to know something about the "central value" or "middle value" of the variable, also referred to as the \textbf{location}\label{Data/Describing Data/Central Tendency/location} of the data. \cite{statistics/book/Statistics-for-Data-Scientists/Maurits-Kaptein}
\end{enumerate}


\subsubsection{(Arithmetic) mean/ average \cite{statistics/book/Statistics-for-Data-Scientists/Maurits-Kaptein}} \label{Data/Describing Data/Central Tendency/(Arithmetic) mean or average}

\begin{table}[H]
    \begin{minipage}{0.3\linewidth}
        $
            \bar{x} = \dfrac{1}{n} \dsum_{i=1}^{n} x_i
        $
    \end{minipage}
    \begin{minipage}{0.65\linewidth}
        \begin{table}[H]
            \begin{tabular}{l l}
                $\bar{x}$ & mean \\
                $x_i$ & item \\
                $n$ & number of items \\
            \end{tabular}
            \caption*{Notations}
        \end{table}
    \end{minipage}
\end{table}



\subsubsection{Mode \cite{statistics/book/Statistics-for-Data-Scientists/Maurits-Kaptein}} \label{Data/Describing Data/Central Tendency/Mode}

\begin{enumerate}
    \item The mode is merely the most frequently occurring value. \cite{statistics/book/Statistics-for-Data-Scientists/Maurits-Kaptein}

    \item There might be multiple modes. \cite{statistics/book/Statistics-for-Data-Scientists/Maurits-Kaptein}
    
\end{enumerate}



\subsubsection{Median \cite{statistics/book/Statistics-for-Data-Scientists/Maurits-Kaptein}} \label{Data/Describing Data/Central Tendency/Median}

\begin{enumerate}
    \item The median is a value that divides the ordered data from small to large (or large to small) into two equal parts: 50\% of the data is below the median and 50\% is above. \cite{statistics/book/Statistics-for-Data-Scientists/Maurits-Kaptein}

    \item The median is not necessarily a value that is present in the data. \cite{statistics/book/Statistics-for-Data-Scientists/Maurits-Kaptein}
\end{enumerate}


\vspace{0.3cm}
\textbf{Steps}:
\begin{enumerate}
    \item sort the data

    \item choose the middle-most value when $n$ is \textbf{odd}\\
        average of the two middle values when $n$ is \textbf{even}
\end{enumerate}



\subsubsection{Quantiles \cite{statistics/book/Statistics-for-Data-Scientists/Maurits-Kaptein}} \label{Data/Describing Data/Central Tendency/Quantiles}

\begin{enumerate}
    \item A quantile $x_q$ is a value that splits the ordered data of a variable $x$ into two parts: \cite{statistics/book/Statistics-for-Data-Scientists/Maurits-Kaptein}
    \begin{enumerate}
        \item $q \cdot 100\%$ of the data is below the value $x_q$

        \item $(1 - q) \cdot 100\%$ of the data is above the value $x_q$
    \end{enumerate}
    
    \item The parameter $q$ can take any value in the interval $[0, 1]$. \cite{statistics/book/Statistics-for-Data-Scientists/Maurits-Kaptein}

    \item Quantiles can be calculated in different ways, depending on the way we "interpolate" between two values. \cite{statistics/book/Statistics-for-Data-Scientists/Maurits-Kaptein} \\
    We could map the ordered values \textit{equally spaced} on the interval $(0, 1)$, where the \textit{i}th ordered value of the data is positioned at the level $q_i = {i}/{(n + 1)}$ in the interval $(0, 1)$, with $n$ being the number of data points. \cite{statistics/book/Statistics-for-Data-Scientists/Maurits-Kaptein} \\
    R uses $q_i = (i - 1)/(n - 1)$ for quantiles. \cite{statistics/book/Statistics-for-Data-Scientists/Maurits-Kaptein} \\
    \textbf{Example}: \cite{statistics/book/Statistics-for-Data-Scientists/Maurits-Kaptein}
    \begin{enumerate}
        \item Data points: $\dCurlyBrac{2, 5, 6, 4}$ ($n=4$)
        \item Sorted Data points: $\dCurlyBrac{2, 4, 5, 6}$ ($n=4$)
        \item Quantiles:\\[0.2cm]
        \begin{tabular}{|l|c|c|c|}
            \hline
            $i$ & $x_i$ & $q_i = i/(n+1) = i/5$ & $q_i = (i-1)/(n-1) = (i-1)/3$ \\ [0.1cm]
            \hline
            $1$ & $2$ & $1/5 = 0.2$ & $0$ \\
            $2$ & $4$ & $2/5 = 0.4$ & $1/3$ \\
            $3$ & $5$ & $3/5 = 0.6$ & $2/3$ \\
            $4$ & $6$ & $4/5 = 0.8$ & $1$ \\
            \hline
        \end{tabular}\\

        \item If $x_i = 3 \Rightarrow q_i = 0.3$
    \end{enumerate}
\end{enumerate}




\subsubsection{Quartiles \cite{statistics/book/Statistics-for-Data-Scientists/Maurits-Kaptein}} \label{Data/Describing Data/Central Tendency/Quartiles}

\begin{enumerate}
    \item When $q = 0.25$, $q = 0.50$, and $q = 0.75$ the quantiles are referred to as the first, second, and third quartiles, respectively. \cite{statistics/book/Statistics-for-Data-Scientists/Maurits-Kaptein}
    \label{Data/Describing Data/Central Tendency/Quartiles/first quartile}
    \label{Data/Describing Data/Central Tendency/Quartiles/second quartile}
    \label{Data/Describing Data/Central Tendency/Quartiles/third quartile}

    \item Splits: \\
    \begin{tabular}{r l l l} % Right-align first column, left-align second column
        1. & $q = 0$ & to & $q = 0.25$ \\
        2. & $q = 0.25$ & to & $q = 0.5$ \\
        3. & $q = 0.5$ & to & $q = 0.75$ \\
        4. & $q = 0.75$ & to & $q = 1$ \\
    \end{tabular}
\end{enumerate}



\subsubsection{Deciles \cite{statistics/book/Statistics-for-Data-Scientists/Maurits-Kaptein}} \label{Data/Describing Data/Central Tendency/Deciles}

\begin{enumerate}
    \item We call quantiles deciles when $q$ is restricted to the set $\dCurlyBrac{0.1, 0.2,\cdots, 0.9}$

    \item Splits: \\
    \begin{tabular}{r l l l} % Right-align first column, left-align second column
        1. & $q = 0$ & to & $q = 0.1$ \\
        2. & $q = 0.1$ & to & $q = 0.2$ \\
        && \vdots & \\
        9. & $q = 0.8$ & to & $q = 0.9$ \\
        10. & $q = 0.9$ & to & $q = 1$ \\
    \end{tabular}
\end{enumerate}



\subsubsection{Percentiles \cite{statistics/book/Statistics-for-Data-Scientists/Maurits-Kaptein}} \label{Data/Describing Data/Central Tendency/Percentiles}

\begin{enumerate}
    \item We call quantiles percentiles when $q$ is restricted to the set $\dCurlyBrac{0.01, 0.02,\cdots, 0.99}$

    \item Splits: \\
    \begin{tabular}{r l l l} % Right-align first column, left-align second column
        1. & $q = 0$ & to & $q = 0.01$ \\
        2. & $q = 0.01$ & to & $q = 0.02$ \\
        & & \vdots & \\
        99. & $q = 0.98$ & to & $q = 0.99$ \\
        100. & $q = 0.99$ & to & $q = 1$ \\
    \end{tabular}
\end{enumerate}








