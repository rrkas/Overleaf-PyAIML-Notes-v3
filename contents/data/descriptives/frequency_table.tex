\subsection{Frequency/ Frequency table \cite{statistics/book/Statistics-for-Data-Scientists/Maurits-Kaptein}}\label{Data/Describing Data/Frequency or Frequency table}

\textbf{Measurement levels}: Nominal and ordinal data

\vspace{0.3cm}

\begin{enumerate}
    \item Frequencies are often uninformative for interval or ratio variables. \cite{statistics/book/Statistics-for-Data-Scientists/Maurits-Kaptein}\\
        if there are lots and lots of different possible values, all of them will have a count of just one. \cite{statistics/book/Statistics-for-Data-Scientists/Maurits-Kaptein}\\
        This is often tackled by discretizing (or "\textbf{binning}”\label{Data/Describing Data/Frequency or Frequency table/binning}) the variable (which, note, effectively "throws away” some of the information in the data). \cite{statistics/book/Statistics-for-Data-Scientists/Maurits-Kaptein}

    
\end{enumerate}


\subsubsection{(Absolute) Frequency/ (Absolute) Frequency table \cite{statistics/book/Statistics-for-Data-Scientists/Maurits-Kaptein}}\label{Data/Describing Data/Frequency or Frequency table/Absolute}

\begin{enumerate}
    \item It refers to the count of occurrences of a particular value or category in a dataset. \cite{common/online/chatgpt}

    \item Simple count, no further processing. \cite{common/online/chatgpt}

    \item \textbf{Use Case}: Helpful in creating bar charts or histograms. \cite{common/online/chatgpt}
\end{enumerate}



\subsubsection{Cumulative Frequency/ Cumulative Frequency table \cite{statistics/book/Statistics-for-Data-Scientists/Maurits-Kaptein}}\label{Data/Describing Data/Frequency or Frequency table/Cumulative}

\begin{enumerate}
    \item It is the running total of frequencies up to a certain value or class. \cite{common/online/chatgpt}

    \item Each cumulative frequency includes its own frequency plus all previous frequencies. \cite{common/online/chatgpt}

    \item \textbf{Use Case}: Useful in percentile calculations and ogive graphs. \cite{common/online/chatgpt}

    \item The cumulative frequency makes more sense for ordinal data than for nominal data, since ordinal data can be ordered in size, which is not possible for nominal data. \cite{statistics/book/Statistics-for-Data-Scientists/Maurits-Kaptein}
\end{enumerate}

\begin{table}[H]
    \begin{minipage}[H]{0.3\linewidth}
    $
        \begin{aligned}
            CF_i 
                &= CF_{i-1} + F_{i} \\
                &= \sum_{k=1}^{i} F_{k}
        \end{aligned}
    $
    \end{minipage}
    \begin{minipage}[H]{0.65\linewidth}
        \begin{table}[H]
            \begin{tabular}{l l}
                $CF_i$ & Cumulative Frequency at the current value \\ 
                $CF_{i-1}$ & Cumulative Frequency at the previous value \\ 
                $F_i$ & Frequency at the current value \\ 
            \end{tabular}
            \caption*{Notations}
        \end{table}
    \end{minipage}
\end{table}


\subsubsection{Relative Frequency/ Relative Frequency table \cite{statistics/book/Statistics-for-Data-Scientists/Maurits-Kaptein}}\label{Data/Describing Data/Frequency or Frequency table/Relative}

\begin{enumerate}
    \item It shows the proportion of each category relative to the total number of observations. \cite{common/online/chatgpt}

    \item Expressed as a fraction, decimal, or percentage. \cite{common/online/chatgpt}

    \item \textbf{Use Case}: Ideal for creating pie charts and understanding distribution proportions. \cite{common/online/chatgpt}
\end{enumerate}


\begin{table}[H]
    \begin{minipage}{0.3\linewidth}
        \[
            \begin{aligned}
                RF_i 
                    &= \dfrac{F_{i}}{\dsum_{k=1}^{N} F_{k}}
            \end{aligned}
        \]
    \end{minipage}
    \begin{minipage}{0.65\linewidth}
        \begin{table}[H]
            \begin{tabular}{l l}
                $RF_i$ & Relative Frequency \\
                $F_i$ & Frequency of the value \\ 
                $N$ & Total number of observations \\ 
            \end{tabular}
            \caption*{Notations}
        \end{table}
    \end{minipage}
\end{table}




\subsubsection{Cumulative Relative Frequency/ Cumulative Relative Frequency table \cite{statistics/book/Statistics-for-Data-Scientists/Maurits-Kaptein}}\label{Data/Describing Data/Frequency or Frequency table/Cumulative Relative}

\begin{enumerate}
    \item Cumulative relative frequency is the accumulation of the relative frequencies of data points up to a certain value. \cite{common/online/chatgpt}

    \item It indicates the proportion of data points that are less than or equal to a particular value. \cite{common/online/chatgpt}

    \item \textbf{Use Cases}:
    \begin{enumerate}
        \item Identifying percentiles and median.

        \item Visualizing with a cumulative relative frequency graph (Ogive).

        \item Understanding data distribution by determining the proportion of values below a specific threshold.
    \end{enumerate}
\end{enumerate}



\begin{table}[H]
    \begin{minipage}{0.3\linewidth}
        \[
            \begin{aligned}
                CRF_i 
                    &= \dfrac{\dsum_{k=1}^{i} F_{k}}{\dsum_{k=1}^{N} F_{k}}
            \end{aligned}
        \]
    \end{minipage}
    \begin{minipage}{0.65\linewidth}
        \begin{table}[H]
            \begin{tabular}{l l}
                $CRF_i$ & Cumulative Relative Frequency \\
                $F_i$ & Frequency of the value \\ 
                $N$ & Total number of observations \\ 
            \end{tabular}
            \caption*{Notations}
        \end{table}
    \end{minipage}
\end{table}




