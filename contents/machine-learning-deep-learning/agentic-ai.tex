\chapter{Agentic AI}

\begin{enumerate}
    \item Agentic AI is an artificial intelligence system that can accomplish a specific goal with \textbf{limited supervision}. 
    \hfill \cite{www.ibm.com/think/topics/agentic-ai}
    
    \item It consists of \textbf{AI agents}—machine learning models that mimic human decision-making to solve problems in real time. 
    \hfill \cite{www.ibm.com/think/topics/agentic-ai}
    
    \item In a \textbf{multiagent system}, each agent performs a specific subtask required to reach the goal and their efforts are coordinated through AI orchestration.
    \hfill \cite{www.ibm.com/think/topics/agentic-ai}

    \item Unlike traditional AI models, which operate within predefined constraints and require human intervention, agentic AI exhibits autonomy, goal-driven behavior and adaptability. 
    \hfill \cite{www.ibm.com/think/topics/agentic-ai}
    
    \item The term “agentic” refers to these models’ agency, or, their capacity to act independently and purposefully.
    \hfill \cite{www.ibm.com/think/topics/agentic-ai}

    \item Agentic AI builds on \textbf{generative AI} (gen AI) techniques by using large language models (LLMs) to function in dynamic environments. 
    While generative models focus on creating content based on learned patterns, agentic AI extends this capability by applying generative outputs toward specific goals.
    \hfill \cite{www.ibm.com/think/topics/agentic-ai}

    \item \textbf{Example}: A generative AI model like OpenAI’s ChatGPT might produce text, images or code, but an agentic AI system can use that generated content to complete complex tasks autonomously by calling external tools. 
    Agents can, for example, not only tell you the best time to climb Mt. Everest given your work schedule, it can also book you a flight and a hotel.
    \hfill \cite{www.ibm.com/think/topics/agentic-ai}
\end{enumerate}






\section{How agentic AI works}

\begin{enumerate}
    \item \textbf{Perception}: Agentic AI begins by collecting data from its environment through sensors, APIs, databases or user interactions. 
    This step ensures that the system has up-to-date information to analyze and act upon.
    \hfill \cite{www.ibm.com/think/topics/agentic-ai}

    \item \textbf{Reasoning}: Once the data is collected, the AI processes it to extract meaningful insights. Using natural language processing (NLP), computer vision or other AI capabilities, it interprets user queries, detects patterns and understands the broader context. 
    This ability helps the AI determine what actions to take based on the situation.
    \hfill \cite{www.ibm.com/think/topics/agentic-ai}

    \item \textbf{Goal setting}: The AI sets objectives based on predefined goals or user inputs. It then develops a strategy to achieve these goals, often by using decision trees, reinforcement learning or other planning algorithms.
    \hfill \cite{www.ibm.com/think/topics/agentic-ai}

    \item \textbf{Decision-making}: AI evaluates multiple possible actions and chooses the optimal one based on factors such as efficiency, accuracy and predicted outcomes. 
    It might use probabilistic models, utility functions or machine learning-based reasoning to determine the best course of action.
    \hfill \cite{www.ibm.com/think/topics/agentic-ai}

    \item \textbf{Execution}: After selecting an action, the AI executes it, either by interacting with external systems (APIs, data, robots) or providing responses to users.
    \hfill \cite{www.ibm.com/think/topics/agentic-ai}

    \item \textbf{Learning and adaptation}: After executing an action, the AI evaluates the outcome, gathering feedback to improve future decisions. 
    Through reinforcement learning or self-supervised learning, the AI refines its strategies over time, making it more effective in handling similar tasks in the future.
    \hfill \cite{www.ibm.com/think/topics/agentic-ai}

    \item \textbf{Orchestration}: AI orchestration is the coordination and management of systems and agents. 
    Orchestration platforms automate AI workflows, track progress toward task completion, manage resource usage, monitor data flow and memory and handle failure events. 
    With the right architecture, dozens, hundreds or even thousands of agents could theoretically work together in harmonious productivity.
    \hfill \cite{www.ibm.com/think/topics/agentic-ai}
\end{enumerate}



\section{Advantages}

\subsection*{Autonomous}

\begin{enumerate}
    \item The most important advancement of agentic systems is that they allow for autonomy to perform tasks without constant human oversight. 
    \hfill \cite{www.ibm.com/think/topics/agentic-ai}
    
    \item Agentic systems can maintain long-term goals, manage multistep problem-solving tasks and track progress over time.
    \hfill \cite{www.ibm.com/think/topics/agentic-ai}
\end{enumerate}

\subsection*{Proactive}

\begin{enumerate}
    \item Agentic systems provide the flexibility of LLMs, which can generate responses or actions based on nuanced, context-dependent understanding, with the structured, deterministic and reliable features of traditional programming. 
    \hfill \cite{www.ibm.com/think/topics/agentic-ai}
    
    \item This approach allows agents to “think” and “do” in a more human-like fashion.
    \hfill \cite{www.ibm.com/think/topics/agentic-ai}

    \item LLMs by themselves can’t directly interact with external tools or databases or set up systems to monitor and collect data in real time, but agents can. 
    \hfill \cite{www.ibm.com/think/topics/agentic-ai}
    
    \item Agents can search the web, call application programming interfaces (APIs) and query databases, then use this information to make decisions and take actions.
    \hfill \cite{www.ibm.com/think/topics/agentic-ai}
\end{enumerate}

\subsection*{Specialized}

\begin{enumerate}
    \item Agents can specialize in specific tasks. Some agents are simple, performing a single repetitive task reliably. 
    \hfill \cite{www.ibm.com/think/topics/agentic-ai}
    
    \item Others can use perception and draw on memory to solve more complex problems. 
    \hfill \cite{www.ibm.com/think/topics/agentic-ai}
    
    \item An agentic architecture might consist of a “conductor” model powered by an LLM that oversees tasks and decisions and supervises other, simpler agents. 
    Such architectures are ideal for sequential workflows but are vulnerable to bottlenecks. 
    \hfill \cite{www.ibm.com/think/topics/agentic-ai}
    
    \item Other architectures are more horizontal, with agents working in harmony as equals in a decentralized fashion, but this architecture can be slower than a vertical hierarchy. 
    \hfill \cite{www.ibm.com/think/topics/agentic-ai}
\end{enumerate}


\subsection*{Adaptable}

\begin{enumerate}
    \item Agents can learn from their experiences, take in feedback and adjust their behavior. 
    \hfill \cite{www.ibm.com/think/topics/agentic-ai}
    
    \item With the right guardrails, agentic systems can improve continuously. 
    \hfill \cite{www.ibm.com/think/topics/agentic-ai}
    
    \item Multiagent systems possess the scalability to eventually handle broadly scoped initiatives.
    \hfill \cite{www.ibm.com/think/topics/agentic-ai}
\end{enumerate}

\subsection*{Intuitive}

\begin{enumerate}
    \item Because agentic systems are powered by LLMs, users can engage with them with natural language prompts. 
    This means that entire software interfaces—think of the many tabs, dropdowns, charts, sliders, pop-ups and other UI elements involved in the SaaS platform of one’s choice—can be replaced by simple language or voice commands.
    \hfill \cite{www.ibm.com/think/topics/agentic-ai}

    \item Theoretically, any software user experience can now be reduced to “talking” with an agent, who can fetch the information one needs and take action based on that information. 
    This productivity benefit can barely be overstated, when one considers the time it takes for workers to learn and master new interfaces and tools.
    \hfill \cite{www.ibm.com/think/topics/agentic-ai}
\end{enumerate}





\section{Challenges}


\begin{enumerate}
    \item The autonomous nature can bring serious consequences if agentic systems go “off the rails.” 
    The usual AI risks apply, but can be magnified in agentic systems.
    \hfill \cite{www.ibm.com/think/topics/agentic-ai}

    \item Many agentic AI systems use reinforcement learning, which involves maximizing a reward function. 
    If the reward system is poorly designed, the AI might exploit loopholes to achieve “high scores” in unintended ways.
    \hfill \cite{www.ibm.com/think/topics/agentic-ai}

    \item Some agentic AI systems can become self-reinforcing, escalating behaviors in an unintended direction. 
    This issue happens when the AI optimizes too aggressively for a particular metric without safeguards.
    \hfill \cite{www.ibm.com/think/topics/agentic-ai}

    \item Because agentic systems are often composed of multiple autonomous agents working together, there are opportunities for failure. 
    \hfill \cite{www.ibm.com/think/topics/agentic-ai}
    
    \item Traffic jams, bottlenecks, resource conflicts—all of these errors have the potential to cascade.
    \hfill \cite{www.ibm.com/think/topics/agentic-ai}
\end{enumerate}




